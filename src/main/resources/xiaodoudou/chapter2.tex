\ruby{新}{あたら}しい\ruby{学校}{がっこう}の\ruby{門}{もん}をくぐる\ruby{前}{まえ}に、トットちゃんのママが、なぜ\ruby{不安}{ふあん}なのかを\ruby{説明}{せつめい}すると、それはトットちゃんが、\ruby{小学校}{しょうがっこう}一\ruby{年}{ねん}なのにかかわらず、すでに\ruby{学校}{がっこう}を\ruby{退学}{たいがく}になったからだった。\ruby{一年生}{いちねんせい}で!!

つい\ruby{先週}{せんしゅう}のことだった。ママはトットちゃんの\ruby{担任}{たんにん}の\ruby{先生}{せんせい}に\ruby{呼}{よ}ばれて、はっきり、こういわれた。

「お\ruby{宅}{たく}のお\ruby{嬢}{じょう}さんがいると、クラス\ruby{中}{じゅう}の\ruby{迷惑}{めいわく}になります。よその\ruby{学校}{がっこう}にお\ruby{連}{つ}れください!」 \ruby{若}{わか}くて\ruby{美}{うつく}しい\ruby{女}{おんな}の\ruby{先生}{せんせい}は、ため\ruby{息}{いき}をつきながら、\ruby{繰}{く}り\ruby{返}{かえ}した。 「\ruby{本当}{ほんとう}に\ruby{困}{こま}ってるんです!」 ママはびっくりした。(\ruby{一体}{いったい}、どんなことを……。クラス\ruby{中}{じゅう}の\ruby{迷惑}{めいわく}になる、どんなことを、あの\ruby{子}{こ}がするんだろうか……)

\ruby{先生}{せんせい}は、カールしたまつ\ruby{毛}{げ}をパチパチさせ、パーマのかかった\ruby{短}{みじか}い\ruby{内巻}{うちまき}の\ruby{毛}{け}を\ruby{手}{て}でなでながら\ruby{説明}{せつめい}に\ruby{取}{と}り\ruby{掛}{か}かった。

「まず、\ruby{授業}{じゅぎょう}\ruby{中}{ちゅう}に、\ruby{机}{つくえ}のフタを、百ぺんくらい、あけたり\ruby{閉}{し}めたりするんです。そこで\ruby{私}{わたし}が、\ruby{用事}{ようじ}がないのに、\ruby{開}{あ}けたり\ruby{閉}{し}めたりしてはいけませんと\ruby{申}{もう}しますと、お\ruby{宅}{たく}のお\ruby{嬢}{じょう}さんは、ノートから、\ruby{筆箱}{ふでばこ}、\ruby{教科書}{きょうかしょ}、\ruby{全部}{ぜんぶ}を\ruby{机}{つくえ}の\ruby{中}{なか}にしまってしまって、\ruby{一}{ひと}つ\ruby{一}{ひと}つ\ruby{取}{と}り\ruby{出}{だ}すんです。たとえば、\ruby{書}{か}き\ruby{取}{と}りをするとしますね。するとお\ruby{嬢}{じょう}さんは、まずフタを\ruby{開}{あ}けて、ノートを\ruby{取}{と}り\ruby{出}{だ}した、と\ruby{思}{おも}うが\ruby{早}{はや}いか、パタン!とフタを\ruby{閉}{し}めてしまいます。そして、すぐにまた\ruby{開}{あ}けて\ruby{頭}{あたま}を\ruby{中}{なか}につっこんで\ruby{筆箱}{ふでばこ}から“ア”を\ruby{書}{か}くための\ruby{鉛筆}{えんぴつ}を\ruby{出}{だ}すと、\ruby{急}{いそ}いで\ruby{閉}{し}めて、“ア”を\ruby{書}{か}きます。ところが、うまく\ruby{書}{か}けなかったり\ruby{間違}{まちが}えたりしますね。そうすると、フタを\ruby{開}{あ}けて、また\ruby{頭}{あたま}を\ruby{突}{つ}っ\ruby{込}{こ}んで、\ruby{消}{け}し\ruby{ゴム}{ごむ}をだし、\ruby{閉}{し}めると、\ruby{急}{いそ}いで\ruby{消}{け}し\ruby{ゴム}{ごむ}を\ruby{使}{つか}い、\ruby{次}{つぎ}に、すごい\ruby{早}{はや}さで\ruby{開}{あ}けて、\ruby{消}{け}し\ruby{ゴム}{ごむ}をしまって、フタを\ruby{閉}{し}めてしまいます。で、すぐ、また\ruby{開}{あ}けるので\ruby{見}{み}てますと、“ア”ひとつだけ\ruby{書}{か}いて、\ruby{道具}{どうぐ}をひとつひとつ、\ruby{全部}{ぜんぶ}しまうんです。\ruby{鉛筆}{えんぴつ}をしまい、\ruby{閉}{し}めて、また\ruby{開}{あ}けてノートをしまい……というふうに。そして、\ruby{次}{つぎ}の“イ”のときに、また、ノートから\ruby{始}{はじ}まって、\ruby{鉛筆}{えんぴつ}、\ruby{消}{け}し\ruby{ゴム}{ごむ}……その\ruby{度}{たび}に,\ruby{私}{わたし}の\ruby{目}{め}の\ruby{前}{まえ}で、\ruby{目}{め}まぐるしく、\ruby{机}{つくえ}のフタが\ruby{開}{ひら}いたり\ruby{閉}{し}まったり。\ruby{私}{わたし}、\ruby{目}{め}が\ruby{回}{まわ}るんです。でも、\ruby{一応}{いちおう}、\ruby{用事}{ようじ}があるんですから、いけないとは\ruby{申}{もう}せませんけど……」 \ruby{先生}{せんせい}のまつ\ruby{毛}{げ}が、その\ruby{時}{とき}を\ruby{思}{おも}い\ruby{出}{だ}したように、パチパチと\ruby{早}{はや}くなった。

そこで\ruby{聞}{き}いて、ママには、トットちゃんが、なんで、\ruby{学校}{がっこう}の\ruby{机}{つくえ}を、そんなに\ruby{開}{あ}けたり\ruby{閉}{し}めたりするのか、ちょっとわかった。というのは、\ruby{初}{はじ}めて\ruby{学校}{がっこう}に\ruby{行}{い}って\ruby{帰}{かえ}ってきた\ruby{日}{ひ}に、トットちゃんが、ひどく\ruby{興奮}{こうふん}して、こうママに\ruby{報告}{ほうこく}したことを\ruby{思}{おも}い\ruby{出}{だ}したからだった。「ねえ、\ruby{学校}{がっこう}って、すごいの。\ruby{家}{いえ}の\ruby{机}{つくえ}の\ruby{引}{ひ}き\ruby{出}{だ}しは、こんな\ruby{風}{ふう}に、\ruby{引}{ひ}っ\ruby{張}{ぱ}るのだけど、\ruby{学校}{がっこう}のはフタが\ruby{上}{うえ}にあがるの。\ruby{ゴミ}{ごみ}\ruby{箱}{ばこ}のフタと\ruby{同}{おな}じなんだけど、もっとツルツルで、いろんなものが、しまえて、とってもいいんだ!」ママには、\ruby{今}{いま}まで\ruby{見}{み}たことのない\ruby{机}{つくえ}の\ruby{前}{まえ}で、トットちゃんが\ruby{面白}{おもしろ}がって、\ruby{開}{あ}けたり\ruby{閉}{し}めたりしてる\ruby{様子}{ようす}が\ruby{目}{め}に\ruby{見}{み}えるようだった。そして、それは、(そんなに\ruby{悪}{わる}いことではないし、\ruby{第}{だい}一、だんだん\ruby{馴}{な}れてくれば、そんなに\ruby{開}{あ}けたり\ruby{閉}{し}めたりしなくなるだろう)と\ruby{考}{かんが}えたけど、\ruby{先生}{せんせい}には、「よく\ruby{注意}{ちゅうい}しますから」といった。ところが、\ruby{先生}{せんせい}には、それまでの\ruby{調子}{ちょうし}より\ruby{声}{こえ}をもうすこし\ruby{高}{たか}くして、こういった。「それだけなら、よろしいんですけど!」ママは、すこし\ruby{身}{み}がちぢむような\ruby{気}{き}がした。\ruby{先生}{せんせい}は、\ruby{体}{からだ}を\ruby{少}{すこ}し\ruby{前}{まえ}にのり\ruby{出}{だ}すといった。「\ruby{机}{つくえ}で\ruby{音}{おと}を\ruby{立}{た}ててないな、と\ruby{思}{おも}うと、\ruby{今度}{こんど}は、\ruby{授業}{じゅぎょう}\ruby{中}{ちゅう}、\ruby{立}{た}ってるんです。ずーっと!」ママは、またびっくりしたので\ruby{聞}{き}いた。「\ruby{立}{た}ってるって、どこにでございましょうか?」\ruby{先生}{せんせい}はすこし\ruby{怒}{おこ}った\ruby{風}{ふう}にいった。「\ruby{教室}{きょうしつ}の\ruby{窓}{まど}のところです!」ママは、わけが\ruby{分}{わ}からないので、\ruby{続}{つづ}けて\ruby{質問}{しつもん}した。「\ruby{窓}{まど}のところで、\ruby{何}{なに}をしてるんでしょうか?」\ruby{先生}{せんせい}は、\ruby{半分}{はんぶん}、\ruby{叫}{さけ}ぶような\ruby{声}{こえ}で\ruby{言}{い}った。「チンドン\ruby{屋}{や}を\ruby{呼}{よ}び\ruby{込}{こ}むためです。」

\ruby{先生}{せんせい}の\ruby{話}{はなし}を、まとめて\ruby{見}{み}ると、こういうことになるらしかった。一\ruby{時間}{じかん}\ruby{目}{め}に、\ruby{机}{つくえ}をパタパタを、かなりやると、それ\ruby{以後}{いご}は、\ruby{机}{つくえ}を\ruby{離}{はな}れて、\ruby{窓}{まど}のところに\ruby{立}{た}って\ruby{外}{そと}を\ruby{見}{み}ている。そこで、\ruby{静}{しず}かにしていてくれるのなら、\ruby{立}{た}っててもいい、と\ruby{先生}{せんせい}が\ruby{思}{おも}った\ruby{矢先}{やさき}に、\ruby{突然}{とつぜん}、トットちゃんは、\ruby{大}{おお}きい\ruby{声}{こえ}で「チンドン\ruby{屋}{や}さーん!」と\ruby{外}{そと}に\ruby{向}{む}かって\ruby{叫}{さけ}んだ。だいたい、この\ruby{教室}{きょうしつ}の\ruby{窓}{まど}というのが、トットちゃんにっとては\ruby{幸福}{こうふく}なことに、\ruby{先生}{せんせい}にとっては\ruby{不幸}{ふこう}なことに、1\ruby{階}{かい}にあり、しかも\ruby{通}{とお}りは\ruby{目}{め}の\ruby{前}{まえ}だった。そして\ruby{境}{さかい}といえば、\ruby{低}{ひく}い、\ruby{生垣}{いけがき}があるだけだったから、トットちゃんは、\ruby{簡単}{かんたん}に、\ruby{通}{とお}りを\ruby{歩}{ある}いてる\ruby{人}{ひと}と、\ruby{話}{はなし}ができるわけだったのだ。さて、\ruby{通}{とお}りかかったチンドン\ruby{屋}{や}さんは、\ruby{呼}{よ}ばれたから\ruby{教室}{きょうしつ}の\ruby{下}{した}まで\ruby{来}{く}る。するとトットちゃんは、うれしそうに、クラス\ruby{中}{じゅう}の\ruby{皆}{みな}に\ruby{呼}{よ}びかけた。「\ruby{来}{き}たわよー」。\ruby{勉強}{べんきょう}してたクラス\ruby{中}{じゅう}の\ruby{子供}{こども}は、\ruby{全員}{ぜんいん}、その\ruby{声}{こえ}で\ruby{窓}{まど}のところに、\ruby{詰}{つ}め\ruby{掛}{か}けて、\ruby{口々}{くちぐち}に\ruby{叫}{さけ}ぶ。「チンドン\ruby{屋}{や}さーん」。すると、トットちゃんは、チンドン\ruby{屋}{や}さんに\ruby{頼}{たの}む。「ねえ、ちょっとだけで、やってみて?」\ruby{学校}{がっこう}のそばを\ruby{通}{とお}る\ruby{時}{とき}は、\ruby{音}{おと}をおさえめにしているチンドン\ruby{屋}{や}さんも、せっかくの\ruby{頼}{たの}みだからというので\ruby{盛大}{せいだい}に\ruby{始}{はじ}める。クラスネットや\ruby{鉦}{かね}や\ruby{太鼓}{たいこ}や、\ruby{三味線}{さみせん}で。その\ruby{間}{あいだ}、\ruby{先生}{せんせい}がどうしてるか、といえば、\ruby{一段落}{いちだんらく}つくまで、ひとり\ruby{教壇}{きょうだん}で、ジーっと\ruby{待}{ま}ってるしかない。(この一\ruby{曲}{きょく}が\ruby{終}{お}わるまでの\ruby{辛抱}{しんぼう}なんだから)と\ruby{自分}{じぶん}に\ruby{言}{い}い\ruby{聞}{き}かせながら。

さて、一\ruby{曲}{きょく}\ruby{終}{お}わると、チンドン\ruby{屋}{や}さんは\ruby{去}{さ}って\ruby{行}{い}き、\ruby{生徒}{せいと}たちは、それぞれの\ruby{席}{せき}に\ruby{戻}{もど}る。ところが、\ruby{驚}{おどろ}いたことに、トットちゃんは、\ruby{窓}{まど}のところから\ruby{動}{うご}かない。「どうして、まだ、そこにいるのですか?」という\ruby{先生}{せんせい}の\ruby{問}{と}いに、トットちゃんは、\ruby{大真面目}{おおまじめ}に\ruby{答}{こた}えた。「だって、また\ruby{違}{ちが}うチンドン\ruby{屋}{や}さんが\ruby{来}{き}たら、お\ruby{話}{はなし}しなきゃならないし。それから、さっきのチンドン\ruby{屋}{や}さんが、また、\ruby{戻}{もど}ってきたら、\ruby{大変}{たいへん}だからです。」

「これじゃ、\ruby{授業}{じゅぎょう}にならない、ということが、おわかりでしょう?」\ruby{話}{はな}してるうちに、\ruby{先生}{せんせい}は、かなり\ruby{感情的}{かんじょうてき}なってきて、ママに\ruby{言}{い}った。ママは、(なるほど、これでは\ruby{先生}{せんせい}も、お\ruby{困}{こま}りだわ)と\ruby{思}{おも}いかけた。とたん、\ruby{先生}{せんせい}は、また\ruby{一段}{いちだん}と\ruby{大}{おお}きな\ruby{声}{こえ}で、こういった。「それに……」ママはびっくりしながらも、\ruby{情}{なさ}けない\ruby{思}{おも}い\ruby{出}{で}\ruby{先生}{せんせい}に\ruby{聞}{き}いた。「まだ、あるんでございましょうか……」\ruby{先生}{せんせい}は、すぐいった。「“まだ”というように、\ruby{数}{かぞ}えられるくらいなら、こうやって、やめていただきたい、とお\ruby{願}{ねが}いはしません!!」それから\ruby{先生}{せんせい}は、\ruby{少}{すこ}し\ruby{息}{いき}を\ruby{静}{しず}めて、ママの\ruby{顔}{かお}を\ruby{見}{み}て\ruby{言}{い}った。「\ruby{昨日}{きのう}のことですが、\ruby{例}{れい}によって、\ruby{窓}{まど}のところに\ruby{立}{た}っているので、またチンドン\ruby{屋}{や}だと\ruby{思}{おも}って\ruby{授業}{じゅぎょう}をしておりましたら、これが、また\ruby{大}{おお}きな\ruby{声}{こえ}で、いきなり、『\ruby{何}{なに}してるの?』と、\ruby{誰}{だれ}かに、\ruby{何}{なに}かを\ruby{聞}{き}いているんですね。\ruby{相手}{あいて}は、\ruby{私}{わたし}のほうから\ruby{見}{み}えませんので、\ruby{誰}{だれ}だろう、と\ruby{思}{おも}っておりますと、また\ruby{大}{おお}きな\ruby{声}{こえ}で、『ねえ、\ruby{何}{なに}をしてるの?』って。それも、\ruby{今度}{こんど}は、\ruby{通}{とお}りにでなく、\ruby{上}{うえ}のほうに\ruby{向}{む}かって\ruby{聞}{き}いてるんです。\ruby{私}{わたし}も\ruby{気}{き}になりまして、\ruby{相手}{あいて}の\ruby{返事}{へんじ}が\ruby{聞}{き}こえるかした、と\ruby{耳}{みみ}を\ruby{澄}{す}ましてみましたが、\ruby{返事}{へんじ}がないんです。お\ruby{嬢}{じょう}さんは、それでも、さかんに、『ねえ、\ruby{何}{なに}してるの?』を\ruby{続}{つづ}けるので、\ruby{授業}{じゅぎょう}にもさしさわりがあるので、\ruby{窓}{まど}のところに\ruby{行}{い}って、お\ruby{嬢}{じょう}さんの\ruby{話}{はな}しかけてる\ruby{相手}{あいて}が\ruby{誰}{だれ}なのか、\ruby{見}{み}てみようと\ruby{思}{おも}いました。\ruby{窓}{まど}から\ruby{顔}{かお}を\ruby{出}{だ}して\ruby{上}{うえ}を\ruby{見}{み}ましたら、なんと、つばめが、\ruby{教室}{きょうしつ}の\ruby{屋根}{やね}の\ruby{下}{した}に、\ruby{巣}{す}を\ruby{作}{つく}っているんです。その、つばめに\ruby{聞}{き}いてるんですね。そりゃ\ruby{私}{わたし}も、\ruby{子供}{こども}の\ruby{気持}{きも}ちが、\ruby{分}{わ}からないわけじゃありませんから、つばめに\ruby{聞}{き}いてることを、\ruby{馬鹿}{ばか}げている、とは\ruby{申}{もう}しません。\ruby{授業}{じゅぎょう}\ruby{中}{ちゅう}に、あんな\ruby{声}{こえ}で、つばめに、『\ruby{何}{なに}をしてるのか?』と\ruby{聞}{き}かなくてもいいと、\ruby{私}{わたし}は\ruby{思}{おも}うんです」そして\ruby{先生}{せんせい}は、ママが、\ruby{一体}{いったい}なんとお\ruby{詫}{わ}びをしよう、と\ruby{口}{くち}を\ruby{開}{あ}きかけたのより、\ruby{早}{はや}く\ruby{言}{い}った。「それから、こういうことも、ございました。\ruby{初}{はじ}めての\ruby{図画}{ずが}の\ruby{時間}{じかん}のことですが、\ruby{国旗}{こっき}を\ruby{描}{えが}いて\ruby{御覧}{ごらん}なさい、と\ruby{私}{わたし}が\ruby{申}{もう}しましたら、\ruby{他}{ほか}の\ruby{子}{こ}は、\ruby{画用紙}{がようし}に、ちゃんと\ruby{日}{ひ}の\ruby{丸}{まる}を\ruby{描}{えが}いたんですが、お\ruby{宅}{たく}のお\ruby{嬢}{じょう}さんは、\ruby{朝日}{あさひ}\ruby{新聞}{しんぶん}の\ruby{模様}{もよう}のような、\ruby{軍艦旗}{ぐんかんき}を\ruby{描}{えが}き\ruby{始}{はじ}めました。それなら、それでいい、と\ruby{思}{おも}ってましたら、\ruby{突然}{とつぜん}、\ruby{旗}{はた}の\ruby{周}{まわ}りに、ふさを、つけ\ruby{始}{はじ}めたんです。ふさ。よく\ruby{青年}{せいねん}\ruby{団}{だん}とか、そういった\ruby{旗}{はた}についてます。あの、ふさです。で、それも、まあ、どこかで\ruby{見}{み}たのだろうから、と\ruby{思}{おも}っておりました。ところが、ちょっと\ruby{目}{め}を\ruby{離}{はな}したキスに、まあ、\ruby{黄色}{きいろ}のふさを、\ruby{机}{つくえ}にまで、どんどん\ruby{描}{えが}いちゃってるんです。だいたい\ruby{画用紙}{がようし}に、ほぼいっぱいに\ruby{旗}{はた}を\ruby{描}{えが}いたんですから、ふさの\ruby{余裕}{よゆう}は、もともと、あまりなかったんですが、それに、\ruby{黄色}{きいろ}のクレヨンで、ゴシゴシふさを\ruby{描}{えが}いたんですね。それが、はみ\ruby{出}{だ}しちゃって、\ruby{画用紙}{がようし}をどかしたら、\ruby{机}{つくえ}に、ひどい\ruby{黄色}{きいろ}のギザギザが\ruby{残}{のこ}ってしまって、ふいても、こすっても、とれません。まあ、\ruby{幸}{さいわ}いなことは、ギザギザが三\ruby{方向}{ほうこう}だけだった、ってことでしょうか?」ママは、ちぢこまりながらも、\ruby{急}{いそ}いで\ruby{質問}{しつもん}した。「\ruby{三方}{さんぼう}\ruby{向}{むか}っていうのは……」\ruby{先生}{せんせい}は、そろそろ\ruby{疲}{つか}れてきた、という\ruby{様子}{ようす}だったが、それでも\ruby{親切}{しんせつ}にいった。「\ruby{旗竿}{はたざお}を\ruby{左}{ひだり}はじに\ruby{描}{えが}きましたから、\ruby{旗}{はた}のギザギザは、\ruby{三方}{さんぼう}だけだったんでございます」ママは、\ruby{少}{すこ}し\ruby{助}{たす}かった、と\ruby{思}{おも}って、「はあ、それで\ruby{三方}{さんぼう}だけ……」といった。すると、\ruby{先生}{せんせい}は、\ruby{次}{つぎ}に、とっても、ゆっくりの\ruby{口調}{くちょう}で、\ruby{一言}{ひとこと}ずつ\ruby{区切}{くぎ}って「ただし、その\ruby{代}{か}わり、\ruby{旗竿}{はたざお}のはじが、やはり、\ruby{机}{つくえ}に、はみ\ruby{出}{だ}して、\ruby{残}{のこ}っております!!」それから\ruby{先生}{せんせい}は\ruby{立}{た}ち\ruby{上}{あ}がると、かなり\ruby{冷}{つめ}たい\ruby{感}{かん}じで、とどめをさすように\ruby{言}{い}った。「それと、\ruby{迷惑}{めいわく}しているのは、\ruby{私}{わたし}だけではございません。\ruby{隣}{となり}の\ruby{一年生}{いちねんせい}の\ruby{受}{う}け\ruby{持}{も}ちの\ruby{先生}{せんせい}もお\ruby{困}{こま}りのことが、あるそうですから……」ママは、\ruby{決心}{けっしん}しないわけには、いかなかった。(\ruby{確}{たし}かに、これじゃ、\ruby{他}{ほか}の\ruby{生徒}{せいと}さんに、ご\ruby{迷惑}{めいわく}すぎる。どこか、\ruby{他}{ほか}の\ruby{学校}{がっこう}を\ruby{探}{さが}して、\ruby{移}{うつ}したほうが、よさそうだ。\ruby{何}{なん}とか、あの\ruby{子}{こ}の\ruby{性格}{せいかく}がわかっていただけて、\ruby{皆}{みな}と\ruby{一緒}{いっしょ}にやっていくことを\ruby{教}{おし}えてくださるような\ruby{学校}{がっこう}に……)そうして、ママが、あっちこっち、かけずりまわって\ruby{見}{み}つけたのが、これから\ruby{行}{い}こうとしている\ruby{学校}{がっこう}、というわけだったのだ。ママは、この\ruby{退学}{たいがく}のことを、トットちゃんに\ruby{話}{はな}していなかった。\ruby{話}{はな}しても、\ruby{何}{なに}がいけなかったのか、わからないだろうし、また、そんなにことで、トットちゃんが、コンプレックスを\ruby{持}{も}つのも、よくないと\ruby{思}{おも}ったから、(いつか、\ruby{大}{おお}きくなったら、\ruby{話}{はな}しましょう)と、きめていた。ただ、トットちゃんには、こういった。「\ruby{新}{あたら}しい\ruby{学校}{がっこう}に\ruby{行}{い}ってみない?いい\ruby{学校}{がっこう}だって\ruby{話}{はなし}よ」トットちゃんは、\ruby{少}{すこ}し\ruby{考}{かんが}えてから、\ruby{言}{い}った。「\ruby{行}{い}くけど……」ママは、(この\ruby{子}{こ}は、\ruby{今}{いま}\ruby{何}{なに}を\ruby{考}{かんが}えてるのだろうか)と\ruby{思}{おも}った。(うすうす、\ruby{退学}{たいがく}のこと、\ruby{気}{き}がついていたんだろうか……)\ruby{次}{つぎ}の\ruby{瞬間}{しゅんかん}、トットちゃんは、ママの\ruby{腕}{うで}の\ruby{中}{なか}に、\ruby{飛}{と}び\ruby{込}{こ}んで\ruby{来}{き}て、いった。「ねえ、\ruby{今度}{こんど}の\ruby{学校}{がっこう}に、いいチンドン\ruby{屋}{や}さん、\ruby{来}{く}るかな?」とにかく、そんなわけで、トットちゃんとママは、\ruby{新}{あたら}しい\ruby{学校}{がっこう}に\ruby{向}{む}かって、\ruby{歩}{ある}いているのだった。


