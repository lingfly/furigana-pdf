トットちゃんは、\ruby{校長}{こうちょう}先生に\ruby{連}{つ}れられて、みんなが、お\ruby{弁当}{べんとう}を\ruby{食}{た}べるところを、見に\ruby{行}{い}くことになった。お\ruby{昼}{ひる}だけは、\ruby{電車}{でんしゃ}でなく、「みんな、\ruby{講堂}{こうどう}に\ruby{集}{あつ}まることになっている」と\ruby{校長}{こうちょう}先生が\ruby{教}{おし}えてくれた。\ruby{講堂}{こうどう}はさっきトットちゃんが上がってきた石の\ruby{階段}{かいだん}の、\ruby{突}{つ}き\ruby{当}{あ}たりにあった。いってみると、\ruby{生徒}{せいと}たちが、\ruby{大騒}{おおさわ}ぎをしながら、\ruby{机}{つくえ}と\ruby{椅子}{いす}を、\ruby{講堂}{こうどう}に、まーるく\ruby{輪}{わ}になるように、\ruby{並}{なら}べているところだった。\ruby{隅}{すみ}っこで、それを見ていたトットちゃんは、\ruby{校長}{こうちょう}先生の\ruby{上着}{うわぎ}を\ruby{引}{ひ}っ\ruby{張}{ぱ}って\ruby{聞}{き}いた。

「\ruby{他}{ほか}の\ruby{生徒}{せいと}は、どこにいるの?」

\ruby{校長}{こうちょう}先生は\ruby{答}{こた}えた。

「これで\ruby{全部}{ぜんぶ}なんだよ」

「\ruby{全部}{ぜんぶ}!?」

トットちゃんは、\ruby{信}{しん}じられない気がした。だって、\ruby{前}{まえ}の学校の一クラスと\ruby{同}{おな}じくらいしか、いないんだもの。そうすると、

「学校中で、五十人くらいなの?」

\ruby{校長}{こうちょう}先生は、「そうだ」といった。トットちゃんは、なにもかも、\ruby{前}{まえ}の学校と\ruby{違}{ちが}ってると\ruby{思}{おも}った。

みんなが\ruby{着席}{ちゃくせき}すると、\ruby{校長}{こうちょう}先生は、

「みんな、\ruby{海}{うみ}のものと、山のもの、もって\ruby{来}{き}たかい?」

と\ruby{聞}{き}いた。

「はーい」

みんな、それぞれの、お\ruby{弁当}{べんとう}の、ふたを\ruby{取}{と}った。

「どれどれ」

\ruby{校長}{こうちょう}先生は、\ruby{机}{つくえ}で\ruby{出来}{でき}た円の中に入ると、ひとりずつ、お\ruby{弁当}{べんとう}をのぞきながら、\ruby{歩}{ある}いている。

\ruby{生徒}{せいと}たちは、\ruby{笑}{わら}ったり、キイキイいったり、にぎやかだった。

「\ruby{海}{うみ}のものと、山のもの、って、なんだろう」

トットちゃんは、おかしくなった。でも、とっても、とっても、この学校は\ruby{変}{か}わっていて、\ruby{面白}{おもしろ}そう。お\ruby{弁当}{べんとう}の\ruby{時間}{じかん}が、こんなに、\ruby{愉快}{ゆかい}で、\ruby{楽}{たの}しいなんて、\ruby{知}{し}らなかった。トットちゃんは、\ruby{明日}{あした}からは、\ruby{自分}{じぶん}も、あの\ruby{机}{つくえ}に\ruby{座}{すわ}って、『\ruby{海}{うみ}のものと、山のもの』の\ruby{弁当}{べんとう}を、\ruby{校長}{こうちょう}先生に見てもらうんだ、と\ruby{思}{おも}うと、もう、\ruby{嬉}{うれ}しさと、\ruby{楽}{たの}しさで、\ruby{胸}{むね}がいっぱいになり、\ruby{叫}{さけ}びそうになった。 お\ruby{弁当}{べんとう}を、のぞきこんでる\ruby{校長}{こうちょう}先生の\ruby{肩}{かた}に、お\ruby{昼}{ひる}の\ruby{光}{ひかり}が、やわらかく\ruby{止}{と}まっていた。


