% !TeX encoding = UTF-8
% !TeX program = LuaLaTeX

\documentclass[12pt,a4paper,oneside,openany]{book}

\usepackage{luatexja-ruby}       %假名标注
\usepackage{xcolor}			 %引入颜色
\usepackage[hidelinks]{hyperref} %给目录添加超链接
\usepackage{titlesec}
\usepackage{indentfirst}         %章节首页首行缩进

% 自定义章节标题样式,调整默认垂直间距
\titleformat{\chapter}       	 % 要格式化的章节命令,如 \chapter、\section、\subsection 等。
[hang]                       	 % 标题的形状,可以是 hang(悬挂格式,默认)、block(块格式)
{\normalfont\Huge\bfseries}   	 % format,标题的整体格式,可以包括字体、大小、粗细等
{Chapter\thechapter}          	 % label,标题编号的格式,如 \thechapter、\thesection 等。
{1em}                         	 % Spacing between label and title
{}  

\headheight = 13pt           	 % 页眉高度
\headsep = 18pt		        	 %  页眉和正文的间距
\topmargin = -20pt
\titlespacing{\chapter}{0pt}{0pt}{10pt} % 调整标题间距

\ltjsetruby{size=0.6}            %设置振假名字号
%\ltjsetruby{fontcmd=\gtfamily}  %设置振假名字体
\ltjsetruby{mode=00}             %设置振假名的「進入」和「突出」模式

\setlength{\parindent}{2em}	 %首行缩进
\marginparwidth = 72pt           %边栏宽度

\linespread{1.8}			 %行距
\selectfont

% 边注
\newcounter{num}[chapter]
\newcommand{\translate}[2]{\addtocounter{num}{1} {\color{orange} #1\textsuperscript{\scriptsize \thenum}}\marginpar{\scriptsize \textsuperscript{\scriptsize \thenum} #2}}

\begin{document}
\tableofcontents

\chapter{初めての駅}
六\ruby{歳}{さい}の\ruby{時}{とき}\ruby{僕}{ぼく}は、「\ruby{体験談}{たいけんだん}」という\ruby{原生林}{げんせいりん}について\ruby{書}{か}かれた\ruby{本}{ほん}で、\ruby{素晴}{すば}らしい\ruby{挿絵}{さしえ}を\ruby{見}{み}たことがある。それは\ruby{大蛇}{だいじゃ}のボアが\ruby{猛獣}{もうじゅう}を\ruby{飲}{の}み\ruby{込}{こ}もうとしている\ruby{絵}{え}だった。\ruby{本}{ほん}にはこんな\ruby{説明}{せつめい}があった。


ボアは\ruby{獲物}{えもの}を\ruby{噛}{か}まずに\ruby{丸}{まる}ごと\ruby{飲}{の}み\ruby{込}{こ}みます。すると\ruby{動}{うご}けなくなるので、\ruby{獲物}{えもの}を\ruby{消化}{しょうか}する\ruby{半年}{はんとし}もの\ruby{間}{あいだ}、ずっと\ruby{眠}{ねむ}って\ruby{過}{す}ごします。

\ruby{僕}{ぼく}はジャングルでの\ruby{冒険}{ぼうけん}についていろいろと\ruby{考}{かんが}え、\ruby{自分}{じぶん}でも\ruby{色鉛筆}{いろえんぴつ}を\ruby{使}{つか}って、\ruby{生}{う}まれて\ruby{初}{はじ}めての\ruby{絵}{え}を\ruby{描}{か}き\ruby{上}{あ}げた。その\ruby{傑作}{けっさく}を\ruby{大人}{おとな}たちに\ruby{見}{み}せ、\ruby{怖}{こわ}いかどうか\ruby{聞}{き}いてみた。すると、こんな\ruby{答}{こた}えが\ruby{返}{かえ}ってきた。

どうして\ruby{帽子}{ぼうし}が\ruby{怖}{こわ}いんだい?

\ruby{帽子}{ぼうし}の\ruby{絵}{え}なんかじゃなかった。ゾウを\ruby{消化}{しょうか}しているボアを\ruby{描}{えが}いたのだ。でも、\ruby{大人}{おとな}にはわからないらしいので、\ruby{今度}{こんど}はボアの\ruby{内側}{うちがわ}の\ruby{絵}{え}を\ruby{描}{か}いてみた。\ruby{大人}{おとな}には\ruby{何時}{なんじ}だって\ruby{説明}{せつめい}が\ruby{必要}{ひつよう}なのだ。\ruby{僕}{ぼく}の\ruby{二番目}{にばんめ}の\ruby{絵}{え}では、ちゃんとボアの\ruby{中}{なか}にいるゾウが\ruby{見}{み}えていた。しかし\ruby{大人}{おとな}たちは\ruby{中}{なか}が\ruby{見}{み}えようが\ruby{見}{み}えまいが、ボアの\ruby{絵}{え}は\ruby{片付}{かたづ}けて、\ruby{地理}{ちり}や\ruby{歴史}{れきし}、\ruby{算数}{さんすう}や\ruby{文法}{ぶんぽう}の\ruby{勉強}{べんきょう}をしなさいと、\ruby{僕}{ぼく}を\ruby{嗜}{たしな}めた。

こうして、6\ruby{歳}{さい}にして\ruby{僕}{ぼく}は\ruby{偉大}{いだい}な\ruby{画家}{がか}になるという\ruby{夢}{ゆめ}を\ruby{諦}{あきら}めた。\ruby{作品}{さくひん}\ruby{第}{だい}一\ruby{号}{ごう}と\ruby{第}{だい}二\ruby{号}{ごう}が\ruby{共}{とも}に\ruby{不評}{ふひょう}で、\ruby{気持}{きも}ちが\ruby{挫}{くじ}けてしまったのだ。

\ruby{大人}{おとな}というのは、\ruby{自分}{じぶん}たちとは\ruby{全}{まった}く\ruby{何}{なに}もわかっていないから、いつも\ruby{子供}{こども}の\ruby{方}{ほう}から\ruby{説明}{せつめい}してあげなきゃいけなくて、うんざりする。\ruby{僕}{ぼく}は\ruby{別}{べつ}の\ruby{仕事}{しごと}を\ruby{選}{えら}ぶ\ruby{必要}{ひつよう}に\ruby{迫}{せま}られて、\ruby{飛行機}{ひこうき}の\ruby{操縦士}{そうじゅうし}になった。そして、\ruby{世界}{せかい}\ruby{中}{じゅう}をあちこち\ruby{飛}{と}び\ruby{回}{まわ}った。\ruby{地理}{ちり}は\ruby{確}{たし}かに\ruby{役}{やく}に\ruby{立}{た}った。\ruby{僕}{ぼく}は\ruby{一目}{ひとめ}で\ruby{中国}{ちゅうごく}とアリゾナを\ruby{見分}{みわ}ける\ruby{事}{こと}ができる。\ruby{夜間飛行}{やかんひこう}で\ruby{迷}{まよ}った\ruby{時}{とき}など、そういう\ruby{知識}{ちしき}があると\ruby{本当}{ほんとう}に\ruby{助}{たす}かる。

これまでの\ruby{人生}{じんせい}で、\ruby{僕}{ぼく}はたくさんの\ruby{重要}{じゅうよう}\ruby{人物}{じんぶつ}と\ruby{知}{し}り\ruby{合}{あ}った。\ruby{随分}{ずいぶん}\ruby{多}{おお}くの\ruby{大人}{おとな}たちと\ruby{一緒}{いっしょ}に\ruby{暮}{く}らしたし、マジカにも\ruby{見}{み}てきた。それでも\ruby{僕}{ぼく}の\ruby{考}{かんが}えはあまり\ruby{変}{か}わらなかった。\ruby{僕}{ぼく}は\ruby{物分}{ものわか}りのよさそうな\ruby{人}{ひと}に\ruby{出会}{であ}った\ruby{時}{とき}には\ruby{必}{かなら}ず、\ruby{肌}{はだ}に\ruby{離}{はな}さず\ruby{持}{も}ち\ruby{歩}{ある}いていた\ruby{作品}{さくひん}\ruby{第}{だい}一\ruby{号}{ごう}を\ruby{見}{み}せ、\ruby{実験}{じっけん}していた。その\ruby{人}{ひと}が\ruby{本当}{ほんとう}に\ruby{物事}{ものごと}の\ruby{分}{わ}かる\ruby{人}{ひと}かどうか、\ruby{知}{し}りたかったから。でも、\ruby{答}{こた}えはいつも\ruby{同}{おな}じだった。

\ruby{帽子}{ぼうし}だね。

その\ruby{後}{あと}\ruby{僕}{ぼく}はボアの\ruby{話}{はなし}も、\ruby{原生林}{げんせいりん}の\ruby{話}{はなし}も、\ruby{星}{ほし}の\ruby{話}{はなし}もしなかった。\ruby{話}{はなし}を\ruby{合}{あ}わせて、ブリッジやゴルフや、\ruby{政治}{せいじ}やネクタイの\ruby{話}{はなし}をした。するとその\ruby{大人}{おとな}は\ruby{話}{はなし}が\ruby{分}{わ}かる\ruby{相手}{あいて}と\ruby{知}{し}り\ruby{合}{あ}えたと\ruby{言}{い}って\ruby{喜}{よろこ}ぶのだ。




\end{document}