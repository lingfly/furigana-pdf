% !TeX encoding = UTF-8
% !TeX program = LuaLaTeX

\documentclass[12pt,a4paper,oneside,openany]{book}

\usepackage{luatexja-ruby}       %假名标注
\usepackage{xcolor}			 %引入颜色
\usepackage[hidelinks]{hyperref} %给目录添加超链接
\usepackage{titlesec}
\usepackage{indentfirst}         %章节首页首行缩进

% 自定义章节标题样式,调整默认垂直间距
\titleformat{\chapter}       	 % 要格式化的章节命令,如 \chapter、\section、\subsection 等。
[hang]                       	 % 标题的形状,可以是 hang(悬挂格式,默认)、block(块格式)
{\normalfont\Huge\bfseries}   	 % format,标题的整体格式,可以包括字体、大小、粗细等
{Chapter\thechapter}          	 % label,标题编号的格式,如 \thechapter、\thesection 等。
{1em}                         	 % Spacing between label and title
{}  

\headheight = 13pt           	 % 页眉高度
\headsep = 30pt		        	 % 页眉和正文的间距
\topmargin = -30pt               % 页眉和页面顶端的间距
\titlespacing{\chapter}{0pt}{-60pt}{10pt} % 调整标题间距
\textheight = 700pt              % 正文高度
\textwidth = 450pt 			 % 正文宽度
\setlength{\hoffset}{-20pt}      % 正文向左偏移20pt

\ltjsetruby{size=0.6}            %设置振假名字号
%\ltjsetruby{fontcmd=\gtfamily}  %设置振假名字体
\ltjsetruby{mode=00}             %设置振假名的「進入」和「突出」模式

\setlength{\parindent}{2em}	 %首行缩进
\marginparwidth = 72pt           %边栏宽度

\linespread{1.8}			 %行距
\selectfont

% 边注
\newcounter{num}[chapter]
\newcommand{\translate}[2]{\addtocounter{num}{1} {\color{orange} #1\textsuperscript{\scriptsize \thenum}}\marginpar{\scriptsize \textsuperscript{\scriptsize \thenum} #2}}

\begin{document}
\tableofcontents

\chapter{初めての駅}
六\ruby{歳}{さい}の\ruby{時}{とき}\ruby{僕}{ぼく}は、「\ruby{体験談}{たいけんだん}」という\ruby{原生林}{げんせいりん}について\ruby{書}{か}かれた\ruby{本}{ほん}で、\ruby{素晴}{すば}らしい\ruby{挿絵}{さしえ}を\ruby{見}{み}たことがある。それは\ruby{大蛇}{だいじゃ}のボアが\ruby{猛獣}{もうじゅう}を\ruby{飲}{の}み\ruby{込}{こ}もうとしている\ruby{絵}{え}だった。\ruby{本}{ほん}にはこんな\ruby{説明}{せつめい}があった。


ボアは\ruby{獲物}{えもの}を\ruby{噛}{か}まずに\ruby{丸}{まる}ごと\ruby{飲}{の}み\ruby{込}{こ}みます。すると\ruby{動}{うご}けなくなるので、\ruby{獲物}{えもの}を\ruby{消化}{しょうか}する\ruby{半年}{はんとし}もの\ruby{間}{あいだ}、ずっと\ruby{眠}{ねむ}って\ruby{過}{す}ごします。

\ruby{僕}{ぼく}はジャングルでの\ruby{冒険}{ぼうけん}についていろいろと\ruby{考}{かんが}え、\ruby{自分}{じぶん}でも\ruby{色鉛筆}{いろえんぴつ}を\ruby{使}{つか}って、\ruby{生}{う}まれて\ruby{初}{はじ}めての\ruby{絵}{え}を\ruby{描}{か}き\ruby{上}{あ}げた。その\ruby{傑作}{けっさく}を\ruby{大人}{おとな}たちに\ruby{見}{み}せ、\ruby{怖}{こわ}いかどうか\ruby{聞}{き}いてみた。すると、こんな\ruby{答}{こた}えが\ruby{返}{かえ}ってきた。

どうして\ruby{帽子}{ぼうし}が\ruby{怖}{こわ}いんだい?

\ruby{帽子}{ぼうし}の\ruby{絵}{え}なんかじゃなかった。ゾウを\ruby{消化}{しょうか}しているボアを\ruby{描}{えが}いたのだ。でも、\ruby{大人}{おとな}にはわからないらしいので、\ruby{今度}{こんど}はボアの\ruby{内側}{うちがわ}の\ruby{絵}{え}を\ruby{描}{か}いてみた。\ruby{大人}{おとな}には\ruby{何時}{なんじ}だって\ruby{説明}{せつめい}が\ruby{必要}{ひつよう}なのだ。\ruby{僕}{ぼく}の\ruby{二番目}{にばんめ}の\ruby{絵}{え}では、ちゃんとボアの\ruby{中}{なか}にいるゾウが\ruby{見}{み}えていた。しかし\ruby{大人}{おとな}たちは\ruby{中}{なか}が\ruby{見}{み}えようが\ruby{見}{み}えまいが、ボアの\ruby{絵}{え}は\ruby{片付}{かたづ}けて、\ruby{地理}{ちり}や\ruby{歴史}{れきし}、\ruby{算数}{さんすう}や\ruby{文法}{ぶんぽう}の\ruby{勉強}{べんきょう}をしなさいと、\ruby{僕}{ぼく}を\ruby{嗜}{たしな}めた。

こうして、6\ruby{歳}{さい}にして\ruby{僕}{ぼく}は\ruby{偉大}{いだい}な\ruby{画家}{がか}になるという\ruby{夢}{ゆめ}を\ruby{諦}{あきら}めた。\ruby{作品}{さくひん}\ruby{第}{だい}一\ruby{号}{ごう}と\ruby{第}{だい}二\ruby{号}{ごう}が\ruby{共}{とも}に\ruby{不評}{ふひょう}で、\ruby{気持}{きも}ちが\ruby{挫}{くじ}けてしまったのだ。

\ruby{大人}{おとな}というのは、\ruby{自分}{じぶん}たちとは\ruby{全}{まった}く\ruby{何}{なに}もわかっていないから、いつも\ruby{子供}{こども}の\ruby{方}{ほう}から\ruby{説明}{せつめい}してあげなきゃいけなくて、うんざりする。\ruby{僕}{ぼく}は\ruby{別}{べつ}の\ruby{仕事}{しごと}を\ruby{選}{えら}ぶ\ruby{必要}{ひつよう}に\ruby{迫}{せま}られて、\ruby{飛行機}{ひこうき}の\ruby{操縦士}{そうじゅうし}になった。そして、\ruby{世界}{せかい}\ruby{中}{じゅう}をあちこち\ruby{飛}{と}び\ruby{回}{まわ}った。\ruby{地理}{ちり}は\ruby{確}{たし}かに\ruby{役}{やく}に\ruby{立}{た}った。\ruby{僕}{ぼく}は\ruby{一目}{ひとめ}で\ruby{中国}{ちゅうごく}とアリゾナを\ruby{見分}{みわ}ける\ruby{事}{こと}ができる。\ruby{夜間飛行}{やかんひこう}で\ruby{迷}{まよ}った\ruby{時}{とき}など、そういう\ruby{知識}{ちしき}があると\ruby{本当}{ほんとう}に\ruby{助}{たす}かる。

これまでの\ruby{人生}{じんせい}で、\ruby{僕}{ぼく}はたくさんの\ruby{重要}{じゅうよう}\ruby{人物}{じんぶつ}と\ruby{知}{し}り\ruby{合}{あ}った。\ruby{随分}{ずいぶん}\ruby{多}{おお}くの\ruby{大人}{おとな}たちと\ruby{一緒}{いっしょ}に\ruby{暮}{く}らしたし、マジカにも\ruby{見}{み}てきた。それでも\ruby{僕}{ぼく}の\ruby{考}{かんが}えはあまり\ruby{変}{か}わらなかった。\ruby{僕}{ぼく}は\ruby{物分}{ものわか}りのよさそうな\ruby{人}{ひと}に\ruby{出会}{であ}った\ruby{時}{とき}には\ruby{必}{かなら}ず、\ruby{肌}{はだ}に\ruby{離}{はな}さず\ruby{持}{も}ち\ruby{歩}{ある}いていた\ruby{作品}{さくひん}\ruby{第}{だい}一\ruby{号}{ごう}を\ruby{見}{み}せ、\ruby{実験}{じっけん}していた。その\ruby{人}{ひと}が\ruby{本当}{ほんとう}に\ruby{物事}{ものごと}の\ruby{分}{わ}かる\ruby{人}{ひと}かどうか、\ruby{知}{し}りたかったから。でも、\ruby{答}{こた}えはいつも\ruby{同}{おな}じだった。

\ruby{帽子}{ぼうし}だね。

その\ruby{後}{あと}\ruby{僕}{ぼく}はボアの\ruby{話}{はなし}も、\ruby{原生林}{げんせいりん}の\ruby{話}{はなし}も、\ruby{星}{ほし}の\ruby{話}{はなし}もしなかった。\ruby{話}{はなし}を\ruby{合}{あ}わせて、ブリッジやゴルフや、\ruby{政治}{せいじ}やネクタイの\ruby{話}{はなし}をした。するとその\ruby{大人}{おとな}は\ruby{話}{はなし}が\ruby{分}{わ}かる\ruby{相手}{あいて}と\ruby{知}{し}り\ruby{合}{あ}えたと\ruby{言}{い}って\ruby{喜}{よろこ}ぶのだ。


\chapter{窓際のトットちゃん}
こうして\ruby{僕}{ぼく}は、六\ruby{年}{ねん}\ruby{前}{まえ}、サハラ\ruby{砂漠}{さばく}で\ruby{飛行機}{ひこうき}が\ruby{故障}{こしょう}するまで、\ruby{心}{こころ}を\ruby{許}{ゆる}して\ruby{話}{はな}せる\ruby{相手}{あいて}に\ruby{出会}{であ}う\ruby{事}{こと}もなく、\ruby{一人}{ひとり}で\ruby{生}{い}きてきた。\ruby{飛行機}{ひこうき}はエンジンのどこかが\ruby{壊}{こわ}れていた。\ruby{整備士}{せいびし}も、\ruby{乗客}{じょうきゃく}も\ruby{乗}{の}せていなかったので、\ruby{僕}{ぼく}は\ruby{難}{むずか}しい\ruby{修理}{しゅうり}の\ruby{仕事}{しごと}を\ruby{一人}{ひとり}でやり\ruby{遂}{と}げるしかなかった。

\ruby{死活問題}{しかつもんだい}だった。\ruby{飲}{の}み\ruby{水}{みず}は一\ruby{週間}{しゅうかん}\ruby{分}{ぶん}あるかないかだった。

\ruby{最初}{さいしょ}の\ruby{夜}{よる}、\ruby{僕}{ぼく}は、\ruby{人}{ひと}の\ruby{住}{す}む\ruby{場所}{ばしょ}から千マイルも\ruby{離}{はな}れた\ruby{砂}{すな}の\ruby{上}{うえ}で\ruby{眠}{ねむ}った。\ruby{大海原}{おおうなばら}を\ruby{筏}{いかだ}で\ruby{漂流}{ひょうりゅう}する\ruby{遭難者}{そうなんしゃ}より、ずっと\ruby{孤独}{こどく}だった。だから、\ruby{夜明}{よあ}けに\ruby{小}{ちい}さな\ruby{可愛}{かわい}らしい\ruby{声}{こえ}で\ruby{起}{お}こされた\ruby{時}{とき}、\ruby{僕}{ぼく}がどんなに\ruby{驚}{おどろ}いたか\ruby{想像}{そうぞう}してみてほしい。その\ruby{声}{こえ}は、こう\ruby{言}{い}った。

お\ruby{願}{ねが}い、\ruby{羊}{ひつじ}の\ruby{絵}{え}を\ruby{描}{か}いて。

えっ?

\ruby{羊}{ひつじ}を\ruby{描}{えが}いて。

\ruby{雷}{かみなり}に\ruby{打}{う}たれたみたいに\ruby{飛}{と}び\ruby{起}{お}きると、\ruby{目}{め}を\ruby{擦}{す}って\ruby{辺}{あた}りを\ruby{見回}{みまわ}した。そこには、とても\ruby{不思議}{ふしぎ}な\ruby{子供}{こども}が\ruby{一人}{ひとり}いて、\ruby{僕}{ぼく}を\ruby{真剣}{しんけん}に\ruby{見}{み}つめていた。\ruby{僕}{ぼく}は\ruby{突然}{とつぜん}\ruby{現}{あらわ}れたその\ruby{子供}{こども}を、\ruby{目}{め}を\ruby{丸}{まる}くして\ruby{見}{み}つめた。\ruby{何度}{なんど}も\ruby{言}{い}うけれど、\ruby{人}{ひと}の\ruby{住}{す}む\ruby{所}{ところ}から千マイルも\ruby{離}{はな}れていたのだ。しかしその\ruby{子}{こ}は\ruby{道}{みち}に\ruby{迷}{まよ}っているようには\ruby{見}{み}えなかった。\ruby{疲}{つか}れや\ruby{餓}{う}えや\ruby{渇}{かわ}きで\ruby{死}{し}にそうになっているようにも、\ruby{怖}{こわ}がっているようにも\ruby{見}{み}えなかった。\ruby{人}{ひと}の\ruby{住}{す}む\ruby{所}{ところ}から千マイルも\ruby{離}{はな}れた\ruby{砂漠}{さばく}の\ruby{真}{ま}ん\ruby{中}{なか}にいながら、\ruby{途方}{とほう}に\ruby{暮}{く}れた\ruby{迷子}{まいご}といった\ruby{様子}{ようす}は\ruby{少}{すこ}しもなかったのだ。

ようやく\ruby{口}{くち}が\ruby{聞}{き}けるようになると、\ruby{僕}{ぼく}はその\ruby{子}{こ}に\ruby{尋}{たず}ねた。

\ruby{君}{きみ}はこんな\ruby{所}{ところ}で\ruby{何}{なに}をしているの?

しかしその\ruby{子}{こ}はとても\ruby{大切}{たいせつ}な\ruby{事}{こと}のように、\ruby{静}{しず}かに\ruby{繰}{く}り\ruby{返}{かえ}すだけ。

お\ruby{願}{ねが}い、\ruby{羊}{ひつじ}の\ruby{絵}{え}を\ruby{描}{か}いて。

バカげた\ruby{話}{はなし}だが、\ruby{人}{ひと}の\ruby{住}{す}む\ruby{所}{ところ}から千マイルも\ruby{離}{はな}れて、\ruby{死}{し}の\ruby{危険}{きけん}にさらされているというのに、\ruby{僕}{ぼく}はその\ruby{子}{こ}に\ruby{言}{い}われるままに、ポケットから\ruby{一枚}{いちまい}の\ruby{紙切}{かみき}れと\ruby{万年筆}{まんねんひつ}を\ruby{取}{と}り\ruby{出}{だ}していた。

だけどそこで、\ruby{僕}{ぼく}が\ruby{一生懸命}{いっしょうけんめい}\ruby{勉強}{べんきょう}してきたのは、\ruby{地理}{ちり}と\ruby{歴史}{れきし}と\ruby{算数}{さんすう}と\ruby{文法}{ぶんぽう}だけだった\ruby{事}{こと}を\ruby{思}{おも}い\ruby{出}{だ}して、\ruby{少}{すこ}し\ruby{不機嫌}{ふきげん}になりながら、\ruby{絵}{え}は\ruby{描}{えが}けないんだと、その\ruby{子}{こ}に\ruby{言}{い}った。

そんなの\ruby{構}{かま}わないよ。\ruby{羊}{ひつじ}を\ruby{描}{えが}いて。

\ruby{僕}{ぼく}は\ruby{羊}{ひつじ}の\ruby{絵}{え}なんか\ruby{描}{えが}いたことはなかったので、\ruby{自分}{じぶん}に\ruby{描}{えが}けるたった\ruby{二}{ふた}つの\ruby{絵}{え}の\ruby{内}{うち}の\ruby{一}{ひと}つを\ruby{描}{えが}いてあげた。ボアの\ruby{外側}{そとがわ}の\ruby{絵}{え}だ。その\ruby{時}{とき}\ruby{男}{おとこ}の\ruby{子}{こ}がこういうのを\ruby{聞}{き}いて、\ruby{僕}{ぼく}はびっくりした。

\ruby{違}{ちが}う、\ruby{違}{ちが}う、ボアに\ruby{飲}{の}み\ruby{込}{こ}まれたゾウなんていらないよ。ボアはとっても\ruby{危険}{きけん}だし、ゾウは\ruby{結構}{けっこう}\ruby{場所塞}{ばしょふさ}ぎだから。\ruby{僕}{ぼく}の\ruby{所}{ところ}はとっても\ruby{小}{ちい}さいんだ。\ruby{欲}{ほ}しいのは\ruby{羊}{ひつじ}、\ruby{羊}{ひつじ}を\ruby{描}{えが}いて。

そこで\ruby{僕}{ぼく}は\ruby{羊}{ひつじ}を\ruby{描}{えが}いた。

ううん、\ruby{駄目}{だめ}だよ。この\ruby{羊}{ひつじ}はひどい\ruby{病気}{びょうき}だ。\ruby{違}{ちが}うのを\ruby{描}{えが}いて。

\ruby{僕}{ぼく}は\ruby{描}{えが}き\ruby{直}{なお}した。\ruby{男}{おとこ}の\ruby{子}{こ}は\ruby{僕}{ぼく}を\ruby{気遣}{きづか}って\ruby{優}{やさ}しく\ruby{微笑}{ほほえ}んだ。

よく\ruby{見}{み}て。これは\ruby{羊}{ひつじ}じゃないでしょう。\ruby{雄羊}{おひつじ}だよね。\ruby{角}{かく}があるもの。

そこで\ruby{僕}{ぼく}はまた\ruby{描}{えが}き\ruby{直}{なお}した。けれどそれも\ruby{前}{まえ}の\ruby{二}{ふた}つと\ruby{同}{おな}じように\ruby{拒絶}{きょぜつ}された。

この\ruby{羊}{ひつじ}は\ruby{年}{とし}を\ruby{取}{と}りすぎてるよ。\ruby{僕}{ぼく}、\ruby{長生}{ながい}きする\ruby{羊}{ひつじ}が\ruby{欲}{ほ}しいの。

\ruby{我慢}{がまん}も\ruby{限界}{げんかい}に\ruby{近付}{ちかづ}いていた。\ruby{修理}{しゅうり}を\ruby{始}{はじ}めなければと\ruby{焦}{あせ}っていた。\ruby{僕}{ぼく}はざっと\ruby{描}{えが}き\ruby{殴}{なぐ}った\ruby{絵}{え}を\ruby{男}{おとこ}の\ruby{子}{こ}に\ruby{投}{な}げ\ruby{渡}{わた}した。

これは\ruby{羊}{ひつじ}の\ruby{箱}{はこ}だ。\ruby{君}{きみ}が\ruby{欲}{ほ}しがっている\ruby{羊}{ひつじ}はこの\ruby{中}{なか}にいるよ。

すると\ruby{驚}{おどろ}いたことに、この\ruby{小}{ちい}さな\ruby{審査}{しんさ}\ruby{員}{いん}の\ruby{顔}{かお}がぱっと\ruby{輝}{かがや}いたのだ。

ぴったりだよ。\ruby{僕}{ぼく}が\ruby{欲}{ほ}しかったのは、この\ruby{羊}{ひつじ}さ。ね、この\ruby{羊}{ひつじ}\ruby{草}{ぐさ}をいっぱい\ruby{食}{た}べるかな。

どうして?

\ruby{僕}{ぼく}の\ruby{所}{ところ}はとっても\ruby{小}{ちい}さいから。

\ruby{大丈夫}{だいじょうぶ}だよ。\ruby{君}{きみ}にあげたのは、とっても\ruby{小}{ちい}さな\ruby{羊}{ひつじ}だからね。

そんなに\ruby{小}{ちい}さくないよ。あれ、\ruby{羊}{ひつじ}は\ruby{寝}{ね}ちゃったみたい。

こうして\ruby{僕}{ぼく}はこの\ruby{小}{ちい}さな\ruby{王子}{おうじ}さまと\ruby{知}{し}り\ruby{合}{あ}いになった。




\chapter{新しい学校}
\ruby{学校}{がっこう}の\ruby{門}{もん}が、はっきり\ruby{見}{み}えるところまで\ruby{来}{き}て、トットちゃんは、\ruby{立}{た}ち\ruby{止}{どま}った。なぜなら、この\ruby{間}{あいだ}まで\ruby{行}{い}っていた\ruby{学校}{がっこう}の\ruby{門}{もん}は、\ruby{立派}{りっぱ}なコンクリートみたいな\ruby{柱}{はしら}で、\ruby{学校}{がっこう}の\ruby{名前}{なまえ}も、\ruby{大}{おお}きく\ruby{書}{か}いてあった。ところが、この\ruby{新}{あたら}しい\ruby{学校}{がっこう}の\ruby{門}{もん}ときたら、\ruby{低}{ひく}い\ruby{木}{き}で、しかも\ruby{葉}{は}っぱが\ruby{生}{は}えていた。「\ruby{地面}{じめん}から\ruby{生}{は}えてる\ruby{門}{もん}ね」と、トットちゃんはママに\ruby{言}{い}った。そうして、こう、\ruby{付}{つ}け\ruby{加}{くわ}えた。「きっと、どんどんはえて、\ruby{今}{いま}に\ruby{電信柱}{でんしんばしら}より\ruby{高}{たか}くなるわ」\ruby{確}{たし}かに、その二\ruby{本}{ほん}の\ruby{門}{もん}は、\ruby{根}{ね}っこのある\ruby{木}{き}だった。トットちゃんは、\ruby{門}{もん}に\ruby{近}{ちか}づくと、いきなり\ruby{顔}{かお}を、\ruby{斜}{なな}めにした。なぜかといえば、\ruby{門}{もん}にぶら\ruby{下}{さ}げてある\ruby{学校}{がっこう}の\ruby{名前}{なまえ}を\ruby{書}{か}いた\ruby{札}{さつ}が、\ruby{風}{かぜ}に\ruby{吹}{ふ}かれたのか、\ruby{斜}{なな}めになっていたからだった。「トモエがくえん」トットちゃんは、\ruby{顔}{かお}を\ruby{斜}{なな}めにしたまま、\ruby{表札}{ひょうさつ}を\ruby{読}{よ}み\ruby{上}{あ}げた。そして、ママに、「トモエって、なあに?」と\ruby{聞}{き}こうとしたときだった。トットちゃんの\ruby{目}{め}の\ruby{端}{はし}に、\ruby{夢}{ゆめ}としか\ruby{思}{おも}えないものが\ruby{見}{み}えたのだった。トットちゃんは、\ruby{身}{み}をかがめると、\ruby{門}{もん}の\ruby{植}{う}え\ruby{込}{こ}みの、\ruby{隙間}{すきま}に\ruby{頭}{あたま}を\ruby{突}{つ}っ\ruby{込}{こ}んで、\ruby{門}{もん}の\ruby{中}{なか}をのぞいてみた。どうしよう、みえたんだけど!「ママ!あれ、\ruby{本当}{ほんとう}の\ruby{電車}{でんしゃ}?\ruby{校庭}{こうてい}に\ruby{並}{なら}んでるの」それは、\ruby{走}{はし}っていない、\ruby{本当}{ほんとう}の\ruby{電車}{でんしゃ}が六\ruby{台}{だい}、\ruby{教室}{きょうしつ}\ruby{用}{よう}に、\ruby{置}{お}かれてあるのだった。トットちゃんは、\ruby{夢}{ゆめ}のように\ruby{思}{おも}った。“\ruby{電車}{でんしゃ}の\ruby{教室}{きょうしつ}……”

\ruby{電車}{でんしゃ}で\ruby{窓}{まど}が、\ruby{朝}{あさ}の\ruby{光}{ひかり}を\ruby{受}{う}けて、キラキラと\ruby{光}{ひか}っていた。\ruby{目}{め}を\ruby{輝}{かがや}かして、のぞいているトットちゃんの、ホッペタも、\ruby{光}{ひか}っていた。   \ruby{気}{き}に\ruby{入}{い}ったわ\ruby{次}{つぎ}の\ruby{瞬間}{しゅんかん}、トットちゃんは、「わーい」と\ruby{歓声}{かんせい}を\ruby{上}{あ}げると、\ruby{電車}{でんしゃ}の\ruby{教室}{きょうしつ}のほうに\ruby{向}{む}かって\ruby{走}{はし}り\ruby{出}{だ}した。そして、\ruby{走}{はし}りながら、ママに\ruby{向}{む}かって\ruby{叫}{さけ}んだ。「ねえ、\ruby{早}{はや}く、\ruby{動}{うご}かない\ruby{電車}{でんしゃ}に\ruby{乗}{の}ってみよう!」ママは、\ruby{驚}{おどろ}いて\ruby{走}{はし}り\ruby{出}{だ}した。もとバスケットバールの\ruby{選手}{せんしゅ}だったままの\ruby{足}{あし}は、トットちゃんより\ruby{速}{はや}かったから、トットちゃんが、\ruby{後}{あと}、ちょっとでドア、というときに、スカートを\ruby{捕}{つか}まえられてしまった。ママは、スカートのはしを、ぎっちり\ruby{握}{にぎ}ったまま、トットちゃんにいった。「ダメよ。この\ruby{電車}{でんしゃ}は、この\ruby{学校}{がっこう}のお\ruby{教室}{きょうしつ}なんだし、あなたは、まだ、この\ruby{学校}{がっこう}に\ruby{入}{はい}れていただいてないんだから。もし、どうしても、この\ruby{電車}{でんしゃ}に\ruby{乗}{の}りたいんだったら、これからお\ruby{目}{め}にかかる\ruby{校長}{こうちょう}\ruby{先生}{せんせい}とちゃんと、お\ruby{話}{はな}してちょうだい。そして、うまくいったら、この\ruby{学校}{がっこう}に\ruby{通}{とお}えるんだから、\ruby{分}{わ}かった?」トットちゃんは、(\ruby{今}{いま}\ruby{乗}{の}れないのは、とても\ruby{残念}{ざんねん}なことだ)と\ruby{思}{おも}ったけど、ママのいう\ruby{通}{とお}りにしようときめたから、\ruby{大}{おお}きな\ruby{声}{こえ}で、「うん」といって、それから、いそいで、つけたした。「\ruby{私}{わたし}、この\ruby{学校}{がっこう}、とっても\ruby{気}{き}に\ruby{入}{い}ったわ」ママは、トットちゃんが\ruby{気}{き}に\ruby{入}{い}ったかどうかより、\ruby{校長}{こうちょう}\ruby{先生}{せんせい}が、トットちゃんを\ruby{気}{き}に\ruby{入}{い}ってくださるかどうか\ruby{問題}{もんだい}なのよ、といいたい\ruby{気}{き}がしたけど、とにかく、トットちゃんのスカートから\ruby{手}{て}を\ruby{離}{はな}し、\ruby{手}{て}をつないで\ruby{校長}{こうちょう}\ruby{室}{しつ}のほうに\ruby{歩}{ある}き\ruby{出}{だ}した。どの\ruby{電車}{でんしゃ}も\ruby{静}{しず}かで、ちょっと\ruby{前}{まえ}に、一\ruby{時間}{じかん}\ruby{目}{め}の\ruby{授業}{じゅぎょう}が\ruby{始}{はじ}まったようだった。あまり\ruby{広}{ひろ}くない\ruby{校庭}{こうてい}の\ruby{周}{まわ}りには、\ruby{塀}{へい}の\ruby{変}{か}わりに、いろんな\ruby{種類}{しゅるい}の\ruby{木}{き}が\ruby{植}{う}わっていて、\ruby{花壇}{かだん}には、\ruby{赤}{あか}や\ruby{黄色}{きいろ}の\ruby{花}{はな}がいっぱい\ruby{咲}{さ}いていた。\ruby{校長}{こうちょう}\ruby{室}{しつ}は、\ruby{電車}{でんしゃ}ではなく、ちょうど、\ruby{門}{もん}から\ruby{正面}{しょうめん}に\ruby{見}{み}える\ruby{扇形}{おうぎがた}に\ruby{広}{ひろ}がった七\ruby{段}{だん}くらいある\ruby{石}{いし}の\ruby{階段}{かいだん}を\ruby{上}{のぼ}った、その\ruby{右手}{みぎて}にあった。トットちゃんは、ママの\ruby{手}{て}を\ruby{振}{ふ}り\ruby{切}{き}ると、\ruby{階段}{かいだん}を\ruby{駆}{か}け\ruby{上}{あ}がって\ruby{行}{い}ったが、\ruby{急}{きゅう}に\ruby{止}{と}まって、\ruby{振}{ふ}り\ruby{向}{む}いた。だから、\ruby{後}{うし}ろから\ruby{行}{い}ったママは、もう\ruby{少}{すこ}しで、トットちゃんと\ruby{正面}{しょうめん}\ruby{衝突}{しょうとつ}するところだった。「どうしたの?」ママは、トットちゃんの\ruby{気}{き}が\ruby{変}{か}わったのかと\ruby{思}{おも}って、\ruby{急}{いそ}いで\ruby{聞}{き}いた。トットちゃんは、ちょうど\ruby{階段}{かいだん}の\ruby{一番}{いちばん}うえに\ruby{立}{た}った\ruby{形}{かたち}だったけど、まじめな\ruby{顔}{かお}をして、\ruby{小声}{こごえ}でママに\ruby{聞}{き}いた。ママは、かなり\ruby{辛抱}{しんぼう}づよい\ruby{人間}{にんげん}だったから……というか,\ruby{面白}{おもしろ}がりやだったから、やはり\ruby{小声}{こごえ}になって、トットちゃんに\ruby{顔}{かお}をつけて、\ruby{聞}{き}いた。「どうして?」トットちゃんは、ますます\ruby{声}{こえ}をひそめて\ruby{言}{い}った。「だってさ、\ruby{校長}{こうちょう}\ruby{先生}{せんせい}って、ママいったけど、こんなに\ruby{電車}{でんしゃ}、いっぱい\ruby{持}{も}ってるんだから、\ruby{本当}{ほんとう}は、\ruby{駅}{えき}の\ruby{人}{ひと}なんじゃないの?」\ruby{確}{たし}かに、\ruby{電車}{でんしゃ}の\ruby{払}{はら}い\ruby{下}{さ}げを\ruby{校舎}{こうしゃ}にしている\ruby{学校}{がっこう}なんてめずらしいから、トットちゃんの\ruby{疑問}{ぎもん}も、もっとものこと、とママも\ruby{思}{おも}ったけど、この\ruby{際}{さい}、\ruby{説明}{せつめい}してるヒマはないので、こういった。「じゃ、あなた、\ruby{校長}{こうちょう}\ruby{先生}{せんせい}に\ruby{伺}{うかが}って\ruby{御覧}{ごらん}なさい、\ruby{自分}{じぶん}で。それと、あなたのパパのことを\ruby{考}{かんが}えてみて?パパはヴァイオリンを\ruby{弾}{ひ}く\ruby{人}{ひと}で、いくつかヴァイオリンを\ruby{持}{も}ってるけど、ヴァイオリン\ruby{屋}{や}さんじゃないでしょう?そういう\ruby{人}{ひと}もいるのよ」トットちゃんは、「そうか」というと、ママと\ruby{手}{て}をつないだ。




\chapter{気に入ったわ}
\ruby{次}{つぎ}の\ruby{瞬間}{しゅんかん}、トットちゃんは、「わーい」と\ruby{歓声}{かんせい}を\ruby{上}{あ}げると、\ruby{電車}{でんしゃ}の\ruby{教室}{きょうしつ}のほうに\ruby{向}{む}かって\ruby{走}{はし}り\ruby{出}{だ}した。そして、\ruby{走}{はし}りながら、ママに\ruby{向}{む}かって\ruby{叫}{さけ}んだ。

「ねえ、\ruby{早}{はや}く、\ruby{動}{うご}かない\ruby{電車}{でんしゃ}に\ruby{乗}{の}ってみよう!」

ママは、\ruby{驚}{おどろ}いて\ruby{走}{はし}り\ruby{出}{だ}した。もとバスケットボールの\ruby{選手}{せんしゅ}だったままの\ruby{足}{あし}は、トットちゃんより\ruby{速}{はや}かったから、トットちゃんが、\ruby{後}{あと}、ちょっとでドア、というときに、スカートを\ruby{捕}{つか}まえられてしまった。ママは、スカートのはしを、ぎっちり\ruby{握}{にぎ}ったまま、トットちゃんにいった。

「ダメよ。この\ruby{電車}{でんしゃ}は、この\ruby{学校}{がっこう}のお\ruby{教室}{きょうしつ}なんだし、あなたは、まだ、この\ruby{学校}{がっこう}に\ruby{入}{はい}れていただいてないんだから。もし、どうしても、この\ruby{電車}{でんしゃ}に\ruby{乗}{の}りたいんだったら、これからお\ruby{目}{め}にかかる\ruby{校長}{こうちょう}\ruby{先生}{せんせい}とちゃんと、お\ruby{話}{はな}してちょうだい。そして、うまくいったら、この\ruby{学校}{がっこう}に\ruby{通}{とお}えるんだから、\ruby{分}{わ}かった?」

トットちゃんは、(\ruby{今}{いま}\ruby{乗}{の}れないのは、とても\ruby{残念}{ざんねん}なことだ)と\ruby{思}{おも}ったけど、ママのいう\ruby{通}{とお}りにしようときめたから、\ruby{大}{おお}きな\ruby{声}{こえ}で、

「うん」

といって、それから、いそいで、つけたした。

「\ruby{私}{わたし}、この\ruby{学校}{がっこう}、とっても\ruby{気}{き}に\ruby{入}{い}ったわ」

ママは、トットちゃんが\ruby{気}{き}に\ruby{入}{い}ったかどうかより、\ruby{校長}{こうちょう}\ruby{先生}{せんせい}が、トットちゃんを\ruby{気}{き}に\ruby{入}{い}ってくださるかどうか\ruby{問題}{もんだい}なのよ、といいたい\ruby{気}{き}がしたけど、とにかく、トットちゃんのスカートから\ruby{手}{て}を\ruby{離}{はな}し、\ruby{手}{て}をつないで\ruby{校長}{こうちょう}\ruby{室}{しつ}のほうに\ruby{歩}{ある}き\ruby{出}{だ}した。

どの\ruby{電車}{でんしゃ}も\ruby{静}{しず}かで、ちょっと\ruby{前}{まえ}に、一\ruby{時間}{じかん}\ruby{目}{め}の\ruby{授業}{じゅぎょう}が\ruby{始}{はじ}まったようだった。あまり\ruby{広}{ひろ}くない\ruby{校庭}{こうてい}の\ruby{周}{まわ}りには、\ruby{塀}{へい}の\ruby{変}{か}わりに、いろんな\ruby{種類}{しゅるい}の\ruby{木}{き}が\ruby{植}{う}わっていて、\ruby{花壇}{かだん}には、\ruby{赤}{あか}や\ruby{黄色}{きいろ}の\ruby{花}{はな}がいっぱい\ruby{咲}{さ}いていた。

\ruby{校長}{こうちょう}\ruby{室}{しつ}は、\ruby{電車}{でんしゃ}ではなく、ちょうど、\ruby{門}{もん}から\ruby{正面}{しょうめん}に\ruby{見}{み}える\ruby{扇形}{おうぎがた}に\ruby{広}{ひろ}がった七\ruby{段}{だん}くらいある\ruby{石}{いし}の\ruby{階段}{かいだん}を\ruby{上}{のぼ}った、その\ruby{右手}{みぎて}にあった。

トットちゃんは、ママの\ruby{手}{て}を\ruby{振}{ふ}り\ruby{切}{き}ると、\ruby{階段}{かいだん}を\ruby{駆}{か}け\ruby{上}{あ}がって\ruby{行}{い}ったが、\ruby{急}{きゅう}に\ruby{止}{と}まって、\ruby{振}{ふ}り\ruby{向}{む}いた。だから、\ruby{後}{うし}ろから\ruby{行}{い}ったママは、もう\ruby{少}{すこ}しで、トットちゃんと\ruby{正面}{しょうめん}\ruby{衝突}{しょうとつ}するところだった。

「どうしたの?」

ママは、トットちゃんの\ruby{気}{き}が\ruby{変}{か}わったのかと\ruby{思}{おも}って、\ruby{急}{いそ}いで\ruby{聞}{き}いた。トットちゃんは、ちょうど\ruby{階段}{かいだん}の\ruby{一番}{いちばん}うえに\ruby{立}{た}った\ruby{形}{かたち}だったけど、まじめな\ruby{顔}{かお}をして、\ruby{小声}{こごえ}でママに\ruby{聞}{き}いた。

「ねえ、これからあいに行く人って、\ruby{駅}{えき}の\ruby{人}{ひと}なんじゃないの?」

ママは、かなり\ruby{辛抱}{しんぼう}づよい\ruby{人間}{にんげん}だったから……というか,\ruby{面白}{おもしろ}がりやだったから、やはり\ruby{小声}{こごえ}になって、トットちゃんに\ruby{顔}{かお}をつけて、\ruby{聞}{き}いた。

「どうして?」

トットちゃんは、ますます\ruby{声}{こえ}をひそめて\ruby{言}{い}った。

「だってさ、\ruby{校長}{こうちょう}\ruby{先生}{せんせい}って、ママいったけど、こんなに\ruby{電車}{でんしゃ}、いっぱい\ruby{持}{も}ってるんだから、\ruby{本当}{ほんとう}は、\ruby{駅}{えき}の\ruby{人}{ひと}なんじゃないの?」

\ruby{確}{たし}かに、\ruby{電車}{でんしゃ}の\ruby{払}{はら}い\ruby{下}{さ}げを\ruby{校舎}{こうしゃ}にしている\ruby{学校}{がっこう}なんてめずらしいから、トットちゃんの\ruby{疑問}{ぎもん}も、もっとものこと、とママも\ruby{思}{おも}ったけど、この\ruby{際}{さい}、\ruby{説明}{せつめい}してるヒマはないので、こういった。

「じゃ、あなた、\ruby{校長}{こうちょう}\ruby{先生}{せんせい}に\ruby{伺}{うかが}って\ruby{御覧}{ごらん}なさい、\ruby{自分}{じぶん}で。それと、あなたのパパのことを\ruby{考}{かんが}えてみて?パパはヴァイオリンを\ruby{弾}{ひ}く\ruby{人}{ひと}で、いくつかヴァイオリンを\ruby{持}{も}ってるけど、ヴァイオリン\ruby{屋}{や}さんじゃないでしょう?そういう\ruby{人}{ひと}もいるのよ」トットちゃんは、「そうか」というと、ママと\ruby{手}{て}をつないだ。

\chapter{校長先生}
トットちゃんとママが入っていくと、\ruby{部屋}{へや}の中にいた男の人が\ruby{椅子}{いす}から立ち上がった。その人は、\ruby{頭}{あたま}の\ruby{毛}{け}が\ruby{薄}{うす}くなっていて、\ruby{前}{まえ}のほうの\ruby{歯}{は}が\ruby{抜}{ぬ}けていて、\ruby{顔}{かお}の\ruby{血色}{けっしょく}がよく、\ruby{背}{せ}はあまり\ruby{高}{たか}くないけど、\ruby{肩}{かた}や\ruby{腕}{うで}が、がっちりしていて、ヨレヨレの\ruby{黒}{くろ}の\ruby{三}{み}つ\ruby{揃}{ぞろ}いを、キチンと\ruby{着}{き}ていた。

トットちゃんは、\ruby{急}{いそ}いで、お\ruby{辞儀}{じぎ}をしてから、\ruby{元気}{げんき}よく\ruby{聞}{き}いた。

「\ruby{校長}{こうちょう}先生か、\ruby{駅}{えき}の人か、どっち?」

ママが、\ruby{慌}{あわ}てて\ruby{説明}{せつめい}しよう、とするまえに、その人は\ruby{笑}{わら}いながら\ruby{答}{こた}えた。

「\ruby{校長}{こうちょう}先生だよ」

トットちゃんは、とってもうれしそうに\ruby{言}{い}った。

「よかった。じゃ、おねがい。\ruby{私}{わたし}、この学校にいりたいの」

\ruby{校長}{こうちょう}先生は、\ruby{椅子}{いす}をトットちゃんに\ruby{勧}{すす}めると、ママのほうを\ruby{向}{む}いて\ruby{言}{い}った。

「じゃ、\ruby{僕}{ぼく}は、これからトットちゃんと\ruby{話}{はなし}がありますから、もう、お\ruby{帰}{かえ}り下さって\ruby{結構}{けっこう}です」

ほんのちょっとの\ruby{間}{あいだ}、トットちゃんは、\ruby{少}{すこ}し\ruby{心細}{こころぼそ}い気がしたけど、なんとなく、(この\ruby{校長}{こうちょう}先生ならいいや)と\ruby{思}{おも}った。ママは、いさぎよく先生にいった。

「じゃ、よろしく、お\ruby{願}{ねが}いします」

そして、ドアを\ruby{閉}{し}めて\ruby{出}{で}て\ruby{行}{い}った。

\ruby{校長}{こうちょう}先生は、トットちゃんの\ruby{前}{まえ}に\ruby{椅子}{いす}を\ruby{引}{ひ}っ\ruby{張}{ぱ}ってきて、とても\ruby{近}{ちか}い\ruby{位置}{いち}に、\ruby{向}{む}かい\ruby{合}{あ}わせに\ruby{腰}{こし}をかけると、こういった。

「さあ、\ruby{何}{なん}でも、先生に\ruby{話}{はな}してごらん。\ruby{話}{はな}したいこと、\ruby{全部}{ぜんぶ}」

「\ruby{話}{はな}したいこと!?」

(なにか\ruby{聞}{き}かれて、お\ruby{返事}{へんじ}するのかな?)と\ruby{思}{おも}っていたトットちゃんは、「\ruby{何}{なん}でも\ruby{話}{はな}していい」と\ruby{聞}{き}いて、ものすごくうれしくなって、すぐ\ruby{話}{はな}し\ruby{始}{はじ}めた。\ruby{順序}{じゅんじょ}も、\ruby{話}{はな}し\ruby{方}{かた}も、\ruby{少}{すこ}しグチャグチャだったけど、\ruby{一生懸命}{いっしょうけんめい}に\ruby{話}{はな}した。

\ruby{今}{いま}\ruby{乗}{の}ってきた\ruby{電車}{でんしゃ}が\ruby{速}{はや}かったこと。

\ruby{駅}{えき}の\ruby{改札口}{かいさつぐち}のおじさんに、お\ruby{願}{ねが}いしたけど、\ruby{切符}{きっぷ}をくれなかったこと。

\ruby{前}{まえ}に\ruby{行}{い}ってた学校の\ruby{受}{う}け\ruby{持}{も}ちの女の先生は、\ruby{顔}{かお}がきれいだということ。

その学校には、つばめの\ruby{巣}{す}があること。

\ruby{家}{いえ}には、ロッキーという\ruby{茶色}{ちゃいろ}の犬がいて“お手”と“ごめんくださいませ”と、ご\ruby{飯}{はん}の\ruby{後}{あと}で、“\ruby{満足}{まんぞく}、\ruby{満足}{まんぞく}”ができること。

\ruby{幼稚}{ようち}\ruby{園}{えん}のとき、ハサミを口の中に入れて、チョキチョキやると、「\ruby{舌}{した}を\ruby{切}{き}ります」と先生が\ruby{怒}{いか}ったけど、\ruby{何回}{なんかい}もやっちゃったっていうこと。

\ruby{洟}{はな}が出てきたときは、いつまでも、ズルズルやってると、ママにしかられるから、なるべく早くかむこと。

パパは、\ruby{海}{うみ}で\ruby{泳}{およ}ぐのが\ruby{上手}{じょうず}で、\ruby{飛}{と}び\ruby{込}{こ}みだって\ruby{出来}{でき}ること。

こういったことを、\ruby{次}{つぎ}から\ruby{次}{つぎ}と、トットちゃんは\ruby{話}{はな}した。先生は、\ruby{笑}{わら}ったり、うなずいたり、「それから?」とかいったりしてくださったから、うれしくて、トットちゃんは、いつまでも\ruby{話}{はな}した。でも、とうとう、\ruby{話}{はなし}がなくなった。トットちゃんは、口をつぐんで\ruby{考}{かんが}えていると、先生はいった。

「もう、ないかい?」

トットちゃんは、これでおしまいにしてしまうのは、\ruby{残念}{ざんねん}だと\ruby{思}{おも}った。

せっかく、\ruby{話}{はなし}を、いっぱい\ruby{聞}{き}いてもらう、いいチャンスなのに。

(なにか、\ruby{話}{はなし}は、ないかなあ……)

\ruby{頭}{あたま}の中が、\ruby{忙}{いそが}しく\ruby{動}{うご}いた。と\ruby{思}{おも}ったら、「よかった!」。\ruby{話}{はなし}が見つかった。

それは、その日、トットちゃんが\ruby{着}{き}てる\ruby{洋服}{ようふく}のことだった。たいがいの\ruby{洋服}{ようふく}は、ママが\ruby{手製}{てせい}で\ruby{作}{つく}ってくれるのだけれど、\ruby{今日}{きょう}のは、\ruby{買}{か}ったものだった。というのも、なにしろトットちゃんが\ruby{夕方}{ゆうがた}、\ruby{外}{そと}から\ruby{帰}{かえ}ってきたとき、どの\ruby{洋服}{ようふく}もビリビリで、ときには、ジャキジャキのときもあったし、どうしてそうなるのか、ママにも\ruby{絶対}{ぜったい}わからないのだけれど、白い\ruby{木綿}{もめん}でゴム入りのパンツまで、ビリビリになっているのだから。トットちゃんの\ruby{話}{はなし}によると、よその\ruby{家}{いえ}の\ruby{庭}{にわ}をつっきって\ruby{垣根}{かきね}をもぐったり、\ruby{原}{はら}っぱの\ruby{鉄条}{てつじょう}\ruby{網}{あみ}をくぐるとき、「こんなになっちゃうんだ」ということなのだけれど、とにかく、そんな\ruby{具合}{ぐあい}で、\ruby{結局}{けっきょく}、\ruby{今朝}{けさ}、\ruby{家}{いえ}をでるとき、ママの\ruby{手製}{てせい}の、しゃれたのは、どれもビリビリで、\ruby{仕方}{しかた}なく、\ruby{前}{まえ}に\ruby{買}{か}ったのを\ruby{着}{き}てきたのだった。それはワンピースで、エンジとグレーの\ruby{細}{こま}かいチェックで、\ruby{布地}{ぬのじ}はジャージーだから、\ruby{悪}{わる}くはないけど、\ruby{衿}{えり}にしてある、花の\ruby{刺繍}{ししゅう}の、赤い\ruby{色}{いろ}が、ママは、「\ruby{趣味}{しゅみ}が\ruby{悪}{わる}い」といっていた。そのことを、トットちゃんは、\ruby{思}{おも}い\ruby{出}{だ}したのだった。だから、\ruby{急}{いそ}いで\ruby{椅子}{いす}から\ruby{降}{お}りると、\ruby{衿}{えり}を手で\ruby{持}{も}ち\ruby{上}{あ}げて、先生のそばに\ruby{行}{い}き、こういった。「この\ruby{衿}{えり}ね、ママ、\ruby{嫌}{きら}いなんだって!」

                                                                                                                                                                                                                                                                                                                                                                                                                                                                                                                                                                                                                                                                          それをいってしまったら、どう\ruby{考}{かんが}えてみても、\ruby{本当}{ほんとう}に、\ruby{話}{はな}しはもう\ruby{無}{な}くなった。トットちゃんは(\ruby{少}{すこ}し\ruby{悲}{かな}しい)と\ruby{思}{おも}った。トットちゃんが、そう\ruby{思}{おも}ったとき、先生が立ち上がった。そして、トットちゃんの\ruby{頭}{あたま}に、大きく\ruby{暖}{あたた}かい手を\ruby{置}{お}くと、「じゃ、これで、\ruby{君}{きみ}は、この学校の\ruby{生徒}{せいと}だよ」そういった。……その\ruby{時}{とき},トットちゃんは、なんだか、生まれて\ruby{初}{はじ}めて、\ruby{本当}{ほんとう}に\ruby{好}{す}きな人にあったような気がした。だって、生まれてから\ruby{今日}{きょう}まで、こんな\ruby{長}{なが}い\ruby{時間}{じかん}、\ruby{自分}{じぶん}の\ruby{話}{はなし}を\ruby{聞}{き}いてくれた人は、いなっかたんだもの。そして、その\ruby{長}{なが}い\ruby{時間}{じかん}の\ruby{間}{あいだ}、\ruby{一度}{いちど}だって、あくびをしたり、\ruby{退屈}{たいくつ}そうにしないで、トットちゃんが\ruby{話}{はな}してるのと\ruby{同}{おな}じように、\ruby{身}{み}を\ruby{乗}{の}り\ruby{出}{だ}して、\ruby{一生懸命}{いっしょうけんめい}、\ruby{聞}{き}いてくれたんだもの。

                                                                                                                                                                                                                                                                                                                                                                                                                                                                                                                                                                                                                                                                          トットちゃんは、このとき、まだ\ruby{時計}{とけい}が\ruby{読}{よ}めなかったんだけど、それでも\ruby{長}{なが}い\ruby{時間}{じかん}、と\ruby{思}{おも}ったくらいなんだから、もし\ruby{読}{よ}めたら、ビックリしたに\ruby{違}{ちが}いない。そして、もっと先生に\ruby{感謝}{かんしゃ}したに\ruby{違}{ちが}いない。というのは、トットちゃんとママが学校に\ruby{着}{つ}いたのが八\ruby{時}{じ}で、\ruby{校長}{こうちょう}\ruby{室}{しつ}で\ruby{全部}{ぜんぶ}の\ruby{話}{はなし}が\ruby{終}{お}わって、トットちゃんが、この学校の生\ruby{徒}{あだ}になった、と\ruby{決}{き}まったとき、先生が\ruby{懐中}{かいちゅう}\ruby{時計}{とけい}を見て、「ああ、お\ruby{弁当}{べんとう}の\ruby{時間}{じかん}だな」といったから、つまり、たっぷり四\ruby{時間}{じかん}、先生は、トットちゃんの\ruby{話}{はなし}を\ruby{聞}{き}いてくれたことになるのだった。\ruby{後}{あと}にも先にも、トットちゃんの\ruby{話}{はなし}を、こんなにちゃんと\ruby{聞}{き}いてくれた\ruby{大人}{おとな}は、いなかった。それにしても、まだ小学校一年生になったばかりのトットちゃんが、四\ruby{時間}{じかん}も、\ruby{一人}{ひとり}でしゃべるぶんの\ruby{話}{はな}しがあったことは、ママや、\ruby{前}{まえ}の学校の先生が\ruby{聞}{き}いたら、きっと、ビックリするに\ruby{違}{ちが}いないことだった。

                                                                                                                                                                                                                                                                                                                                                                                                                                                                                                                                                                                                                                                                          このとき、トットちゃんは、まだ\ruby{退学}{たいがく}のことはもちろん、\ruby{周}{まわ}りの\ruby{大人}{おとな}が、手こずってることも、気がついていなかったし、もともと\ruby{性格}{せいかく}も\ruby{陽気}{ようき}で、\ruby{忘}{わす}れっぽいタチだったから、\ruby{無邪気}{むじゃき}に見えた。でも、トットちゃんの中のどこかに、なんとなく、\ruby{疎外感}{そがいかん}のような、\ruby{他}{ほか}の\ruby{子供}{こども}と\ruby{違}{ちが}って、ひとりだけ、ちょっと、\ruby{冷}{つめ}たい目で見られているようなものを、おぼろげには\ruby{感}{かん}じていた。それが、この\ruby{校長}{こうちょう}先生といると、\ruby{安心}{あんしん}で、\ruby{暖}{あたた}かくて、\ruby{気持}{きも}ちがよかった。(この人となら、ずーっと\ruby{一緒}{いっしょ}にいてもいい)これが、\ruby{校長}{こうちょう}先生、\ruby{小林宗作}{こばやしそうさく}\ruby{氏}{し}に、\ruby{初}{はじ}めて\ruby{遭}{あ}った日、トットちゃんが\ruby{感}{かん}じた、\ruby{感想}{かんそう}だった。そして、\ruby{有難}{ありがた}いことに、\ruby{校長}{こうちょう}先生も、トットちゃんと、\ruby{同}{おな}じ\ruby{感想}{かんそう}を、その\ruby{時}{とき}、\ruby{持}{も}っていたのだった。




\chapter{お弁当}
トットちゃんは、\ruby{校長}{こうちょう}先生に\ruby{連}{つ}れられて、みんなが、お\ruby{弁当}{べんとう}を\ruby{食}{た}べるところを、見に\ruby{行}{い}くことになった。お\ruby{昼}{ひる}だけは、\ruby{電車}{でんしゃ}でなく、「みんな、\ruby{講堂}{こうどう}に\ruby{集}{あつ}まることになっている」と\ruby{校長}{こうちょう}先生が\ruby{教}{おし}えてくれた。\ruby{講堂}{こうどう}はさっきトットちゃんが上がってきた石の\ruby{階段}{かいだん}の、\ruby{突}{つ}き\ruby{当}{あ}たりにあった。いってみると、\ruby{生徒}{せいと}たちが、\ruby{大騒}{おおさわ}ぎをしながら、\ruby{机}{つくえ}と\ruby{椅子}{いす}を、\ruby{講堂}{こうどう}に、まーるく\ruby{輪}{わ}になるように、\ruby{並}{なら}べているところだった。\ruby{隅}{すみ}っこで、それを見ていたトットちゃんは、\ruby{校長}{こうちょう}先生の\ruby{上着}{うわぎ}を\ruby{引}{ひ}っ\ruby{張}{ぱ}って\ruby{聞}{き}いた。

「\ruby{他}{ほか}の\ruby{生徒}{せいと}は、どこにいるの?」

\ruby{校長}{こうちょう}先生は\ruby{答}{こた}えた。

「これで\ruby{全部}{ぜんぶ}なんだよ」

「\ruby{全部}{ぜんぶ}!?」

トットちゃんは、\ruby{信}{しん}じられない気がした。だって、\ruby{前}{まえ}の学校の一クラスと\ruby{同}{おな}じくらいしか、いないんだもの。そうすると、

「学校中で、五十人くらいなの?」

\ruby{校長}{こうちょう}先生は、「そうだ」といった。トットちゃんは、なにもかも、\ruby{前}{まえ}の学校と\ruby{違}{ちが}ってると\ruby{思}{おも}った。

みんなが\ruby{着席}{ちゃくせき}すると、\ruby{校長}{こうちょう}先生は、

「みんな、\ruby{海}{うみ}のものと、山のもの、もって\ruby{来}{き}たかい?」

と\ruby{聞}{き}いた。

「はーい」

みんな、それぞれの、お\ruby{弁当}{べんとう}の、ふたを\ruby{取}{と}った。

「どれどれ」

\ruby{校長}{こうちょう}先生は、\ruby{机}{つくえ}で\ruby{出来}{でき}た円の中に入ると、ひとりず、お\ruby{弁当}{べんとう}をのぞきながら、\ruby{歩}{ある}いている。

\ruby{生徒}{せいと}たちは、\ruby{笑}{わら}ったり、キイキイいったり、にぎやかだった。

「\ruby{海}{うみ}のものと、山のもの、って、なんだろう」

トットちゃんは、おかしくなった。でも、とっても、とっても、この学校は\ruby{変}{か}わっていて、\ruby{面白}{おもしろ}そう。お\ruby{弁当}{べんとう}の\ruby{時間}{じかん}が、こんなに、\ruby{愉快}{ゆかい}で、\ruby{楽}{たの}しいなんて、\ruby{知}{し}らなかった。トットちゃんは、\ruby{明日}{あした}からは、\ruby{自分}{じぶん}も、あの\ruby{机}{つくえ}に\ruby{座}{すわ}って、『\ruby{海}{うみ}のものと、山のもの』の\ruby{弁当}{べんとう}を、\ruby{校長}{こうちょう}先生に見てもらうんだ、と\ruby{思}{おも}うと、もう、\ruby{嬉}{うれ}しさと、\ruby{楽}{たの}しさで、\ruby{胸}{むね}がいっぱいになり、\ruby{叫}{さけ}びそうになった。 お\ruby{弁当}{べんとう}を、のぞきこんでる\ruby{校長}{こうちょう}先生の\ruby{肩}{かた}に、お\ruby{昼}{ひる}の\ruby{光}{ひかり}が、やわらかく\ruby{止}{と}まっていた。




\chapter{今日から学校に行く}
きのう、「\ruby{今日}{きょう}から、\ruby{君}{きみ}は、もう、この学校の\ruby{生徒}{せいと}だよ」、そう\ruby{校長}{こうちょう}先生に\ruby{言}{い}われたトットちゃんにとって、こんなに\ruby{次}{つぎ}の日が\ruby{待}{ま}ち\ruby{遠}{どお}しい、ってことは、\ruby{今}{いま}までになかった。だから、いつもなら\ruby{朝}{あさ}、ママが\ruby{叩}{たた}き\ruby{起}{お}こしても、まだベッドの上でぼんやりしてることの\ruby{多}{おお}いトットちゃんが、この日ばかりは、\ruby{誰}{だれ}からも\ruby{起}{お}こされない\ruby{前}{まえ}に、もうソックスまではいて、ランドセルを\ruby{背負}{しょ}って、みんなの\ruby{起}{お}きるのを\ruby{待}{ま}っていた。

ロッキーは、\ruby{途中}{とちゅう}までは、耳をピンと立てて\ruby{神妙}{しんみょう}に\ruby{聞}{き}いていたけど、\ruby{説明}{せつめい}の\ruby{終}{お}わりのところで、\ruby{定期}{ていき}を、ちょっと、なめてみて、それから、あくびをした。それでも、トットちゃんは、\ruby{一生懸命}{いっしょうけんめい}に\ruby{話}{はな}し\ruby{続}{つづ}けた。

「\ruby{電車}{でんしゃ}の\ruby{教室}{きょうしつ}は、\ruby{動}{うご}かないから、お\ruby{教室}{きょうしつ}では、\ruby{定期}{ていき}はいらないと\ruby{思}{おも}うんだ。とにかく、\ruby{今日}{きょう}は\ruby{持}{も}ってるのよ」

たしかにロッキーは、\ruby{今}{いま}まで、\ruby{歩}{ある}いて\ruby{通}{かよ}う学校の\ruby{門}{もん}まで、\ruby{毎日}{まいにち}、トットちゃんと\ruby{一緒}{いっしょ}に\ruby{行}{い}って、\ruby{後}{あと}は、\ruby{一人}{ひとり}で\ruby{家}{いえ}に\ruby{帰}{かえ}ってきていたから、\ruby{今日}{きょう}も、そのつもりでいた。

トットちゃんは、\ruby{定期}{ていき}をロッキーの\ruby{首}{くび}からはずすと、\ruby{大切}{たいせつ}そうに\ruby{自分}{じぶん}の\ruby{首}{くび}にかけると、パパとママに、もう\ruby{一度}{いちど}、 『\ruby{行}{い}ってまいりまーす』というと、\ruby{今度}{こんど}は\ruby{振}{ふ}り\ruby{返}{かえ}らずに、ランドセルをカタカタいわせて\ruby{走}{はし}り\ruby{出}{だ}した。ロッキーも、からだをのびのびさせながら、\ruby{並}{なら}んで\ruby{走}{はし}り\ruby{出}{だ}した。

\ruby{駅}{えき}までの\ruby{道}{みち}は、\ruby{前}{まえ}の学校に\ruby{行}{い}く\ruby{道}{みち}と、ほとんど\ruby{変}{か}わらなかった。だから、\ruby{途中}{とちゅう}でトットちゃんは、\ruby{顔見知}{かおみし}りの犬や\ruby{猫}{ねこ}や、\ruby{前}{まえ}の\ruby{同級}{どうきゅう}生と、すれ\ruby{違}{ちが}った。トットちゃんは、その\ruby{度}{たび}に、「\ruby{定期}{ていき}を見せて、\ruby{驚}{おどろ}かせてやろうかな?」と\ruby{思}{おも}ったけど、(もし\ruby{遅}{おそ}くなったら\ruby{大変}{たいへん}だから、\ruby{今日}{きょう}は、よそう……)と\ruby{決}{き}めて、どんどん\ruby{歩}{ある}いた。

\ruby{駅}{えき}のところに\ruby{来}{き}て、いつもなら左に\ruby{行}{い}くトットちゃんが、右に\ruby{曲}{ま}がったので、\ruby{可哀}{かわい}そうにロッキーは、とても\ruby{心配}{しんぱい}そうに\ruby{立}{た}ち\ruby{止}{どま}って、キョロキョロした。トットちゃんは、\ruby{改札口}{かいさつぐち}のところまで\ruby{行}{い}ったんだけど、\ruby{戻}{もど}ってきて、まだ\ruby{不思議}{ふしぎ}そうな\ruby{顔}{かお}をしてるロッキーにいった。

「もう、\ruby{前}{まえ}の学校には\ruby{行}{い}かないのよ。\ruby{新}{あたら}しい学校に\ruby{行}{い}くんだから」

それからトットちゃんは、ロッキーの\ruby{顔}{かお}に、\ruby{自分}{じぶん}の\ruby{顔}{かお}をくっつけ、ついでにロッキーの耳の中の、においをかいだ。(いつもと\ruby{同}{おな}じくらい、くさいけれど、\ruby{私}{わたし}には、いい、におい!)そう\ruby{思}{おも}うと\ruby{顔}{かお}を\ruby{離}{はな}して、「バイバイ」というと、\ruby{定期}{ていき}を\ruby{駅}{えき}の人に見せて、ちょっと\ruby{高}{たか}い\ruby{駅}{えき}の\ruby{階段}{かいだん}を、\ruby{登}{のぼ}り\ruby{始}{はじ}めた。ロッキーは、小さい\ruby{声}{こえ}で\ruby{鳴}{な}いて、トットちゃんが\ruby{階段}{かいだん}を上がっていくのを、いつまでも\ruby{見送}{みおく}っていた。

この\ruby{家}{いえ}の中で、いちばん、きちんと\ruby{時間}{じかん}を\ruby{守}{まも}るシェパードのロッキーは、トットちゃんの、いつもと\ruby{違}{ちが}う\ruby{行動}{こうどう}に、\ruby{怪訝}{けげん}そうな目を\ruby{向}{む}けながら、それでも、大きく\ruby{伸}{の}びをすると、トットちゃんにぴったりとくっついて、(\ruby{何}{なに}か\ruby{始}{はじ}まるらしい)ことを\ruby{期待}{きたい}した。

ママ\ruby{大変}{たいへん}だった。\ruby{大忙}{おおいそが}しで、『\ruby{海}{うみ}のものと山のもの』のお\ruby{弁当}{べんとう}を\ruby{作}{つく}り、トットちゃんに\ruby{朝}{あさ}ごはんを\ruby{食}{た}べさせ、\ruby{毛糸}{けいと}で\ruby{編}{あ}んだヒモを\ruby{通}{とお}した、セルロイドの\ruby{定期入}{ていきい}れを、トットちゃんの\ruby{首}{くび}にかけた。これは\ruby{定期}{ていき}を、なくさないためだった、パパは「いい子でね」と\ruby{頭}{あたま}をヒシャヒシャにしたまま\ruby{言}{い}った。「もちろん!」と、トットちゃんは\ruby{言}{い}うと、\ruby{玄関}{げんかん}で\ruby{靴}{くつ}を\ruby{履}{は}き、\ruby{戸}{と}を\ruby{開}{あ}けると、クルリと\ruby{家}{いえ}の中を\ruby{向}{む}き、\ruby{丁寧}{ていねい}にお\ruby{辞儀}{じぎ}をして、こういった。

「みなさま、\ruby{行}{い}ってまいります」

\ruby{見送}{みおく}りに立っていたママは、ちょっと\ruby{涙}{なみだ}でそうになった。それは、こんなに生き生きとしてお\ruby{行儀}{ぎょうぎ}よく、\ruby{素直}{すなお}で、\ruby{楽}{たの}しそうにしてるトットちゃんが、つい、このあいだ、「\ruby{退学}{たいがく}になった」、ということを\ruby{思}{おも}い\ruby{出}{だ}したからだった。(\ruby{新}{あたら}しい学校で、うまくいくといい……)ママは\ruby{心}{こころ}からそう\ruby{祈}{いの}った。

ところが、\ruby{次}{つぎ}の\ruby{瞬間}{しゅんかん}、ママは、\ruby{飛}{と}び\ruby{上}{あ}がるほど\ruby{驚}{おどろ}いた。というのは、トットちゃんが、せっかくママが\ruby{首}{くび}からかけた\ruby{定期}{ていき}を、ロッキーの\ruby{首}{くび}にかけているのを見たからだった。ママは、(\ruby{一体}{いったい}どうなるのだろう?)と\ruby{思}{おも}ったけど、だまって、\ruby{成}{な}り\ruby{行}{ゆ}きを見ることにした。トットちゃんは、\ruby{定期}{ていき}をロッキーの\ruby{首}{くび}にかけると、しゃがんで、ロッキーに、こういった。

「いい?この\ruby{定期}{ていき}のヒモは、あんたに、\ruby{合}{あ}わないのよ」

\ruby{確}{たし}かに、ロッキーにはヒモが\ruby{長}{なが}く、\ruby{定期}{ていき}は\ruby{地面}{じめん}を\ruby{引}{ひ}きずっていた。

「わかった?これは\ruby{私}{わたし}の\ruby{定期}{ていき}で、あんたのじゃないから、あんたは\ruby{電車}{でんしゃ}に\ruby{乗}{の}れないの。\ruby{校長}{こうちょう}先生に\ruby{聞}{き}いてみるけど、\ruby{駅}{えき}の人にも。で『いい』っていったら、あんたも学校に\ruby{来}{こ}られるんだけど、どうかなあ」




\chapter{電車の教室}
すぐに\ruby{僕}{ぼく}は\ruby{王子}{おうじ}さまの\ruby{花}{はな}の\ruby{事}{こと}を、もっとよく\ruby{知}{し}るようになった。\ruby{王子}{おうじ}さまの\ruby{星}{ほし}にはもともと\ruby{花}{はな}びらが\ruby{一重}{ひとえ}の\ruby{素朴}{そぼく}な\ruby{花}{はな}が\ruby{場所}{ばしょ}もとらず、\ruby{邪魔}{じゃま}にもならずに\ruby{咲}{さ}いていた。ところがある\ruby{日}{ひ}、どこからともなく\ruby{運}{はこ}ばれてきた\ruby{種}{たね}が\ruby{芽}{め}を\ruby{出}{だ}した。\ruby{王子}{おうじ}さまは\ruby{他}{ほか}のものとは\ruby{似}{に}ても\ruby{似}{に} つかないその\ruby{芽}{め}を\ruby{見}{み}つけて、\ruby{注意深}{ちゅういぶか}く\ruby{観察}{かんさつ}していた。\ruby{新種}{しんしゅ}のバオバブかもしれないからだ。

しかしそれはすぐに\ruby{伸}{の}びるのをやめ、\ruby{花}{はな}を\ruby{咲}{さ}かせる\ruby{準備}{じゅんび}を\ruby{始}{はじ}めた。ふっくらと\ruby{大}{おお}きく\ruby{艶}{あで}やかに\ruby{蕾}{つぼみ}が\ruby{育}{そだ}っていくのを\ruby{見}{み}て、\ruby{王子}{おうじ}さまは\ruby{奇跡}{きせき}のようなものが\ruby{現}{あらわ}れてくるのを\ruby{感}{かん}じていた。

しかし\ruby{花}{はな}は\ruby{緑}{みどり}の\ruby{部屋}{へや}に\ruby{隠}{かく}れたまま、\ruby{美}{うつく}しい\ruby{装}{よそお}いにかかりきりだった。\ruby{慎重}{しんちょう}に\ruby{色}{いろ}を\ruby{選}{えら}び、ゆっくり\ruby{衣装}{いしょう}を\ruby{纏}{まと}い、\ruby{花}{はな}びらを\ruby{一枚}{いちまい}ずつ\ruby{整}{ととの}える。\ruby{雛罌粟}{ひなげし}のように\ruby{皺}{しわ}くちゃな\ruby{姿}{すがた}は\ruby{見}{み}せたくなかった。これ\ruby{以上}{いじょう}はない\ruby{輝}{かがや}きを\ruby{放}{はな}つ\ruby{美}{うつく}しい\ruby{姿}{すがた}で\ruby{華麗}{かれい}に\ruby{登場}{とうじょう}したかった。そう、\ruby{花}{はな}はとてもお\ruby{洒落}{しゃれ}だった。

\ruby{謎}{なぞ}めいた\ruby{準備}{じゅんび}は\ruby{何日}{なんにち}も\ruby{続}{つづ}いた。そしてある\ruby{朝}{あさ}、ぴったり\ruby{日}{ひ}の\ruby{出}{で}の\ruby{時間}{じかん}に、\ruby{花}{はな}は\ruby{姿}{すがた}を\ruby{現}{あらわ}した。

そして、あれほど\ruby{念入}{ねんい}りに\ruby{装}{よそお}いを\ruby{凝}{こ}らしておきながら、\ruby{欠伸}{あくび}を\ruby{噛}{か}み\ruby{殺}{ころ}してこう\ruby{言}{い}った。

ああ、たった\ruby{今}{いま}\ruby{目}{め}が\ruby{覚}{さ}めたばかり、ごめんなさいね。\ruby{髪}{かみ}がぼそぼそだわ。

しかし\ruby{王子}{おうじ}さまは\ruby{感動}{かんどう}を\ruby{抑}{おさ}える\ruby{事}{こと}ができなかった。

なんて\ruby{綺麗}{きれい}なんだ、\ruby{君}{きみ}は。

でしょう?

\ruby{花}{はな}は\ruby{静}{しず}かに\ruby{答}{こた}えた。

\ruby{私}{わたし}はお\ruby{日様}{ひさま}と\ruby{一緒}{いっしょ}に\ruby{生}{う}まれたんですもの。

\ruby{王子}{おうじ}さまは\ruby{花}{はな}があまり\ruby{謙虚}{けんきょ}ではない\ruby{事}{こと}に\ruby{気付}{きづ}いたが、それでも\ruby{目}{め}が\ruby{眩}{くら}むほど\ruby{美}{うるわ}しかった。

そろそろ\ruby{朝食}{ちょうしょく}のお\ruby{時間}{じかん}ね、お\ruby{願}{ねが}いしてもよろしいかしら?

\ruby{王子}{おうじ}さまはすっかりドギマギしていたが、\ruby{如雨露}{じょうろ}に\ruby{新鮮}{しんせん}な\ruby{水}{みず}を\ruby{汲}{く}んできて、たっぷり\ruby{花}{はな}にかけてあげた。\ruby{花}{はな}はすぐに\ruby{気}{き}まぐれな\ruby{自惚}{うぬぼ}れで\ruby{王子}{おうじ}さまを\ruby{困}{こま}らせるようになった。\ruby{例}{たと}えばある\ruby{日}{ひ}、\ruby{自分}{じぶん}の四\ruby{本}{ほん}の\ruby{刺}{とげ}の\ruby{話}{はなし}をしながらこう\ruby{言}{い}った。

たとえ\ruby{虎}{とら}が\ruby{来}{き}ても\ruby{大丈夫}{だいじょうぶ}よ。\ruby{鋭}{するど}い\ruby{爪}{つめ}で。。

\ruby{僕}{ぼく}の\ruby{星}{ほし}には\ruby{虎}{とら}はいないよ。それに、\ruby{虎}{とら}は\ruby{草}{くさ}を\ruby{食}{た}べないし。

\ruby{私}{わたし}、\ruby{草}{くさ}ではないんですけど。

ごめんなさい。

\ruby{虎}{とら}なんかちっとも\ruby{怖}{こわ}くないけれど、\ruby{風}{かぜ}が\ruby{吹}{ふ}き\ruby{込}{こ}むのは\ruby{苦手}{にがて}なの。あなた、\ruby{衝立}{ついたて}はないのかしら。

\ruby{風}{かぜ}が\ruby{吹}{ふ}き\ruby{込}{こ}むのが\ruby{苦手}{にがて}だなんて、\ruby{植物}{しょくぶつ}なのに、\ruby{困}{こま}った\ruby{事}{こと}だな。この\ruby{花}{はな}は\ruby{結構}{けっこう}\ruby{気難}{きむずか}し\ruby{屋}{おく}さんだぞ。

\ruby{暗}{くら}くなったら、ガラスの\ruby{覆}{おお}いを\ruby{被}{かぶ}せてちょうだい?この\ruby{星}{ほし}はとても\ruby{寒}{さむ}いわ。\ruby{作}{つく}りが\ruby{悪}{わる}いのね。\ruby{前}{まえ}に\ruby{私}{わたし}がいた\ruby{所}{ところ}は。。。

\ruby{花}{はな}はいきなり\ruby{口}{ぐち}を\ruby{噤}{つぐ}んだ。\ruby{種}{たね}の\ruby{状態}{じょうたい}で\ruby{来}{き}たのだから、\ruby{他}{ほか}の\ruby{世界}{せかい}の\ruby{事}{こと}など\ruby{何一}{なにひと}つ\ruby{知}{し}っているはずがない。\ruby{花}{はな}はすぐにばれる\ruby{嘘}{うそ}をついてしまった\ruby{事}{こと}が\ruby{恥}{は}ずかしくて、\ruby{悪}{わる}いのは\ruby{王子}{おうじ}さまのせいにしようと、二\ruby{度}{ど}三\ruby{度}{ど}せきをしたで、\ruby{衝立}{ついたて}は?

\ruby{探}{さが}しに\ruby{行}{い}こうとしていたら、\ruby{君}{きみ}が\ruby{話}{はな}しかけてきたんでしょう。

すると\ruby{花}{はな}はわざとまたせきをして\ruby{王子}{おうじ}さまの\ruby{良心}{りょうしん}を\ruby{疼}{うず}かせた。

こうして\ruby{王子}{おうじ}さまは\ruby{心}{こころ}から\ruby{愛}{あい}していたにも\ruby{関}{かか}わらず、じきに\ruby{花}{はな}の\ruby{事}{こと}を\ruby{信用}{しんよう}できなくなっていった。\ruby{些細}{ささい}な\ruby{言葉}{ことば}を\ruby{一一}{いちいち}\ruby{深刻}{しんこく}に\ruby{受}{う}け\ruby{止}{と}め、そのたびに\ruby{不幸}{ふこう}になった。

\ruby{花}{はな}の\ruby{言}{い}う\ruby{事}{こと}なんか、\ruby{聞}{き}かないほうがよかったんだよ。ただ\ruby{眺}{なが}めたり、\ruby{香}{かお}りを\ruby{楽}{たの}しんでいればいいんだ。あの\ruby{花}{はな}は\ruby{僕}{ぼく}の\ruby{星}{ほし}をいい\ruby{香}{かお}りで\ruby{満}{み}たしてくれた。それなのに\ruby{僕}{ぼく}はそれを\ruby{楽}{たの}しめなかった。\ruby{虎}{とら}の\ruby{爪}{つめ}の\ruby{話}{はなし}にしても、\ruby{僕}{ぼく}はうんざりしたけれど、\ruby{花}{はな}にして\ruby{見}{み}れば、ほろりとさせるつもりだったのかもしれない。あの\ruby{頃}{ころ}の\ruby{僕}{ぼく}は\ruby{何}{なに}もわかっていなかったんだね。\ruby{言葉}{ことば}ではなく、\ruby{振}{ふ}る\ruby{舞}{ま}いで\ruby{判断}{はんだん}しなくちゃいけなかったんだ。\ruby{花}{はな}は\ruby{僕}{ぼく}の\ruby{星}{ほし}をいい\ruby{香}{かお}りで\ruby{満}{み}たし、\ruby{明}{あか}るくしてくれた。\ruby{僕}{ぼく}は\ruby{逃}{に}げちゃいけなかったんだ。つまらない\ruby{見}{み}せかけに\ruby{隠}{かく}れた\ruby{花}{はな}の\ruby{優}{やさ}しさに\ruby{気付}{きづ}くべきだった。\ruby{花}{はな}って\ruby{本当}{ほんとう}に\ruby{矛盾}{むじゅん}しているからね、でも\ruby{僕}{ぼく}はまだ\ruby{子供}{こども}で、あの\ruby{花}{はな}の\ruby{愛}{あい}し\ruby{方}{かた}がわからなかったんだ。




\end{document}