トットちゃんが、きのう、\ruby{校長}{こうちょう}先生から\ruby{教}{おし}えていただいた、\ruby{自分}{じぶん}の\ruby{教室}{きょうしつ}である、\ruby{電車}{でんしゃ}のドアに手をかけたとき、まだ\ruby{校庭}{こうてい}には、\ruby{誰}{だれ}の\ruby{姿}{すがた}も見えなかった。\ruby{今}{いま}と\ruby{違}{ちが}って、\ruby{昔}{むかし}の\ruby{電車}{でんしゃ}は、\ruby{外}{そと}から\ruby{開}{ひら}くように、ドアに\ruby{取手}{とりて}がついていた。\ruby{両手}{りょうて}で、その\ruby{取手}{とりて}を\ruby{持}{も}って、右に\ruby{引}{ひ}くと、ドアは、すぐ\ruby{開}{ひら}いた。トットちゃんは、ドキドキしながら、そーっと、\ruby{首}{くび}を\ruby{突}{つ}っ\ruby{込}{こ}んで、中を見てみた。

「わあーい」

これなら、\ruby{勉強}{べんきょう}しながら、いつも\ruby{旅行}{りょこう}をしてるみたいじゃない。\ruby{網棚}{あみだな}もあるし、\ruby{窓}{まど}も\ruby{全部}{ぜんぶ}、そのままだし。\ruby{違}{ちが}うところは、\ruby{運転手}{うんてんしゅ}さんの\ruby{席}{せき}のところに\ruby{黒板}{こくばん}があるのと、\ruby{電車}{でんしゃ}の\ruby{長}{なが}い\ruby{腰掛}{こしかけ}を、はずして、\ruby{生徒}{せいと}\ruby{用}{よう}の\ruby{机}{つくえ}と\ruby{腰掛}{こしかけ}が\ruby{進行}{しんこう}\ruby{方向}{ほうこう}に\ruby{向}{む}いて\ruby{並}{なら}んでいるのと、つり\ruby{革}{かわ}が\ruby{無}{な}いところだけ。\ruby{後}{あと}は、\ruby{天井}{てんじょう}も\ruby{床}{ゆか}も、\ruby{全部}{ぜんぶ}、\ruby{電車}{でんしゃ}のままになっていた。トットちゃんは\ruby{靴}{くつ}を\ruby{脱}{ぬ}いで中に入り、\ruby{誰}{だれ}でも\ruby{腰掛}{こしか}けていたいくらい、\ruby{気持}{きも}ちのいい\ruby{椅子}{いす}だった。トットちゃんは、うれしくて、(こんな気に入った学校は、\ruby{絶対}{ぜったい}に、お休みなんかしないで、ずーっとくる)と,\ruby{強}{つよ}く\ruby{心}{こころ}に\ruby{思}{おも}った。 それからトットちゃんは、\ruby{窓}{まど}から\ruby{外}{そと}を見ていた。すると、\ruby{動}{うご}いていないはずの\ruby{電車}{でんしゃ}なのに、\ruby{校庭}{こうてい}の花や木が、\ruby{少}{すこ}し\ruby{風}{ふう}に\ruby{揺}{ゆ}れているせいか、\ruby{電車}{でんしゃ}が\ruby{走}{はし}っているような\ruby{気持}{きも}ちになった。

「ああ、\ruby{嬉}{うれ}しいなあー」

トットちゃんは、とうとう\ruby{声}{こえ}に出して、そういった。それから、\ruby{顔}{かお}をぺったり\ruby{ガラス}{がらす}\ruby{窓}{まど}にくっつけると、いつも、\ruby{嬉}{うれ}しいとき、そうするように、デタラメ\ruby{歌}{うた}を、うたいはじめた。 とても うれし うれし とても どうしてかっていえば……そこまで\ruby{歌}{うた}ったとき、\ruby{誰}{だれ}かが\ruby{乗}{の}り\ruby{込}{こ}んできた。女の子だった。その子は、ノートと\ruby{筆箱}{ふでばこ}をランとセルから出して\ruby{机}{つくえ}の上に\ruby{置}{お}くと、\ruby{背伸}{せの}びをして、\ruby{網棚}{あみだな}にランドセルをのせた。それから\ruby{草履}{ぞうり}\ruby{袋}{ぶくろ}も、のせた。トットちゃんは\ruby{歌}{うた}をやめて、\ruby{急}{いそ}いで、まねをした。\ruby{次}{つぎ}に、男の子が\ruby{乗}{の}ってきた。その子は、ドアのところから、バスケットボールのように、ランドセルを、\ruby{網棚}{あみだな}に\ruby{投}{な}げ\ruby{込}{こ}んだ。\ruby{網棚}{あみだな}の、\ruby{網}{あみ}は、大きく\ruby{波}{なみ}うつと、ランドセルを、\ruby{投}{な}げ\ruby{出}{だ}した。ランドセルは、\ruby{床}{ゆか}に\ruby{落}{お}ちた。その男の子は、「\ruby{失敗}{しっぱい}!」というと、またもや、\ruby{同}{おな}じところから、\ruby{網棚}{あみだな}めがけて、\ruby{投}{な}げ\ruby{込}{こ}んだ。\ruby{今度}{こんど}は、うまく、おさまった。『\ruby{成功}{せいこう}!』と、その子は\ruby{叫}{さけ}ぶと、すぐ、「\ruby{失敗}{しっぱい}!」といって、\ruby{机}{つくえ}によじ\ruby{登}{のぼ}ると、\ruby{網棚}{あみだな}のランドセルを\ruby{開}{あ}けて、\ruby{筆箱}{ふでばこ}やノートを出した。そういうのを出すのを\ruby{忘}{わす}れたから、\ruby{失敗}{しっぱい}だったに\ruby{違}{ちが}いなかった。

こうして、九人の\ruby{生徒}{せいと}が、トットちゃんの\ruby{電車}{でんしゃ}に\ruby{乗}{の}り\ruby{込}{こ}んできて、それが、トモエ\ruby{学園}{がくえん}の、一年生の\ruby{全員}{ぜんいん}だった。 そしてそれは、\ruby{同}{おな}じ\ruby{電車}{でんしゃ}で\ruby{旅}{たび}をする、\ruby{仲間}{なかま}だった。


