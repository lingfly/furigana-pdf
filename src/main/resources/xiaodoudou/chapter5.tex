トットちゃんとママが入っていくと、\ruby{部屋}{へや}の中にいた男の人が\ruby{椅子}{いす}から立ち上がった。その人は、\ruby{頭}{あたま}の\ruby{毛}{け}が\ruby{薄}{うす}くなっていて、\ruby{前}{まえ}のほうの\ruby{歯}{は}が\ruby{抜}{ぬ}けていて、\ruby{顔}{かお}の\ruby{血色}{けっしょく}がよく、\ruby{背}{せ}はあまり\ruby{高}{たか}くないけど、\ruby{肩}{かた}や\ruby{腕}{うで}が、がっちりしていて、ヨレヨレの\ruby{黒}{くろ}の\ruby{三}{み}つ\ruby{揃}{ぞろ}いを、キチンと\ruby{着}{き}ていた。トットちゃんは、\ruby{急}{いそ}いで、お\ruby{辞儀}{じぎ}をしてから、\ruby{元気}{げんき}よく\ruby{聞}{き}いた。「\ruby{校長}{こうちょう}先生か、\ruby{駅}{えき}の人か、どっち?」「\ruby{校長}{こうちょう}先生だよ」トットちゃんは、とってもうれしそうに\ruby{言}{い}った。「よかった。じゃ、おねがい。\ruby{私}{わたし}、この学校にいりたいの」\ruby{校長}{こうちょう}先生は、\ruby{椅子}{いす}をトットちゃんに\ruby{勧}{すす}めると、ママのほうを\ruby{向}{む}いて\ruby{言}{い}った。「じゃ、\ruby{僕}{ぼく}は、これからトットちゃんと\ruby{話}{はなし}がありますから、もう、お\ruby{帰}{かえ}り下さって\ruby{結構}{けっこう}です」ほんのちょっとの\ruby{間}{あいだ}、トットちゃんは、\ruby{少}{すこ}し\ruby{心細}{こころぼそ}い気がしたけど、なんとなく、(この\ruby{校長}{こうちょう}先生ならいいや)と\ruby{思}{おも}った。ママは、いさぎよく先生にいった。「じゃ、よろしく、お\ruby{願}{ねが}いします」そして、ドアを\ruby{閉}{し}めて\ruby{出}{で}て\ruby{行}{い}った。\ruby{校長}{こうちょう}先生は、トットちゃんの\ruby{前}{まえ}に\ruby{椅子}{いす}を\ruby{引}{ひ}っ\ruby{張}{ぱ}ってきて、とても\ruby{近}{ちか}い\ruby{位置}{いち}に、\ruby{向}{む}かい\ruby{合}{あ}わせに\ruby{腰}{こし}をかけると、こういった。「さあ、\ruby{何}{なん}でも、先生に\ruby{話}{はな}してごらん。\ruby{話}{はな}したいこと、\ruby{全部}{ぜんぶ}」「\ruby{話}{はな}したいこと!?」(なにか\ruby{聞}{き}かれて、お\ruby{返事}{へんじ}するのかな?)と\ruby{思}{おも}っていたトットちゃんは、「\ruby{何}{なん}でも\ruby{話}{はな}していい」と\ruby{聞}{き}いて、ものすごくうれしくなって、すぐ\ruby{話}{はな}し\ruby{始}{はじ}めた。\ruby{順序}{じゅんじょ}も、\ruby{話}{はな}し\ruby{方}{かた}も、\ruby{少}{すこ}しグチャグチャだったけど、\ruby{一生懸命}{いっしょうけんめい}に\ruby{話}{はな}した。\ruby{今}{いま}\ruby{乗}{の}ってきた\ruby{電車}{でんしゃ}が\ruby{速}{はや}かったこと。

\ruby{駅}{えき}の\ruby{改札口}{かいさつぐち}のおじさんに、お\ruby{願}{ねが}いしたけど、\ruby{切符}{きっぷ}をくれなかったこと。\ruby{前}{まえ}に\ruby{行}{い}ってた学校の\ruby{受}{う}け\ruby{持}{も}ちの女の先生は、\ruby{顔}{かお}がきれいだということ。その学校には、つばめの\ruby{巣}{す}があること。\ruby{家}{いえ}には、ロッキーという\ruby{茶色}{ちゃいろ}の犬がいて“お手”と“ごめんくださいませ”と、ご\ruby{飯}{はん}の\ruby{後}{あと}で、“\ruby{満足}{まんぞく}、\ruby{満足}{まんぞく}”ができること。\ruby{幼稚}{ようち}\ruby{園}{えん}のとき、ハサミを口の中に入れて、チョキチョキやると、「\ruby{舌}{した}を\ruby{切}{き}ります」と先生が\ruby{怒}{いか}ったけど、\ruby{何回}{なんかい}もやっちゃったっていうこと。\ruby{洟}{はな}が出てきたときは、いつまでも、ズルズルやってると、ママにしかられるから、なるべく早くかむこと。パパは、\ruby{海}{うみ}で\ruby{泳}{およ}ぐのが\ruby{上手}{じょうず}で、\ruby{飛}{と}び\ruby{込}{こ}みだって\ruby{出来}{でき}ること。こういったことを、\ruby{次}{つぎ}から\ruby{次}{つぎ}と、トットちゃんは\ruby{話}{はな}した。先生は、\ruby{笑}{わら}ったり、うなずいたり、「これから?」とかいったりしてくださったから、うれしくて、トットちゃんは、いつまでも\ruby{話}{はな}した。でも、とうとう、\ruby{話}{はなし}がなくなった。トットちゃんは、口をつぐんで\ruby{考}{かんが}えていると、先生はいった。「もう、ないかい?」トットちゃんは、これでおしまいにしてしまうのは、\ruby{残念}{ざんねん}だと\ruby{思}{おも}った。せっかく、\ruby{話}{はなし}を、いっぱい\ruby{聞}{き}いてもらう、いいチャンスなのに。(なにか、\ruby{話}{はなし}は、ないかなあ……)\ruby{頭}{あたま}の中が、\ruby{忙}{いそが}しく\ruby{動}{うご}いた。と\ruby{思}{おも}ったら、「よかった!」。\ruby{話}{はなし}が見つかった。それは、その日、トットちゃんが\ruby{着}{き}てる\ruby{洋服}{ようふく}のことだった。たいがいの\ruby{洋服}{ようふく}は、ママが\ruby{手製}{てせい}で\ruby{作}{つく}ってくれるのだけれど、\ruby{今日}{きょう}のは、\ruby{買}{か}ったものだった。というのも、なにしろトットちゃんが\ruby{夕方}{ゆうがた}、\ruby{外}{そと}から\ruby{帰}{かえ}ってきたとき、どの\ruby{洋服}{ようふく}もビリビリで、ときには、ジャキジャキのときもあったし、どうしてそうなるのか、ママにも\ruby{絶対}{ぜったい}わからないのだけれど、白い\ruby{木綿}{もめん}でゴム入りのパンツまで、ビリビリになっているのだから。トットちゃんの\ruby{話}{はなし}によると、よその\ruby{家}{いえ}の\ruby{庭}{にわ}をつっきって\ruby{垣根}{かきね}をもぐったり、\ruby{原}{はら}っぱの\ruby{鉄条}{てつじょう}\ruby{網}{あみ}をくぐるとき、「こんなになっちゃうんだ」ということなのだけれど、とにかく、そんな\ruby{具合}{ぐあい}で、\ruby{結局}{けっきょく}、\ruby{今朝}{けさ}、\ruby{家}{いえ}をでるとき、ママの\ruby{手製}{てせい}の、しゃれたのは、どれもビリビリで、\ruby{仕方}{しかた}なく、\ruby{前}{まえ}に\ruby{買}{か}ったのを\ruby{着}{き}てきたのだった。それはワンピースで、エンジとグレーの\ruby{細}{こま}かいチェックで、\ruby{布地}{ぬのじ}はジャージーだから、\ruby{悪}{わる}くはないけど、\ruby{衿}{えり}にしてある、花の\ruby{刺繍}{ししゅう}の、赤い\ruby{色}{いろ}が、ママは、「\ruby{趣味}{しゅみ}が\ruby{悪}{わる}い」といっていた。そのことを、トットちゃんは、\ruby{思}{おも}い\ruby{出}{だ}したのだった。だから、\ruby{急}{いそ}いで\ruby{椅子}{いす}から\ruby{降}{お}りると、\ruby{衿}{えり}を手で\ruby{持}{も}ち\ruby{上}{あ}げて、先生のそばに\ruby{行}{い}き、こういった。「この\ruby{衿}{えり}ね、ママ、\ruby{嫌}{きら}いなんだって!」

                                                                                                                                                                                                                                                                                                                                                                                                                                                                                                                                                                                                                                                                          それをいってしまったら、どう\ruby{考}{かんが}えてみても、\ruby{本当}{ほんとう}に、\ruby{話}{はな}しはもう\ruby{無}{な}くなった。トットちゃんは(\ruby{少}{すこ}し\ruby{悲}{かな}しい)と\ruby{思}{おも}った。トットちゃんが、そう\ruby{思}{おも}ったとき、先生が立ち上がった。そして、トットちゃんの\ruby{頭}{あたま}に、大きく\ruby{暖}{あたた}かい手を\ruby{置}{お}くと、「じゃ、これで、\ruby{君}{きみ}は、この学校の\ruby{生徒}{せいと}だよ」そういった。……その\ruby{時}{とき},トットちゃんは、なんだか、生まれて\ruby{初}{はじ}めて、\ruby{本当}{ほんとう}に\ruby{好}{す}きな人にあったような気がした。だって、生まれてから\ruby{今日}{きょう}まで、こんな\ruby{長}{なが}い\ruby{時間}{じかん}、\ruby{自分}{じぶん}の\ruby{話}{はなし}を\ruby{聞}{き}いてくれた人は、いなっかたんだもの。そして、その\ruby{長}{なが}い\ruby{時間}{じかん}の\ruby{間}{あいだ}、\ruby{一度}{いちど}だって、あくびをしたり、\ruby{退屈}{たいくつ}そうにしないで、トットちゃんが\ruby{話}{はな}してるのと\ruby{同}{おな}じように、\ruby{身}{み}を\ruby{乗}{の}り\ruby{出}{だ}して、\ruby{一生懸命}{いっしょうけんめい}、\ruby{聞}{き}いてくれたんだもの。

                                                                                                                                                                                                                                                                                                                                                                                                                                                                                                                                                                                                                                                                          トットちゃんは、このとき、まだ\ruby{時計}{とけい}が\ruby{読}{よ}めなかったんだけど、それでも\ruby{長}{なが}い\ruby{時間}{じかん}、と\ruby{思}{おも}ったくらいなんだから、もし\ruby{読}{よ}めたら、ビックリしたに\ruby{違}{ちが}いない。そして、もっと先生に\ruby{感謝}{かんしゃ}したに\ruby{違}{ちが}いない。というのは、トットちゃんとママが学校に\ruby{着}{つ}いたのが八\ruby{時}{じ}で、\ruby{校長}{こうちょう}\ruby{室}{しつ}で\ruby{全部}{ぜんぶ}の\ruby{話}{はなし}が\ruby{終}{お}わって、トットちゃんが、この学校の生\ruby{徒}{あだ}になった、と\ruby{決}{き}まったとき、先生が\ruby{懐中}{かいちゅう}\ruby{時計}{とけい}を見て、「ああ、お\ruby{弁当}{べんとう}の\ruby{時間}{じかん}だな」といったから、つまり、たっぷり四\ruby{時間}{じかん}、先生は、トットちゃんの\ruby{話}{はなし}を\ruby{聞}{き}いてくれたことになるのだった。\ruby{後}{あと}にも先にも、トットちゃんの\ruby{話}{はなし}を、こんなにちゃんと\ruby{聞}{き}いてくれた\ruby{大人}{おとな}は、いなかった。それにしても、まだ小学校一年生になったばかりのトットちゃんが、四\ruby{時間}{じかん}も、\ruby{一人}{ひとり}でしゃべるぶんの\ruby{話}{はな}しがあったことは、ママや、\ruby{前}{まえ}の学校の先生が\ruby{聞}{き}いたら、きっと、ビックリするに\ruby{違}{ちが}いないことだった。

                                                                                                                                                                                                                                                                                                                                                                                                                                                                                                                                                                                                                                                                          このとき、トットちゃんは、まだ\ruby{退学}{たいがく}のことはもちろん、\ruby{周}{まわ}りの\ruby{大人}{おとな}が、手こずってることも、気がついていなかったし、もともと\ruby{性格}{せいかく}も\ruby{陽気}{ようき}で、\ruby{忘}{わす}れっぽいタチだったから、\ruby{無邪気}{むじゃき}に見えた。でも、トットちゃんの中のどこかに、なんとなく、\ruby{疎外感}{そがいかん}のような、\ruby{他}{ほか}の\ruby{子供}{こども}と\ruby{違}{ちが}って、ひとりだけ、ちょっと、\ruby{冷}{つめ}たい目で見られているようなものを、おぼろげには\ruby{感}{かん}じていた。それが、この\ruby{校長}{こうちょう}先生といると、\ruby{安心}{あんしん}で、\ruby{暖}{あたた}かくて、\ruby{気持}{きも}ちがよかった。(この人となら、ずーっと\ruby{一緒}{いっしょ}にいてもいい)これが、\ruby{校長}{こうちょう}先生、\ruby{小林宗作}{こばやしそうさく}\ruby{氏}{し}に、\ruby{初}{はじ}めて\ruby{遭}{あ}った日、トットちゃんが\ruby{感}{かん}じた、\ruby{感想}{かんそう}だった。そして、\ruby{有難}{ありがた}いことに、\ruby{校長}{こうちょう}先生も、トットちゃんと、\ruby{同}{おな}じ\ruby{感想}{かんそう}を、その\ruby{時}{とき}、\ruby{持}{も}っていたのだった。


