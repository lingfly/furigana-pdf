お\ruby{教室}{きょうしつ}が\ruby{本当}{ほんとう}の\ruby{電車}{でんしゃ}で、“かわってる”と\ruby{思}{おも}ったトットちゃんが、\ruby{次}{つぎ}に“かわってる”と\ruby{思}{おも}ったのは、\ruby{教室}{きょうしつ}で\ruby{座}{すわ}る\ruby{場所}{ばしょ}だった。\ruby{前}{まえ}の学校は、\ruby{誰}{だれ}かさんは、どの\ruby{机}{つくえ}、\ruby{隣}{となり}は\ruby{誰}{だれ}、\ruby{前}{まえ}は\ruby{誰}{だれ}、と\ruby{決}{き}まっていた。ところが、この学校は、どこでも、\ruby{次}{つぎ}の日の\ruby{気分}{きぶん}や\ruby{都合}{つごう}で、\ruby{毎日}{まいにち}、\ruby{好}{す}きなところに\ruby{座}{すわ}っていいのだった。

そこでトットちゃんは、さんざん\ruby{考}{かんが}え、そして\ruby{見回}{みまわ}したあげく、\ruby{朝}{あさ}、トットちゃんの\ruby{次}{つぎ}に\ruby{教室}{きょうしつ}に入ってきた女の子の\ruby{隣}{となり}に\ruby{座}{すわ}ることに\ruby{決}{き}めた。なぜなら、この子が、\ruby{長}{なが}い耳をした\ruby{兎}{うさぎ}の\ruby{絵}{え}のついた、ジャンパースカートをはいていたからだった。

でも、なによりも“かわっていた”のは、この学校の、\ruby{授業}{じゅぎょう}のやりかただった。

\ruby{普通}{ふつう}の学校は、一\ruby{時間}{じかん}目が\ruby{国語}{こくご}なら、\ruby{国語}{こくご}をやって、二\ruby{時間}{じかん}目が\ruby{算数}{さんすう}なら、\ruby{算数}{さんすう}、という\ruby{風}{かぜ}に、\ruby{時間}{じかん}\ruby{割}{わり}の\ruby{通}{とお}りの\ruby{順番}{じゅんばん}なのだけど、この学校は、まるっきり\ruby{違}{ちが}っていた。

\ruby{何}{なに}しろ、一\ruby{時間}{じかん}目が\ruby{始}{はじ}まるときに、その日、一日やる\ruby{時間}{じかん}\ruby{割}{わり}の、\ruby{全部}{ぜんぶ}の\ruby{科目}{かもく}の\ruby{問題}{もんだい}を、女の先生が、\ruby{黒板}{こくばん}にいっぱいに\ruby{書}{か}いちゃって、

「さあ、どれでも\ruby{好}{す}きなのから、\ruby{始}{はじ}めてください」

といったんだ。だから\ruby{生徒}{せいと}は、\ruby{国語}{こくご}であろうと、\ruby{算数}{さんすう}であろうと、\ruby{自分}{じぶん}の\ruby{好}{す}きなのから\ruby{始}{はじ}めていっこうに、かまわないのだった。だから、\ruby{作文}{さくぶん}の\ruby{好}{す}きな子が、\ruby{作文}{さくぶん}を\ruby{書}{か}いていると、\ruby{後}{うし}ろでは、\ruby{物理}{ぶつり}の\ruby{好}{す}きな子が、アルコールランプに火をつけて、フラスコをブクブクやったり、\ruby{何}{なに}かを\ruby{爆発}{ばくはつ}させてる、なんていう\ruby{光景}{こうけい}は、どの\ruby{教室}{きょうしつ}でもみられることだった。この\ruby{授業}{じゅぎょう}のやり\ruby{方}{かた}は、\ruby{上級}{じょうきゅう}になるにしたがって、その\ruby{子供}{こども}の\ruby{興味}{きょうみ}を\ruby{持}{も}っているもの、\ruby{興味}{きょうみ}の\ruby{持}{も}ち\ruby{方}{かた}、\ruby{物}{もの}の\ruby{考}{かんが}え\ruby{方}{かた}、そして、\ruby{個性}{こせい}、といったものが、先生に、はっきり\ruby{分}{わ}かってくるから、先生にとって、\ruby{生徒}{せいと}を\ruby{知}{し}る上で、\ruby{何}{なに}よりの\ruby{勉強法}{べんきょうほう}だった。

また、\ruby{生徒}{せいと}にとっても、\ruby{好}{す}きな\ruby{学科}{がっか}からやっていい、というのは、\ruby{嬉}{うれ}しいことだったし、\ruby{嫌}{きら}いな\ruby{学科}{がっか}にしても、学校が\ruby{終}{お}わる\ruby{時間}{じかん}までに、やればいいのだから、\ruby{何}{なん}とか、やりくり\ruby{出来}{でき}た。\ruby{従}{したが}って、\ruby{自習}{じしゅう}の\ruby{形式}{けいしき}が\ruby{多}{おお}く、いよいよ、\ruby{分}{わ}からなくなってくると、先生のところに\ruby{聞}{き}きに\ruby{行}{い}くか、\ruby{自分}{じぶん}の\ruby{席}{せき}に先生に\ruby{来}{き}ていただいて、\ruby{納得}{なっとく}の\ruby{行}{い}くまで、\ruby{教}{おし}えてもらう。そして、\ruby{例}{れい}\ruby{題}{だい}をもらって、また\ruby{自習}{じしゅう}に入る。これは\ruby{本当}{ほんとう}の\ruby{勉強}{べんきょう}だった。だから、先生の\ruby{話}{はなし}や\ruby{説明}{せつめい}を、ボンヤリ\ruby{聞}{き}く、といった\ruby{事}{こと}は、\ruby{無}{な}いにひとしかった。トットちゃん\ruby{達}{たち}、一年生は、まだ\ruby{自習}{じしゅう}をするほどの\ruby{勉強}{べんきょう}を\ruby{始}{はじ}めていなかったけど、それでも、\ruby{自分}{じぶん}の\ruby{好}{す}きな\ruby{科目}{かもく}から\ruby{勉強}{べんきょう}する、ということには、かわりなかった。カタカナを\ruby{書}{か}く子、\ruby{絵}{え}を\ruby{描}{か}く子。本を\ruby{読}{よ}んでる子。中には、\ruby{体操}{たいそう}をしている子もいた。トットちゃんの\ruby{隣}{となり}の女の子は、もう、ひらがなが\ruby{書}{か}けるらしく、ノートに\ruby{写}{うつ}していた。トットちゃんは、\ruby{何}{なに}もかもが\ruby{珍}{めずら}しくて、ワクワクしちゃって、みんなみたいに、すぐ\ruby{勉強}{べんきょう}、というわけにはいかなかった。そんな\ruby{時}{とき}、トットちゃんの\ruby{後}{うし}ろの\ruby{机}{つくえ}の男の子が立ち上がって、\ruby{黒板}{こくばん}のほうに\ruby{歩}{ある}き\ruby{出}{だ}した。ノートを\ruby{持}{も}って。\ruby{黒板}{こくばん}の\ruby{横}{よこ}の\ruby{机}{つくえ}で、\ruby{他}{ほか}の子に\ruby{何}{なに}かを\ruby{教}{おし}えている先生のところに\ruby{行}{い}くらしかった。その子の\ruby{歩}{ある}くのを、\ruby{後}{うし}ろから見たトットちゃんは、それまでキョロキョロしてた\ruby{動作}{どうさ}をピタリと\ruby{止}{と}めて、\ruby{頬杖}{ほおづえ}をつき、ジーっと、その子を見つめた。その子は、\ruby{歩}{ある}くとき、足を\ruby{引}{ひ}きずっていた。とっても、\ruby{歩}{ある}くとき、\ruby{体}{からだ}が\ruby{揺}{ゆ}れた。\ruby{始}{はじ}めは、わざとしているのか、と\ruby{思}{おも}ったくらいだった。でも、やっぱり、わざとじゃなくて、そういう\ruby{風}{かぜ}になっちゃうんだ、と、しばらく見ていたトットちゃんに\ruby{分}{わ}かった。その子が、\ruby{自分}{じぶん}の\ruby{机}{つくえ}に\ruby{戻}{もど}ってくるのを、トットちゃんは、さっきの、\ruby{頬杖}{ほおづえ}のまま、見た。目と目が\ruby{合}{あ}った。その男の子は、トットちゃんを見ると、ニコリと\ruby{笑}{わら}った。トットちゃんも、あわてて、ニコリとした。その子が、\ruby{後}{うし}ろの\ruby{席}{せき}に\ruby{座}{すわ}ると、――\ruby{座}{すわ}るのも、\ruby{他}{ほか}の子より、\ruby{時間}{じかん}がかかったんだけど――トットちゃんは、クルリと\ruby{振}{ふ}り\ruby{向}{む}いて、その子に\ruby{聞}{き}いた。「どうして、そんな\ruby{風}{ふう}に\ruby{歩}{ある}くの?」その子は、\ruby{優}{やさ}しい\ruby{声}{こえ}で\ruby{静}{しず}かに\ruby{答}{こた}えた。とても\ruby{利口}{りこう}そうな\ruby{声}{こえ}だった。「\ruby{僕}{ぼく}、\ruby{小児}{しょうに}\ruby{麻痺}{まひ}なんだ」「しょうにまひ?」トットちゃんは、それまで、そういう\ruby{言葉}{ことば}を\ruby{聴}{き}いたことが\ruby{無}{な}かったから、\ruby{聞}{き}き\ruby{返}{かえ}した。その子は、\ruby{少}{すこ}し小さい\ruby{声}{こえ}でいった。「そう、\ruby{小児}{しょうに}\ruby{麻痺}{まひ}。足だけじゃないよ。手だって……」そういうと、その子は、\ruby{長}{なが}い\ruby{指}{ゆび}と\ruby{指}{ゆび}が、くっついて、\ruby{曲}{ま}がったみたいになった手を出した。トットちゃんは、その左手を見ながら、「\ruby{直}{なお}らないの?」と\ruby{心配}{しんぱい}になって\ruby{聞}{き}いた。その子は、\ruby{黙}{だま}っていた。トットちゃんは、\ruby{悪}{わる}いことを\ruby{聞}{き}いたのかと\ruby{悲}{かな}しくなった。すると、その子は、\ruby{明}{あか}るい\ruby{声}{こえ}で\ruby{言}{い}った。「\ruby{僕}{ぼく}の\ruby{名前}{なまえ}は、やまもとやすあき。\ruby{君}{きみ}は?」トットちゃんは、その子が\ruby{元気}{げんき}な\ruby{声}{こえ}を出したので、\ruby{嬉}{うれ}しくなって、大きな\ruby{声}{こえ}で\ruby{言}{い}った。「トットちゃんよ」こうして、山本\ruby{泰明}{やすあき}ちゃんと、トットちゃんのお\ruby{友達}{ともだち}づきあいが\ruby{始}{はじ}まった。\ruby{電車}{でんしゃ}の中は、\ruby{暖}{あたた}かい\ruby{日差}{ひざ}しで、\ruby{暑}{あつ}いくらいだった。\ruby{誰}{だれ}かが、\ruby{窓}{まど}を\ruby{開}{ひら}けた。\ruby{新}{あたら}しい\ruby{春}{はる}の\ruby{風}{かぜ}が、\ruby{電車}{でんしゃ}の中を\ruby{通}{とお}り\ruby{抜}{ぬ}け、\ruby{子供}{こども}たちの\ruby{髪}{かみ}の\ruby{毛}{け}が\ruby{歌}{うた}っているように、とびはねた。トットちゃんの、トモエでの\ruby{第}{だい}一目は、こんな\ruby{風}{ふう}に\ruby{始}{はじ}まったのだった。


