で、\ruby{普通}{ふつう}なら、これで、「いただきまーす」になるんだけど、このトモエ\ruby{学園}{がくえん}は、ここで、\ruby{合唱}{がっしょう}が入るのが、また、\ruby{変}{か}わっていた。\ruby{校長}{こうちょう}先生は、\ruby{音楽家}{おんがくか}でもあったから、『お\ruby{弁当}{べんとう}を\ruby{食}{た}べる\ruby{前}{まえ}に\ruby{歌}{うた}う\ruby{歌}{うた}』というのを\ruby{作}{つく}った。ただし、これは、\ruby{作曲}{さっきょく}が、イギリス人で、\ruby{歌詞}{かし}だけが、\ruby{校長}{こうちょう}先生だった。というより、\ruby{本当}{ほんとう}は、もともとあった\ruby{曲}{きょく}に、先生が\ruby{替}{か}え\ruby{歌}{うた}をつけた、というのが、正しいのだけれど。もともとの\ruby{曲}{きょく}は、あの\ruby{有名}{ゆうめい}な、『\ruby{船}{ふね}をこげよ(Row Boat)』 ロー ロー ロー ユアー ボート ジェントリー ダウン ザ ストゥリーム メリリー メリリー メリリー メリリー ライス イズ バット ア ドリームで、これに\ruby{校長}{こうちょう}先生がつけた\ruby{歌詞}{かし}は、\ruby{次}{つぎ}のようだった。 よーく \ruby{噛}{か}めよ たべものを \ruby{噛}{か}めよ  \ruby{噛}{か}めよ  \ruby{噛}{か}めよ  \ruby{噛}{か}めよ たべものを そして、これを\ruby{歌}{うた}い\ruby{終}{お}わると、\ruby{初}{はじ}めて、「いただきまーす」になるのだった。 “ロー ロー ロー ユアー ボート”のメロディーに、“よく、\ruby{噛}{か}めよ”は、ぴったりとあった。だから、この学校の\ruby{卒業}{そつぎょう}生は、ずいぶんと大きくなるまで、このメロディーは、お\ruby{弁当}{べんとう}の\ruby{前}{まえ}の\ruby{歌}{うた}う\ruby{歌}{うた}だ、と\ruby{信}{しん}じていたくらいだった。\ruby{校長}{こうちょう}先生は、\ruby{自分}{じぶん}の\ruby{歯}{は}が\ruby{抜}{ぬ}けていたので、この\ruby{歌}{うた}を\ruby{作}{つく}ったのかもしれないけど、\ruby{本当}{ほんとう}は、「よく\ruby{噛}{か}めよ」というより、お\ruby{食事}{しょくじ}は、\ruby{時間}{じかん}をかけて、\ruby{楽}{たの}しく、いろんなお\ruby{話}{はな}しをしながら、ゆっくり\ruby{食}{た}べるものだ、と、いつも\ruby{生徒}{せいと}に\ruby{話}{はな}していたから、そのことを\ruby{忘}{わす}れないように、この\ruby{歌}{うた}を\ruby{作}{つく}ったのかもしれなかった。さて、みんなは、大きな\ruby{声}{こえ}で、この\ruby{歌}{うた}を\ruby{歌}{うた}うと、「いただきまーす」といって、『\ruby{海}{うみ}のものと山のもの』に、とりかかった。トットちゃんも、もちろん、\ruby{同}{おな}じようにした。 \ruby{講堂}{こうどう}は\ruby{一瞬}{いっしゅん}だけ、\ruby{静}{しず}かになった。


