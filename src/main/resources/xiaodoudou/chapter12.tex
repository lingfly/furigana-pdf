お\ruby{弁当}{べんとう}の\ruby{後}{あと}、みんなと\ruby{校庭}{こうてい}で\ruby{走}{はし}り\ruby{回}{まわ}ったトットちゃんが、\ruby{電車}{でんしゃ}の\ruby{教室}{きょうしつ}に\ruby{戻}{もど}ると、女の先生が、「\ruby{皆}{みな}さん、\ruby{今日}{きょう}は、とてもよく\ruby{勉強}{べんきょう}したから、\ruby{午後}{ごご}は、\ruby{何}{なに}をしたい?」と\ruby{聞}{き}いた。トットちゃんが、(えーと、\ruby{私}{わたし}のしたいこと、って\ruby{言}{い}えば……)なんて\ruby{考}{かんが}えるより\ruby{前}{まえ}に、みんなが口々に「\ruby{散歩}{さんぽ}!」といった。すると先生は、 「じゃ、\ruby{行}{い}きましょう」といって立ち上がり、みんなも、\ruby{電車}{でんしゃ}のドアを\ruby{開}{ひら}けて、\ruby{靴}{くつ}を\ruby{履}{は}いて、\ruby{飛}{と}び\ruby{出}{だ}した。トットちゃんは、パパと犬のロッキーと、\ruby{散歩}{さんぽ}に\ruby{行}{い}ったことはあるけど、学校で、\ruby{散歩}{さんぽ}に\ruby{行}{い}く、って\ruby{知}{し}らなかったから、ビックリした。でも、\ruby{散歩}{さんぽ}は\ruby{大好}{だいす}きだから、トットちゃんも、\ruby{急}{いそ}いで\ruby{靴}{くつ}を\ruby{履}{は}いた。あとで\ruby{分}{わ}かったことだけど、先生が\ruby{朝}{あさ}の一\ruby{時間}{じかん}目に、その日、一日やる\ruby{時間}{じかん}\ruby{割}{わり}の\ruby{問題}{もんだい}を\ruby{黒板}{こくばん}に\ruby{書}{か}いて、みんなが、\ruby{頑張}{がんば}って、\ruby{午前中}{ごぜんちゅう}に、\ruby{全部}{ぜんぶ}やっちゃうと、\ruby{午後}{ごご}は、たいがい\ruby{散歩}{さんぽ}になるのだった。これは一年生でも、六年生でも\ruby{同}{おな}じだった。学校の\ruby{門}{もん}を出ると、女の先生を、\ruby{真}{ま}ん\ruby{中}{なか}にして、九人の一年生は、小さい川に\ruby{沿}{そ}って\ruby{歩}{ある}き\ruby{出}{だ}した。川の\ruby{両側}{りょうがわ}には、ついこの\ruby{間}{あいだ}まで\ruby{満開}{まんかい}だった、\ruby{桜}{さくら}の大きい木が、ずーっと\ruby{並}{なら}んでいた。そして、\ruby{見渡}{みわた}す\ruby{限}{かぎ}り、\ruby{菜}{な}の\ruby{花畑}{はなばたけ}だった。\ruby{今}{いま}では、川も\ruby{埋}{う}め\ruby{立}{た}てられ、\ruby{団地}{だんち}やお\ruby{店}{みせ}でギュウヅメの\ruby{自由}{じゆう}の\ruby{丘}{おか}も、この\ruby{頃}{ころ}は、ほとんどが\ruby{畑}{はたけ}だった。「お\ruby{散歩}{さんぽ}は、\ruby{九品仏}{くほんぶつ}よ」と、\ruby{兎}{うさぎ}の\ruby{絵}{え}のジャンパー?スカートの、女の子がいった。この子は、“サッコちゃん”という\ruby{名前}{なまえ}だった。それからサッコちゃんは、「\ruby{九品仏}{くほんぶつ}の\ruby{池}{いけ}のそばで、この\ruby{前}{まえ}、\ruby{蛇}{へび}を見たわよ」とか、「\ruby{九品仏}{くほんぶつ}のお\ruby{寺}{てら}の\ruby{古}{ふる}い\ruby{井戸}{いど}の中に、\ruby{流}{なが}れ\ruby{星}{ぼし}が\ruby{落}{お}ちてるんだって」とか\ruby{教}{おし}えてくれた。みんなは、\ruby{勝手}{かって}に、おしゃべりしながら\ruby{歩}{ある}いていく。空は青く、\ruby{蝶々}{ちょうちょ}が、いっぱい、あっちにも、こっちにも、ヒラヒラしていた。十\ruby{分}{ふん}くらい\ruby{歩}{ある}いたところで、女の先生は、足を\ruby{止}{と}めた。そして、\ruby{黄色}{きいろ}い\ruby{菜}{な}の\ruby{花}{はな}を\ruby{指}{さ}して、「これは、\ruby{菜}{な}の\ruby{花}{はな}ね。どうして、お花が\ruby{咲}{さ}くか、\ruby{分}{わ}かる?」といった。そして、それから、メシベとオシベの\ruby{話}{はな}しをした。\ruby{生徒}{せいと}は、みんな\ruby{道}{みち}にしゃがんで、\ruby{菜}{な}の\ruby{花}{はな}を\ruby{観察}{かんさつ}した。先生は、\ruby{蝶々}{ちょうちょ}も、花を\ruby{咲}{さ}かせるお\ruby{手伝}{てつだ}いをしている、といった。\ruby{本当}{ほんとう}に、\ruby{蝶々}{ちょうちょ}は、お\ruby{手伝}{てつだ}いをしているらしく、\ruby{忙}{いそが}しそうだった。それから、また先生は\ruby{歩}{ある}き\ruby{出}{だ}したから、みんなも、\ruby{観察}{かんさつ}はおしまいにして、立ち上がった。\ruby{誰}{だれ}かが、「オシベと、アカンベは\ruby{違}{ちが}うよね」とか、いった。トットちゃんは、(\ruby{違}{ちが}うんじゃないかなあー!)と\ruby{思}{おも}ったけど、よく、わかんなかった。でも、オシベとメシベが\ruby{大切}{たいせつ}、ってことは、みんなと\ruby{同}{おな}じように、よく\ruby{分}{わ}かった。そして、また\ruby{十分}{じゅうぶん}くらい\ruby{歩}{ある}くと、見たいもののほうに、キャアキャアいって\ruby{走}{はし}っていった。サッコちゃんが、「\ruby{流}{なが}れ\ruby{星}{ぼし}の\ruby{井戸}{いど}を見に\ruby{行}{い}かない?」といったので、もちろん、トットちゃんは、「うん」といって、サッコちゃんの\ruby{後}{あと}について\ruby{走}{はし}った。\ruby{井戸}{いど}っていっても、石みたいので\ruby{出来}{でき}ていて、\ruby{二人}{ふたり}の\ruby{胸}{むね}のところくらいまであり、木のふたがしてあった、\ruby{二人}{ふたり}でふたを\ruby{取}{と}って、下をのぞくと中は\ruby{真}{ま}っ\ruby{暗}{くら}で、よく見ると、コンクリートの\ruby{固}{かた}まりか、石の\ruby{固}{かた}まりみたいのが入っているだけで、トットちゃんが\ruby{想像}{そうぞう}してたみたいな、キラキラ\ruby{光}{ひか}る\ruby{星}{ほし}は、どこにも見えなかった。\ruby{長}{なが}いこと、\ruby{頭}{あたま}を\ruby{井戸}{いど}の中に\ruby{突}{つ}っ\ruby{込}{こ}んでいたトットちゃんは、\ruby{頭}{あたま}を上げると、サッコちゃんに\ruby{聞}{き}いた。「お\ruby{星}{ほし}さま、見た?」サッコちゃんは、\ruby{頭}{あたま}を\ruby{振}{ふ}ると「\ruby{一度}{いちど}も、ないの」といった。トットちゃんは、どうして\ruby{光}{ひか}らないか、お\ruby{考}{かんが}えた。そして、いった。「お\ruby{星}{ほし}さま、\ruby{今}{いま}、\ruby{寝}{ね}てるんじゃないの?」サッコちゃんは、大きい目を、もっと大きくしていった。「お\ruby{星}{ほし}さまって、\ruby{寝}{ね}るの?」トットちゃんは、あまり\ruby{確信}{かくしん}が\ruby{無}{な}かったから、早口でいった。「お\ruby{星}{ほし}さまは、\ruby{昼間}{ひるま}、\ruby{寝}{ね}てて、\ruby{夜}{よる}、\ruby{起}{お}きて、\ruby{光}{ひか}るんじゃないか、って\ruby{思}{おも}うんだ」それから、みんなで、\ruby{仁王}{におう}さまのお\ruby{腹}{なか}を見て\ruby{笑}{わら}ったり、\ruby{薄暗}{うすぐら}いお\ruby{堂}{どう}の中の\ruby{仏}{ほとけ}さまを、(\ruby{少}{すこ}し、こわい)と\ruby{思}{おも}いながらも、のぞいたり、\ruby{天狗}{てんぐ}さまの大きな\ruby{足跡}{あしあと}の\ruby{残}{のこ}ってる石に、\ruby{自分}{じぶん}の足を\ruby{乗}{の}せて\ruby{比}{くら}べてみたり、\ruby{池}{いけ}の\ruby{周}{まわ}りを\ruby{回}{まわ}って、ボートに\ruby{乗}{の}っている人に、「こんちは」といったり、お\ruby{墓}{はか}の\ruby{周}{まわ}りの、\ruby{黒}{くろ}いツルツルの、あぶら石を\ruby{借}{か}りて、\ruby{石蹴}{いしけ}りをしたり、もう\ruby{満足}{まんぞく}するぐらい、\ruby{遊}{あそ}んだ。\ruby{特}{とく}に、\ruby{初}{はじ}めてのトットちゃんは、もう\ruby{興奮}{こうふん}して、\ruby{次}{つぎ}から\ruby{次}{つぎ}と、\ruby{何}{なに}かを\ruby{発見}{はっけん}しては、\ruby{叫}{さけ}び\ruby{声}{ごえ}を上げた。\ruby{春}{はる}の\ruby{日差}{ひざ}しが、\ruby{少}{すこ}し\ruby{傾}{かたむ}いた。先生は、「\ruby{帰}{かえ}りましょう」といって、また、みんな、\ruby{菜}{な}の\ruby{花}{はな}と\ruby{桜}{さくら}の\ruby{木}{こ}の\ruby{間}{ま}も\ruby{道}{みち}を、\ruby{並}{なら}んで、学校に\ruby{向}{む}かった。\ruby{子供}{こども}たちにとって、\ruby{自由}{じゆう}で、お\ruby{遊}{あそ}びの\ruby{時間}{じかん}と見える、この『\ruby{散歩}{さんぽ}』が、\ruby{実}{じつ}は、\ruby{貴重}{きちょう}は、\ruby{理科}{りか}か、\ruby{歴史}{れきし}か、\ruby{生物}{せいぶつ}の\ruby{勉強}{べんきょう}になっているのだ、ということを、\ruby{子供}{こども}たちは気がついていなかった。トットちゃんは、もう、すっかり、みんなと\ruby{友達}{ともだち}になっていて、\ruby{前}{まえ}から、ずーっと\ruby{一緒}{いっしょ}にいるような気になっていた。だから、\ruby{帰}{かえ}り\ruby{道}{みち}に「\ruby{明日}{あした}も、\ruby{散歩}{さんぽ}にしよう!」と、みんなに大きい\ruby{声}{こえ}で\ruby{言}{い}った。みんなは、とびはねながら、いった。「そうしよう」\ruby{蝶々}{ちょうちょ}は、まだまだ\ruby{忙}{いそが}しそうで、\ruby{鳥}{とり}の\ruby{声}{こえ}が、\ruby{近}{ちか}くや\ruby{遠}{とお}くに\ruby{聞}{き}こえていた。トットちゃんの\ruby{胸}{むね}は、なんか、うれしいもので、いっぱいだった。


