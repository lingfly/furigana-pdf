\ruby{学校}{がっこう}の\ruby{門}{もん}が、はっきり\ruby{見}{み}えるところまで\ruby{来}{き}て、トットちゃんは、\ruby{立}{た}ち\ruby{止}{どま}った。なぜなら、この\ruby{間}{あいだ}まで\ruby{行}{い}っていた\ruby{学校}{がっこう}の\ruby{門}{もん}は、\ruby{立派}{りっぱ}なコンクリートみたいな\ruby{柱}{はしら}で、\ruby{学校}{がっこう}の\ruby{名前}{なまえ}も、\ruby{大}{おお}きく\ruby{書}{か}いてあった。ところが、この\ruby{新}{あたら}しい\ruby{学校}{がっこう}の\ruby{門}{もん}ときたら、\ruby{低}{ひく}い\ruby{木}{き}で、しかも\ruby{葉}{は}っぱが\ruby{生}{は}えていた。

「\ruby{地面}{じめん}から\ruby{生}{は}えてる\ruby{門}{もん}ね」

と、トットちゃんはママに\ruby{言}{い}った。そうして、こう、\ruby{付}{つ}け\ruby{加}{くわ}えた。

「きっと、どんどんはえて、\ruby{今}{いま}に\ruby{電信柱}{でんしんばしら}より\ruby{高}{たか}くなるわ」

\ruby{確}{たし}かに、その二\ruby{本}{ほん}の\ruby{門}{もん}は、\ruby{根}{ね}っこのある\ruby{木}{き}だった。トットちゃんは、\ruby{門}{もん}に\ruby{近}{ちか}づくと、いきなり\ruby{顔}{かお}を、\ruby{斜}{なな}めにした。なぜかといえば、\ruby{門}{もん}にぶら\ruby{下}{さ}げてある\ruby{学校}{がっこう}の\ruby{名前}{なまえ}を\ruby{書}{か}いた\ruby{札}{さつ}が、\ruby{風}{かぜ}に\ruby{吹}{ふ}かれたのか、\ruby{斜}{なな}めになっていたからだった。

「トモエがくえん」トットちゃんは、\ruby{顔}{かお}を\ruby{斜}{なな}めにしたまま、\ruby{表札}{ひょうさつ}を\ruby{読}{よ}み\ruby{上}{あ}げた。そして、ママに、

「トモエって、なあに?」

と\ruby{聞}{き}こうとしたときだった。トットちゃんの\ruby{目}{め}の\ruby{端}{はし}に、\ruby{夢}{ゆめ}としか\ruby{思}{おも}えないものが\ruby{見}{み}えたのだった。トットちゃんは、\ruby{身}{み}をかがめると、\ruby{門}{もん}の\ruby{植}{う}え\ruby{込}{こ}みの、\ruby{隙間}{すきま}に\ruby{頭}{あたま}を\ruby{突}{つ}っ\ruby{込}{こ}んで、\ruby{門}{もん}の\ruby{中}{なか}をのぞいてみた。どうしよう、みえたんだけど!

「ママ!あれ、\ruby{本当}{ほんとう}の\ruby{電車}{でんしゃ}?\ruby{校庭}{こうてい}に\ruby{並}{なら}んでるの」

それは、\ruby{走}{はし}っていない、\ruby{本当}{ほんとう}の\ruby{電車}{でんしゃ}が六\ruby{台}{だい}、\ruby{教室}{きょうしつ}\ruby{用}{よう}に、\ruby{置}{お}かれてあるのだった。トットちゃんは、\ruby{夢}{ゆめ}のように\ruby{思}{おも}った。“\ruby{電車}{でんしゃ}の\ruby{教室}{きょうしつ}……”

\ruby{電車}{でんしゃ}で\ruby{窓}{まど}が、\ruby{朝}{あさ}の\ruby{光}{ひかり}を\ruby{受}{う}けて、キラキラと\ruby{光}{ひか}っていた。\ruby{目}{め}を\ruby{輝}{かがや}かして、のぞいているトットちゃんの、ホッペタも、\ruby{光}{ひか}っていた。


