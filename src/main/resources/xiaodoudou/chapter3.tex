\ruby{学校}{がっこう}の\ruby{門}{もん}が、はっきり\ruby{見}{み}えるところまで\ruby{来}{き}て、トットちゃんは、\ruby{立}{た}ち\ruby{止}{どま}った。なぜなら、この\ruby{間}{あいだ}まで\ruby{行}{い}っていた\ruby{学校}{がっこう}の\ruby{門}{もん}は、\ruby{立派}{りっぱ}なコンクリートみたいな\ruby{柱}{はしら}で、\ruby{学校}{がっこう}の\ruby{名前}{なまえ}も、\ruby{大}{おお}きく\ruby{書}{か}いてあった。ところが、この\ruby{新}{あたら}しい\ruby{学校}{がっこう}の\ruby{門}{もん}ときたら、\ruby{低}{ひく}い\ruby{木}{き}で、しかも\ruby{葉}{は}っぱが\ruby{生}{は}えていた。「\ruby{地面}{じめん}から\ruby{生}{は}えてる\ruby{門}{もん}ね」と、トットちゃんはママに\ruby{言}{い}った。そうして、こう、\ruby{付}{つ}け\ruby{加}{くわ}えた。「きっと、どんどんはえて、\ruby{今}{いま}に\ruby{電信柱}{でんしんばしら}より\ruby{高}{たか}くなるわ」\ruby{確}{たし}かに、その二\ruby{本}{ほん}の\ruby{門}{もん}は、\ruby{根}{ね}っこのある\ruby{木}{き}だった。トットちゃんは、\ruby{門}{もん}に\ruby{近}{ちか}づくと、いきなり\ruby{顔}{かお}を、\ruby{斜}{なな}めにした。なぜかといえば、\ruby{門}{もん}にぶら\ruby{下}{さ}げてある\ruby{学校}{がっこう}の\ruby{名前}{なまえ}を\ruby{書}{か}いた\ruby{札}{さつ}が、\ruby{風}{かぜ}に\ruby{吹}{ふ}かれたのか、\ruby{斜}{なな}めになっていたからだった。「トモエがくえん」トットちゃんは、\ruby{顔}{かお}を\ruby{斜}{なな}めにしたまま、\ruby{表札}{ひょうさつ}を\ruby{読}{よ}み\ruby{上}{あ}げた。そして、ママに、「トモエって、なあに?」と\ruby{聞}{き}こうとしたときだった。トットちゃんの\ruby{目}{め}の\ruby{端}{はし}に、\ruby{夢}{ゆめ}としか\ruby{思}{おも}えないものが\ruby{見}{み}えたのだった。トットちゃんは、\ruby{身}{み}をかがめると、\ruby{門}{もん}の\ruby{植}{う}え\ruby{込}{こ}みの、\ruby{隙間}{すきま}に\ruby{頭}{あたま}を\ruby{突}{つ}っ\ruby{込}{こ}んで、\ruby{門}{もん}の\ruby{中}{なか}をのぞいてみた。どうしよう、みえたんだけど!「ママ!あれ、\ruby{本当}{ほんとう}の\ruby{電車}{でんしゃ}?\ruby{校庭}{こうてい}に\ruby{並}{なら}んでるの」それは、\ruby{走}{はし}っていない、\ruby{本当}{ほんとう}の\ruby{電車}{でんしゃ}が六\ruby{台}{だい}、\ruby{教室}{きょうしつ}\ruby{用}{よう}に、\ruby{置}{お}かれてあるのだった。トットちゃんは、\ruby{夢}{ゆめ}のように\ruby{思}{おも}った。“\ruby{電車}{でんしゃ}の\ruby{教室}{きょうしつ}……”

\ruby{電車}{でんしゃ}で\ruby{窓}{まど}が、\ruby{朝}{あさ}の\ruby{光}{ひかり}を\ruby{受}{う}けて、キラキラと\ruby{光}{ひか}っていた。\ruby{目}{め}を\ruby{輝}{かがや}かして、のぞいているトットちゃんの、ホッペタも、\ruby{光}{ひか}っていた。   \ruby{気}{き}に\ruby{入}{い}ったわ\ruby{次}{つぎ}の\ruby{瞬間}{しゅんかん}、トットちゃんは、「わーい」と\ruby{歓声}{かんせい}を\ruby{上}{あ}げると、\ruby{電車}{でんしゃ}の\ruby{教室}{きょうしつ}のほうに\ruby{向}{む}かって\ruby{走}{はし}り\ruby{出}{だ}した。そして、\ruby{走}{はし}りながら、ママに\ruby{向}{む}かって\ruby{叫}{さけ}んだ。「ねえ、\ruby{早}{はや}く、\ruby{動}{うご}かない\ruby{電車}{でんしゃ}に\ruby{乗}{の}ってみよう!」ママは、\ruby{驚}{おどろ}いて\ruby{走}{はし}り\ruby{出}{だ}した。もとバスケットバールの\ruby{選手}{せんしゅ}だったままの\ruby{足}{あし}は、トットちゃんより\ruby{速}{はや}かったから、トットちゃんが、\ruby{後}{あと}、ちょっとでドア、というときに、スカートを\ruby{捕}{つか}まえられてしまった。ママは、スカートのはしを、ぎっちり\ruby{握}{にぎ}ったまま、トットちゃんにいった。「ダメよ。この\ruby{電車}{でんしゃ}は、この\ruby{学校}{がっこう}のお\ruby{教室}{きょうしつ}なんだし、あなたは、まだ、この\ruby{学校}{がっこう}に\ruby{入}{はい}れていただいてないんだから。もし、どうしても、この\ruby{電車}{でんしゃ}に\ruby{乗}{の}りたいんだったら、これからお\ruby{目}{め}にかかる\ruby{校長}{こうちょう}\ruby{先生}{せんせい}とちゃんと、お\ruby{話}{はな}してちょうだい。そして、うまくいったら、この\ruby{学校}{がっこう}に\ruby{通}{とお}えるんだから、\ruby{分}{わ}かった?」トットちゃんは、(\ruby{今}{いま}\ruby{乗}{の}れないのは、とても\ruby{残念}{ざんねん}なことだ)と\ruby{思}{おも}ったけど、ママのいう\ruby{通}{とお}りにしようときめたから、\ruby{大}{おお}きな\ruby{声}{こえ}で、「うん」といって、それから、いそいで、つけたした。「\ruby{私}{わたし}、この\ruby{学校}{がっこう}、とっても\ruby{気}{き}に\ruby{入}{い}ったわ」ママは、トットちゃんが\ruby{気}{き}に\ruby{入}{い}ったかどうかより、\ruby{校長}{こうちょう}\ruby{先生}{せんせい}が、トットちゃんを\ruby{気}{き}に\ruby{入}{い}ってくださるかどうか\ruby{問題}{もんだい}なのよ、といいたい\ruby{気}{き}がしたけど、とにかく、トットちゃんのスカートから\ruby{手}{て}を\ruby{離}{はな}し、\ruby{手}{て}をつないで\ruby{校長}{こうちょう}\ruby{室}{しつ}のほうに\ruby{歩}{ある}き\ruby{出}{だ}した。どの\ruby{電車}{でんしゃ}も\ruby{静}{しず}かで、ちょっと\ruby{前}{まえ}に、一\ruby{時間}{じかん}\ruby{目}{め}の\ruby{授業}{じゅぎょう}が\ruby{始}{はじ}まったようだった。あまり\ruby{広}{ひろ}くない\ruby{校庭}{こうてい}の\ruby{周}{まわ}りには、\ruby{塀}{へい}の\ruby{変}{か}わりに、いろんな\ruby{種類}{しゅるい}の\ruby{木}{き}が\ruby{植}{う}わっていて、\ruby{花壇}{かだん}には、\ruby{赤}{あか}や\ruby{黄色}{きいろ}の\ruby{花}{はな}がいっぱい\ruby{咲}{さ}いていた。\ruby{校長}{こうちょう}\ruby{室}{しつ}は、\ruby{電車}{でんしゃ}ではなく、ちょうど、\ruby{門}{もん}から\ruby{正面}{しょうめん}に\ruby{見}{み}える\ruby{扇形}{おうぎがた}に\ruby{広}{ひろ}がった七\ruby{段}{だん}くらいある\ruby{石}{いし}の\ruby{階段}{かいだん}を\ruby{上}{のぼ}った、その\ruby{右手}{みぎて}にあった。トットちゃんは、ママの\ruby{手}{て}を\ruby{振}{ふ}り\ruby{切}{き}ると、\ruby{階段}{かいだん}を\ruby{駆}{か}け\ruby{上}{あ}がって\ruby{行}{い}ったが、\ruby{急}{きゅう}に\ruby{止}{と}まって、\ruby{振}{ふ}り\ruby{向}{む}いた。だから、\ruby{後}{うし}ろから\ruby{行}{い}ったママは、もう\ruby{少}{すこ}しで、トットちゃんと\ruby{正面}{しょうめん}\ruby{衝突}{しょうとつ}するところだった。「どうしたの?」ママは、トットちゃんの\ruby{気}{き}が\ruby{変}{か}わったのかと\ruby{思}{おも}って、\ruby{急}{いそ}いで\ruby{聞}{き}いた。トットちゃんは、ちょうど\ruby{階段}{かいだん}の\ruby{一番}{いちばん}うえに\ruby{立}{た}った\ruby{形}{かたち}だったけど、まじめな\ruby{顔}{かお}をして、\ruby{小声}{こごえ}でママに\ruby{聞}{き}いた。ママは、かなり\ruby{辛抱}{しんぼう}づよい\ruby{人間}{にんげん}だったから……というか,\ruby{面白}{おもしろ}がりやだったから、やはり\ruby{小声}{こごえ}になって、トットちゃんに\ruby{顔}{かお}をつけて、\ruby{聞}{き}いた。「どうして?」トットちゃんは、ますます\ruby{声}{こえ}をひそめて\ruby{言}{い}った。「だってさ、\ruby{校長}{こうちょう}\ruby{先生}{せんせい}って、ママいったけど、こんなに\ruby{電車}{でんしゃ}、いっぱい\ruby{持}{も}ってるんだから、\ruby{本当}{ほんとう}は、\ruby{駅}{えき}の\ruby{人}{ひと}なんじゃないの?」\ruby{確}{たし}かに、\ruby{電車}{でんしゃ}の\ruby{払}{はら}い\ruby{下}{さ}げを\ruby{校舎}{こうしゃ}にしている\ruby{学校}{がっこう}なんてめずらしいから、トットちゃんの\ruby{疑問}{ぎもん}も、もっとものこと、とママも\ruby{思}{おも}ったけど、この\ruby{際}{さい}、\ruby{説明}{せつめい}してるヒマはないので、こういった。「じゃ、あなた、\ruby{校長}{こうちょう}\ruby{先生}{せんせい}に\ruby{伺}{うかが}って\ruby{御覧}{ごらん}なさい、\ruby{自分}{じぶん}で。それと、あなたのパパのことを\ruby{考}{かんが}えてみて?パパはヴァイオリンを\ruby{弾}{ひ}く\ruby{人}{ひと}で、いくつかヴァイオリンを\ruby{持}{も}ってるけど、ヴァイオリン\ruby{屋}{や}さんじゃないでしょう?そういう\ruby{人}{ひと}もいるのよ」トットちゃんは、「そうか」というと、ママと\ruby{手}{て}をつないだ。


