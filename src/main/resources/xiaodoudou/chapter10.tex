さて、トットちゃんが\ruby{待}{ま}ちに\ruby{待}{ま}った『\ruby{海}{うみ}のものと山のもの』のお\ruby{弁当}{べんとう}の\ruby{時間}{じかん}が\ruby{来}{こ}た。この『\ruby{海}{うみ}のものと山のもの』って、\ruby{何}{なに}か、といえば、それは、\ruby{校長}{こうちょう}先生が\ruby{考}{かんが}えた、お\ruby{弁当}{べんとう}のおかずのことだった。\ruby{普通}{ふつう}なら、お\ruby{弁当}{べんとう}のおかずについて、「\ruby{子供}{こども}が\ruby{好}{す}き\ruby{嫌}{きら}いをしないように、\ruby{工夫}{くふう}してください」とか、「\ruby{栄養}{えいよう}が、\ruby{片寄}{かたよ}らないようにお\ruby{願}{ねが}いします」とか、\ruby{言}{い}うところだけど、\ruby{校長}{こうちょう}先生はひとこと、「\ruby{海}{うみ}のものと山のものを\ruby{持}{も}たせてください」と、\ruby{子供}{こども}たちの\ruby{家}{いえ}の人に、\ruby{頼}{たの}んだ、というわけだった。

山は……\ruby{例}{たと}えば、お\ruby{野菜}{やさい}とか、お\ruby{肉}{にく}とか(お\ruby{肉}{にく}は山で\ruby{取}{と}れるってわけじゃないけど、大きく\ruby{分}{わ}けると、\ruby{牛}{うし}とか\ruby{豚}{ぶた}とかニワトリとかは、\ruby{陸}{りく}に\ruby{住}{す}んでいるのだから、山のほうに入るって\ruby{考}{かんが}え)、\ruby{海}{うみ}は、お\ruby{魚}{さかな}とか、\ruby{佃煮}{つくだに}とか。この二\ruby{種類}{しゅるい}を、\ruby{必}{かなら}ずお\ruby{弁当}{べんとう}のおかずに入れてほしい、というのだった。(こんなに\ruby{簡単}{かんたん}に、\ruby{必要}{ひつよう}なことを\ruby{表現}{ひょうげん}できる\ruby{大人}{おとな}は、\ruby{校長}{こうちょう}先生のほかには、そういない)とトットちゃんのママは、ひどく\ruby{感心}{かんしん}していた。しかも、ママにとっても、\ruby{海}{うみ}と山とに、\ruby{分}{わ}けてもらっただけで、おかずを\ruby{考}{かんが}えるのが、とても\ruby{面倒}{めんどう}なことじゃなく\ruby{思}{おも}えてきたから、\ruby{不思議}{ふしぎ}だった。それに\ruby{校長}{こうちょう}先生は、\ruby{海}{うみ}と山といっても、“\ruby{無理}{むり}しないこと”“\ruby{贅沢}{ぜいたく}しないこと”といってくださったから、山は“キンピラゴボウと\ruby{玉子焼}{たまごやき}”で\ruby{海}{うみ}は“おかか”という\ruby{風}{ふう}でよかったし、もっと\ruby{簡単}{かんたん}な\ruby{海}{うみ}と山を\ruby{例}{れい}にすれば、“お\ruby{海苔}{のり}と\ruby{梅干}{うめぼし}”でよかったのだ。

そして\ruby{子供}{こども}たちは、トットちゃんが\ruby{始}{はじ}めてみたときに、とっても、うらやましく\ruby{思}{おも}ったように、お\ruby{弁当}{べんとう}の\ruby{時間}{じかん}に、\ruby{校長}{こうちょう}先生が、\ruby{自分}{じぶん}たちのお\ruby{弁当箱}{べんとうばこ}の中をのぞいて、「\ruby{海}{うみ}のものと、山のものは、あるかい?」と、ひとりずつ\ruby{確}{たし}かめてくださるのが、\ruby{嬉}{うれ}しかったし、それから、\ruby{自分}{じぶん}たちも、どれが\ruby{海}{うみ}で、どれが山かを\ruby{発見}{はっけん}するのも、ものすごいスリルだった。でも、たまには、\ruby{母親}{ははおや}が\ruby{忙}{いそが}しかったり、あれこれ手が\ruby{回}{まわ}らなくて、山だけだったり、\ruby{海}{うみ}だけという子もいた。そういう\ruby{時}{とき}は、どうなるのか、といえば、その子は\ruby{心配}{しんぱい}しないでいいのだった。なぜなら、お\ruby{弁当}{べんとう}の中をのぞいて\ruby{歩}{ある}く\ruby{校長}{こうちょう}先生の\ruby{後}{あと}から、白い、\ruby{割烹}{かっぽう}\ruby{前掛}{まえか}けをかけた、\ruby{校長}{こうちょう}先生の\ruby{奥}{おく}さんが、\ruby{両手}{りょうて}に、おなべをひとつずつ\ruby{持}{も}って、ついて\ruby{歩}{ある}いていた。そして先生がどっちか足りないこの\ruby{前}{まえ}で、「\ruby{海}{うみ}!」というと、\ruby{奥}{おく}さんは、\ruby{海}{うみ}のおなべから、ちくわの\ruby{煮}{に}たのを、二\ruby{個}{こ}くらい、お\ruby{弁当箱}{べんとうばこ}のふたに、\ruby{乗}{の}せてくださったし、先生が、「山!」といえば、もう\ruby{片方}{かたほう}の、山のおなべから、おいもの\ruby{煮}{に}ころがしが、\ruby{飛}{と}び\ruby{出}{だ}す、という\ruby{風}{ふう}だったから。こんなわけだったので、どの\ruby{子供}{こども}たちも「ちくわが\ruby{嫌}{きら}い」なんて、そんなことは、\ruby{言}{い}わなかったし、(\ruby{誰}{だれ}のおかずが\ruby{上等}{じょうとう}で、\ruby{誰}{だれ}のおかずが、いつも、みっともない)なんて\ruby{思}{おも}わなくて、\ruby{海}{うみ}と山とが\ruby{揃}{そろ}った、ということが、\ruby{嬉}{うれ}しくて、お\ruby{互}{たが}いに\ruby{笑}{わら}いあったり、\ruby{叫}{さけ}んだりするのだった。トットちゃんにも、やっと『\ruby{海}{うみ}のものと山のもの』が、なんだか\ruby{分}{わ}かった。そしたら、(ママが、\ruby{今朝}{けさ}、大\ruby{急行}{きゅうこう}で\ruby{作}{つく}ってくれたお\ruby{弁当}{べんとう}は、\ruby{大丈夫}{だいじょうぶ}かな?)と\ruby{少}{すこ}し\ruby{心配}{しんぱい}になった。でも、ふたを\ruby{取}{と}ったとき、トットちゃんが、「わあーい」といいそうになって、口お\ruby{押}{お}さえたくらい、それは、それは、ステキなお\ruby{弁当}{べんとう}だった。\ruby{黄色}{きいろ}のいり\ruby{卵}{たまご}、グリンピース、\ruby{茶色}{ちゃいろ}のデンブ、ピンク\ruby{色}{いろ}の、タラコをパラパラに\ruby{炒}{い}ったの、そんな、いろんな\ruby{色}{いろ}が、お\ruby{花畑}{はなばたけ}みたいな\ruby{模様}{もよう}になっていたのだもの。\ruby{校長}{こうちょう}先生は、トットちゃんのを、のぞきこむと、「きれいだね」といった。トットちゃんは、\ruby{嬉}{うれ}しくなって、「ママは、とっても、おかず\ruby{上手}{じょうず}なの」といった。\ruby{校長}{こうちょう}先生は、「そうかい」といってから、\ruby{茶色}{ちゃいろ}のデンブをさして、トットちゃんに、「これは、\ruby{海}{うみ}かい?山かい?」と\ruby{聞}{き}いた、トットちゃんは、デンブを、ジーっと見て、「これは、どっちだろう」と\ruby{考}{かんが}えた。(\ruby{色}{いろ}からすると、山みたいだけど、だって、土みたいな\ruby{色}{いろ}だからさ。でも……わかんない)そう\ruby{思}{おも}ったので、「わかりません」と\ruby{答}{こた}えた。すると、\ruby{校長}{こうちょう}先生は、大きな\ruby{声}{こえ}で、「デンブは、\ruby{海}{うみ}と山と、どっちだい?」と、みんなに\ruby{聞}{き}いた。ちょっと\ruby{考}{かんが}える\ruby{間}{ま}があって、みんな\ruby{一斉}{いっせい}に、「山!」とか、『\ruby{海}{うみ}!』とか\ruby{叫}{さけ}んで、どっちとも\ruby{決}{き}まらなかった。みんなが\ruby{叫}{さけ}び\ruby{終}{お}わると、\ruby{校長}{こうちょう}先生は、いった。「いいかい、デンブは、\ruby{海}{うみ}だよ」「なんで」と、\ruby{肥}{ふと}った男の子が\ruby{聞}{き}いた。\ruby{校長}{こうちょう}先生は、\ruby{机}{つくえ}の\ruby{輪}{わ}の\ruby{真}{ま}ん中に立つと、「デンブは、\ruby{魚}{さかな}の\ruby{身}{み}をほぐして、\ruby{細}{こま}かくして、\ruby{炒}{い}って\ruby{作}{つく}ったものだからさ」と\ruby{説明}{せつめい}した。「ふーん」と、みんなは、\ruby{感心}{かんしん}した\ruby{声}{こえ}を出した。そのとき\ruby{誰}{だれ}かが、「先生、トットちゃんのデンブ、見てもいい?」と\ruby{聞}{き}いた。\ruby{校長}{こうちょう}先生が、「いいよ」というと、学校中の子が、ゾロゾロ立ってきて、トットちゃんのデンブを見た。

デンブは\ruby{知}{し}ってて、\ruby{食}{た}べたことはあっても、\ruby{今}{いま}の\ruby{話}{はなし}で、\ruby{急}{きゅう}に\ruby{興味}{きょうみ}が出てきた子も、また、\ruby{自分}{じぶん}の\ruby{家}{いえ}のデンブと、トットちゃんのと、\ruby{少}{すこ}し、かわっているのかな?と\ruby{思}{おも}って、見たい子もいるに\ruby{違}{ちが}いなかった。デンブを見にきた子の中には、においをかぐ子もいたので、トットちゃんは、\ruby{鼻息}{はないき}で、デンブが\ruby{飛}{と}ばないか、と\ruby{心配}{しんぱい}になったくらいだった。でも、\ruby{初}{はじ}めてのお\ruby{弁当}{べんとう}の\ruby{時間}{じかん}は、\ruby{少}{すこ}しドキドキはしたけど、\ruby{楽}{たの}しくて、『\ruby{海}{うみ}のものと山のもの』を\ruby{考}{かんが}えるのも\ruby{面白}{おもしろ}いし、デンブがお\ruby{魚}{さかな}って\ruby{分}{わ}かったし、ママは、『\ruby{海}{うみ}のものと山のもの』を、ちゃんと入れてくれたし、トットちゃんは、(ぜんぶ、よかったな)と、\ruby{嬉}{うれ}しくなった。そして、\ruby{次}{つぎ}に、\ruby{嬉}{うれ}しいのは、ママの\ruby{弁当}{べんとう}は、\ruby{食}{た}べると、おいしいことだった。


