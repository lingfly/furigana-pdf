トットちゃんには、\ruby{本当}{ほんとう}に、\ruby{新}{あたら}しい\ruby{驚}{おどろ}きで、いっぱいの、トモエ\ruby{学園}{がくえん}での\ruby{毎日}{まいにち}が\ruby{過}{す}ぎていった。

\ruby{相変}{あいか}わらず、\ruby{学校}{がっこう}に\ruby{早}{はや}く\ruby{行}{い}きたくて、\ruby{朝}{あさ}が\ruby{待}{ま}ちきれなかった。そして、\ruby{帰}{かえ}ってくると、\ruby{犬}{いぬ}のロッキーと、ママとパパに、「\ruby{今日}{きょう}、\ruby{学校}{がっこう}で、どんなことをして、どのくらい\ruby{面白}{おもしろ}かった」とか、「もう、びっくりしちゃった」とか、しまいには、ママが、

「\ruby{話}{はなし}は、ちょっとお\ruby{休}{やす}みして、おやつにしたら?」

というまで、\ruby{話}{はなし}をやめなかった。そして、これは、どんなにトットちゃんが、\ruby{学校}{がっこう}に\ruby{馴}{な}れてもやっぱり、\ruby{毎日}{まいにち}ように、\ruby{話}{はな}すことは、\ruby{山}{やま}のように、あったのだった。

(でも、こんなに\ruby{話}{はな}すことがたくさんあるってことは、\ruby{有難}{ありがた}いこと)

と、ママは、\ruby{心}{こころ}から、\ruby{嬉}{うれ}しく\ruby{思}{おも}っていた。

ある\ruby{日}{ひ}、トットちゃんは、\ruby{学校}{がっこう}に\ruby{行}{い}く\ruby{電車}{でんしゃ}の\ruby{中}{なか}で、\ruby{突然}{とつぜん}、

「あれ?オモエに\ruby{校歌}{こうか}って、あったかな?」

と\ruby{考}{かんが}えた。そう\ruby{思}{おも}ったら、もう、\ruby{早}{はや}く\ruby{学校}{がっこう}に\ruby{着}{つ}きたくなって、まだ、あと\ruby{二}{ふた}つも\ruby{駅}{えき}があるのに、ドアのところに\ruby{立}{た}って、\ruby{自由}{じゆう}が\ruby{丘}{おか}に\ruby{電車}{でんしゃ}が\ruby{着}{つ}いたら、すぐ\ruby{出}{で}られるように、ヨーイ・ドンの\ruby{格好}{かっこう}で\ruby{待}{ま}った。ひとつ\ruby{前}{まえ}の\ruby{駅}{えき}で、ドアが\ruby{開}{あ}いたとき、\ruby{乗}{の}り\ruby{込}{こ}もうとした、おばさんは、\ruby{女}{おんな}の\ruby{子}{こ}が、ドアのところで、ヨーイ・ドンの\ruby{形}{かたち}になってるので、\ruby{降}{お}りるのか、と\ruby{思}{おも}ったら、そのままの\ruby{形}{かたち}で\ruby{動}{うご}かないので、

「どうなっちゃってるのかね」

といいながら、\ruby{乗}{の}り\ruby{込}{こ}んできた。

こんな\ruby{具合}{ぐあい}だったから、\ruby{駅}{えき}に\ruby{着}{つ}いたときの、トットちゃんの\ruby{早}{はや}く\ruby{降}{お}りたことといったら、なかった。\ruby{若}{わか}い\ruby{男}{おとこ}の\ruby{車掌}{しゃしょう}さんが、しゃれたポーズで、まだ、\ruby{完全}{かんぜん}に\ruby{止}{と}まっていない\ruby{電車}{でんしゃ}から、プラットホームに\ruby{片足}{かたあし}をつけて、おりながら、

「\ruby{自由}{じゆう}が\ruby{丘}{おか}!お\ruby{降}{ふ}りの\ruby{方}{ほう}は……」

といったとき、もう、トットちゃんの\ruby{姿}{すがた}は、\ruby{改札口}{かいさつぐち}から、\ruby{見}{み}えなくなっていた。

\ruby{学校}{がっこう}に\ruby{着}{つ}いて、\ruby{電車}{でんしゃ}の\ruby{教室}{きょうしつ}に\ruby{入}{はい}ると、トットちゃんは、\ruby{先}{さき}に\ruby{来}{き}ていた、\ruby{山内}{やまのうち}\ruby{泰二}{たいじ}\ruby{君}{くん}に、すぐ\ruby{聞}{き}いた。

「ねえ、タイちゃん。この\ruby{学校}{がっこう}って、\ruby{校歌}{こうか}ある?」

\ruby{物理}{ぶつり}の\ruby{好}{す}きなタイちゃんは、とても、\ruby{考}{かんが}えそうな\ruby{声}{こえ}で\ruby{答}{こた}えた。

「ないんじゃないかな?」

「ふーん」

と、トットちゃんは、\ruby{少}{すこ}し、もったいをつけて、それから、

「あったほうが、いいと\ruby{思}{おも}うんだ。\ruby{前}{まえ}の\ruby{学校}{がっこう}なんて、すごいのが、あったんだから!」

といって、\ruby{大}{おお}きな\ruby{声}{こえ}で\ruby{歌}{うた}い\ruby{始}{はじ}めた。

「せんぞくいけはあさけれどいじんのむねをふかくくみ(\ruby{洗足池}{せんぞくいけ}は\ruby{浅}{あさ}けれど、\ruby{偉人}{いじん}の\ruby{胸}{むね}を\ruby{深}{ふか}く\ruby{汲}{く}み)」

これが、まえの\ruby{学校}{がっこう}の\ruby{校歌}{こうか}だった。ほんの\ruby{少}{すこ}ししか\ruby{通}{かよ}わなかったし、\ruby{一年生}{いちねんせい}には、\ruby{難}{むずか}しい\ruby{言葉}{ことば}だったけど、トットちゃんは、ちゃんと、\ruby{覚}{おぼ}えていた。(ただし、この\ruby{部分}{ぶぶん}だけだったけど)

\ruby{聞}{き}き\ruby{終}{お}わると、\ruby{泰}{たい}ちゃんは、\ruby{少}{すこ}し\ruby{感心}{かんしん}したように、\ruby{頭}{あたま}を二\ruby{回}{かい}くらい、\ruby{軽}{かる}く\ruby{振}{ふ}ると、

「ふーん」

といった。その\ruby{頃}{ころ}には、\ruby{他}{ほか}の\ruby{生徒}{せいと}も\ruby{着}{き}ていて、みんなも、トットちゃんの、\ruby{難}{むずか}しい\ruby{言葉}{ことば}に\ruby{尊敬}{そんけい}と、\ruby{憧}{あこが}れを\ruby{持}{も}ったらしく、

「ふーん」

といった。トットちゃんは、いった。「ねえ、\ruby{校長}{こうちょう}\ruby{先生}{せんせい}に、\ruby{校歌}{こうか}、\ruby{作}{つく}ってもらおうよ」

みんなも、そう\ruby{思}{おも}ったところだったから、

「そうしよう、そうしよう」

といって、みんなで、ゾロゾロ\ruby{校長}{こうちょう}\ruby{室}{しつ}に\ruby{行}{い}った。\ruby{校長}{こうちょう}\ruby{先生}{せんせい}は、トットちゃんの\ruby{歌}{うた}を\ruby{聞}{き}き、みんなの\ruby{希望}{きぼう}を\ruby{聞}{き}くと、「よし、じゃ、\ruby{明日}{あした}の\ruby{朝}{あさ}までに\ruby{作}{つく}っておくよ」といった。みんなは、「\ruby{約束}{やくそく}だよ」といって、また、ゾロゾロ\ruby{教室}{きょうしつ}に\ruby{戻}{もど}った。さて、\ruby{次}{つぎ}の\ruby{日}{ひ}の\ruby{朝}{あさ}だった。\ruby{各}{かく}\ruby{教室}{きょうしつ}に、\ruby{校長}{こうちょう}\ruby{先生}{せんせい}から、“みんな、\ruby{校庭}{こうてい}に\ruby{集}{あつ}まるように”という、ことづけがあった。トットちゃん\ruby{達}{たち}は、\ruby{期待}{きたい}でむねを、ワクワクさせながら\ruby{校庭}{こうてい}に\ruby{集}{あつ}まった。\ruby{校長}{こうちょう}\ruby{先生}{せんせい}は、\ruby{校庭}{こうてい}の\ruby{真}{ま}ん\ruby{中}{なか}に、\ruby{黒板}{こくばん}を\ruby{運}{はこ}び\ruby{出}{だ}すと、いった。「いいかい、\ruby{君}{きみ}\ruby{達}{たち}の\ruby{学校}{がっこう}、トモエの\ruby{校歌}{こうか}だよ」そして\ruby{黒板}{こくばん}に、五\ruby{線}{せん}を\ruby{書}{か}くと、\ruby{次}{つぎ}のように、オタマジャクシを\ruby{並}{なら}べた。それから、\ruby{校長}{こうちょう}\ruby{先生}{せんせい}は、\ruby{手}{て}を\ruby{指揮者}{しきしゃ}のように、\ruby{大}{おお}きく\ruby{上}{あ}げると、「さあ、\ruby{一緒}{いっしょ}に\ruby{歌}{うた}おう!」といって、\ruby{手}{て}を\ruby{振}{ふ}り\ruby{下}{お}ろした。\ruby{全校}{ぜんこう}\ruby{生徒}{せいと}、五十\ruby{人}{にん}は、みんな、\ruby{先生}{せんせい}の\ruby{声}{こえ}に\ruby{合}{あ}わせて、\ruby{歌}{うた}った。「トモエ、トモエ、トーモエ!」「……これだけ?」ちょっとした\ruby{間}{ま}があって、トットちゃんが\ruby{聞}{き}いた。\ruby{校長}{こうちょう}\ruby{先生}{せんせい}は、\ruby{得意}{とくい}そうに\ruby{答}{こた}えた。「そうだよ」トットちゃんは、ひどく、がっかりした\ruby{声}{こえ}で、\ruby{先生}{せんせい}に\ruby{言}{い}った。「もっと、むずかしいのが、よかったんだ。センゾクイケハアサケレドーみたいなの」\ruby{先生}{せんせい}は、\ruby{顔}{かお}を\ruby{真}{ま}っ\ruby{赤}{か}にして、\ruby{笑}{わら}いながらいった。「いいかい?これ、いいと\ruby{思}{おも}うけどな」\ruby{結局}{けっきょく}、\ruby{他}{ほか}の\ruby{子供}{こども}\ruby{達}{たち}も、「こんなカンタンすぎるのなら、いらない」といって、\ruby{断}{ことわ}った。\ruby{先生}{せんせい}は、ちょっと\ruby{残念}{ざんねん}そうだったけど、\ruby{別}{べつ}に\ruby{怒}{いか}りもしないで、\ruby{黒板}{こくばん}けしで、\ruby{消}{け}してしまった。トットちゃんは、すこし(\ruby{先生}{せんせい}に\ruby{悪}{わる}かったかな)と\ruby{思}{おも}ったけど(ほしかったのは、もっと\ruby{偉}{えら}そうなヤツだったんだもの、\ruby{仕方}{しかた}がないや)と\ruby{考}{かんが}えた。

\ruby{本当}{ほんとう}は、こんなに\ruby{簡単}{かんたん}で『\ruby{学校}{がっこう}を、そして\ruby{子供}{こども}たち』を\ruby{愛}{あい}する\ruby{校長}{こうちょう}\ruby{先生}{せんせい}の\ruby{気持}{きも}ちがこもった\ruby{校歌}{こうか}はなかったのに、\ruby{子供}{こども}\ruby{達}{たち}には、まだ、それが\ruby{分}{わ}からなかった。そして、その\ruby{後}{あと}、\ruby{子供}{こども}たちも\ruby{校歌}{こうか}のことは\ruby{忘}{わす}れ、\ruby{先生}{せんせい}も\ruby{要}{い}らないと\ruby{思}{おも}ったのか、\ruby{黒板}{こくばん}けしで\ruby{消}{け}したまま、\ruby{最後}{さいご}まで、トモエには、\ruby{校歌}{こうか}って、なかった。


