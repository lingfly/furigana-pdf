


\markright{初めての駅}

%\gtfamily                       %设置主体文字字体
\linespread{1.8}
\selectfont
\ruby{自由}{じゆう}が\ruby{丘}{おか}の\ruby{駅}{えき}で、\ruby{大井町線}{おおいまちせん}から\ruby{降}{お}りると、\ruby{ママ}{まま}は、\ruby{トット}{とっと}ちゃんの\ruby{手}{て}を\ruby{引}{ひ}っ\ruby{張}{ぱ}って、\ruby{改札口}{かいさつぐち}を\ruby{出}{で}ようとした。\ruby{トット}{とっと}ちゃんは、それまで、あまり\ruby{電車}{でんしゃ}に\ruby{乗}{の}ったことがなかったから、\ruby{大切}{たいせつ}に\ruby{握}{にぎ}っていた\ruby{切符}{きっぷ}をあげちゃうのは、もったいないなと\ruby{思}{おも}った。

そこで、\ruby{改札口}{かいさつぐち}のおじさんに、「この\ruby{切符}{きっぷ}、もらっちゃいけない?」と\ruby{聞}{き}いた。おじさんは「\ruby{ダメ}{だめ}だよ」というと、\ruby{トット}{とっと}ちゃんの\ruby{手}{て}から、\ruby{切符}{きっぷ}を\ruby{取}{と}り\ruby{上}{あ}げた。\ruby{トット}{とっと}ちゃんは、\ruby{改札口}{かいさつぐち}の\ruby{箱}{はこ}にいっぱい\ruby{溜}{た}まっている\ruby{切符}{きっぷ}をさして\ruby{聞}{き}いた。「これ、\ruby{全部}{ぜんぶ}、おじさんの?」おじさんは、\ruby{他}{ほか}の\ruby{出}{で}て\ruby{行}{い}く\ruby{人}{ひと}の\ruby{切符}{きっぷ}をひったくりながら\ruby{答}{こた}えた。「おじさんのじゃないよ、\ruby{駅}{えき}のだから」「へーえ……」\ruby{トット}{とっと}ちゃんは、\ruby{未練}{みれん}がましく、\ruby{箱}{はこ}を\ruby{覗}{のぞ}き\ruby{込}{こ}みながら\ruby{言}{い}った。「\ruby{私}{わたし}、\ruby{大人}{おとな}になったら、\ruby{切符}{きっぷ}を\ruby{売}{う}る\ruby{人}{ひと}になろうと\ruby{思}{おも}うわ」おじさんは、はじめて、\ruby{トット}{とっと}ちゃんを\ruby{チラリ}{ちらり}と\ruby{見}{み}て、いった。「うちの\ruby{男}{おとこ}の\ruby{子}{こ}も、\ruby{駅}{えき}で\ruby{働}{はたら}きたいって、いってるから、\ruby{一緒}{いっしょ}にやるといいよ」

\ruby{トット}{とっと}ちゃんは、\ruby{少}{すこ}し\ruby{離}{はな}れて、おじさんを\ruby{見}{み}た。おじさんは\ruby{肥}{ふと}っていて、\ruby{眼鏡}{めがね}をかけていて、よく\ruby{見}{み}ると、やさしそうなところもあった。「ふん……」\ruby{トット}{とっと}ちゃんは、\ruby{手}{て}を\ruby{腰}{こし}に\ruby{当}{あ}てて、\ruby{観察}{かんさつ}しながら\ruby{言}{い}った。「おじさんとこの\ruby{子}{こ}と、\ruby{一緒}{いっしょ}にやってもいいけど、\ruby{考}{かんが}えとくわ。あたし、これから\ruby{新}{あたら}しい\ruby{学校}{がっこう}に\ruby{行}{い}くんで、\ruby{忙}{いそが}しいから」そういうと、\ruby{トット}{とっと}ちゃんは、\ruby{待}{ま}ってる\ruby{ママ}{まま}のところに\ruby{走}{はし}っていった。そして、こう\ruby{叫}{さけ}んだ。「\ruby{私}{わたし}、\ruby{切符}{きっぷ}\ruby{屋}{や}さんになろうと\ruby{思}{おも}うんだ!」\ruby{ママ}{まま}は、\ruby{驚}{おどろ}きもしないで、いった。「でも、\ruby{スパイ}{すぱい}になるって\ruby{言}{い}ってたのは、どうするの?」

\ruby{トット}{とっと}ちゃんは、\ruby{ママ}{まま}に\ruby{手}{て}を\ruby{取}{と}られて\ruby{歩}{ある}き\ruby{出}{だ}しながら、\ruby{考}{かんが}えた。(そうだわ。\ruby{昨日}{きのう}までは、\ruby{絶対}{ぜったい}に\ruby{スパイ}{すぱい}になろう、って\ruby{決}{き}めてたのに。でも、いまの\ruby{切符}{きっぷ}をいっぱい\ruby{箱}{はこ}にしまっておく\ruby{人}{ひと}になるのも、とても、いいと\ruby{思}{おも}うわ)「そうだ!」\ruby{トット}{とっと}ちゃんは、いいことを\ruby{思}{おも}いついて、\ruby{ママ}{まま}の\ruby{顔}{かお}をのぞきながら、\ruby{大声}{おおごえ}をはりあげていった。「ねえ、\ruby{本当}{ほんとう}は\ruby{スパイ}{すぱい}なんだけど、\ruby{切符}{きっぷ}\ruby{屋}{や}さんなのは、どう?」\ruby{ママ}{まま}は\ruby{答}{こた}えなかった。

\ruby{本当}{ほんとう}のことを\ruby{言}{い}うと、\ruby{ママ}{まま}はとても\ruby{不安}{ふあん}だったのだ。もし、これから\ruby{行}{い}く\ruby{小学校}{しょうがっこう}で、\ruby{トット}{とっと}ちゃんのことを、あずかってくれなかったら……。\ruby{小}{ちい}さい\ruby{花}{はな}のついた、\ruby{フェルト}{ふぇると}の\ruby{帽子}{ぼうし}をかぶっている、\ruby{ママ}{まま}の、きれいな\ruby{顔}{かお}が、\ruby{少}{すこ}しまじめになった。そして、\ruby{道}{みち}を\ruby{飛}{と}び\ruby{跳}{は}ねながら、\ruby{何}{なに}かを\ruby{早口}{はやくち}でしゃべってるとっとちゃんを\ruby{見}{み}た。\ruby{トット}{とっと}ちゃんは、\ruby{ママ}{まま}の\ruby{心配}{しんぱい}を\ruby{知}{し}らなかったから、\ruby{顔}{かお}があうと、うれしそうに\ruby{笑}{わら}っていった。「ねえ、\ruby{私}{わたし}、やっぱり、どっちもやめて、\ruby{チンドン}{ちんどん}\ruby{屋}{や}さんになる!!」\ruby{ママ}{まま}は、\ruby{多少}{たしょう}、\ruby{絶望的}{ぜつぼうてき}な\ruby{気分}{きぶん}で\ruby{言}{い}った。「さあ、\ruby{遅}{おく}れるわ。\ruby{校長}{こうちょう}\ruby{先生}{せんせい}が\ruby{待}{ま}ってらしゃるんだから。もう、おしゃべりしないで、\ruby{前}{まえ}を\ruby{向}{む}いて、\ruby{歩}{ある}いてちょうだい」\ruby{二人}{ふたり}の\ruby{目}{め}の\ruby{前}{まえ}に、\ruby{小}{ちい}さい\ruby{学校}{がっこう}の\ruby{門}{もん}が\ruby{見}{み}えてきた。




