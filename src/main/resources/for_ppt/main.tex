% !TeX encoding = UTF-8
% !TeX program = LuaLaTeX

\documentclass[12pt,a4paper,oneside,openany]{book}

\usepackage{luatexja-ruby}       %假名标注
\usepackage{xcolor}			 %引入颜色
\usepackage[hidelinks]{hyperref} %给目录添加超链接
\usepackage{titlesec}
\usepackage{indentfirst}         %章节首页首行缩进


% 自定义章节标题样式,调整默认垂直间距
\titleformat{\chapter}       	 % 要格式化的章节命令,如 \chapter、\section、\subsection 等。
[hang]                       	 % 标题的形状,可以是 hang(悬挂格式,默认)、block(块格式)
{\normalfont\Huge\bfseries}   	 % format,标题的整体格式,可以包括字体、大小、粗细等
{Chapter\thechapter}          	 % label,标题编号的格式,如 \thechapter、\thesection 等。
{1em}                         	 % Spacing between label and title
{}

\headheight = 13pt           	 % 页眉高度
\headsep = 30pt		        	 % 页眉和正文的间距
\topmargin = -30pt               % 页眉和页面顶端的间距
\titlespacing{\chapter}{0pt}{-60pt}{10pt} % 调整标题间距
\textheight = 700pt              % 正文高度
\textwidth = 300pt 			 % 正文宽度
\setlength{\hoffset}{-20pt}      % 正文向左偏移20pt

\ltjsetruby{size=0.6}            %设置振假名字号
%\ltjsetruby{fontcmd=\gtfamily}  %设置振假名字体
\ltjsetruby{mode=00}             %设置振假名的「進入」和「突出」模式

%\setlength{\parindent}{2em}	 %首行缩进
\marginparwidth = 72pt           %边栏宽度

\linespread{2.0}			 %行距
\selectfont

% 边注
\newcounter{num}[chapter]
\newcommand{\translate}[2]{\addtocounter{num}{1} {\color{orange} #1\textsuperscript{\scriptsize \thenum}}\marginpar{\scriptsize \textsuperscript{\scriptsize \thenum} #2}}

\begin{document}
\tableofcontents

トットちゃんには、本当に、新しい驚きで、いっぱいの、トモエ学園での毎日が過ぎていった。相変わらず、学校に早く行きたくて、朝が待ちきれなかった。そして、帰ってくると、犬のロッキーと、ママとパパに、「今日、学校で、どんなことをして、どのくらい面白かった」とか、「もう、びっくりしちゃった」とか、しまいには、ママが、「話は、ちょっとお休みして、おやつにしたら?」というまで、話をやめなかった。そして、これは、どんなにトットちゃんが、学校に馴れてもやっぱり、毎日ように、話すことは、山のように、あったのだった。(でも、こんなに話すことがたくさんあるってことは、有難いこと)と、ママは、心から、嬉しく思っていた。ある日、トットちゃんは、学校に行く電車の中で、突然、「あれ?オモエに校歌って、あったかな?」と考えた。そう思ったら、もう、早く学校に着きたくなって、まだ、あと二つも駅があるのに、ドアのところに立って、自由が丘に電車が着いたら、すぐ出られるように、ヨーイ・ドンの格好で待った。ひとつ前の駅で、ドアが開いたとき、乗り込もうとした、おばさんは、女の子が、ドアのところで、ヨーイ・ドンの形になってるので、降りるのか、と思ったら、そのままの形で動かないので、「どうなっちゃってるのかね」といいながら、乗り込んできた。こんな具合だったから、駅に着いたときの、トットちゃんの早く降りたことといったら、なかった。若い男の車掌さんが、しゃれたポーズで、まだ、完全に止まっていない電車から、プラットホームに片足をつけて、おりながら、「自由が丘!お降りの方は……」といったとき、もう、トットちゃんの姿は、改札口から、見えなくなっていた。学校に着いて、電車の教室に入ると、トットちゃんは、先に来ていた、山内泰二君に、すぐ聞いた。「ねえ、タイちゃん。この学校って、校歌ある?」物理の好きなタイちゃんは、とても、考えそうな声で答えた。「ないんじゃないかな?」「ふーん」と、トットちゃんは、少し、もったいをつけて、それから、「あったほうが、いいと思うんだ。前の学校なんて、すごいのが、あったんだから!」といって、大きな声で歌い始めた。「せんぞくいけはあさけれどいじんのむねをふかくくみ(洗足池は浅けれど、偉人の胸を深く汲み)」これが、まえの学校の校歌だった。ほんの少ししか通わなかったし、一年生には、難しい言葉だったけど、トットちゃんは、ちゃんと、覚えていた。(ただし、この部分だけだったけど)聞き終わると、泰ちゃんは、少し感心したように、頭を二回くらい、軽く振ると、「ふーん」といった。その頃には、他の生徒も着ていて、みんなも、トットちゃんの、難しい言葉に尊敬と、憧れを持ったらしく、「ふーん」といった。トットちゃんは、いった。「ねえ、校長先生に、校歌、作ってもらおうよ」みんなも、そう思ったところだったから、「そうしよう、そうしよう」といって、みんなで、ゾロゾロ校長室に行った。校長先生は、トットちゃんの歌を聞き、みんなの希望を聞くと、「よし、じゃ、明日の朝までに作っておくよ」といった。みんなは、「約束だよ」といって、また、ゾロゾロ教室に戻った。さて、次の日の朝だった。各教室に、校長先生から、“みんな、校庭に集まるように”という、ことづけがあった。トットちゃん達は、期待でむねを、ワクワクさせながら校庭に集まった。校長先生は、校庭の真ん中に、黒板を運び出すと、いった。「いいかい、君達の学校、トモエの校歌だよ」そして黒板に、五線を書くと、次のように、オタマジャクシを並べた。それから、校長先生は、手を指揮者のように、大きく上げると、「さあ、一緒に歌おう!」といって、手を振り下ろした。全校生徒、五十人は、みんな、先生の声に合わせて、歌った。「トモエ、トモエ、トーモエ!」「……これだけ?」ちょっとした間があって、トットちゃんが聞いた。校長先生は、得意そうに答えた。「そうだよ」トットちゃんは、ひどく、がっかりした声で、先生に言った。「もっと、むずかしいのが、よかったんだ。センゾクイケハアサケレドーみたいなの」先生は、顔を真っ赤にして、笑いながらいった。「いいかい?これ、いいと思うけどな」結局、他の子供達も、「こんなカンタンすぎるのなら、いらない」といって、断った。先生は、ちょっと残念そうだったけど、別に怒りもしないで、黒板けしで、消してしまった。トットちゃんは、すこし(先生に悪かったかな)と思ったけど(ほしかったのは、もっと偉そうなヤツだったんだもの、仕方がないや)と考えた。
本当は、こんなに簡単で『学校を、そして子供たち』を愛する校長先生の気持ちがこもった校歌はなかったのに、子供達には、まだ、それが分からなかった。そして、その後、子供たちも校歌のことは忘れ、先生も要らないと思ったのか、黒板けしで消したまま、最後まで、トモエには、校歌って、なかった。

\end{document}