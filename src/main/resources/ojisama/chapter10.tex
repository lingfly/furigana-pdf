\ruby{王子}{おうじ}さまは\ruby{小}{しょう}\ruby{惑星}{わくせい}325、326、327、328、329、330の\ruby{近}{ちか}くを\ruby{通}{とお}りかかった。そこで\ruby{仕事}{しごと}を\ruby{探}{さが}したり、\ruby{見聞}{けんぶん}を\ruby{広}{ひろ}げるため、それらの\ruby{小}{しょう}\ruby{惑星}{わくせい}を\ruby{一}{ひと}つずつ\ruby{訪}{たず}ねる\ruby{事}{こと}にした。\ruby{最初}{さいしょ}の\ruby{星}{ほし}には\ruby{王様}{おうさま}が\ruby{住}{す}んでいた。\ruby{緋色}{ひいろ}の\ruby{衣}{ころも}に\ruby{白点}{はくてん}の\ruby{毛皮}{けがわ}を\ruby{纏}{まと}い、\ruby{質素}{しっそ}だが、\ruby{威厳}{いげん}のある\ruby{玉座}{ぎょくざ}に\ruby{腰掛}{こしか}けていた。

\ruby{王子}{おうじ}\ruby{様}{さま}を\ruby{見}{み}かけると、\ruby{大}{おお}きな\ruby{声}{こえ}で\ruby{言}{い}いました。

「や、\ruby{家来}{けらい}が\ruby{来}{き}たなあ!」

\ruby{王子}{おうじ}\ruby{様}{さま}は、\ruby{一度}{いちど}も\ruby{僕}{ぼく}に\ruby{会}{あ}ったことがないのに、どうして\ruby{見}{み}\ruby{覚}{おぼ}えがあるのだろうと\ruby{考}{かんが}えました。\ruby{王様}{おうさま}にかかれば、\ruby{世界}{せかい}はとてもあっさりしたものになる。\ruby{誰}{だれ}も\ruby{彼}{かれ}もみんな、\ruby{家来}{けらい}。\ruby{王子}{おうじ}\ruby{様}{さま}はそれを\ruby{知}{し}らなかったんだ。

「\ruby{近}{ちか}く\ruby{寄}{よ}りなさい。そのほうがもっとよく\ruby{見}{み}えるように。」

\ruby{王様}{おうさま}はやっと\ruby{誰}{だれ}かに\ruby{王様}{おうさま}らしくできると、\ruby{嬉}{うれ}しくてたまらなかった。

\ruby{王子}{おうじ}\ruby{様}{さま}はどこかに\ruby{座}{すわ}ろうと、\ruby{周}{まわ}りを\ruby{見}{み}た。でも、\ruby{星}{ほし}は\ruby{大}{おお}きな\ruby{毛皮}{けがわ}の\ruby{裾}{すそ}で、どこもいっぱいだった。\ruby{王子}{おうじ}\ruby{様}{さま}は\ruby{仕方}{しかた}なく\ruby{立}{た}ちっぱなし、しかもへとへとだったから、あくびが\ruby{出}{で}た。

「\ruby{王}{おう}の\ruby{前}{まえ}であくびとは、\ruby{作法}{さほう}がなっとらん!」と、\ruby{王様}{おうさま}は\ruby{言}{い}った。「ダメであるぞ!」

「\ruby{我慢}{がまん}できないんです。」と、\ruby{王子}{おうじ}\ruby{様}{さま}は\ruby{迷惑}{めいわく}そうに\ruby{返事}{へんじ}をした。「\ruby{僕}{ぼく}、\ruby{長}{なが}い\ruby{旅}{たび}をしてきたんでしょう?それに、\ruby{眠}{ねむ}らなかったものですから…」

「そうか。では、あくびをしなさい。\ruby{命令}{めいれい}する。わしはもう\ruby{何年}{なんねん}か\ruby{人}{ひと}のあくびをするのを\ruby{見}{み}たことがない。あくびというものは\ruby{面白}{おもしろ}いものだなあ。さあ、あくびしなさい、もう\ruby{一度}{いちど}、\ruby{命令}{めいれい}じゃ。」

「\ruby{胸}{むね}がドキドキして、もうできなくなりました。」と、\ruby{王子}{おうじ}\ruby{様}{さま}は、\ruby{顔}{かお}を\ruby{真}{ま}っ\ruby{赤}{か}にした。

「これはこれは…では、こう\ruby{命令}{めいれい}する。あるときはあくびをし、あるときは…」

\ruby{王様}{おうさま}は\ruby{何}{なに}か\ruby{口}{ぐち}の\ruby{中}{なか}でもぐもぐ\ruby{言}{い}って、\ruby{気}{き}を\ruby{揉}{も}んでいる\ruby{様子}{ようす}でした。

なぜなら、\ruby{王様}{おうさま}はなんでも\ruby{自分}{じぶん}の\ruby{思}{おも}い\ruby{通}{どお}りにしたくて、そこから\ruby{外}{はず}れるものは\ruby{許}{ゆる}せなかった。いわゆる、\ruby{絶対}{ぜったい}の\ruby{王様}{おうさま}ってやつ。でも、\ruby{根}{ね}は\ruby{優}{やさ}しかったので、\ruby{物分}{ものわか}りのいいことしか\ruby{言}{い}いつけなかった。

\ruby{王様}{おうさま}にはこんな\ruby{口}{くち}\ruby{癖}{ぐせ}がある。

「わしが\ruby{大将}{たいしょう}に\ruby{海}{うみ}の\ruby{鳥}{とり}になれと\ruby{命令}{めいれい}したとする。その\ruby{大将}{たいしょう}がわしの\ruby{命令}{めいれい}に\ruby{従}{したが}わないとしても、\ruby{大将}{たいしょう}がいけないわけではないだろう。わしがいけないのだろう。」

「\ruby{座}{すわ}っていい?」と、\ruby{王子}{おうじ}\ruby{様}{さま}は\ruby{気}{き}まずそうに\ruby{言}{い}った。

「うん、\ruby{座}{すわ}んなさい、\ruby{命令}{めいれい}する。」\ruby{王様}{おうさま}は\ruby{毛皮}{けがわ}の\ruby{裾}{すそ}を\ruby{厳}{おごそ}かに\ruby{引}{ひ}いて、\ruby{言}{い}いつけた。

でも、\ruby{王子}{おうじ}\ruby{様}{さま}にはよくわからないことがあった。この\ruby{星}{ほし}はすごくちいちゃい、\ruby{王様}{おうさま}は\ruby{一体}{いったい}、\ruby{何}{なに}を\ruby{治}{おさ}めてるんだろうか。

「\ruby{陛下}{へいか}、すいませんが、\ruby{質問}{しつもん}が…」

「\ruby{訪}{たず}ねなさい、\ruby{命令}{めいれい}する!」と、\ruby{王様}{おうさま}は\ruby{慌}{あわ}てて\ruby{言}{い}った。

「\ruby{陛下}{へいか}は\ruby{何}{なに}を\ruby{治}{おさ}めてるんですか。」

「すべてである。」と、\ruby{王様}{おうさま}は\ruby{当}{あ}たり\ruby{前}{まえ}のように\ruby{答}{こた}えた。

「すべて?」

\ruby{王様}{おうさま}はそっと\ruby{指}{ゆび}を\ruby{出}{だ}して、\ruby{自分}{じぶん}の\ruby{星}{ほし}と、ほかの\ruby{惑星}{わくせい}とか\ruby{星}{ほし}とか、みんなを\ruby{指}{さ}した。

「あれをみんな?」と、\ruby{王子}{おうじ}\ruby{様}{さま}は\ruby{言}{い}った。

「うん、あれをみんな。」と、\ruby{王様}{おうさま}は\ruby{答}{こた}えた。なぜなら、\ruby{絶対}{ぜったい}の\ruby{王様}{おうさま}であるだけでなく、\ruby{宇宙}{うちゅう}の\ruby{王様}{おうさま}でもあったからだ。

「じゃあ、\ruby{星}{ほし}はみんな、\ruby{陛下}{へいか}にしたがっているわけですね?」

「そうだとも。すぐにも\ruby{従}{したが}う。わしは\ruby{不}{ふ}\ruby{規律}{きりつ}を\ruby{許}{ゆる}さんのじゃ。」

あまりにすごい\ruby{力}{ちから}なので、\ruby{王子}{おうじ}\ruby{様}{さま}はびっくりした。\ruby{自分}{じぶん}にもしそれだけの\ruby{力}{ちから}があれば、40\ruby{回}{かい}と\ruby{言}{い}わず、72\ruby{回}{かい}、いや、100\ruby{回}{かい}でも、いやいや、200\ruby{回}{かい}でも、\ruby{夕暮}{ゆうぐ}れがたった一\ruby{日}{にち}の\ruby{間}{あいだ}に\ruby{見}{み}られるんじゃないか。しかも、\ruby{椅子}{いす}も\ruby{動}{うご}かずに。

そう\ruby{考}{かんが}えたとき、ちょっと\ruby{切}{せつ}なくなった。そういえば、\ruby{自分}{じぶん}の\ruby{小}{ちい}さな\ruby{星}{ほし}を\ruby{捨}{す}ててきたんだって。だから、\ruby{思}{おも}い\ruby{切}{き}って、\ruby{王様}{おうさま}にお\ruby{願}{ねが}いをしてみた。

「\ruby{夕暮}{ゆうぐ}れが\ruby{見}{み}たいんです。どうかお\ruby{願}{ねが}いします。\ruby{夕暮}{ゆうぐ}れろって\ruby{言}{い}ってください。」

「わしが\ruby{大将}{たいしょう}に\ruby{向}{む}かって、\ruby{蝶々}{ちょうちょ}みたいに\ruby{花}{はな}から\ruby{花}{はな}へ\ruby{飛}{と}べとか、\ruby{悲劇}{ひげき}を\ruby{書}{か}けとか、\ruby{海}{うみ}の\ruby{鳥}{とり}になれとか、\ruby{命令}{めいれい}するとする。そして、その\ruby{大将}{たいしょう}が\ruby{命令}{めいれい}を\ruby{実行}{じっこう}しないとしたら、\ruby{大将}{たいしょう}とわしと、どっちが\ruby{間違}{まちが}ってるだろうかね。」

「\ruby{王様}{おうさま}の\ruby{方}{ほう}です。」と、\ruby{王子}{おうじ}\ruby{様}{さま}はきっぱり\ruby{言}{い}った。

「その\ruby{通}{とお}り。\ruby{人}{ひと}には\ruby{銘々}{めいめい}その\ruby{人}{ひと}のできることをしてもらわなきゃならん。\ruby{道理}{どうり}の\ruby{土台}{どだい}あっての\ruby{権力}{けんりょく}じゃ。もし、お\ruby{前}{まえ}が\ruby{人民}{じんみん}たちに、\ruby{海}{うみ}に\ruby{行}{い}って\ruby{飛}{と}び\ruby{込}{こ}めと\ruby{命令}{めいれい}したら、\ruby{人民}{じんみん}たちは\ruby{革命}{かくめい}を\ruby{起}{お}こすだろう。わしは\ruby{無理}{むり}の\ruby{命令}{めいれい}をしないのだから、みんなをわしに\ruby{服従}{ふくじゅう}させる\ruby{権力}{けんりょく}があるのじゃ。」

「じゃあ、\ruby{僕}{ぼく}の\ruby{夕暮}{ゆうぐ}れは?」と、\ruby{王子}{おうじ}\ruby{様}{さま}は\ruby{迫}{せま}った。なぜなら、\ruby{王子}{おうじ}\ruby{様}{さま}は\ruby{一度}{いちど}\ruby{聞}{き}いたことは\ruby{絶対}{ぜったい}\ruby{忘}{わす}れない。

「うーん、\ruby{夕日}{ゆうひ}は\ruby{見}{み}せてあげる。わしが\ruby{命令}{めいれい}してやる。だが、\ruby{都合}{つごう}がよくなるまで、\ruby{待}{ま}つとしよう。それがわしの\ruby{政治}{せいじ}のことじゃ。」

「それはいつ?」と、\ruby{王子}{おうじ}\ruby{様}{さま}は\ruby{尋}{たず}ねる。

「うーん…」と、\ruby{王様}{おうさま}は\ruby{言}{い}って、\ruby{分厚}{ぶあつ}い\ruby{暦}{こよみ}を\ruby{調}{しら}べた。「うーん、そうだなあ。\ruby{大体}{だいたい}、\ruby{午後}{ごご}7\ruby{時}{じ}40\ruby{分}{ふん}ぐらいである。まあ、\ruby{見}{み}ていなさい、\ruby{万事}{ばんじ}わしの\ruby{命令}{めいれい}\ruby{通}{どお}りになるから。」

\ruby{王子}{おうじ}\ruby{様}{さま}はあくびをした。\ruby{夕暮}{ゆうぐ}れに\ruby{会}{あ}えなくて、\ruby{残念}{ざんねん}だった。それに、ちょっともううんざりだった。

「ここですることはもうないから。」と、\ruby{王子}{おうじ}\ruby{様}{さま}は\ruby{王様}{おうさま}に\ruby{言}{い}った。「そろそろ\ruby{行}{い}くよ。」

「\ruby{行}{い}くな、\ruby{行}{い}くな!」と、\ruby{王様}{おうさま}は\ruby{言}{い}った。\ruby{家来}{けらい}ができて、それだけ\ruby{嬉}{うれ}しかったんだ。

「\ruby{行}{い}ってはならん!そちを\ruby{大臣}{だいじん}にしてやるぞ!」

「それで\ruby{何}{なに}をするの?」

「うーん、\ruby{人}{ひと}を\ruby{裁}{さば}くであるぞ!」

「でも、\ruby{裁}{さば}くにしても、\ruby{人}{ひと}がいないよ。」

「そりゃ\ruby{分}{わ}からん。わしはまだ、わしの\ruby{国}{くに}を\ruby{回}{まわ}ってみたことがないんでね。\ruby{年}{とし}を\ruby{取}{と}ったし、\ruby{馬車}{ばしゃ}を\ruby{置}{お}く\ruby{場所}{ばしょ}がないんで、\ruby{歩}{ある}くのが\ruby{疲}{つか}れるよ。」

「うーん~でも\ruby{僕}{ぼく}はもう\ruby{見}{み}たよ。」と、\ruby{王子}{おうじ}\ruby{様}{さま}は\ruby{屈}{かが}んで、もう\ruby{一度}{いちど}チラリっと\ruby{星}{ほし}の\ruby{向}{む}こう\ruby{側}{がわ}を\ruby{見}{み}た。「あっちには\ruby{人}{ひと}っ\ruby{子一人}{こひとり}いない。」

「なら、\ruby{自分}{じぶん}を\ruby{裁}{さば}くである。」と、\ruby{王様}{おうさま}は\ruby{答}{こた}えた。「もっと\ruby{難}{むずか}しいぞ、\ruby{自分}{じぶん}を\ruby{裁}{さば}くほうが、\ruby{人}{ひと}を\ruby{裁}{さば}くよりも、はるかに\ruby{難}{むずか}しい。うまく\ruby{自分}{じぶん}を\ruby{裁}{さば}くことができたなら、それは、\ruby{正真}{しょうしん}\ruby{正銘}{しょうめい}\ruby{賢者}{けんじゃ}の\ruby{証}{あかし}だ。」

すると、\ruby{王子}{おうじ}\ruby{様}{さま}は\ruby{言}{い}った。

「\ruby{僕}{ぼく}、どこにいたって、\ruby{自分}{じぶん}を\ruby{裁}{さば}けます。ここに\ruby{住}{す}む\ruby{必要}{ひつよう}はありません。」

「ええとね、わしの\ruby{星}{ほし}には、\ruby{年}{とし}とったねずみがどこかにいるようじゃ。\ruby{夜}{よる}、\ruby{物}{もの}\ruby{音}{おと}がするからな。そのヨボヨボのねずみを\ruby{裁}{さば}けばよい。ときとき、\ruby{死刑}{しけい}にするんである。そうすれば、その\ruby{命}{いのち}はそちの\ruby{裁}{さば}き\ruby{次第}{しだい}である。だが、いつも\ruby{許}{ゆる}してやることだ。一\ruby{匹}{ひき}しかいないねずみなんだからね。」

また、\ruby{王子}{おうじ}\ruby{様}{さま}は\ruby{返事}{へんじ}をする。

「\ruby{僕}{ぼく}、\ruby{死刑}{しけい}にするの\ruby{嫌}{きら}いだし、もう、さっさと\ruby{行}{い}きたいんです。」

「ならん!」と、\ruby{王様}{おうさま}は\ruby{言}{い}う。

もう、\ruby{王子}{おうじ}\ruby{様}{さま}はいつでも\ruby{行}{おこな}けたんだけど、\ruby{年寄}{としよ}りの\ruby{王様}{おうさま}をしょんぼりさせたくなかった。

「もし\ruby{陛下}{へいか}が、\ruby{言}{い}う\ruby{通}{とお}りになるのをお\ruby{望}{のぞ}みなら、\ruby{物分}{ものわか}りのいいことを\ruby{言}{い}いつけられるはずです。ほら、一\ruby{分}{ふん}\ruby{以内}{いない}に\ruby{出発}{しゅっぱつ}せよ、とか。\ruby{僕}{ぼく}には、\ruby{都合良}{つごうよ}くなっているように\ruby{思}{おも}うんですけど。」

\ruby{王様}{おうさま}は\ruby{何}{なに}も\ruby{言}{い}わかなった。

\ruby{王子}{おうじ}\ruby{様}{さま}はどうしようかと\ruby{思}{おも}ったけど、ため\ruby{息}{いき}をついて、ついに\ruby{星}{ほし}を\ruby{後}{あと}にした。

「そちをほかの\ruby{星}{ほし}へ\ruby{使}{つか}わせるぞ!」そのとき、\ruby{王様}{おうさま}は\ruby{慌}{あわ}ててこう\ruby{言}{い}った。まったくもって、\ruby{偉}{えら}そうな\ruby{言}{い}い\ruby{方}{かた}だった。

\ruby{大人}{おとな}の\ruby{人}{ひと}って、\ruby{相当}{そうとう}\ruby{変}{か}わってるなあ。と、\ruby{王子}{おうじ}\ruby{様}{さま}は\ruby{旅}{たび}を\ruby{続}{つづ}けながら、そう\ruby{思}{おも}った。


