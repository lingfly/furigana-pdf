五\ruby{番目}{ばんめ}の\ruby{星}{ほし}はとても\ruby{珍}{めずら}しい\ruby{星}{ほし}だった。\ruby{星}{ほし}のうちで、\ruby{一番}{いちばん}\ruby{小}{ちい}さな\ruby{星}{ほし}だった。そこには、ちょうど、\ruby{街灯}{がいとう}と\ruby{点灯}{てんとう}\ruby{人}{ひと}がいられるぐらいの\ruby{場所}{ばしょ}しかなかった。\ruby{王子}{おうじ}\ruby{様}{さま}はどうやってもわからなかった。\ruby{空}{そら}のこんな\ruby{場所}{ばしょ}で、\ruby{星}{ほし}に\ruby{家}{いえ}もないし、\ruby{人}{ひと}もいないのに、\ruby{街灯}{がいとう}と\ruby{点灯}{てんとう}\ruby{人}{ひと}がいて、\ruby{何}{なに}のためになるんだろうか。それでも、\ruby{王子}{おうじ}\ruby{様}{さま}は、\ruby{心}{こころ}の\ruby{中}{なか}でこう\ruby{思}{おも}った。

この\ruby{人}{ひと}はバカバカしいかもしれない。でも、\ruby{王様}{おうさま}、\ruby{自惚}{うぬぼ}れ\ruby{男}{おとこ}、\ruby{実業家}{じつぎょうか}や、\ruby{大酒飲}{おおざけの}みよりは、バカバカしくない。そうだとしても、この\ruby{人}{ひと}のやってることには、\ruby{意味}{いみ}がある。\ruby{明}{あ}かりをつけるってことは、\ruby{例}{たと}えるなら、\ruby{星}{ほし}とか、\ruby{花}{はな}とかが、ひとつ\ruby{新}{あたら}しく\ruby{生}{う}まれるってこと。だから、\ruby{灯}{とも}りを\ruby{消}{け}すのは、\ruby{星}{ほし}とか\ruby{花}{はな}をお\ruby{休}{やす}みさせるってこと。とっても\ruby{素敵}{すてき}なお\ruby{勤}{つと}め。\ruby{素敵}{すてき}だから、\ruby{本当}{ほんとう}に、\ruby{誰}{だれ}かのためになる。

\ruby{王子}{おうじ}\ruby{様}{さま}は、\ruby{星}{ほし}に\ruby{足}{あし}を\ruby{踏}{ふ}み\ruby{入}{い}れたとき、\ruby{丁寧}{ていねい}に\ruby{点灯}{てんとう}\ruby{人}{じん}にお\ruby{辞儀}{じぎ}をした。

「こんにちは。なぜ、いま、\ruby{街頭}{がいとう}の\ruby{火}{ひ}を\ruby{消}{け}したの?」

「\ruby{命令}{めいれい}だよ。やあ、おはよう。」と、\ruby{点灯}{てんとう}\ruby{人}{じん}が\ruby{答}{こた}えた。

「どうな\ruby{命令}{めいれい}?」

「\ruby{街頭}{がいとう}の\ruby{火}{ひ}を\ruby{消}{け}すことだよ。やあ、こんばんは。」と\ruby{言}{い}って、\ruby{点灯}{てんとう}\ruby{人}{じん}はまた\ruby{火}{ひ}をつけた。

「だけど、なぜ、また\ruby{火}{ひ}をつけたの?」

「\ruby{命令}{めいれい}だよ。」と、\ruby{点灯}{てんとう}\ruby{人}{じん}が\ruby{答}{こた}えた。

「わからないなあ。」と、\ruby{王子}{おうじ}\ruby{様}{さま}が\ruby{言}{い}った。

「わからなくていいよ、\ruby{命令}{めいれい}は\ruby{命令}{めいれい}だよ。やあ、おはよう。」と\ruby{言}{い}って、\ruby{点灯}{てんとう}\ruby{人}{じん}は\ruby{街頭}{がいとう}の\ruby{火}{ひ}を\ruby{消}{け}した。

それから、おでこを\ruby{赤}{あか}いチェックのハンカチで\ruby{拭}{ふ}いた。

「なにしろ、とんでもない\ruby{仕事}{しごと}だよ。\ruby{昔}{むかし}は\ruby{理屈}{りくつ}にあってたんだがね。\ruby{朝}{あさ}になると\ruby{火}{ひ}を\ruby{消}{け}す、\ruby{夕方}{ゆうがた}になると、\ruby{火}{ひ}をつける。\ruby{昼間}{ひるま}は\ruby{休}{やす}めたし、\ruby{夜}{よる}は\ruby{眠}{ねむ}ったもんだ。」

「で、そのあと、\ruby{命令}{めいれい}が\ruby{変}{か}わったってこと?」

「\ruby{命令}{めいれい}は\ruby{変}{か}わりゃしないよ。それが\ruby{本当}{ほんとう}、ひどい\ruby{話}{はなし}なんだ。この\ruby{星}{ほし}は\ruby{年々}{としどし}、\ruby{回}{まわ}るのがどんどん\ruby{早}{はや}くなるのに、\ruby{命令}{めいれい}は\ruby{変}{か}わらないときてるんだからなあ。」

「つまり?」

「つまり、\ruby{今}{いま}じゃ、この\ruby{星}{ほし}のやつが、一\ruby{分間}{ふんかん}に\ruby{一周}{ひとまわ}りすることになってるんで、\ruby{俺}{おれ}ときたら、一\ruby{秒}{びょう}も\ruby{休}{やす}めなくなったんだよ。一\ruby{分間}{ふんかん}に\ruby{一度}{いちど}、\ruby{火}{ひ}をつけたり、\ruby{消}{け}したりするんだからな。」

「\ruby{変}{へん}だなあ。一\ruby{分間}{ふんかん}が一\ruby{日}{にち}だなんて。」

「ちっとも\ruby{変}{へん}なことなんかないよ。\ruby{俺}{おれ}たちは、もう一\ruby{ヶ月}{かげつ}も\ruby{話}{はな}してるんだぜ。」と、\ruby{点灯}{てんとう}\ruby{人}{じん}が\ruby{言}{い}った。

「一\ruby{ヶ月}{かげつ}?」

「そうだよ。30\ruby{分}{ふん}、だから、30\ruby{日}{にち}さ。やあ、こんばんは。」\ruby{点灯}{てんとう}\ruby{人}{じん}は、また\ruby{街灯}{がいとう}に\ruby{火}{ひ}をつけた。

\ruby{王子}{おうじ}\ruby{様}{さま}は、\ruby{相手}{あいて}の\ruby{顔}{かお}をじっと\ruby{見}{み}た。そして、こんなにも\ruby{命令}{めいれい}をよく\ruby{守}{まも}る\ruby{点灯}{てんとう}\ruby{人}{じん}が\ruby{好}{す}きになった。\ruby{王子}{おうじ}\ruby{様}{さま}は、\ruby{夕暮}{ゆうぐ}れを\ruby{見}{み}たいとき、\ruby{自分}{じぶん}から\ruby{椅子}{いす}を\ruby{動}{うご}かしていたことを\ruby{思}{おも}い\ruby{出}{だ}した。

\ruby{王子}{おうじ}\ruby{様}{さま}は、この\ruby{友達}{ともだち}を\ruby{助}{たす}けたかった。

「ねえ、\ruby{休}{やすみ}みたい\ruby{時}{とき}に\ruby{休}{やす}めるこつ、\ruby{知}{し}ってるよ。」

「いつだって\ruby{休}{やす}みたいよ!」と、\ruby{点灯}{てんとう}\ruby{人}{じん}が\ruby{言}{い}った。

\ruby{人}{ひと}っていうのは、\ruby{真面目}{まじめ}にやってても、\ruby{怠}{なま}けたいものなんだ。

\ruby{王子}{おうじ}\ruby{様}{さま}は、\ruby{言葉}{ことば}を\ruby{続}{つづ}けた。

「\ruby{君}{きみ}の\ruby{星}{ほし}は、\ruby{本当}{ほんとう}に\ruby{小}{ちい}さいんだから、3\ruby{歩}{ほ}\ruby{歩}{ある}けば、ぐるりっと\ruby{回}{まわ}ってしまえるよ。\ruby{相当}{そうとう}ゆっくり\ruby{歩}{ある}いてさえいたら、しょっちゅうお\ruby{日様}{ひさま}を\ruby{眺}{なが}めていられるわけだよ。\ruby{休}{やす}みたくなったら、\ruby{歩}{ある}くんだな。そしたら、\ruby{君}{きみ}がほしいが\ruby{思}{おも}うだけ、\ruby{昼間}{ひるま}が\ruby{続}{つづ}くよ。」

「そうしたかたって、\ruby{俺}{おれ}は\ruby{大}{たい}して\ruby{助}{たす}からないなあ。\ruby{俺}{おれ}がこの\ruby{世}{よ}で\ruby{好}{す}きなのは、\ruby{眠}{ねむ}ることだよ。」

「そりゃ\ruby{困}{こま}ったね。」と、\ruby{王子}{おうじ}\ruby{様}{さま}は\ruby{言}{い}った。

「うん、\ruby{困}{こま}ったよ。やあ、おはよう。」そして、\ruby{点灯}{てんとう}\ruby{人}{じん}は、\ruby{街頭}{がいとう}の\ruby{火}{ひ}を\ruby{消}{け}した。

\ruby{王子}{おうじ}\ruby{様}{さま}は、ずっと\ruby{遠}{とお}くへ\ruby{旅}{たび}を\ruby{続}{つづ}けながら、こんなふうに\ruby{思}{おも}った。

あの\ruby{人}{ひと}、ほかのみんなから、\ruby{馬鹿}{ばか}にされるだろうな。\ruby{王様}{おうさま}、\ruby{自惚}{うぬぼ}れ\ruby{男}{おとこ}、\ruby{大酒飲}{おおざけの}み、\ruby{実業家}{じつぎょうか}から。でも、\ruby{僕}{ぼく}からしてみれば、たった\ruby{一人}{ひとり}、あの\ruby{人}{ひと}だけは、\ruby{変}{へん}だと\ruby{思}{おも}わなかった。それっていうのも、もしかすると、あの\ruby{人}{ひと}が\ruby{自分}{じぶん}じゃないことのために、あくせくしていたからかも。

\ruby{王子}{おうじ}\ruby{様}{さま}は\ruby{残念}{ざんねん}そうにため\ruby{息}{いき}をついた。さらに\ruby{考}{かんが}える。

たった\ruby{一人}{ひとり}、あの\ruby{人}{ひと}だけ、\ruby{僕}{ぼく}は\ruby{友達}{ともだち}になれると\ruby{思}{おも}った。でも、あの\ruby{人}{ひと}の\ruby{星}{ほし}は、\ruby{本当}{ほんとう}に\ruby{小}{ちい}さすぎて、\ruby{二人}{ふたり}も\ruby{入}{はい}らない…ただ、\ruby{王子}{おうじ}\ruby{様}{さま}としては、そうとは\ruby{思}{おも}いたくなかったんだけど、\ruby{実}{じつ}は、この\ruby{星}{ほし}のことも、\ruby{残念}{ざんねん}に\ruby{思}{おも}っていたんだ。だって、なんといっても、24\ruby{時間}{じかん}に1440\ruby{回}{かい}も\ruby{夕暮}{ゆうぐ}れが\ruby{見}{み}られるっていう、\ruby{恵}{めぐ}まれた\ruby{星}{ほし}なんだから。


