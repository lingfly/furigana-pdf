こうして\ruby{僕}{ぼく}は\ruby{二}{ふた}つ\ruby{目}{め}のとても\ruby{大切}{たいせつ}な\ruby{事}{こと}を\ruby{知}{し}った。\ruby{王子}{おうじ}さまのいた\ruby{星}{ほし}は、\ruby{家}{いえ}\ruby{一軒}{いっけん}よりやや\ruby{大}{おお}きいくらいの\ruby{大}{おお}きさなのだ。それほど\ruby{驚}{おどろ}きはしなかった。\ruby{地球}{ちきゅう}や\ruby{木星}{もくせい}、\ruby{火星}{かせい}、\ruby{金星}{きんせい}の\ruby{様}{よう}に、\ruby{名前}{なまえ}のある\ruby{巨大}{きょだい}な\ruby{星}{ほし}\ruby{以外}{いがい}にも、\ruby{望遠鏡}{ぼうえんきょう}でも\ruby{見}{み}つからないほど\ruby{小}{ちい}さな\ruby{星}{ほし}が、\ruby{何}{なに}百とあることを\ruby{知}{し}っていたからだ。\ruby{天文}{てんもん}\ruby{学者}{がくしゃ}がそんな\ruby{星}{ほし}を\ruby{発見}{はっけん}すると、\ruby{名前}{なまえ}の\ruby{替}{か}わりに\ruby{番号}{ばんごう}をつける。

\ruby{例}{たと}えば、\ruby{小}{しょう}\ruby{惑星}{わくせい}32 5 といった\ruby{様}{よう}に。\ruby{王子}{おうじ}さまがやってきた\ruby{星}{ほし}は、\ruby{小}{しょう}\ruby{惑星}{わくせい}B612だと\ruby{思}{おも}う。1909\ruby{年}{ねん}に、トルコの\ruby{天文}{てんもん}\ruby{学者}{がくしゃ}が\ruby{一度}{いちど}だけ\ruby{望遠鏡}{ぼうえんきょう}で\ruby{観測}{かんそく}した\ruby{星}{ほし}だ。\ruby{天文}{てんもん}\ruby{学者}{がくしゃ}は\ruby{国際}{こくさい}\ruby{天文学}{てんもんがく}\ruby{会}{かい}で、\ruby{自分}{じぶん}の\ruby{発見}{はっけん}について\ruby{堂々}{どうどう}と\ruby{発表}{はっぴょう}した。しかしその\ruby{時}{とき}は\ruby{服装}{ふくそう}のせいで、\ruby{誰}{だれ}にも\ruby{信}{しん}じてもらえなかった。\ruby{大人}{おとな}なんてそんなもんだ。しかし、\ruby{小}{しょう}\ruby{惑星}{わくせい}B612に\ruby{名誉}{めいよ}\ruby{挽回}{ばんかい}の\ruby{幸運}{こううん}が\ruby{訪}{おとず}れた。トルコの\ruby{独裁者}{どくさいしゃ}が\ruby{国民}{こくみん}にヨーロッパ\ruby{風}{ふう}の\ruby{服}{ふく}を\ruby{着}{き}るように\ruby{命令}{めいれい}し、\ruby{従}{したが}わなければ\ruby{死刑}{しけい}という\ruby{事}{こと}になったのだ。そこで\ruby{天文}{てんもん}\ruby{学者}{がくしゃ}は、1920\ruby{年}{ねん}、\ruby{今度}{こんど}はもっと\ruby{専念}{せんねん}された\ruby{服装}{ふくそう}で\ruby{同}{おな}じ\ruby{発表}{はっぴょう}を\ruby{繰}{く}り\ruby{返}{かえ}した。この\ruby{時}{とき}は\ruby{皆}{みな}が\ruby{彼}{かれ}の\ruby{言}{い}う\ruby{事}{こと}を\ruby{信}{しん}じた。

この\ruby{星}{ほし}の\ruby{事}{こと}をこんなに\ruby{詳}{くわ}しく\ruby{話}{はな}して、\ruby{番}{ばん}\ruby{号}{ごう}まで\ruby{教}{おし}えるのは、\ruby{大人}{おとな}たちのせいだ。\ruby{大人}{おとな}は\ruby{数字}{すうじ}が\ruby{好}{す}きだ。\ruby{数字}{すうじ}\ruby{以外}{いがい}には\ruby{興味}{きょうみ}がない。\ruby{新}{あたら}しい\ruby{友達}{ともだち}の\ruby{事}{こと}を\ruby{話}{はな}しても、どんな\ruby{声}{こえ}か、どんな\ruby{遊}{あそ}びが\ruby{好}{す}きか、ちょうちょう\ruby{集}{あつ}めているか、といった\ruby{大切}{たいせつ}な\ruby{事}{こと}は\ruby{何}{なに}も\ruby{聞}{き}いてこない。\ruby{何歳}{なんさい}か、\ruby{何人}{なんにん}\ruby{兄弟}{きょうだい}か、お\ruby{父}{とう}さんの\ruby{年収}{ねんしゅう}はいくらか、といった\ruby{数字}{すうじ}のことばかり\ruby{聞}{き}いてきて、それですっかり\ruby{知}{し}ったつもりになる。

\ruby{王子}{おうじ}さまは\ruby{本当}{ほんとう}にいたよ。\ruby{可愛}{かわい}かったし、\ruby{笑}{わら}っていたし、\ruby{羊}{ひつじ}を\ruby{欲}{ほ}しがっていた。だって、\ruby{羊}{ひつじ}を\ruby{欲}{ほ}しがるって\ruby{事}{こと}は、\ruby{間違}{まちが}えなくその\ruby{人}{ひと}が\ruby{本当}{ほんとう}にいるって\ruby{事}{こと}の\ruby{証拠}{しょうこ}だからね。

こんなふうに\ruby{話}{はな}しても、\ruby{大人}{おとな}は\ruby{肩}{かた}を\ruby{竦}{すく}め、\ruby{子供扱}{こどもあつか}いするだけだ。しかし、\ruby{王子}{おうじ}さまが\ruby{来}{き}た\ruby{星}{ほし}は\ruby{小}{しょう}\ruby{惑星}{わくせい}B612だよ、と\ruby{言}{い}えば、\ruby{大人}{おとな}は\ruby{納得}{なっとく}して、それ\ruby{以上}{いじょう}\ruby{余計}{よけい}な\ruby{事}{こと}は\ruby{聞}{き}いてこない。

\ruby{大人}{おとな}なんてそんなもんだ。でも、\ruby{悪}{わる}く\ruby{思}{おも}ってはいけないよ。\ruby{子供}{こども}は\ruby{大人}{おとな}に\ruby{対}{たい}して、\ruby{広}{ひろ}い\ruby{心}{こころ}で\ruby{接}{せっ}してあげなきゃね。でも、\ruby{生}{い}きるという\ruby{事}{こと}がどういう\ruby{事}{こと}なのか、よくわかっている\ruby{僕}{ぼく}たちには、\ruby{数字}{すうじ}なんかどうでもいい。

\ruby{本当}{ほんとう}だったら\ruby{僕}{ぼく}は、この\ruby{物語}{ものがたり}をお\ruby{伽話}{とぎばなし}のように\ruby{始}{はじ}めたかった。\ruby{昔々}{むかしむかし}、\ruby{自分}{じぶん}よりほんの\ruby{少}{すこ}し\ruby{大}{おお}きいだけの\ruby{星}{ほし}に\ruby{暮}{く}らしている\ruby{小}{ちい}さな\ruby{王子}{おうじ}さまがいました。\ruby{王子}{おうじ}さまは\ruby{友達}{ともだち}をほしがっていました。\ruby{生}{い}きるという\ruby{事}{こと}がどういう\ruby{事}{こと}なのかわかっている\ruby{人}{ひと}には、こういう\ruby{言}{い}い\ruby{方}{かた}のほうがずっと\ruby{本当}{ほんとう}らしく\ruby{聞}{き}こえるだろう。\ruby{僕}{ぼく}はこの\ruby{本}{ほん}を\ruby{軽々}{かるがる}しく\ruby{読}{よ}まれたくない。こういった\ruby{思}{おも}い\ruby{出話}{でばなし}を\ruby{語}{かた}る\ruby{事}{こと}は、\ruby{僕}{ぼく}にとって\ruby{本当}{ほんとう}に\ruby{辛}{から}い。\ruby{僕}{ぼく}の\ruby{友達}{ともだち}が\ruby{羊}{ひつじ}を\ruby{連}{つ}れていってしまって、もう6\ruby{年}{ねん}になる。こうして\ruby{彼}{かれ}の\ruby{事}{こと}を\ruby{書}{か}くのは、\ruby{彼}{かれ}を\ruby{忘}{わす}れないためだ。\ruby{友達}{ともだち}を\ruby{忘}{わす}れてしまうのは\ruby{悲}{かな}しい、\ruby{誰}{だれ}にでも\ruby{友達}{ともだち}がいるわけではない。それに、\ruby{僕}{ぼく}も\ruby{数字}{すうじ}にしか\ruby{興味}{きょうみ}のない\ruby{大人}{おとな}になってしまうかもしれない。そうならないために\ruby{僕}{ぼく}は、\ruby{絵}{え}の\ruby{具}{ぐ}\ruby{箱}{はこ}と\ruby{鉛筆}{えんぴつ}を\ruby{買}{か}った。6\ruby{歳}{さい}でボアの\ruby{外側}{そとがわ}と\ruby{内側}{うちがわ}を\ruby{描}{えが}いて\ruby{以来}{いらい}、\ruby{何}{なに}も\ruby{描}{えが}いていなかった\ruby{僕}{ぼく}にとって、この\ruby{年}{とし}でもう\ruby{一度}{いちど}\ruby{絵}{え}を\ruby{描}{か}くのは\ruby{大変}{たいへん}な\ruby{事}{こと}だった。できるだけ、\ruby{本物}{ほんもの}そっくりな\ruby{肖像画}{しょうぞうが}を\ruby{描}{えが}いてみるつもりだ。

でも、ちゃんと\ruby{描}{えが}けるかどうかは、\ruby{自信}{じしん}がない。\ruby{一枚}{いちまい}いいものが\ruby{描}{えが}けても、その\ruby{次}{つぎ}はまるで\ruby{似}{に}ていないかもしれない。\ruby{背丈}{せたけ}が\ruby{難}{むずか}しいし、\ruby{服}{ふく}の\ruby{色}{いろ}も\ruby{迷}{まよ}ってしまう。\ruby{手探}{てさぐ}りでやってみるが、もっと\ruby{大事}{だいじ}な\ruby{細}{こま}かい\ruby{部分}{ぶぶん}を\ruby{間違}{まちが}えてしまうかもしれない。でも、そこは\ruby{大目}{おおめ}に\ruby{見}{み}てほしい。\ruby{王子}{おうじ}さまは\ruby{詳}{くわ}しい\ruby{事}{こと}は\ruby{何}{なに}も\ruby{説明}{せつめい}してくれなかったのだ。おそらく\ruby{彼}{かれ}は\ruby{僕}{ぼく}の\ruby{事}{こと}を\ruby{自分}{じぶん}と\ruby{同}{おな}じ\ruby{仲間}{なかま}だと\ruby{思}{おも}ったのだろう。しかし\ruby{残念}{ざんねん}ながら\ruby{僕}{ぼく}は、\ruby{箱}{はこ}の\ruby{中}{なか}の\ruby{羊}{ひつじ}を\ruby{見}{み}る\ruby{事}{こと}ができない。\ruby{少}{すこ}しばかり\ruby{大人}{おとな}になってしまったのかもしれない。\ruby{年}{とし}を\ruby{取}{と}ったのだ。


