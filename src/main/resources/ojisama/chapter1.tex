% !TeX encoding = UTF-8
% !TeX program = LuaLaTeX

\documentclass[18pt,a4paper,oneside,openany]{book}
% \raggedright
%\pagestyle{headings}
\setlength{\parindent}{2em}


\usepackage{luatexja-ruby}
\usepackage{indentfirst}

\ltjsetruby{size=0.6}            %设置振假名字号
%\ltjsetruby{fontcmd=\gtfamily}  %设置振假名字体
\ltjsetruby{mode=00}             %设置振假名的「進入」和「突出」模式

\begin{document}


\tableofcontents


\chapter{六歳の時}
\markright{六歳の時}


%\gtfamily                       %设置主体文字字体
%\normalsize
\linespread{1.8}
\selectfont

六\ruby{歳}{さい}の\ruby{時}{とき}\ruby{僕}{ぼく}は、「\ruby{体験談}{たいけんだん}」という\ruby{原生林}{げんせいりん}について\ruby{書}{か}かれた\ruby{本}{ほん}で、\ruby{素晴}{すば}らしい\ruby{挿絵}{さしえ}を\ruby{見}{み}たことがある。それは\ruby{大蛇}{だいじゃ}の\ruby{ボア}{ぼあ}が\ruby{猛獣}{もうじゅう}を\ruby{飲}{の}み\ruby{込}{こ}もうとしている\ruby{絵}{え}だった。\ruby{本}{ほん}にはこんな\ruby{説明}{せつめい}があった。

\ruby{ボア}{ぼあ}は\ruby{獲物}{えもの}を\ruby{噛}{か}まずに\ruby{丸}{まる}ごと\ruby{飲}{の}み\ruby{込}{こ}みます。すると\ruby{動}{うご}けなくなるので、\ruby{獲物}{えもの}を\ruby{消化}{しょうか}する\ruby{半年}{はんとし}もの\ruby{間}{かん}、ずっと\ruby{眠}{ねむ}って\ruby{過}{す}ごします。

\ruby{僕}{ぼく}は\ruby{ジャングル}{じゃんぐる}での\ruby{冒険}{ぼうけん}についていろいろと\ruby{考}{かんが}え、\ruby{自分}{じぶん}でも\ruby{色鉛筆}{いろえんぴつ}を\ruby{使}{つか}って、\ruby{生}{う}まれて\ruby{初}{はじ}めての\ruby{絵}{え}を\ruby{描}{か}き\ruby{上}{あ}げた。その\ruby{傑作}{けっさく}を\ruby{大人}{おとな}たちに\ruby{見}{み}せ、\ruby{怖}{こわ}いかどうか\ruby{聞}{き}いてみた。すると、こんな\ruby{答}{こた}えが\ruby{返}{かえ}ってきた。

どうして\ruby{帽子}{ぼうし}が\ruby{怖}{こわ}いんだい?

\ruby{帽子}{ぼうし}の\ruby{絵}{え}なんかじゃなかった。\ruby{ゾウ}{ぞう}を\ruby{消化}{しょうか}している\ruby{ボア}{ぼあ}を\ruby{描}{えが}いたのだ。でも、\ruby{大人}{おとな}にはわからないらしいので、\ruby{今度}{こんど}は\ruby{ボア}{ぼあ}の\ruby{内側}{うちがわ}の\ruby{絵}{え}を\ruby{描}{か}いてみた。\ruby{大人}{おとな}には\ruby{何時}{なんじ}だって\ruby{説明}{せつめい}が\ruby{必要}{ひつよう}なのだ。\ruby{僕}{ぼく}の\ruby{二番目}{にばんめ}の\ruby{絵}{え}では、ちゃんと\ruby{ボア}{ぼあ}の\ruby{中}{なか}にいる\ruby{ゾウ}{ぞう}が\ruby{見}{み}えていた。しかし\ruby{大人}{おとな}たちは\ruby{中}{なか}が\ruby{見}{み}えようが\ruby{見}{み}えまいが、\ruby{ボア}{ぼあ}の\ruby{絵}{え}は\ruby{片付}{かたづ}けて、\ruby{地理}{ちり}や\ruby{歴史}{れきし}、\ruby{算数}{さんすう}や\ruby{文法}{ぶんぽう}の\ruby{勉強}{べんきょう}をしなさいと、\ruby{僕}{ぼく}を\ruby{嗜}{たしな}めた。

こうして、6\ruby{歳}{さい}にして\ruby{僕}{ぼく}は\ruby{偉大}{いだい}な\ruby{画家}{がか}になるという\ruby{夢}{ゆめ}を\ruby{諦}{あきら}めた。\ruby{作品}{さくひん}\ruby{第}{だい}一\ruby{号}{ごう}と\ruby{第}{だい}二\ruby{号}{ごう}が\ruby{共}{とも}に\ruby{不評}{ふひょう}で、\ruby{気持}{きも}ちが\ruby{挫}{くじ}けてしまったのだ。

\ruby{大人}{おとな}というのは、\ruby{自分}{じぶん}たちとは\ruby{全}{まった}く\ruby{何}{なに}もわかっていないから、いつも\ruby{子供}{こども}の\ruby{方}{ほう}から\ruby{説明}{せつめい}してあげなきゃいけなくて、うんざりする。\ruby{僕}{ぼく}は\ruby{別}{べつ}の\ruby{仕事}{しごと}を\ruby{選}{えら}ぶ\ruby{必要}{ひつよう}に\ruby{迫}{せま}られて、\ruby{飛行機}{ひこうき}の\ruby{操縦士}{そうじゅうし}になった。そして、\ruby{世界}{せかい}\ruby{中}{ちゅう}をあちこち\ruby{飛}{と}び\ruby{回}{まわ}った。\ruby{地理}{ちり}は\ruby{確}{たし}かに\ruby{役}{やく}に\ruby{立}{た}った。\ruby{僕}{ぼく}は\ruby{一目}{ひとめ}で\ruby{中国}{ちゅうごく}と\ruby{アリゾナ}{ありぞな}を\ruby{見分}{みわ}ける\ruby{事}{こと}ができる。\ruby{夜間飛行}{やかんひこう}で\ruby{迷}{まよ}った\ruby{時}{とき}など、そういう\ruby{知識}{ちしき}があると\ruby{本当}{ほんとう}に\ruby{助}{たす}かる。

これまでの\ruby{人生}{じんせい}で、\ruby{僕}{ぼく}はたくさんの\ruby{重要}{じゅうよう}\ruby{人物}{じんぶつ}と\ruby{知}{し}り\ruby{合}{あ}った。\ruby{随分}{ずいぶん}\ruby{多}{おお}くの\ruby{大人}{おとな}たちと\ruby{一緒}{いっしょ}に\ruby{暮}{く}らしたし、\ruby{マジカ}{まじか}にも\ruby{見}{み}てきた。それでも\ruby{僕}{ぼく}の\ruby{考}{かんが}えはあまり\ruby{変}{か}わらなかった。\ruby{僕}{ぼく}は\ruby{物分}{ものわか}りのよさそうな\ruby{人}{ひと}に\ruby{出会}{であ}った\ruby{時}{とき}には\ruby{必}{かなら}ず、\ruby{肌}{はだ}に\ruby{離}{はな}さず\ruby{持}{も}ち\ruby{歩}{ある}いていた\ruby{作品}{さくひん}\ruby{第}{だい}一\ruby{号}{ごう}を\ruby{見}{み}せ、\ruby{実験}{じっけん}していた。その\ruby{人}{ひと}が\ruby{本当}{ほんとう}に\ruby{物事}{ものごと}の\ruby{分}{わ}かる\ruby{人}{ひと}かどうか、\ruby{知}{し}りたかったから。でも、\ruby{答}{こた}えはいつも\ruby{同}{おな}じだった。

\ruby{帽子}{ぼうし}だね。

その\ruby{後}{あと}\ruby{僕}{ぼく}は\ruby{ボア}{ぼあ}の\ruby{話}{はなし}も、\ruby{原生林}{げんせいりん}の\ruby{話}{はなし}も、\ruby{星}{ほし}の\ruby{話}{はなし}もしなかった。\ruby{話}{はなし}を\ruby{合}{あ}わせて、\ruby{ブリッジ}{ぶりっじ}や\ruby{ゴルフ}{ごるふ}や、\ruby{政治}{せいじ}や\ruby{ネクタイ}{ねくたい}の\ruby{話}{はなし}をした。するとその\ruby{大人}{おとな}は\ruby{話}{はなし}が\ruby{分}{わ}かる\ruby{相手}{あいて}と\ruby{知}{し}り\ruby{合}{あ}えたと\ruby{言}{い}って\ruby{喜}{よろこ}ぶのだ。

\chapter{Chapter2}

萨芬奋斗奋斗

\end{document}

