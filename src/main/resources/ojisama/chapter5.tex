\ruby{日}{ひ}を\ruby{追}{お}うごとに\ruby{僕}{ぼく}は\ruby{王子}{おうじ}さまの\ruby{星}{ほし}の\ruby{事}{こと}や、そこからの\ruby{旅立}{たびだ}ち、これまでの\ruby{旅}{たび}について\ruby{知}{し}るようになっていった。\ruby{王子}{おうじ}さまが\ruby{偶々}{たまたま}\ruby{口}{くち}にした\ruby{言葉}{ことば}で、\ruby{少}{すこ}しずつ\ruby{様子}{ようす}がわかってきた。こうして\ruby{三日目}{みっかめ}に、バオバブをめぐる\ruby{大}{だい}\ruby{騒動}{そうどう}を\ruby{知}{し}った。これも\ruby{羊}{ひつじ}のおかげだった。\ruby{王子}{おうじ}さまが\ruby{急}{きゅう}に\ruby{心配}{しんぱい}になったらしくて、こう\ruby{聞}{き}いてきたのだ。

\ruby{羊}{ひつじ}が\ruby{小}{ちい}さな\ruby{木}{き}も\ruby{食}{た}べるって、\ruby{本当}{ほんとう}なんでしょう?

うん、\ruby{本当}{ほんとう}だよ。

ああ、よかった。

\ruby{羊}{ひつじ}が\ruby{小}{ちい}さな\ruby{木}{き}を\ruby{食}{た}べる\ruby{事}{こと}がなぜそんなに\ruby{大事}{だいじ}な\ruby{事}{こと}なのか、\ruby{僕}{ぼく}にはわからなかった。しかし、\ruby{王子}{おうじ}さまは\ruby{更}{さら}にこう\ruby{聞}{き}いてきた。

だったら、バオバブも\ruby{食}{た}べるよね。

\ruby{僕}{ぼく}は\ruby{王子}{おうじ}さまにバオバブは\ruby{小}{ちい}さな\ruby{木}{き}じゃなくて、\ruby{教会}{きょうかい}の\ruby{建物}{たてもの}と\ruby{同}{おな}じくらい\ruby{大}{おお}きな\ruby{木}{き}だから、ゾウの\ruby{群}{む}れを\ruby{丸}{まる}ごと\ruby{連}{つ}れてきても、たった一\ruby{本}{ほん}のバオバブも\ruby{食}{た}べきれないだろうと\ruby{教}{おし}えてあげた。ゾウの\ruby{群}{む}れを\ruby{思}{おも}い\ruby{描}{えが}いて、\ruby{王子}{おうじ}さまは\ruby{笑}{わら}った。

\ruby{上}{うえ}に\ruby{上}{うえ}に\ruby{積}{つ}み\ruby{重}{かさ}ねなきゃいけないね。

しかし、\ruby{続}{つづ}けてなかなか\ruby{鋭}{するど}い\ruby{指摘}{してき}をした。

バオバブだって、\ruby{大}{おお}きくなる\ruby{前}{まえ}は、\ruby{小}{ちい}さいんだよね。

そりゃそうだよ。それにしても、どうして\ruby{羊}{ひつじ}に\ruby{小}{ちい}さなバオバブを\ruby{食}{た}べてもらいたいんだい?

\ruby{何}{なに}を\ruby{言}{い}ってるの?そんなの\ruby{当}{あ}たり\ruby{前}{まえ}でしょう。

\ruby{僕}{ぼく}は\ruby{一人}{ひとり}でこの\ruby{難問}{なんもん}を\ruby{解}{と}き\ruby{明}{あ}かす\ruby{事}{こと}になり、\ruby{散々}{さんざん}\ruby{頭}{あたま}を\ruby{捻}{ひね}った。つまり、こういう\ruby{事}{こと}だ。\ruby{王子}{おうじ}さまの\ruby{星}{ほし}には、\ruby{他}{ほか}の\ruby{星}{ほし}と\ruby{同}{おな}じように、よい\ruby{草}{くさ}と\ruby{悪}{わる}い\ruby{草}{くさ}があった。よい\ruby{草}{くさ}はよい\ruby{種}{たね}から\ruby{育}{そだ}ち、\ruby{悪}{わる}い\ruby{草}{くさ}は\ruby{悪}{わる}い\ruby{種}{たね}から\ruby{育}{そだ}つ。しかし、\ruby{種}{たね}は\ruby{目}{め}に\ruby{見}{み}えない。\ruby{土}{つち}の\ruby{中}{なか}でひっそりと\ruby{眠}{ねむ}っている。その\ruby{一}{ひと}つが\ruby{気}{き}まぐれに\ruby{目}{め}を\ruby{覚}{さ}ますと、\ruby{伸}{の}びをして、おずおずとあどけない\ruby{小}{ちい}さな\ruby{茎}{くき}を\ruby{太陽}{たいよう}に\ruby{向}{む}かって\ruby{伸}{の}ばし\ruby{始}{はじ}める。それが \ruby{赤蕪}{あかかぶ} やバラだったら、そのままにしておいて\ruby{構}{かま}わない。でも、\ruby{悪}{わる}い\ruby{草}{くさ}だと\ruby{分}{わ}かったら、すぐに\ruby{抜}{ぬ}き\ruby{取}{と}らなくてはいけない。\ruby{王子}{おうじ}さまの\ruby{星}{ほし}には、そんな\ruby{恐}{おそ}ろしい\ruby{種}{たね}があった。バオバブの\ruby{種}{たね}だ。\ruby{星}{ほし}の\ruby{土}{つち}はどこもかしこもバオバブの\ruby{種}{たね}だらけだった。\ruby{少}{すこ}しでも\ruby{抜}{ぬ}くのが\ruby{遅}{おく}れると、バオバブはもう\ruby{手}{て}がつけられなくなる。\ruby{星}{ほし}\ruby{全体}{ぜんたい}を\ruby{覆}{おお}いつくし、\ruby{根}{ね}っ\ruby{子}{こ}がつき\ruby{抜}{ぬ}け、\ruby{穴}{あな}を\ruby{開}{あ}けてしまう。\ruby{小}{ちい}さな\ruby{星}{ほし}だと\ruby{殖}{ふえ}\ruby{過}{す}ぎたバオバブで\ruby{破裂}{はれつ}してしまう。

\ruby{決}{き}まりにできるかどうかだね。\ruby{毎朝}{まいあさ}、\ruby{自分}{じぶん}の\ruby{身支度}{みじたく}が\ruby{済}{す}んだら、\ruby{星}{ほし}の\ruby{手入}{てい}れに\ruby{取}{と}り\ruby{掛}{か}かる。

\ruby{芽}{め}を\ruby{出}{だ}したばかりのバラとバオバブはよく\ruby{似}{に}ているんだけど、それを\ruby{見分}{みわ}けて、バオバブだと\ruby{分}{わ}かったら、すぐに\ruby{抜}{ぬ}いてしまう。\ruby{手間}{てま}はかかるけど、とっても\ruby{簡単}{かんたん}な\ruby{事}{こと}だよ。\ruby{偶}{ぐう}には\ruby{仕事}{しごと}を\ruby{後回}{あとまわ}しにしても\ruby{大丈夫}{だいじょうぶ}な\ruby{時}{とき}ってあるけど、バオバブでそんな\ruby{事}{こと}をしたら、\ruby{取}{と}り\ruby{返}{かえ}しがつかなくなるんだ。\ruby{例}{たと}えばね、ある\ruby{星}{ほし}に\ruby{怠}{なま}け\ruby{者}{もの}が\ruby{住}{す}んでいたんだけど、その\ruby{人}{ひと}は三本\ruby{さんぼん}のバオバブをほったらかしにしていたばかりに……\ruby{僕}{ぼく}は\ruby{王子}{おうじ}さまの\ruby{話}{はな}す\ruby{通}{とお}りにその\ruby{星}{ほし}の\ruby{絵}{え}を\ruby{描}{か}いた。\ruby{星}{ほし}より\ruby{巨大}{きょだい}な三\ruby{本}{ほん}のバオバブと\ruby{途方}{とほう}に\ruby{暮}{く}れる\ruby{怠}{なま}け\ruby{者}{もの}、お\ruby{説教}{せっきょう}\ruby{臭}{くさ}い\ruby{事}{こと}を\ruby{言}{い}うのはあまり\ruby{好}{す}きじゃないけれど、バオバブの\ruby{脅威}{きょうい}は\ruby{地球}{ちきゅう}ではほとんど\ruby{知}{し}られていないし、\ruby{小}{しょう}\ruby{惑星}{わくせい}で\ruby{道}{みち}に\ruby{迷}{まよ}った\ruby{人}{ひと}が\ruby{危険}{きけん}な\ruby{目}{め}に\ruby{遭}{あ}う\ruby{可能性}{かのうせい}は、あまりにも\ruby{大}{おお}きい。だから\ruby{僕}{ぼく}は\ruby{一度}{いちど}だけ\ruby{普段}{ふだん}の\ruby{慎}{つつし}みを\ruby{忘}{わす}れて、こう\ruby{言}{い}っておこう。

おーい、\ruby{子供}{こども}たち、バオバブに\ruby{気}{き}をつけろ。

\ruby{僕}{ぼく}は\ruby{友人}{ゆうじん}たちに\ruby{警告}{けいこく}を\ruby{与}{あた}えるために、\ruby{一生懸命}{いっしょうけんめい}この\ruby{絵}{え}を\ruby{仕上}{しあ}げた。\ruby{苦労}{くろう}して\ruby{描}{えが}いた\ruby{価値}{かち}はあった。\ruby{他}{た}はこれほどうまくいかなかった。バオバブを\ruby{描}{えが}いた\ruby{時}{とき}は、\ruby{切羽詰}{せっぱつま}って\ruby{気持}{きも}ちが\ruby{高}{たか}ぶっていたのだ。


