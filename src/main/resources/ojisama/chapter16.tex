そんなわけで、七\ruby{番目}{ばんめ}の\ruby{星}{ほし}が\ruby{地球}{ちきゅう}だった。この\ruby{地球}{ちきゅう}というのは、どこにでもある\ruby{星}{ほし}なんかじゃない。\ruby{数}{かぞ}えてみると、\ruby{王様}{おうさま}が(もちろん、\ruby{黒}{くろ}い\ruby{顔}{かお}の\ruby{王様}{おうさま}も\ruby{入}{い}れて)110\ruby{人}{にん}、\ruby{地理}{ちり}\ruby{学者}{がくしゃ}が7000\ruby{人}{にん}、\ruby{実業家}{じつぎょうか}が90\ruby{万}{まん}\ruby{人}{にん}、\ruby{大酒飲}{おおざけの}みが750\ruby{万}{まん}\ruby{人}{にん}、\ruby{自惚}{うぬぼ}れが3\ruby{億}{おく}1千100\ruby{万}{まん}\ruby{人}{にん}で、\ruby{合}{あ}わせて\ruby{大体}{だいたい}20\ruby{億}{おく}の\ruby{大人}{おとな}の\ruby{人}{ひと}がいる。

\ruby{地球}{ちきゅう}の\ruby{大}{おお}きさをわかりやすくする、こんな\ruby{話}{はなし}がある。\ruby{電気}{でんき}が\ruby{使}{つか}われるまでは、\ruby{六}{むっ}つの\ruby{大陸}{たいりく}ひっくるめて、なんと、46\ruby{万}{まん}2511\ruby{人}{にん}もの、\ruby{大勢}{おおぜい}の\ruby{点灯}{てんとう}\ruby{人}{ひと}がいなきゃならなかった。

\ruby{遠}{とお}くから\ruby{眺}{なが}めると、\ruby{大変}{たいへん}\ruby{見}{み}ものだ。この\ruby{大勢}{おおぜい}の\ruby{動}{うご}きは、バレーのダンサーみたいに、きちっきちっとしていた。

まずはニュージーランドとオーストラリアの\ruby{点灯}{てんとう}\ruby{人}{じん}の\ruby{出番}{でばん}が\ruby{来}{く}る。そこで、\ruby{自分}{じぶん}のランプをつけると、この\ruby{人}{ひと}たちは\ruby{眠}{ねむ}りにつく。すると、\ruby{次}{つぎ}は\ruby{中国}{ちゅうごく}とシベリアの\ruby{番}{ばん}が\ruby{来}{き}て、この\ruby{動}{うご}きに\ruby{加}{くわ}わって、\ruby{終}{お}わると、\ruby{裏}{うら}に\ruby{引}{ひ}っ\ruby{込}{こ}む。それから、ロシアとインドの\ruby{点灯}{てんとう}\ruby{人}{じん}の\ruby{番}{ばん}になる。

\ruby{次}{つぎ}はアフリカとヨーロッパ。それから\ruby{南}{みなみ}アメリカ、それから\ruby{北}{きた}\ruby{アメリカ}{あめりか}。しかも、この\ruby{人}{ひと}たちは、\ruby{自分}{じぶん}の\ruby{出}{で}る\ruby{順}{じゅん}を、\ruby{絶対}{ぜったい}に\ruby{間違}{まちが}えない。

でも、\ruby{北極}{ほっきょく}にひとつだけ、\ruby{南極}{なんきょく}にもひとつだけ、\ruby{街灯}{がいとう}があるんだけど、そこの\ruby{二人}{ふたり}の\ruby{点灯}{てんとう}\ruby{人}{じん}は、のんべんだらりとした\ruby{毎}{まい}\ruby{日}{ひ}を\ruby{送}{おく}っていた。だって、一\ruby{年}{ねん}に二\ruby{回}{かい}、\ruby{働}{はたら}くだけでいいんだから。


