% !TeX encoding = UTF-8
% !TeX program = LuaLaTeX

\documentclass[12pt,a4paper,oneside,openany]{book}

\usepackage{luatexja-ruby}       %假名标注
\usepackage{xcolor}			 %引入颜色
\usepackage[hidelinks]{hyperref} %给目录添加超链接
\usepackage{titlesec}
\usepackage{indentfirst}         %章节首页首行缩进

% 自定义章节标题样式,调整默认垂直间距
\titleformat{\chapter}       	 % 要格式化的章节命令,如 \chapter、\section、\subsection 等。
[hang]                       	 % 标题的形状,可以是 hang(悬挂格式,默认)、block(块格式)
{\normalfont\Huge\bfseries}   	 % format,标题的整体格式,可以包括字体、大小、粗细等
{Chapter\thechapter}          	 % label,标题编号的格式,如 \thechapter、\thesection 等。
{1em}                         	 % Spacing between label and title
{}  

\headheight = 13pt           	 % 页眉高度
\headsep = 30pt		        	 % 页眉和正文的间距
\topmargin = -30pt               % 页眉和页面顶端的间距
\titlespacing{\chapter}{0pt}{-60pt}{10pt} % 调整标题间距
\textheight = 700pt              % 正文高度
\textwidth = 450pt 			 % 正文宽度
\setlength{\hoffset}{-20pt}      % 正文向左偏移20pt

\ltjsetruby{size=0.6}            %设置振假名字号
%\ltjsetruby{fontcmd=\gtfamily}  %设置振假名字体
\ltjsetruby{mode=00}             %设置振假名的「進入」和「突出」模式

\setlength{\parindent}{2em}	 %首行缩进
\marginparwidth = 72pt           %边栏宽度

\linespread{1.8}			 %行距
\selectfont

% 边注
\newcounter{num}[chapter]
\newcommand{\translate}[2]{\addtocounter{num}{1} {\color{orange} #1\textsuperscript{\scriptsize \thenum}}\marginpar{\scriptsize \textsuperscript{\scriptsize \thenum} #2}}

\begin{document}
\tableofcontents

\chapter{序章}
六\ruby{歳}{さい}の\ruby{時}{とき}\ruby{僕}{ぼく}は、「\ruby{体験談}{たいけんだん}」という\ruby{原生林}{げんせいりん}について\ruby{書}{か}かれた\ruby{本}{ほん}で、\ruby{素晴}{すば}らしい\ruby{挿絵}{さしえ}を\ruby{見}{み}たことがある。それは\ruby{大蛇}{だいじゃ}のボアが\ruby{猛獣}{もうじゅう}を\ruby{飲}{の}み\ruby{込}{こ}もうとしている\ruby{絵}{え}だった。\ruby{本}{ほん}にはこんな\ruby{説明}{せつめい}があった。


ボアは\ruby{獲物}{えもの}を\ruby{噛}{か}まずに\ruby{丸}{まる}ごと\ruby{飲}{の}み\ruby{込}{こ}みます。すると\ruby{動}{うご}けなくなるので、\ruby{獲物}{えもの}を\ruby{消化}{しょうか}する\ruby{半年}{はんとし}もの\ruby{間}{あいだ}、ずっと\ruby{眠}{ねむ}って\ruby{過}{す}ごします。

\ruby{僕}{ぼく}はジャングルでの\ruby{冒険}{ぼうけん}についていろいろと\ruby{考}{かんが}え、\ruby{自分}{じぶん}でも\ruby{色鉛筆}{いろえんぴつ}を\ruby{使}{つか}って、\ruby{生}{う}まれて\ruby{初}{はじ}めての\ruby{絵}{え}を\ruby{描}{か}き\ruby{上}{あ}げた。その\ruby{傑作}{けっさく}を\ruby{大人}{おとな}たちに\ruby{見}{み}せ、\ruby{怖}{こわ}いかどうか\ruby{聞}{き}いてみた。すると、こんな\ruby{答}{こた}えが\ruby{返}{かえ}ってきた。

どうして\ruby{帽子}{ぼうし}が\ruby{怖}{こわ}いんだい?

\ruby{帽子}{ぼうし}の\ruby{絵}{え}なんかじゃなかった。ゾウを\ruby{消化}{しょうか}しているボアを\ruby{描}{えが}いたのだ。でも、\ruby{大人}{おとな}にはわからないらしいので、\ruby{今度}{こんど}はボアの\ruby{内側}{うちがわ}の\ruby{絵}{え}を\ruby{描}{か}いてみた。\ruby{大人}{おとな}には\ruby{何時}{なんじ}だって\ruby{説明}{せつめい}が\ruby{必要}{ひつよう}なのだ。\ruby{僕}{ぼく}の\ruby{二番目}{にばんめ}の\ruby{絵}{え}では、ちゃんとボアの\ruby{中}{なか}にいるゾウが\ruby{見}{み}えていた。しかし\ruby{大人}{おとな}たちは\ruby{中}{なか}が\ruby{見}{み}えようが\ruby{見}{み}えまいが、ボアの\ruby{絵}{え}は\ruby{片付}{かたづ}けて、\ruby{地理}{ちり}や\ruby{歴史}{れきし}、\ruby{算数}{さんすう}や\ruby{文法}{ぶんぽう}の\ruby{勉強}{べんきょう}をしなさいと、\ruby{僕}{ぼく}を\ruby{嗜}{たしな}めた。

こうして、6\ruby{歳}{さい}にして\ruby{僕}{ぼく}は\ruby{偉大}{いだい}な\ruby{画家}{がか}になるという\ruby{夢}{ゆめ}を\ruby{諦}{あきら}めた。\ruby{作品}{さくひん}\ruby{第}{だい}一\ruby{号}{ごう}と\ruby{第}{だい}二\ruby{号}{ごう}が\ruby{共}{とも}に\ruby{不評}{ふひょう}で、\ruby{気持}{きも}ちが\ruby{挫}{くじ}けてしまったのだ。

\ruby{大人}{おとな}というのは、\ruby{自分}{じぶん}たちとは\ruby{全}{まった}く\ruby{何}{なに}もわかっていないから、いつも\ruby{子供}{こども}の\ruby{方}{ほう}から\ruby{説明}{せつめい}してあげなきゃいけなくて、うんざりする。\ruby{僕}{ぼく}は\ruby{別}{べつ}の\ruby{仕事}{しごと}を\ruby{選}{えら}ぶ\ruby{必要}{ひつよう}に\ruby{迫}{せま}られて、\ruby{飛行機}{ひこうき}の\ruby{操縦士}{そうじゅうし}になった。そして、\ruby{世界}{せかい}\ruby{中}{じゅう}をあちこち\ruby{飛}{と}び\ruby{回}{まわ}った。\ruby{地理}{ちり}は\ruby{確}{たし}かに\ruby{役}{やく}に\ruby{立}{た}った。\ruby{僕}{ぼく}は\ruby{一目}{ひとめ}で\ruby{中国}{ちゅうごく}とアリゾナを\ruby{見分}{みわ}ける\ruby{事}{こと}ができる。\ruby{夜間飛行}{やかんひこう}で\ruby{迷}{まよ}った\ruby{時}{とき}など、そういう\ruby{知識}{ちしき}があると\ruby{本当}{ほんとう}に\ruby{助}{たす}かる。

これまでの\ruby{人生}{じんせい}で、\ruby{僕}{ぼく}はたくさんの\ruby{重要}{じゅうよう}\ruby{人物}{じんぶつ}と\ruby{知}{し}り\ruby{合}{あ}った。\ruby{随分}{ずいぶん}\ruby{多}{おお}くの\ruby{大人}{おとな}たちと\ruby{一緒}{いっしょ}に\ruby{暮}{く}らしたし、マジカにも\ruby{見}{み}てきた。それでも\ruby{僕}{ぼく}の\ruby{考}{かんが}えはあまり\ruby{変}{か}わらなかった。\ruby{僕}{ぼく}は\ruby{物分}{ものわか}りのよさそうな\ruby{人}{ひと}に\ruby{出会}{であ}った\ruby{時}{とき}には\ruby{必}{かなら}ず、\ruby{肌}{はだ}に\ruby{離}{はな}さず\ruby{持}{も}ち\ruby{歩}{ある}いていた\ruby{作品}{さくひん}\ruby{第}{だい}一\ruby{号}{ごう}を\ruby{見}{み}せ、\ruby{実験}{じっけん}していた。その\ruby{人}{ひと}が\ruby{本当}{ほんとう}に\ruby{物事}{ものごと}の\ruby{分}{わ}かる\ruby{人}{ひと}かどうか、\ruby{知}{し}りたかったから。でも、\ruby{答}{こた}えはいつも\ruby{同}{おな}じだった。

\ruby{帽子}{ぼうし}だね。

その\ruby{後}{あと}\ruby{僕}{ぼく}はボアの\ruby{話}{はなし}も、\ruby{原生林}{げんせいりん}の\ruby{話}{はなし}も、\ruby{星}{ほし}の\ruby{話}{はなし}もしなかった。\ruby{話}{はなし}を\ruby{合}{あ}わせて、ブリッジやゴルフや、\ruby{政治}{せいじ}やネクタイの\ruby{話}{はなし}をした。するとその\ruby{大人}{おとな}は\ruby{話}{はなし}が\ruby{分}{わ}かる\ruby{相手}{あいて}と\ruby{知}{し}り\ruby{合}{あ}えたと\ruby{言}{い}って\ruby{喜}{よろこ}ぶのだ。


\chapter{綿羊}
こうして\ruby{僕}{ぼく}は、六\ruby{年}{ねん}\ruby{前}{まえ}、サハラ\ruby{砂漠}{さばく}で\ruby{飛行機}{ひこうき}が\ruby{故障}{こしょう}するまで、\ruby{心}{こころ}を\ruby{許}{ゆる}して\ruby{話}{はな}せる\ruby{相手}{あいて}に\ruby{出会}{であ}う\ruby{事}{こと}もなく、\ruby{一人}{ひとり}で\ruby{生}{い}きてきた。\ruby{飛行機}{ひこうき}はエンジンのどこかが\ruby{壊}{こわ}れていた。\ruby{整備士}{せいびし}も、\ruby{乗客}{じょうきゃく}も\ruby{乗}{の}せていなかったので、\ruby{僕}{ぼく}は\ruby{難}{むずか}しい\ruby{修理}{しゅうり}の\ruby{仕事}{しごと}を\ruby{一人}{ひとり}でやり\ruby{遂}{と}げるしかなかった。

\ruby{死活問題}{しかつもんだい}だった。\ruby{飲}{の}み\ruby{水}{みず}は一\ruby{週間}{しゅうかん}\ruby{分}{ぶん}あるかないかだった。

\ruby{最初}{さいしょ}の\ruby{夜}{よる}、\ruby{僕}{ぼく}は、\ruby{人}{ひと}の\ruby{住}{す}む\ruby{場所}{ばしょ}から千マイルも\ruby{離}{はな}れた\ruby{砂}{すな}の\ruby{上}{うえ}で\ruby{眠}{ねむ}った。\ruby{大海原}{おおうなばら}を\ruby{筏}{いかだ}で\ruby{漂流}{ひょうりゅう}する\ruby{遭難者}{そうなんしゃ}より、ずっと\ruby{孤独}{こどく}だった。だから、\ruby{夜明}{よあ}けに\ruby{小}{ちい}さな\ruby{可愛}{かわい}らしい\ruby{声}{こえ}で\ruby{起}{お}こされた\ruby{時}{とき}、\ruby{僕}{ぼく}がどんなに\ruby{驚}{おどろ}いたか\ruby{想像}{そうぞう}してみてほしい。その\ruby{声}{こえ}は、こう\ruby{言}{い}った。

お\ruby{願}{ねが}い、\ruby{羊}{ひつじ}の\ruby{絵}{え}を\ruby{描}{か}いて。

えっ?

\ruby{羊}{ひつじ}を\ruby{描}{えが}いて。

\ruby{雷}{かみなり}に\ruby{打}{う}たれたみたいに\ruby{飛}{と}び\ruby{起}{お}きると、\ruby{目}{め}を\ruby{擦}{す}って\ruby{辺}{あた}りを\ruby{見回}{みまわ}した。そこには、とても\ruby{不思議}{ふしぎ}な\ruby{子供}{こども}が\ruby{一人}{ひとり}いて、\ruby{僕}{ぼく}を\ruby{真剣}{しんけん}に\ruby{見}{み}つめていた。\ruby{僕}{ぼく}は\ruby{突然}{とつぜん}\ruby{現}{あらわ}れたその\ruby{子供}{こども}を、\ruby{目}{め}を\ruby{丸}{まる}くして\ruby{見}{み}つめた。\ruby{何度}{なんど}も\ruby{言}{い}うけれど、\ruby{人}{ひと}の\ruby{住}{す}む\ruby{所}{ところ}から千マイルも\ruby{離}{はな}れていたのだ。しかしその\ruby{子}{こ}は\ruby{道}{みち}に\ruby{迷}{まよ}っているようには\ruby{見}{み}えなかった。\ruby{疲}{つか}れや\ruby{餓}{う}えや\ruby{渇}{かわ}きで\ruby{死}{し}にそうになっているようにも、\ruby{怖}{こわ}がっているようにも\ruby{見}{み}えなかった。\ruby{人}{ひと}の\ruby{住}{す}む\ruby{所}{ところ}から千マイルも\ruby{離}{はな}れた\ruby{砂漠}{さばく}の\ruby{真}{ま}ん\ruby{中}{なか}にいながら、\ruby{途方}{とほう}に\ruby{暮}{く}れた\ruby{迷子}{まいご}といった\ruby{様子}{ようす}は\ruby{少}{すこ}しもなかったのだ。

ようやく\ruby{口}{くち}が\ruby{聞}{き}けるようになると、\ruby{僕}{ぼく}はその\ruby{子}{こ}に\ruby{尋}{たず}ねた。

\ruby{君}{きみ}はこんな\ruby{所}{ところ}で\ruby{何}{なに}をしているの?

しかしその\ruby{子}{こ}はとても\ruby{大切}{たいせつ}な\ruby{事}{こと}のように、\ruby{静}{しず}かに\ruby{繰}{く}り\ruby{返}{かえ}すだけ。

お\ruby{願}{ねが}い、\ruby{羊}{ひつじ}の\ruby{絵}{え}を\ruby{描}{か}いて。

バカげた\ruby{話}{はなし}だが、\ruby{人}{ひと}の\ruby{住}{す}む\ruby{所}{ところ}から千マイルも\ruby{離}{はな}れて、\ruby{死}{し}の\ruby{危険}{きけん}にさらされているというのに、\ruby{僕}{ぼく}はその\ruby{子}{こ}に\ruby{言}{い}われるままに、ポケットから\ruby{一枚}{いちまい}の\ruby{紙切}{かみき}れと\ruby{万年筆}{まんねんひつ}を\ruby{取}{と}り\ruby{出}{だ}していた。

だけどそこで、\ruby{僕}{ぼく}が\ruby{一生懸命}{いっしょうけんめい}\ruby{勉強}{べんきょう}してきたのは、\ruby{地理}{ちり}と\ruby{歴史}{れきし}と\ruby{算数}{さんすう}と\ruby{文法}{ぶんぽう}だけだった\ruby{事}{こと}を\ruby{思}{おも}い\ruby{出}{だ}して、\ruby{少}{すこ}し\ruby{不機嫌}{ふきげん}になりながら、\ruby{絵}{え}は\ruby{描}{えが}けないんだと、その\ruby{子}{こ}に\ruby{言}{い}った。

そんなの\ruby{構}{かま}わないよ。\ruby{羊}{ひつじ}を\ruby{描}{えが}いて。

\ruby{僕}{ぼく}は\ruby{羊}{ひつじ}の\ruby{絵}{え}なんか\ruby{描}{えが}いたことはなかったので、\ruby{自分}{じぶん}に\ruby{描}{えが}けるたった\ruby{二}{ふた}つの\ruby{絵}{え}の\ruby{内}{うち}の\ruby{一}{ひと}つを\ruby{描}{えが}いてあげた。ボアの\ruby{外側}{そとがわ}の\ruby{絵}{え}だ。その\ruby{時}{とき}\ruby{男}{おとこ}の\ruby{子}{こ}がこういうのを\ruby{聞}{き}いて、\ruby{僕}{ぼく}はびっくりした。

\ruby{違}{ちが}う、\ruby{違}{ちが}う、ボアに\ruby{飲}{の}み\ruby{込}{こ}まれたゾウなんていらないよ。ボアはとっても\ruby{危険}{きけん}だし、ゾウは\ruby{結構}{けっこう}\ruby{場所塞}{ばしょふさ}ぎだから。\ruby{僕}{ぼく}の\ruby{所}{ところ}はとっても\ruby{小}{ちい}さいんだ。\ruby{欲}{ほ}しいのは\ruby{羊}{ひつじ}、\ruby{羊}{ひつじ}を\ruby{描}{えが}いて。

そこで\ruby{僕}{ぼく}は\ruby{羊}{ひつじ}を\ruby{描}{えが}いた。

ううん、\ruby{駄目}{だめ}だよ。この\ruby{羊}{ひつじ}はひどい\ruby{病気}{びょうき}だ。\ruby{違}{ちが}うのを\ruby{描}{えが}いて。

\ruby{僕}{ぼく}は\ruby{描}{えが}き\ruby{直}{なお}した。\ruby{男}{おとこ}の\ruby{子}{こ}は\ruby{僕}{ぼく}を\ruby{気遣}{きづか}って\ruby{優}{やさ}しく\ruby{微笑}{ほほえ}んだ。

よく\ruby{見}{み}て。これは\ruby{羊}{ひつじ}じゃないでしょう。\ruby{雄羊}{おひつじ}だよね。\ruby{角}{かく}があるもの。

そこで\ruby{僕}{ぼく}はまた\ruby{描}{えが}き\ruby{直}{なお}した。けれどそれも\ruby{前}{まえ}の\ruby{二}{ふた}つと\ruby{同}{おな}じように\ruby{拒絶}{きょぜつ}された。

この\ruby{羊}{ひつじ}は\ruby{年}{とし}を\ruby{取}{と}りすぎてるよ。\ruby{僕}{ぼく}、\ruby{長生}{ながい}きする\ruby{羊}{ひつじ}が\ruby{欲}{ほ}しいの。

\ruby{我慢}{がまん}も\ruby{限界}{げんかい}に\ruby{近付}{ちかづ}いていた。\ruby{修理}{しゅうり}を\ruby{始}{はじ}めなければと\ruby{焦}{あせ}っていた。\ruby{僕}{ぼく}はざっと\ruby{描}{えが}き\ruby{殴}{なぐ}った\ruby{絵}{え}を\ruby{男}{おとこ}の\ruby{子}{こ}に\ruby{投}{な}げ\ruby{渡}{わた}した。

これは\ruby{羊}{ひつじ}の\ruby{箱}{はこ}だ。\ruby{君}{きみ}が\ruby{欲}{ほ}しがっている\ruby{羊}{ひつじ}はこの\ruby{中}{なか}にいるよ。

すると\ruby{驚}{おどろ}いたことに、この\ruby{小}{ちい}さな\ruby{審査}{しんさ}\ruby{員}{いん}の\ruby{顔}{かお}がぱっと\ruby{輝}{かがや}いたのだ。

ぴったりだよ。\ruby{僕}{ぼく}が\ruby{欲}{ほ}しかったのは、この\ruby{羊}{ひつじ}さ。ね、この\ruby{羊}{ひつじ}\ruby{草}{ぐさ}をいっぱい\ruby{食}{た}べるかな。

どうして?

\ruby{僕}{ぼく}の\ruby{所}{ところ}はとっても\ruby{小}{ちい}さいから。

\ruby{大丈夫}{だいじょうぶ}だよ。\ruby{君}{きみ}にあげたのは、とっても\ruby{小}{ちい}さな\ruby{羊}{ひつじ}だからね。

そんなに\ruby{小}{ちい}さくないよ。あれ、\ruby{羊}{ひつじ}は\ruby{寝}{ね}ちゃったみたい。

こうして\ruby{僕}{ぼく}はこの\ruby{小}{ちい}さな\ruby{王子}{おうじ}さまと\ruby{知}{し}り\ruby{合}{あ}いになった。




\chapter{花儿}
\ruby{学校}{がっこう}の\ruby{門}{もん}が、はっきり\ruby{見}{み}えるところまで\ruby{来}{き}て、トットちゃんは、\ruby{立}{た}ち\ruby{止}{どま}った。なぜなら、この\ruby{間}{あいだ}まで\ruby{行}{い}っていた\ruby{学校}{がっこう}の\ruby{門}{もん}は、\ruby{立派}{りっぱ}なコンクリートみたいな\ruby{柱}{はしら}で、\ruby{学校}{がっこう}の\ruby{名前}{なまえ}も、\ruby{大}{おお}きく\ruby{書}{か}いてあった。ところが、この\ruby{新}{あたら}しい\ruby{学校}{がっこう}の\ruby{門}{もん}ときたら、\ruby{低}{ひく}い\ruby{木}{き}で、しかも\ruby{葉}{は}っぱが\ruby{生}{は}えていた。「\ruby{地面}{じめん}から\ruby{生}{は}えてる\ruby{門}{もん}ね」と、トットちゃんはママに\ruby{言}{い}った。そうして、こう、\ruby{付}{つ}け\ruby{加}{くわ}えた。「きっと、どんどんはえて、\ruby{今}{いま}に\ruby{電信柱}{でんしんばしら}より\ruby{高}{たか}くなるわ」\ruby{確}{たし}かに、その二\ruby{本}{ほん}の\ruby{門}{もん}は、\ruby{根}{ね}っこのある\ruby{木}{き}だった。トットちゃんは、\ruby{門}{もん}に\ruby{近}{ちか}づくと、いきなり\ruby{顔}{かお}を、\ruby{斜}{なな}めにした。なぜかといえば、\ruby{門}{もん}にぶら\ruby{下}{さ}げてある\ruby{学校}{がっこう}の\ruby{名前}{なまえ}を\ruby{書}{か}いた\ruby{札}{さつ}が、\ruby{風}{かぜ}に\ruby{吹}{ふ}かれたのか、\ruby{斜}{なな}めになっていたからだった。「トモエがくえん」トットちゃんは、\ruby{顔}{かお}を\ruby{斜}{なな}めにしたまま、\ruby{表札}{ひょうさつ}を\ruby{読}{よ}み\ruby{上}{あ}げた。そして、ママに、「トモエって、なあに?」と\ruby{聞}{き}こうとしたときだった。トットちゃんの\ruby{目}{め}の\ruby{端}{はし}に、\ruby{夢}{ゆめ}としか\ruby{思}{おも}えないものが\ruby{見}{み}えたのだった。トットちゃんは、\ruby{身}{み}をかがめると、\ruby{門}{もん}の\ruby{植}{う}え\ruby{込}{こ}みの、\ruby{隙間}{すきま}に\ruby{頭}{あたま}を\ruby{突}{つ}っ\ruby{込}{こ}んで、\ruby{門}{もん}の\ruby{中}{なか}をのぞいてみた。どうしよう、みえたんだけど!「ママ!あれ、\ruby{本当}{ほんとう}の\ruby{電車}{でんしゃ}?\ruby{校庭}{こうてい}に\ruby{並}{なら}んでるの」それは、\ruby{走}{はし}っていない、\ruby{本当}{ほんとう}の\ruby{電車}{でんしゃ}が六\ruby{台}{だい}、\ruby{教室}{きょうしつ}\ruby{用}{よう}に、\ruby{置}{お}かれてあるのだった。トットちゃんは、\ruby{夢}{ゆめ}のように\ruby{思}{おも}った。“\ruby{電車}{でんしゃ}の\ruby{教室}{きょうしつ}……”

\ruby{電車}{でんしゃ}で\ruby{窓}{まど}が、\ruby{朝}{あさ}の\ruby{光}{ひかり}を\ruby{受}{う}けて、キラキラと\ruby{光}{ひか}っていた。\ruby{目}{め}を\ruby{輝}{かがや}かして、のぞいているトットちゃんの、ホッペタも、\ruby{光}{ひか}っていた。   \ruby{気}{き}に\ruby{入}{い}ったわ\ruby{次}{つぎ}の\ruby{瞬間}{しゅんかん}、トットちゃんは、「わーい」と\ruby{歓声}{かんせい}を\ruby{上}{あ}げると、\ruby{電車}{でんしゃ}の\ruby{教室}{きょうしつ}のほうに\ruby{向}{む}かって\ruby{走}{はし}り\ruby{出}{だ}した。そして、\ruby{走}{はし}りながら、ママに\ruby{向}{む}かって\ruby{叫}{さけ}んだ。「ねえ、\ruby{早}{はや}く、\ruby{動}{うご}かない\ruby{電車}{でんしゃ}に\ruby{乗}{の}ってみよう!」ママは、\ruby{驚}{おどろ}いて\ruby{走}{はし}り\ruby{出}{だ}した。もとバスケットバールの\ruby{選手}{せんしゅ}だったままの\ruby{足}{あし}は、トットちゃんより\ruby{速}{はや}かったから、トットちゃんが、\ruby{後}{あと}、ちょっとでドア、というときに、スカートを\ruby{捕}{つか}まえられてしまった。ママは、スカートのはしを、ぎっちり\ruby{握}{にぎ}ったまま、トットちゃんにいった。「ダメよ。この\ruby{電車}{でんしゃ}は、この\ruby{学校}{がっこう}のお\ruby{教室}{きょうしつ}なんだし、あなたは、まだ、この\ruby{学校}{がっこう}に\ruby{入}{はい}れていただいてないんだから。もし、どうしても、この\ruby{電車}{でんしゃ}に\ruby{乗}{の}りたいんだったら、これからお\ruby{目}{め}にかかる\ruby{校長}{こうちょう}\ruby{先生}{せんせい}とちゃんと、お\ruby{話}{はな}してちょうだい。そして、うまくいったら、この\ruby{学校}{がっこう}に\ruby{通}{とお}えるんだから、\ruby{分}{わ}かった?」トットちゃんは、(\ruby{今}{いま}\ruby{乗}{の}れないのは、とても\ruby{残念}{ざんねん}なことだ)と\ruby{思}{おも}ったけど、ママのいう\ruby{通}{とお}りにしようときめたから、\ruby{大}{おお}きな\ruby{声}{こえ}で、「うん」といって、それから、いそいで、つけたした。「\ruby{私}{わたし}、この\ruby{学校}{がっこう}、とっても\ruby{気}{き}に\ruby{入}{い}ったわ」ママは、トットちゃんが\ruby{気}{き}に\ruby{入}{い}ったかどうかより、\ruby{校長}{こうちょう}\ruby{先生}{せんせい}が、トットちゃんを\ruby{気}{き}に\ruby{入}{い}ってくださるかどうか\ruby{問題}{もんだい}なのよ、といいたい\ruby{気}{き}がしたけど、とにかく、トットちゃんのスカートから\ruby{手}{て}を\ruby{離}{はな}し、\ruby{手}{て}をつないで\ruby{校長}{こうちょう}\ruby{室}{しつ}のほうに\ruby{歩}{ある}き\ruby{出}{だ}した。どの\ruby{電車}{でんしゃ}も\ruby{静}{しず}かで、ちょっと\ruby{前}{まえ}に、一\ruby{時間}{じかん}\ruby{目}{め}の\ruby{授業}{じゅぎょう}が\ruby{始}{はじ}まったようだった。あまり\ruby{広}{ひろ}くない\ruby{校庭}{こうてい}の\ruby{周}{まわ}りには、\ruby{塀}{へい}の\ruby{変}{か}わりに、いろんな\ruby{種類}{しゅるい}の\ruby{木}{き}が\ruby{植}{う}わっていて、\ruby{花壇}{かだん}には、\ruby{赤}{あか}や\ruby{黄色}{きいろ}の\ruby{花}{はな}がいっぱい\ruby{咲}{さ}いていた。\ruby{校長}{こうちょう}\ruby{室}{しつ}は、\ruby{電車}{でんしゃ}ではなく、ちょうど、\ruby{門}{もん}から\ruby{正面}{しょうめん}に\ruby{見}{み}える\ruby{扇形}{おうぎがた}に\ruby{広}{ひろ}がった七\ruby{段}{だん}くらいある\ruby{石}{いし}の\ruby{階段}{かいだん}を\ruby{上}{のぼ}った、その\ruby{右手}{みぎて}にあった。トットちゃんは、ママの\ruby{手}{て}を\ruby{振}{ふ}り\ruby{切}{き}ると、\ruby{階段}{かいだん}を\ruby{駆}{か}け\ruby{上}{あ}がって\ruby{行}{い}ったが、\ruby{急}{きゅう}に\ruby{止}{と}まって、\ruby{振}{ふ}り\ruby{向}{む}いた。だから、\ruby{後}{うし}ろから\ruby{行}{い}ったママは、もう\ruby{少}{すこ}しで、トットちゃんと\ruby{正面}{しょうめん}\ruby{衝突}{しょうとつ}するところだった。「どうしたの?」ママは、トットちゃんの\ruby{気}{き}が\ruby{変}{か}わったのかと\ruby{思}{おも}って、\ruby{急}{いそ}いで\ruby{聞}{き}いた。トットちゃんは、ちょうど\ruby{階段}{かいだん}の\ruby{一番}{いちばん}うえに\ruby{立}{た}った\ruby{形}{かたち}だったけど、まじめな\ruby{顔}{かお}をして、\ruby{小声}{こごえ}でママに\ruby{聞}{き}いた。ママは、かなり\ruby{辛抱}{しんぼう}づよい\ruby{人間}{にんげん}だったから……というか,\ruby{面白}{おもしろ}がりやだったから、やはり\ruby{小声}{こごえ}になって、トットちゃんに\ruby{顔}{かお}をつけて、\ruby{聞}{き}いた。「どうして?」トットちゃんは、ますます\ruby{声}{こえ}をひそめて\ruby{言}{い}った。「だってさ、\ruby{校長}{こうちょう}\ruby{先生}{せんせい}って、ママいったけど、こんなに\ruby{電車}{でんしゃ}、いっぱい\ruby{持}{も}ってるんだから、\ruby{本当}{ほんとう}は、\ruby{駅}{えき}の\ruby{人}{ひと}なんじゃないの?」\ruby{確}{たし}かに、\ruby{電車}{でんしゃ}の\ruby{払}{はら}い\ruby{下}{さ}げを\ruby{校舎}{こうしゃ}にしている\ruby{学校}{がっこう}なんてめずらしいから、トットちゃんの\ruby{疑問}{ぎもん}も、もっとものこと、とママも\ruby{思}{おも}ったけど、この\ruby{際}{さい}、\ruby{説明}{せつめい}してるヒマはないので、こういった。「じゃ、あなた、\ruby{校長}{こうちょう}\ruby{先生}{せんせい}に\ruby{伺}{うかが}って\ruby{御覧}{ごらん}なさい、\ruby{自分}{じぶん}で。それと、あなたのパパのことを\ruby{考}{かんが}えてみて?パパはヴァイオリンを\ruby{弾}{ひ}く\ruby{人}{ひと}で、いくつかヴァイオリンを\ruby{持}{も}ってるけど、ヴァイオリン\ruby{屋}{や}さんじゃないでしょう?そういう\ruby{人}{ひと}もいるのよ」トットちゃんは、「そうか」というと、ママと\ruby{手}{て}をつないだ。




\chapter{第四章}
\ruby{次}{つぎ}の\ruby{瞬間}{しゅんかん}、トットちゃんは、「わーい」と\ruby{歓声}{かんせい}を\ruby{上}{あ}げると、\ruby{電車}{でんしゃ}の\ruby{教室}{きょうしつ}のほうに\ruby{向}{む}かって\ruby{走}{はし}り\ruby{出}{だ}した。そして、\ruby{走}{はし}りながら、ママに\ruby{向}{む}かって\ruby{叫}{さけ}んだ。

「ねえ、\ruby{早}{はや}く、\ruby{動}{うご}かない\ruby{電車}{でんしゃ}に\ruby{乗}{の}ってみよう!」

ママは、\ruby{驚}{おどろ}いて\ruby{走}{はし}り\ruby{出}{だ}した。もとバスケットボールの\ruby{選手}{せんしゅ}だったままの\ruby{足}{あし}は、トットちゃんより\ruby{速}{はや}かったから、トットちゃんが、\ruby{後}{あと}、ちょっとでドア、というときに、スカートを\ruby{捕}{つか}まえられてしまった。ママは、スカートのはしを、ぎっちり\ruby{握}{にぎ}ったまま、トットちゃんにいった。

「ダメよ。この\ruby{電車}{でんしゃ}は、この\ruby{学校}{がっこう}のお\ruby{教室}{きょうしつ}なんだし、あなたは、まだ、この\ruby{学校}{がっこう}に\ruby{入}{はい}れていただいてないんだから。もし、どうしても、この\ruby{電車}{でんしゃ}に\ruby{乗}{の}りたいんだったら、これからお\ruby{目}{め}にかかる\ruby{校長}{こうちょう}\ruby{先生}{せんせい}とちゃんと、お\ruby{話}{はな}してちょうだい。そして、うまくいったら、この\ruby{学校}{がっこう}に\ruby{通}{とお}えるんだから、\ruby{分}{わ}かった?」

トットちゃんは、(\ruby{今}{いま}\ruby{乗}{の}れないのは、とても\ruby{残念}{ざんねん}なことだ)と\ruby{思}{おも}ったけど、ママのいう\ruby{通}{とお}りにしようときめたから、\ruby{大}{おお}きな\ruby{声}{こえ}で、

「うん」

といって、それから、いそいで、つけたした。

「\ruby{私}{わたし}、この\ruby{学校}{がっこう}、とっても\ruby{気}{き}に\ruby{入}{い}ったわ」

ママは、トットちゃんが\ruby{気}{き}に\ruby{入}{い}ったかどうかより、\ruby{校長}{こうちょう}\ruby{先生}{せんせい}が、トットちゃんを\ruby{気}{き}に\ruby{入}{い}ってくださるかどうか\ruby{問題}{もんだい}なのよ、といいたい\ruby{気}{き}がしたけど、とにかく、トットちゃんのスカートから\ruby{手}{て}を\ruby{離}{はな}し、\ruby{手}{て}をつないで\ruby{校長}{こうちょう}\ruby{室}{しつ}のほうに\ruby{歩}{ある}き\ruby{出}{だ}した。

どの\ruby{電車}{でんしゃ}も\ruby{静}{しず}かで、ちょっと\ruby{前}{まえ}に、一\ruby{時間}{じかん}\ruby{目}{め}の\ruby{授業}{じゅぎょう}が\ruby{始}{はじ}まったようだった。あまり\ruby{広}{ひろ}くない\ruby{校庭}{こうてい}の\ruby{周}{まわ}りには、\ruby{塀}{へい}の\ruby{変}{か}わりに、いろんな\ruby{種類}{しゅるい}の\ruby{木}{き}が\ruby{植}{う}わっていて、\ruby{花壇}{かだん}には、\ruby{赤}{あか}や\ruby{黄色}{きいろ}の\ruby{花}{はな}がいっぱい\ruby{咲}{さ}いていた。

\ruby{校長}{こうちょう}\ruby{室}{しつ}は、\ruby{電車}{でんしゃ}ではなく、ちょうど、\ruby{門}{もん}から\ruby{正面}{しょうめん}に\ruby{見}{み}える\ruby{扇形}{おうぎがた}に\ruby{広}{ひろ}がった七\ruby{段}{だん}くらいある\ruby{石}{いし}の\ruby{階段}{かいだん}を\ruby{上}{のぼ}った、その\ruby{右手}{みぎて}にあった。

トットちゃんは、ママの\ruby{手}{て}を\ruby{振}{ふ}り\ruby{切}{き}ると、\ruby{階段}{かいだん}を\ruby{駆}{か}け\ruby{上}{あ}がって\ruby{行}{い}ったが、\ruby{急}{きゅう}に\ruby{止}{と}まって、\ruby{振}{ふ}り\ruby{向}{む}いた。だから、\ruby{後}{うし}ろから\ruby{行}{い}ったママは、もう\ruby{少}{すこ}しで、トットちゃんと\ruby{正面}{しょうめん}\ruby{衝突}{しょうとつ}するところだった。

「どうしたの?」

ママは、トットちゃんの\ruby{気}{き}が\ruby{変}{か}わったのかと\ruby{思}{おも}って、\ruby{急}{いそ}いで\ruby{聞}{き}いた。トットちゃんは、ちょうど\ruby{階段}{かいだん}の\ruby{一番}{いちばん}うえに\ruby{立}{た}った\ruby{形}{かたち}だったけど、まじめな\ruby{顔}{かお}をして、\ruby{小声}{こごえ}でママに\ruby{聞}{き}いた。

「ねえ、これからあいに行く人って、\ruby{駅}{えき}の\ruby{人}{ひと}なんじゃないの?」

ママは、かなり\ruby{辛抱}{しんぼう}づよい\ruby{人間}{にんげん}だったから……というか,\ruby{面白}{おもしろ}がりやだったから、やはり\ruby{小声}{こごえ}になって、トットちゃんに\ruby{顔}{かお}をつけて、\ruby{聞}{き}いた。

「どうして?」

トットちゃんは、ますます\ruby{声}{こえ}をひそめて\ruby{言}{い}った。

「だってさ、\ruby{校長}{こうちょう}\ruby{先生}{せんせい}って、ママいったけど、こんなに\ruby{電車}{でんしゃ}、いっぱい\ruby{持}{も}ってるんだから、\ruby{本当}{ほんとう}は、\ruby{駅}{えき}の\ruby{人}{ひと}なんじゃないの?」

\ruby{確}{たし}かに、\ruby{電車}{でんしゃ}の\ruby{払}{はら}い\ruby{下}{さ}げを\ruby{校舎}{こうしゃ}にしている\ruby{学校}{がっこう}なんてめずらしいから、トットちゃんの\ruby{疑問}{ぎもん}も、もっとものこと、とママも\ruby{思}{おも}ったけど、この\ruby{際}{さい}、\ruby{説明}{せつめい}してるヒマはないので、こういった。

「じゃ、あなた、\ruby{校長}{こうちょう}\ruby{先生}{せんせい}に\ruby{伺}{うかが}って\ruby{御覧}{ごらん}なさい、\ruby{自分}{じぶん}で。それと、あなたのパパのことを\ruby{考}{かんが}えてみて?パパはヴァイオリンを\ruby{弾}{ひ}く\ruby{人}{ひと}で、いくつかヴァイオリンを\ruby{持}{も}ってるけど、ヴァイオリン\ruby{屋}{や}さんじゃないでしょう?そういう\ruby{人}{ひと}もいるのよ」トットちゃんは、「そうか」というと、ママと\ruby{手}{て}をつないだ。

\chapter{第五章}
トットちゃんとママが入っていくと、\ruby{部屋}{へや}の中にいた男の人が\ruby{椅子}{いす}から立ち上がった。その人は、\ruby{頭}{あたま}の\ruby{毛}{け}が\ruby{薄}{うす}くなっていて、\ruby{前}{まえ}のほうの\ruby{歯}{は}が\ruby{抜}{ぬ}けていて、\ruby{顔}{かお}の\ruby{血色}{けっしょく}がよく、\ruby{背}{せ}はあまり\ruby{高}{たか}くないけど、\ruby{肩}{かた}や\ruby{腕}{うで}が、がっちりしていて、ヨレヨレの\ruby{黒}{くろ}の\ruby{三}{み}つ\ruby{揃}{ぞろ}いを、キチンと\ruby{着}{き}ていた。

トットちゃんは、\ruby{急}{いそ}いで、お\ruby{辞儀}{じぎ}をしてから、\ruby{元気}{げんき}よく\ruby{聞}{き}いた。

「\ruby{校長}{こうちょう}先生か、\ruby{駅}{えき}の人か、どっち?」

ママが、\ruby{慌}{あわ}てて\ruby{説明}{せつめい}しよう、とするまえに、その人は\ruby{笑}{わら}いながら\ruby{答}{こた}えた。

「\ruby{校長}{こうちょう}先生だよ」

トットちゃんは、とってもうれしそうに\ruby{言}{い}った。

「よかった。じゃ、おねがい。\ruby{私}{わたし}、この学校にいりたいの」

\ruby{校長}{こうちょう}先生は、\ruby{椅子}{いす}をトットちゃんに\ruby{勧}{すす}めると、ママのほうを\ruby{向}{む}いて\ruby{言}{い}った。

「じゃ、\ruby{僕}{ぼく}は、これからトットちゃんと\ruby{話}{はなし}がありますから、もう、お\ruby{帰}{かえ}り下さって\ruby{結構}{けっこう}です」

ほんのちょっとの\ruby{間}{あいだ}、トットちゃんは、\ruby{少}{すこ}し\ruby{心細}{こころぼそ}い気がしたけど、なんとなく、(この\ruby{校長}{こうちょう}先生ならいいや)と\ruby{思}{おも}った。ママは、いさぎよく先生にいった。

「じゃ、よろしく、お\ruby{願}{ねが}いします」

そして、ドアを\ruby{閉}{し}めて\ruby{出}{で}て\ruby{行}{い}った。

\ruby{校長}{こうちょう}先生は、トットちゃんの\ruby{前}{まえ}に\ruby{椅子}{いす}を\ruby{引}{ひ}っ\ruby{張}{ぱ}ってきて、とても\ruby{近}{ちか}い\ruby{位置}{いち}に、\ruby{向}{む}かい\ruby{合}{あ}わせに\ruby{腰}{こし}をかけると、こういった。

「さあ、\ruby{何}{なん}でも、先生に\ruby{話}{はな}してごらん。\ruby{話}{はな}したいこと、\ruby{全部}{ぜんぶ}」

「\ruby{話}{はな}したいこと!?」

(なにか\ruby{聞}{き}かれて、お\ruby{返事}{へんじ}するのかな?)と\ruby{思}{おも}っていたトットちゃんは、「\ruby{何}{なん}でも\ruby{話}{はな}していい」と\ruby{聞}{き}いて、ものすごくうれしくなって、すぐ\ruby{話}{はな}し\ruby{始}{はじ}めた。\ruby{順序}{じゅんじょ}も、\ruby{話}{はな}し\ruby{方}{かた}も、\ruby{少}{すこ}しグチャグチャだったけど、\ruby{一生懸命}{いっしょうけんめい}に\ruby{話}{はな}した。

\ruby{今}{いま}\ruby{乗}{の}ってきた\ruby{電車}{でんしゃ}が\ruby{速}{はや}かったこと。

\ruby{駅}{えき}の\ruby{改札口}{かいさつぐち}のおじさんに、お\ruby{願}{ねが}いしたけど、\ruby{切符}{きっぷ}をくれなかったこと。

\ruby{前}{まえ}に\ruby{行}{い}ってた学校の\ruby{受}{う}け\ruby{持}{も}ちの女の先生は、\ruby{顔}{かお}がきれいだということ。

その学校には、つばめの\ruby{巣}{す}があること。

\ruby{家}{いえ}には、ロッキーという\ruby{茶色}{ちゃいろ}の犬がいて“お手”と“ごめんくださいませ”と、ご\ruby{飯}{はん}の\ruby{後}{あと}で、“\ruby{満足}{まんぞく}、\ruby{満足}{まんぞく}”ができること。

\ruby{幼稚}{ようち}\ruby{園}{えん}のとき、ハサミを口の中に入れて、チョキチョキやると、「\ruby{舌}{した}を\ruby{切}{き}ります」と先生が\ruby{怒}{いか}ったけど、\ruby{何回}{なんかい}もやっちゃったっていうこと。

\ruby{洟}{はな}が出てきたときは、いつまでも、ズルズルやってると、ママにしかられるから、なるべく早くかむこと。

パパは、\ruby{海}{うみ}で\ruby{泳}{およ}ぐのが\ruby{上手}{じょうず}で、\ruby{飛}{と}び\ruby{込}{こ}みだって\ruby{出来}{でき}ること。

こういったことを、\ruby{次}{つぎ}から\ruby{次}{つぎ}と、トットちゃんは\ruby{話}{はな}した。先生は、\ruby{笑}{わら}ったり、うなずいたり、「それから?」とかいったりしてくださったから、うれしくて、トットちゃんは、いつまでも\ruby{話}{はな}した。でも、とうとう、\ruby{話}{はなし}がなくなった。トットちゃんは、口をつぐんで\ruby{考}{かんが}えていると、先生はいった。

「もう、ないかい?」

トットちゃんは、これでおしまいにしてしまうのは、\ruby{残念}{ざんねん}だと\ruby{思}{おも}った。

せっかく、\ruby{話}{はなし}を、いっぱい\ruby{聞}{き}いてもらう、いいチャンスなのに。

(なにか、\ruby{話}{はなし}は、ないかなあ……)

\ruby{頭}{あたま}の中が、\ruby{忙}{いそが}しく\ruby{動}{うご}いた。と\ruby{思}{おも}ったら、「よかった!」。\ruby{話}{はなし}が見つかった。

それは、その日、トットちゃんが\ruby{着}{き}てる\ruby{洋服}{ようふく}のことだった。たいがいの\ruby{洋服}{ようふく}は、ママが\ruby{手製}{てせい}で\ruby{作}{つく}ってくれるのだけれど、\ruby{今日}{きょう}のは、\ruby{買}{か}ったものだった。というのも、なにしろトットちゃんが\ruby{夕方}{ゆうがた}、\ruby{外}{そと}から\ruby{帰}{かえ}ってきたとき、どの\ruby{洋服}{ようふく}もビリビリで、ときには、ジャキジャキのときもあったし、どうしてそうなるのか、ママにも\ruby{絶対}{ぜったい}わからないのだけれど、白い\ruby{木綿}{もめん}でゴム入りのパンツまで、ビリビリになっているのだから。トットちゃんの\ruby{話}{はなし}によると、よその\ruby{家}{いえ}の\ruby{庭}{にわ}をつっきって\ruby{垣根}{かきね}をもぐったり、\ruby{原}{はら}っぱの\ruby{鉄条}{てつじょう}\ruby{網}{あみ}をくぐるとき、「こんなになっちゃうんだ」ということなのだけれど、とにかく、そんな\ruby{具合}{ぐあい}で、\ruby{結局}{けっきょく}、\ruby{今朝}{けさ}、\ruby{家}{いえ}をでるとき、ママの\ruby{手製}{てせい}の、しゃれたのは、どれもビリビリで、\ruby{仕方}{しかた}なく、\ruby{前}{まえ}に\ruby{買}{か}ったのを\ruby{着}{き}てきたのだった。それはワンピースで、エンジとグレーの\ruby{細}{こま}かいチェックで、\ruby{布地}{ぬのじ}はジャージーだから、\ruby{悪}{わる}くはないけど、\ruby{衿}{えり}にしてある、花の\ruby{刺繍}{ししゅう}の、赤い\ruby{色}{いろ}が、ママは、「\ruby{趣味}{しゅみ}が\ruby{悪}{わる}い」といっていた。そのことを、トットちゃんは、\ruby{思}{おも}い\ruby{出}{だ}したのだった。だから、\ruby{急}{いそ}いで\ruby{椅子}{いす}から\ruby{降}{お}りると、\ruby{衿}{えり}を手で\ruby{持}{も}ち\ruby{上}{あ}げて、先生のそばに\ruby{行}{い}き、こういった。「この\ruby{衿}{えり}ね、ママ、\ruby{嫌}{きら}いなんだって!」

                                                                                                                                                                                                                                                                                                                                                                                                                                                                                                                                                                                                                                                                          それをいってしまったら、どう\ruby{考}{かんが}えてみても、\ruby{本当}{ほんとう}に、\ruby{話}{はな}しはもう\ruby{無}{な}くなった。トットちゃんは(\ruby{少}{すこ}し\ruby{悲}{かな}しい)と\ruby{思}{おも}った。トットちゃんが、そう\ruby{思}{おも}ったとき、先生が立ち上がった。そして、トットちゃんの\ruby{頭}{あたま}に、大きく\ruby{暖}{あたた}かい手を\ruby{置}{お}くと、「じゃ、これで、\ruby{君}{きみ}は、この学校の\ruby{生徒}{せいと}だよ」そういった。……その\ruby{時}{とき},トットちゃんは、なんだか、生まれて\ruby{初}{はじ}めて、\ruby{本当}{ほんとう}に\ruby{好}{す}きな人にあったような気がした。だって、生まれてから\ruby{今日}{きょう}まで、こんな\ruby{長}{なが}い\ruby{時間}{じかん}、\ruby{自分}{じぶん}の\ruby{話}{はなし}を\ruby{聞}{き}いてくれた人は、いなっかたんだもの。そして、その\ruby{長}{なが}い\ruby{時間}{じかん}の\ruby{間}{あいだ}、\ruby{一度}{いちど}だって、あくびをしたり、\ruby{退屈}{たいくつ}そうにしないで、トットちゃんが\ruby{話}{はな}してるのと\ruby{同}{おな}じように、\ruby{身}{み}を\ruby{乗}{の}り\ruby{出}{だ}して、\ruby{一生懸命}{いっしょうけんめい}、\ruby{聞}{き}いてくれたんだもの。

                                                                                                                                                                                                                                                                                                                                                                                                                                                                                                                                                                                                                                                                          トットちゃんは、このとき、まだ\ruby{時計}{とけい}が\ruby{読}{よ}めなかったんだけど、それでも\ruby{長}{なが}い\ruby{時間}{じかん}、と\ruby{思}{おも}ったくらいなんだから、もし\ruby{読}{よ}めたら、ビックリしたに\ruby{違}{ちが}いない。そして、もっと先生に\ruby{感謝}{かんしゃ}したに\ruby{違}{ちが}いない。というのは、トットちゃんとママが学校に\ruby{着}{つ}いたのが八\ruby{時}{じ}で、\ruby{校長}{こうちょう}\ruby{室}{しつ}で\ruby{全部}{ぜんぶ}の\ruby{話}{はなし}が\ruby{終}{お}わって、トットちゃんが、この学校の生\ruby{徒}{あだ}になった、と\ruby{決}{き}まったとき、先生が\ruby{懐中}{かいちゅう}\ruby{時計}{とけい}を見て、「ああ、お\ruby{弁当}{べんとう}の\ruby{時間}{じかん}だな」といったから、つまり、たっぷり四\ruby{時間}{じかん}、先生は、トットちゃんの\ruby{話}{はなし}を\ruby{聞}{き}いてくれたことになるのだった。\ruby{後}{あと}にも先にも、トットちゃんの\ruby{話}{はなし}を、こんなにちゃんと\ruby{聞}{き}いてくれた\ruby{大人}{おとな}は、いなかった。それにしても、まだ小学校一年生になったばかりのトットちゃんが、四\ruby{時間}{じかん}も、\ruby{一人}{ひとり}でしゃべるぶんの\ruby{話}{はな}しがあったことは、ママや、\ruby{前}{まえ}の学校の先生が\ruby{聞}{き}いたら、きっと、ビックリするに\ruby{違}{ちが}いないことだった。

                                                                                                                                                                                                                                                                                                                                                                                                                                                                                                                                                                                                                                                                          このとき、トットちゃんは、まだ\ruby{退学}{たいがく}のことはもちろん、\ruby{周}{まわ}りの\ruby{大人}{おとな}が、手こずってることも、気がついていなかったし、もともと\ruby{性格}{せいかく}も\ruby{陽気}{ようき}で、\ruby{忘}{わす}れっぽいタチだったから、\ruby{無邪気}{むじゃき}に見えた。でも、トットちゃんの中のどこかに、なんとなく、\ruby{疎外感}{そがいかん}のような、\ruby{他}{ほか}の\ruby{子供}{こども}と\ruby{違}{ちが}って、ひとりだけ、ちょっと、\ruby{冷}{つめ}たい目で見られているようなものを、おぼろげには\ruby{感}{かん}じていた。それが、この\ruby{校長}{こうちょう}先生といると、\ruby{安心}{あんしん}で、\ruby{暖}{あたた}かくて、\ruby{気持}{きも}ちがよかった。(この人となら、ずーっと\ruby{一緒}{いっしょ}にいてもいい)これが、\ruby{校長}{こうちょう}先生、\ruby{小林宗作}{こばやしそうさく}\ruby{氏}{し}に、\ruby{初}{はじ}めて\ruby{遭}{あ}った日、トットちゃんが\ruby{感}{かん}じた、\ruby{感想}{かんそう}だった。そして、\ruby{有難}{ありがた}いことに、\ruby{校長}{こうちょう}先生も、トットちゃんと、\ruby{同}{おな}じ\ruby{感想}{かんそう}を、その\ruby{時}{とき}、\ruby{持}{も}っていたのだった。




\chapter{第六章}
トットちゃんは、\ruby{校長}{こうちょう}先生に\ruby{連}{つ}れられて、みんなが、お\ruby{弁当}{べんとう}を\ruby{食}{た}べるところを、見に\ruby{行}{い}くことになった。お\ruby{昼}{ひる}だけは、\ruby{電車}{でんしゃ}でなく、「みんな、\ruby{講堂}{こうどう}に\ruby{集}{あつ}まることになっている」と\ruby{校長}{こうちょう}先生が\ruby{教}{おし}えてくれた。\ruby{講堂}{こうどう}はさっきトットちゃんが上がってきた石の\ruby{階段}{かいだん}の、\ruby{突}{つ}き\ruby{当}{あ}たりにあった。いってみると、\ruby{生徒}{せいと}たちが、\ruby{大騒}{おおさわ}ぎをしながら、\ruby{机}{つくえ}と\ruby{椅子}{いす}を、\ruby{講堂}{こうどう}に、まーるく\ruby{輪}{わ}になるように、\ruby{並}{なら}べているところだった。\ruby{隅}{すみ}っこで、それを見ていたトットちゃんは、\ruby{校長}{こうちょう}先生の\ruby{上着}{うわぎ}を\ruby{引}{ひ}っ\ruby{張}{ぱ}って\ruby{聞}{き}いた。

「\ruby{他}{ほか}の\ruby{生徒}{せいと}は、どこにいるの?」

\ruby{校長}{こうちょう}先生は\ruby{答}{こた}えた。

「これで\ruby{全部}{ぜんぶ}なんだよ」

「\ruby{全部}{ぜんぶ}!?」

トットちゃんは、\ruby{信}{しん}じられない気がした。だって、\ruby{前}{まえ}の学校の一クラスと\ruby{同}{おな}じくらいしか、いないんだもの。そうすると、

「学校中で、五十人くらいなの?」

\ruby{校長}{こうちょう}先生は、「そうだ」といった。トットちゃんは、なにもかも、\ruby{前}{まえ}の学校と\ruby{違}{ちが}ってると\ruby{思}{おも}った。

みんなが\ruby{着席}{ちゃくせき}すると、\ruby{校長}{こうちょう}先生は、

「みんな、\ruby{海}{うみ}のものと、山のもの、もって\ruby{来}{き}たかい?」

と\ruby{聞}{き}いた。

「はーい」

みんな、それぞれの、お\ruby{弁当}{べんとう}の、ふたを\ruby{取}{と}った。

「どれどれ」

\ruby{校長}{こうちょう}先生は、\ruby{机}{つくえ}で\ruby{出来}{でき}た円の中に入ると、ひとりず、お\ruby{弁当}{べんとう}をのぞきながら、\ruby{歩}{ある}いている。

\ruby{生徒}{せいと}たちは、\ruby{笑}{わら}ったり、キイキイいったり、にぎやかだった。

「\ruby{海}{うみ}のものと、山のもの、って、なんだろう」

トットちゃんは、おかしくなった。でも、とっても、とっても、この学校は\ruby{変}{か}わっていて、\ruby{面白}{おもしろ}そう。お\ruby{弁当}{べんとう}の\ruby{時間}{じかん}が、こんなに、\ruby{愉快}{ゆかい}で、\ruby{楽}{たの}しいなんて、\ruby{知}{し}らなかった。トットちゃんは、\ruby{明日}{あした}からは、\ruby{自分}{じぶん}も、あの\ruby{机}{つくえ}に\ruby{座}{すわ}って、『\ruby{海}{うみ}のものと、山のもの』の\ruby{弁当}{べんとう}を、\ruby{校長}{こうちょう}先生に見てもらうんだ、と\ruby{思}{おも}うと、もう、\ruby{嬉}{うれ}しさと、\ruby{楽}{たの}しさで、\ruby{胸}{むね}がいっぱいになり、\ruby{叫}{さけ}びそうになった。 お\ruby{弁当}{べんとう}を、のぞきこんでる\ruby{校長}{こうちょう}先生の\ruby{肩}{かた}に、お\ruby{昼}{ひる}の\ruby{光}{ひかり}が、やわらかく\ruby{止}{と}まっていた。




\chapter{第七章}
きのう、「\ruby{今日}{きょう}から、\ruby{君}{きみ}は、もう、この学校の\ruby{生徒}{せいと}だよ」、そう\ruby{校長}{こうちょう}先生に\ruby{言}{い}われたトットちゃんにとって、こんなに\ruby{次}{つぎ}の日が\ruby{待}{ま}ち\ruby{遠}{どお}しい、ってことは、\ruby{今}{いま}までになかった。だから、いつもなら\ruby{朝}{あさ}、ママが\ruby{叩}{たた}き\ruby{起}{お}こしても、まだベッドの上でぼんやりしてることの\ruby{多}{おお}いトットちゃんが、この日ばかりは、\ruby{誰}{だれ}からも\ruby{起}{お}こされない\ruby{前}{まえ}に、もうソックスまではいて、ランドセルを\ruby{背負}{しょ}って、みんなの\ruby{起}{お}きるのを\ruby{待}{ま}っていた。

ロッキーは、\ruby{途中}{とちゅう}までは、耳をピンと立てて\ruby{神妙}{しんみょう}に\ruby{聞}{き}いていたけど、\ruby{説明}{せつめい}の\ruby{終}{お}わりのところで、\ruby{定期}{ていき}を、ちょっと、なめてみて、それから、あくびをした。それでも、トットちゃんは、\ruby{一生懸命}{いっしょうけんめい}に\ruby{話}{はな}し\ruby{続}{つづ}けた。

「\ruby{電車}{でんしゃ}の\ruby{教室}{きょうしつ}は、\ruby{動}{うご}かないから、お\ruby{教室}{きょうしつ}では、\ruby{定期}{ていき}はいらないと\ruby{思}{おも}うんだ。とにかく、\ruby{今日}{きょう}は\ruby{持}{も}ってるのよ」

たしかにロッキーは、\ruby{今}{いま}まで、\ruby{歩}{ある}いて\ruby{通}{かよ}う学校の\ruby{門}{もん}まで、\ruby{毎日}{まいにち}、トットちゃんと\ruby{一緒}{いっしょ}に\ruby{行}{い}って、\ruby{後}{あと}は、\ruby{一人}{ひとり}で\ruby{家}{いえ}に\ruby{帰}{かえ}ってきていたから、\ruby{今日}{きょう}も、そのつもりでいた。

トットちゃんは、\ruby{定期}{ていき}をロッキーの\ruby{首}{くび}からはずすと、\ruby{大切}{たいせつ}そうに\ruby{自分}{じぶん}の\ruby{首}{くび}にかけると、パパとママに、もう\ruby{一度}{いちど}、 『\ruby{行}{い}ってまいりまーす』というと、\ruby{今度}{こんど}は\ruby{振}{ふ}り\ruby{返}{かえ}らずに、ランドセルをカタカタいわせて\ruby{走}{はし}り\ruby{出}{だ}した。ロッキーも、からだをのびのびさせながら、\ruby{並}{なら}んで\ruby{走}{はし}り\ruby{出}{だ}した。

\ruby{駅}{えき}までの\ruby{道}{みち}は、\ruby{前}{まえ}の学校に\ruby{行}{い}く\ruby{道}{みち}と、ほとんど\ruby{変}{か}わらなかった。だから、\ruby{途中}{とちゅう}でトットちゃんは、\ruby{顔見知}{かおみし}りの犬や\ruby{猫}{ねこ}や、\ruby{前}{まえ}の\ruby{同級}{どうきゅう}生と、すれ\ruby{違}{ちが}った。トットちゃんは、その\ruby{度}{たび}に、「\ruby{定期}{ていき}を見せて、\ruby{驚}{おどろ}かせてやろうかな?」と\ruby{思}{おも}ったけど、(もし\ruby{遅}{おそ}くなったら\ruby{大変}{たいへん}だから、\ruby{今日}{きょう}は、よそう……)と\ruby{決}{き}めて、どんどん\ruby{歩}{ある}いた。

\ruby{駅}{えき}のところに\ruby{来}{き}て、いつもなら左に\ruby{行}{い}くトットちゃんが、右に\ruby{曲}{ま}がったので、\ruby{可哀}{かわい}そうにロッキーは、とても\ruby{心配}{しんぱい}そうに\ruby{立}{た}ち\ruby{止}{どま}って、キョロキョロした。トットちゃんは、\ruby{改札口}{かいさつぐち}のところまで\ruby{行}{い}ったんだけど、\ruby{戻}{もど}ってきて、まだ\ruby{不思議}{ふしぎ}そうな\ruby{顔}{かお}をしてるロッキーにいった。

「もう、\ruby{前}{まえ}の学校には\ruby{行}{い}かないのよ。\ruby{新}{あたら}しい学校に\ruby{行}{い}くんだから」

それからトットちゃんは、ロッキーの\ruby{顔}{かお}に、\ruby{自分}{じぶん}の\ruby{顔}{かお}をくっつけ、ついでにロッキーの耳の中の、においをかいだ。(いつもと\ruby{同}{おな}じくらい、くさいけれど、\ruby{私}{わたし}には、いい、におい!)そう\ruby{思}{おも}うと\ruby{顔}{かお}を\ruby{離}{はな}して、「バイバイ」というと、\ruby{定期}{ていき}を\ruby{駅}{えき}の人に見せて、ちょっと\ruby{高}{たか}い\ruby{駅}{えき}の\ruby{階段}{かいだん}を、\ruby{登}{のぼ}り\ruby{始}{はじ}めた。ロッキーは、小さい\ruby{声}{こえ}で\ruby{鳴}{な}いて、トットちゃんが\ruby{階段}{かいだん}を上がっていくのを、いつまでも\ruby{見送}{みおく}っていた。

この\ruby{家}{いえ}の中で、いちばん、きちんと\ruby{時間}{じかん}を\ruby{守}{まも}るシェパードのロッキーは、トットちゃんの、いつもと\ruby{違}{ちが}う\ruby{行動}{こうどう}に、\ruby{怪訝}{けげん}そうな目を\ruby{向}{む}けながら、それでも、大きく\ruby{伸}{の}びをすると、トットちゃんにぴったりとくっついて、(\ruby{何}{なに}か\ruby{始}{はじ}まるらしい)ことを\ruby{期待}{きたい}した。

ママ\ruby{大変}{たいへん}だった。\ruby{大忙}{おおいそが}しで、『\ruby{海}{うみ}のものと山のもの』のお\ruby{弁当}{べんとう}を\ruby{作}{つく}り、トットちゃんに\ruby{朝}{あさ}ごはんを\ruby{食}{た}べさせ、\ruby{毛糸}{けいと}で\ruby{編}{あ}んだヒモを\ruby{通}{とお}した、セルロイドの\ruby{定期入}{ていきい}れを、トットちゃんの\ruby{首}{くび}にかけた。これは\ruby{定期}{ていき}を、なくさないためだった、パパは「いい子でね」と\ruby{頭}{あたま}をヒシャヒシャにしたまま\ruby{言}{い}った。「もちろん!」と、トットちゃんは\ruby{言}{い}うと、\ruby{玄関}{げんかん}で\ruby{靴}{くつ}を\ruby{履}{は}き、\ruby{戸}{と}を\ruby{開}{あ}けると、クルリと\ruby{家}{いえ}の中を\ruby{向}{む}き、\ruby{丁寧}{ていねい}にお\ruby{辞儀}{じぎ}をして、こういった。

「みなさま、\ruby{行}{い}ってまいります」

\ruby{見送}{みおく}りに立っていたママは、ちょっと\ruby{涙}{なみだ}でそうになった。それは、こんなに生き生きとしてお\ruby{行儀}{ぎょうぎ}よく、\ruby{素直}{すなお}で、\ruby{楽}{たの}しそうにしてるトットちゃんが、つい、このあいだ、「\ruby{退学}{たいがく}になった」、ということを\ruby{思}{おも}い\ruby{出}{だ}したからだった。(\ruby{新}{あたら}しい学校で、うまくいくといい……)ママは\ruby{心}{こころ}からそう\ruby{祈}{いの}った。

ところが、\ruby{次}{つぎ}の\ruby{瞬間}{しゅんかん}、ママは、\ruby{飛}{と}び\ruby{上}{あ}がるほど\ruby{驚}{おどろ}いた。というのは、トットちゃんが、せっかくママが\ruby{首}{くび}からかけた\ruby{定期}{ていき}を、ロッキーの\ruby{首}{くび}にかけているのを見たからだった。ママは、(\ruby{一体}{いったい}どうなるのだろう?)と\ruby{思}{おも}ったけど、だまって、\ruby{成}{な}り\ruby{行}{ゆ}きを見ることにした。トットちゃんは、\ruby{定期}{ていき}をロッキーの\ruby{首}{くび}にかけると、しゃがんで、ロッキーに、こういった。

「いい?この\ruby{定期}{ていき}のヒモは、あんたに、\ruby{合}{あ}わないのよ」

\ruby{確}{たし}かに、ロッキーにはヒモが\ruby{長}{なが}く、\ruby{定期}{ていき}は\ruby{地面}{じめん}を\ruby{引}{ひ}きずっていた。

「わかった?これは\ruby{私}{わたし}の\ruby{定期}{ていき}で、あんたのじゃないから、あんたは\ruby{電車}{でんしゃ}に\ruby{乗}{の}れないの。\ruby{校長}{こうちょう}先生に\ruby{聞}{き}いてみるけど、\ruby{駅}{えき}の人にも。で『いい』っていったら、あんたも学校に\ruby{来}{こ}られるんだけど、どうかなあ」




\chapter{第八章}
すぐに\ruby{僕}{ぼく}は\ruby{王子}{おうじ}さまの\ruby{花}{はな}の\ruby{事}{こと}を、もっとよく\ruby{知}{し}るようになった。\ruby{王子}{おうじ}さまの\ruby{星}{ほし}にはもともと\ruby{花}{はな}びらが\ruby{一重}{ひとえ}の\ruby{素朴}{そぼく}な\ruby{花}{はな}が\ruby{場所}{ばしょ}もとらず、\ruby{邪魔}{じゃま}にもならずに\ruby{咲}{さ}いていた。ところがある\ruby{日}{ひ}、どこからともなく\ruby{運}{はこ}ばれてきた\ruby{種}{たね}が\ruby{芽}{め}を\ruby{出}{だ}した。\ruby{王子}{おうじ}さまは\ruby{他}{ほか}のものとは\ruby{似}{に}ても\ruby{似}{に} つかないその\ruby{芽}{め}を\ruby{見}{み}つけて、\ruby{注意深}{ちゅういぶか}く\ruby{観察}{かんさつ}していた。\ruby{新種}{しんしゅ}のバオバブかもしれないからだ。

しかしそれはすぐに\ruby{伸}{の}びるのをやめ、\ruby{花}{はな}を\ruby{咲}{さ}かせる\ruby{準備}{じゅんび}を\ruby{始}{はじ}めた。ふっくらと\ruby{大}{おお}きく\ruby{艶}{あで}やかに\ruby{蕾}{つぼみ}が\ruby{育}{そだ}っていくのを\ruby{見}{み}て、\ruby{王子}{おうじ}さまは\ruby{奇跡}{きせき}のようなものが\ruby{現}{あらわ}れてくるのを\ruby{感}{かん}じていた。

しかし\ruby{花}{はな}は\ruby{緑}{みどり}の\ruby{部屋}{へや}に\ruby{隠}{かく}れたまま、\ruby{美}{うつく}しい\ruby{装}{よそお}いにかかりきりだった。\ruby{慎重}{しんちょう}に\ruby{色}{いろ}を\ruby{選}{えら}び、ゆっくり\ruby{衣装}{いしょう}を\ruby{纏}{まと}い、\ruby{花}{はな}びらを\ruby{一枚}{いちまい}ずつ\ruby{整}{ととの}える。\ruby{雛罌粟}{ひなげし}のように\ruby{皺}{しわ}くちゃな\ruby{姿}{すがた}は\ruby{見}{み}せたくなかった。これ\ruby{以上}{いじょう}はない\ruby{輝}{かがや}きを\ruby{放}{はな}つ\ruby{美}{うつく}しい\ruby{姿}{すがた}で\ruby{華麗}{かれい}に\ruby{登場}{とうじょう}したかった。そう、\ruby{花}{はな}はとてもお\ruby{洒落}{しゃれ}だった。

\ruby{謎}{なぞ}めいた\ruby{準備}{じゅんび}は\ruby{何日}{なんにち}も\ruby{続}{つづ}いた。そしてある\ruby{朝}{あさ}、ぴったり\ruby{日}{ひ}の\ruby{出}{で}の\ruby{時間}{じかん}に、\ruby{花}{はな}は\ruby{姿}{すがた}を\ruby{現}{あらわ}した。

そして、あれほど\ruby{念入}{ねんい}りに\ruby{装}{よそお}いを\ruby{凝}{こ}らしておきながら、\ruby{欠伸}{あくび}を\ruby{噛}{か}み\ruby{殺}{ころ}してこう\ruby{言}{い}った。

ああ、たった\ruby{今}{いま}\ruby{目}{め}が\ruby{覚}{さ}めたばかり、ごめんなさいね。\ruby{髪}{かみ}がぼそぼそだわ。

しかし\ruby{王子}{おうじ}さまは\ruby{感動}{かんどう}を\ruby{抑}{おさ}える\ruby{事}{こと}ができなかった。

なんて\ruby{綺麗}{きれい}なんだ、\ruby{君}{きみ}は。

でしょう?

\ruby{花}{はな}は\ruby{静}{しず}かに\ruby{答}{こた}えた。

\ruby{私}{わたし}はお\ruby{日様}{ひさま}と\ruby{一緒}{いっしょ}に\ruby{生}{う}まれたんですもの。

\ruby{王子}{おうじ}さまは\ruby{花}{はな}があまり\ruby{謙虚}{けんきょ}ではない\ruby{事}{こと}に\ruby{気付}{きづ}いたが、それでも\ruby{目}{め}が\ruby{眩}{くら}むほど\ruby{美}{うるわ}しかった。

そろそろ\ruby{朝食}{ちょうしょく}のお\ruby{時間}{じかん}ね、お\ruby{願}{ねが}いしてもよろしいかしら?

\ruby{王子}{おうじ}さまはすっかりドギマギしていたが、\ruby{如雨露}{じょうろ}に\ruby{新鮮}{しんせん}な\ruby{水}{みず}を\ruby{汲}{く}んできて、たっぷり\ruby{花}{はな}にかけてあげた。\ruby{花}{はな}はすぐに\ruby{気}{き}まぐれな\ruby{自惚}{うぬぼ}れで\ruby{王子}{おうじ}さまを\ruby{困}{こま}らせるようになった。\ruby{例}{たと}えばある\ruby{日}{ひ}、\ruby{自分}{じぶん}の四\ruby{本}{ほん}の\ruby{刺}{とげ}の\ruby{話}{はなし}をしながらこう\ruby{言}{い}った。

たとえ\ruby{虎}{とら}が\ruby{来}{き}ても\ruby{大丈夫}{だいじょうぶ}よ。\ruby{鋭}{するど}い\ruby{爪}{つめ}で。。

\ruby{僕}{ぼく}の\ruby{星}{ほし}には\ruby{虎}{とら}はいないよ。それに、\ruby{虎}{とら}は\ruby{草}{くさ}を\ruby{食}{た}べないし。

\ruby{私}{わたし}、\ruby{草}{くさ}ではないんですけど。

ごめんなさい。

\ruby{虎}{とら}なんかちっとも\ruby{怖}{こわ}くないけれど、\ruby{風}{かぜ}が\ruby{吹}{ふ}き\ruby{込}{こ}むのは\ruby{苦手}{にがて}なの。あなた、\ruby{衝立}{ついたて}はないのかしら。

\ruby{風}{かぜ}が\ruby{吹}{ふ}き\ruby{込}{こ}むのが\ruby{苦手}{にがて}だなんて、\ruby{植物}{しょくぶつ}なのに、\ruby{困}{こま}った\ruby{事}{こと}だな。この\ruby{花}{はな}は\ruby{結構}{けっこう}\ruby{気難}{きむずか}し\ruby{屋}{おく}さんだぞ。

\ruby{暗}{くら}くなったら、ガラスの\ruby{覆}{おお}いを\ruby{被}{かぶ}せてちょうだい?この\ruby{星}{ほし}はとても\ruby{寒}{さむ}いわ。\ruby{作}{つく}りが\ruby{悪}{わる}いのね。\ruby{前}{まえ}に\ruby{私}{わたし}がいた\ruby{所}{ところ}は。。。

\ruby{花}{はな}はいきなり\ruby{口}{ぐち}を\ruby{噤}{つぐ}んだ。\ruby{種}{たね}の\ruby{状態}{じょうたい}で\ruby{来}{き}たのだから、\ruby{他}{ほか}の\ruby{世界}{せかい}の\ruby{事}{こと}など\ruby{何一}{なにひと}つ\ruby{知}{し}っているはずがない。\ruby{花}{はな}はすぐにばれる\ruby{嘘}{うそ}をついてしまった\ruby{事}{こと}が\ruby{恥}{は}ずかしくて、\ruby{悪}{わる}いのは\ruby{王子}{おうじ}さまのせいにしようと、二\ruby{度}{ど}三\ruby{度}{ど}せきをしたで、\ruby{衝立}{ついたて}は?

\ruby{探}{さが}しに\ruby{行}{い}こうとしていたら、\ruby{君}{きみ}が\ruby{話}{はな}しかけてきたんでしょう。

すると\ruby{花}{はな}はわざとまたせきをして\ruby{王子}{おうじ}さまの\ruby{良心}{りょうしん}を\ruby{疼}{うず}かせた。

こうして\ruby{王子}{おうじ}さまは\ruby{心}{こころ}から\ruby{愛}{あい}していたにも\ruby{関}{かか}わらず、じきに\ruby{花}{はな}の\ruby{事}{こと}を\ruby{信用}{しんよう}できなくなっていった。\ruby{些細}{ささい}な\ruby{言葉}{ことば}を\ruby{一一}{いちいち}\ruby{深刻}{しんこく}に\ruby{受}{う}け\ruby{止}{と}め、そのたびに\ruby{不幸}{ふこう}になった。

\ruby{花}{はな}の\ruby{言}{い}う\ruby{事}{こと}なんか、\ruby{聞}{き}かないほうがよかったんだよ。ただ\ruby{眺}{なが}めたり、\ruby{香}{かお}りを\ruby{楽}{たの}しんでいればいいんだ。あの\ruby{花}{はな}は\ruby{僕}{ぼく}の\ruby{星}{ほし}をいい\ruby{香}{かお}りで\ruby{満}{み}たしてくれた。それなのに\ruby{僕}{ぼく}はそれを\ruby{楽}{たの}しめなかった。\ruby{虎}{とら}の\ruby{爪}{つめ}の\ruby{話}{はなし}にしても、\ruby{僕}{ぼく}はうんざりしたけれど、\ruby{花}{はな}にして\ruby{見}{み}れば、ほろりとさせるつもりだったのかもしれない。あの\ruby{頃}{ころ}の\ruby{僕}{ぼく}は\ruby{何}{なに}もわかっていなかったんだね。\ruby{言葉}{ことば}ではなく、\ruby{振}{ふ}る\ruby{舞}{ま}いで\ruby{判断}{はんだん}しなくちゃいけなかったんだ。\ruby{花}{はな}は\ruby{僕}{ぼく}の\ruby{星}{ほし}をいい\ruby{香}{かお}りで\ruby{満}{み}たし、\ruby{明}{あか}るくしてくれた。\ruby{僕}{ぼく}は\ruby{逃}{に}げちゃいけなかったんだ。つまらない\ruby{見}{み}せかけに\ruby{隠}{かく}れた\ruby{花}{はな}の\ruby{優}{やさ}しさに\ruby{気付}{きづ}くべきだった。\ruby{花}{はな}って\ruby{本当}{ほんとう}に\ruby{矛盾}{むじゅん}しているからね、でも\ruby{僕}{ぼく}はまだ\ruby{子供}{こども}で、あの\ruby{花}{はな}の\ruby{愛}{あい}し\ruby{方}{かた}がわからなかったんだ。




\chapter{第九章}
お\ruby{教室}{きょうしつ}が\ruby{本当}{ほんとう}の\ruby{電車}{でんしゃ}で、“かわってる”と\ruby{思}{おも}ったトットちゃんが、\ruby{次}{つぎ}に“かわってる”と\ruby{思}{おも}ったのは、\ruby{教室}{きょうしつ}で\ruby{座}{すわ}る\ruby{場所}{ばしょ}だった。\ruby{前}{まえ}の学校は、\ruby{誰}{だれ}かさんは、どの\ruby{机}{つくえ}、\ruby{隣}{となり}は\ruby{誰}{だれ}、\ruby{前}{まえ}は\ruby{誰}{だれ}、と\ruby{決}{き}まっていた。ところが、この学校は、どこでも、\ruby{次}{つぎ}の日の\ruby{気分}{きぶん}や\ruby{都合}{つごう}で、\ruby{毎日}{まいにち}、\ruby{好}{す}きなところに\ruby{座}{すわ}っていいのだった。

そこでトットちゃんは、さんざん\ruby{考}{かんが}え、そして\ruby{見回}{みまわ}したあげく、\ruby{朝}{あさ}、トットちゃんの\ruby{次}{つぎ}に\ruby{教室}{きょうしつ}に入ってきた女の子の\ruby{隣}{となり}に\ruby{座}{すわ}ることに\ruby{決}{き}めた。なぜなら、この子が、\ruby{長}{なが}い耳をした\ruby{兎}{うさぎ}の\ruby{絵}{え}のついた、ジャンパースカートをはいていたからだった。

でも、なによりも“かわっていた”のは、この学校の、\ruby{授業}{じゅぎょう}のやりかただった。

\ruby{普通}{ふつう}の学校は、一\ruby{時間}{じかん}目が\ruby{国語}{こくご}なら、\ruby{国語}{こくご}をやって、二\ruby{時間}{じかん}目が\ruby{算数}{さんすう}なら、\ruby{算数}{さんすう}、という\ruby{風}{かぜ}に、\ruby{時間}{じかん}\ruby{割}{わり}の\ruby{通}{とお}りの\ruby{順番}{じゅんばん}なのだけど、この学校は、まるっきり\ruby{違}{ちが}っていた。

\ruby{何}{なに}しろ、一\ruby{時間}{じかん}目が\ruby{始}{はじ}まるときに、その日、一日やる\ruby{時間}{じかん}\ruby{割}{わり}の、\ruby{全部}{ぜんぶ}の\ruby{科目}{かもく}の\ruby{問題}{もんだい}を、女の先生が、\ruby{黒板}{こくばん}にいっぱいに\ruby{書}{か}いちゃって、

「さあ、どれでも\ruby{好}{す}きなのから、\ruby{始}{はじ}めてください」

といったんだ。だから\ruby{生徒}{せいと}は、\ruby{国語}{こくご}であろうと、\ruby{算数}{さんすう}であろうと、\ruby{自分}{じぶん}の\ruby{好}{す}きなのから\ruby{始}{はじ}めていっこうに、かまわないのだった。だから、\ruby{作文}{さくぶん}の\ruby{好}{す}きな子が、\ruby{作文}{さくぶん}を\ruby{書}{か}いていると、\ruby{後}{うし}ろでは、\ruby{物理}{ぶつり}の\ruby{好}{す}きな子が、アルコールランプに火をつけて、フラスコをブクブクやったり、\ruby{何}{なに}かを\ruby{爆発}{ばくはつ}させてる、なんていう\ruby{光景}{こうけい}は、どの\ruby{教室}{きょうしつ}でもみられることだった。この\ruby{授業}{じゅぎょう}のやり\ruby{方}{かた}は、\ruby{上級}{じょうきゅう}になるにしたがって、その\ruby{子供}{こども}の\ruby{興味}{きょうみ}を\ruby{持}{も}っているもの、\ruby{興味}{きょうみ}の\ruby{持}{も}ち\ruby{方}{かた}、\ruby{物}{もの}の\ruby{考}{かんが}え\ruby{方}{かた}、そして、\ruby{個性}{こせい}、といったものが、先生に、はっきり\ruby{分}{わ}かってくるから、先生にとって、\ruby{生徒}{せいと}を\ruby{知}{し}る上で、\ruby{何}{なに}よりの\ruby{勉強法}{べんきょうほう}だった。

また、\ruby{生徒}{せいと}にとっても、\ruby{好}{す}きな\ruby{学科}{がっか}からやっていい、というのは、\ruby{嬉}{うれ}しいことだったし、\ruby{嫌}{きら}いな\ruby{学科}{がっか}にしても、学校が\ruby{終}{お}わる\ruby{時間}{じかん}までに、やればいいのだから、\ruby{何}{なん}とか、やりくり\ruby{出来}{でき}た。\ruby{従}{したが}って、\ruby{自習}{じしゅう}の\ruby{形式}{けいしき}が\ruby{多}{おお}く、いよいよ、\ruby{分}{わ}からなくなってくると、先生のところに\ruby{聞}{き}きに\ruby{行}{い}くか、\ruby{自分}{じぶん}の\ruby{席}{せき}に先生に\ruby{来}{き}ていただいて、\ruby{納得}{なっとく}の\ruby{行}{い}くまで、\ruby{教}{おし}えてもらう。そして、\ruby{例}{れい}\ruby{題}{だい}をもらって、また\ruby{自習}{じしゅう}に入る。これは\ruby{本当}{ほんとう}の\ruby{勉強}{べんきょう}だった。だから、先生の\ruby{話}{はなし}や\ruby{説明}{せつめい}を、ボンヤリ\ruby{聞}{き}く、といった\ruby{事}{こと}は、\ruby{無}{な}いにひとしかった。トットちゃん\ruby{達}{たち}、一年生は、まだ\ruby{自習}{じしゅう}をするほどの\ruby{勉強}{べんきょう}を\ruby{始}{はじ}めていなかったけど、それでも、\ruby{自分}{じぶん}の\ruby{好}{す}きな\ruby{科目}{かもく}から\ruby{勉強}{べんきょう}する、ということには、かわりなかった。カタカナを\ruby{書}{か}く子、\ruby{絵}{え}を\ruby{描}{か}く子。本を\ruby{読}{よ}んでる子。中には、\ruby{体操}{たいそう}をしている子もいた。トットちゃんの\ruby{隣}{となり}の女の子は、もう、ひらがなが\ruby{書}{か}けるらしく、ノートに\ruby{写}{うつ}していた。トットちゃんは、\ruby{何}{なに}もかもが\ruby{珍}{めずら}しくて、ワクワクしちゃって、みんなみたいに、すぐ\ruby{勉強}{べんきょう}、というわけにはいかなかった。そんな\ruby{時}{とき}、トットちゃんの\ruby{後}{うし}ろの\ruby{机}{つくえ}の男の子が立ち上がって、\ruby{黒板}{こくばん}のほうに\ruby{歩}{ある}き\ruby{出}{だ}した。ノートを\ruby{持}{も}って。\ruby{黒板}{こくばん}の\ruby{横}{よこ}の\ruby{机}{つくえ}で、\ruby{他}{ほか}の子に\ruby{何}{なに}かを\ruby{教}{おし}えている先生のところに\ruby{行}{い}くらしかった。その子の\ruby{歩}{ある}くのを、\ruby{後}{うし}ろから見たトットちゃんは、それまでキョロキョロしてた\ruby{動作}{どうさ}をピタリと\ruby{止}{と}めて、\ruby{頬杖}{ほおづえ}をつき、ジーっと、その子を見つめた。その子は、\ruby{歩}{ある}くとき、足を\ruby{引}{ひ}きずっていた。とっても、\ruby{歩}{ある}くとき、\ruby{体}{からだ}が\ruby{揺}{ゆ}れた。\ruby{始}{はじ}めは、わざとしているのか、と\ruby{思}{おも}ったくらいだった。でも、やっぱり、わざとじゃなくて、そういう\ruby{風}{かぜ}になっちゃうんだ、と、しばらく見ていたトットちゃんに\ruby{分}{わ}かった。その子が、\ruby{自分}{じぶん}の\ruby{机}{つくえ}に\ruby{戻}{もど}ってくるのを、トットちゃんは、さっきの、\ruby{頬杖}{ほおづえ}のまま、見た。目と目が\ruby{合}{あ}った。その男の子は、トットちゃんを見ると、ニコリと\ruby{笑}{わら}った。トットちゃんも、あわてて、ニコリとした。その子が、\ruby{後}{うし}ろの\ruby{席}{せき}に\ruby{座}{すわ}ると、――\ruby{座}{すわ}るのも、\ruby{他}{ほか}の子より、\ruby{時間}{じかん}がかかったんだけど――トットちゃんは、クルリと\ruby{振}{ふ}り\ruby{向}{む}いて、その子に\ruby{聞}{き}いた。「どうして、そんな\ruby{風}{ふう}に\ruby{歩}{ある}くの?」その子は、\ruby{優}{やさ}しい\ruby{声}{こえ}で\ruby{静}{しず}かに\ruby{答}{こた}えた。とても\ruby{利口}{りこう}そうな\ruby{声}{こえ}だった。「\ruby{僕}{ぼく}、\ruby{小児}{しょうに}\ruby{麻痺}{まひ}なんだ」「しょうにまひ?」トットちゃんは、それまで、そういう\ruby{言葉}{ことば}を\ruby{聴}{き}いたことが\ruby{無}{な}かったから、\ruby{聞}{き}き\ruby{返}{かえ}した。その子は、\ruby{少}{すこ}し小さい\ruby{声}{こえ}でいった。「そう、\ruby{小児}{しょうに}\ruby{麻痺}{まひ}。足だけじゃないよ。手だって……」そういうと、その子は、\ruby{長}{なが}い\ruby{指}{ゆび}と\ruby{指}{ゆび}が、くっついて、\ruby{曲}{ま}がったみたいになった手を出した。トットちゃんは、その左手を見ながら、「\ruby{直}{なお}らないの?」と\ruby{心配}{しんぱい}になって\ruby{聞}{き}いた。その子は、\ruby{黙}{だま}っていた。トットちゃんは、\ruby{悪}{わる}いことを\ruby{聞}{き}いたのかと\ruby{悲}{かな}しくなった。すると、その子は、\ruby{明}{あか}るい\ruby{声}{こえ}で\ruby{言}{い}った。「\ruby{僕}{ぼく}の\ruby{名前}{なまえ}は、やまもとやすあき。\ruby{君}{きみ}は?」トットちゃんは、その子が\ruby{元気}{げんき}な\ruby{声}{こえ}を出したので、\ruby{嬉}{うれ}しくなって、大きな\ruby{声}{こえ}で\ruby{言}{い}った。「トットちゃんよ」こうして、山本\ruby{泰明}{やすあき}ちゃんと、トットちゃんのお\ruby{友達}{ともだち}づきあいが\ruby{始}{はじ}まった。\ruby{電車}{でんしゃ}の中は、\ruby{暖}{あたた}かい\ruby{日差}{ひざ}しで、\ruby{暑}{あつ}いくらいだった。\ruby{誰}{だれ}かが、\ruby{窓}{まど}を\ruby{開}{ひら}けた。\ruby{新}{あたら}しい\ruby{春}{はる}の\ruby{風}{かぜ}が、\ruby{電車}{でんしゃ}の中を\ruby{通}{とお}り\ruby{抜}{ぬ}け、\ruby{子供}{こども}たちの\ruby{髪}{かみ}の\ruby{毛}{け}が\ruby{歌}{うた}っているように、とびはねた。トットちゃんの、トモエでの\ruby{第}{だい}一目は、こんな\ruby{風}{ふう}に\ruby{始}{はじ}まったのだった。




\chapter{第十章}
\ruby{王子}{おうじ}さまは\ruby{小}{しょう}\ruby{惑星}{わくせい}325、326、327、328、329、330の\ruby{近}{ちか}くを\ruby{通}{とお}りかかった。そこで\ruby{仕事}{しごと}を\ruby{探}{さが}したり、\ruby{見聞}{けんぶん}を\ruby{広}{ひろ}げるため、それらの\ruby{小}{しょう}\ruby{惑星}{わくせい}を\ruby{一}{ひと}つずつ\ruby{訪}{たず}ねる\ruby{事}{こと}にした。\ruby{最初}{さいしょ}の\ruby{星}{ほし}には\ruby{王様}{おうさま}が\ruby{住}{す}んでいた。\ruby{緋色}{ひいろ}の\ruby{衣}{ころも}に\ruby{白点}{はくてん}の\ruby{毛皮}{けがわ}を\ruby{纏}{まと}い、\ruby{質素}{しっそ}だが、\ruby{威厳}{いげん}のある\ruby{玉座}{ぎょくざ}に\ruby{腰掛}{こしか}けていた。

\ruby{王子}{おうじ}\ruby{様}{さま}を\ruby{見}{み}かけると、\ruby{大}{おお}きな\ruby{声}{こえ}で\ruby{言}{い}いました。

「や、\ruby{家来}{けらい}が\ruby{来}{き}たなあ!」

\ruby{王子}{おうじ}\ruby{様}{さま}は、\ruby{一度}{いちど}も\ruby{僕}{ぼく}に\ruby{会}{あ}ったことがないのに、どうして\ruby{見}{み}\ruby{覚}{おぼ}えがあるのだろうと\ruby{考}{かんが}えました。\ruby{王様}{おうさま}にかかれば、\ruby{世界}{せかい}はとてもあっさりしたものになる。\ruby{誰}{だれ}も\ruby{彼}{かれ}もみんな、\ruby{家来}{けらい}。\ruby{王子}{おうじ}\ruby{様}{さま}はそれを\ruby{知}{し}らなかったんだ。

「\ruby{近}{ちか}く\ruby{寄}{よ}りなさい。そのほうがもっとよく\ruby{見}{み}えるように。」

\ruby{王様}{おうさま}はやっと\ruby{誰}{だれ}かに\ruby{王様}{おうさま}らしくできると、\ruby{嬉}{うれ}しくてたまらなかった。

\ruby{王子}{おうじ}\ruby{様}{さま}はどこかに\ruby{座}{すわ}ろうと、\ruby{周}{まわ}りを\ruby{見}{み}た。でも、\ruby{星}{ほし}は\ruby{大}{おお}きな\ruby{毛皮}{けがわ}の\ruby{裾}{すそ}で、どこもいっぱいだった。\ruby{王子}{おうじ}\ruby{様}{さま}は\ruby{仕方}{しかた}なく\ruby{立}{た}ちっぱなし、しかもへとへとだったから、あくびが\ruby{出}{で}た。

「\ruby{王}{おう}の\ruby{前}{まえ}であくびとは、\ruby{作法}{さほう}がなっとらん!」と、\ruby{王様}{おうさま}は\ruby{言}{い}った。「ダメであるぞ!」

「\ruby{我慢}{がまん}できないんです。」と、\ruby{王子}{おうじ}\ruby{様}{さま}は\ruby{迷惑}{めいわく}そうに\ruby{返事}{へんじ}をした。「\ruby{僕}{ぼく}、\ruby{長}{なが}い\ruby{旅}{たび}をしてきたんでしょう?それに、\ruby{眠}{ねむ}らなかったものですから…」

「そうか。では、あくびをしなさい。\ruby{命令}{めいれい}する。わしはもう\ruby{何年}{なんねん}か\ruby{人}{ひと}のあくびをするのを\ruby{見}{み}たことがない。あくびというものは\ruby{面白}{おもしろ}いものだなあ。さあ、あくびしなさい、もう\ruby{一度}{いちど}、\ruby{命令}{めいれい}じゃ。」

「\ruby{胸}{むね}がドキドキして、もうできなくなりました。」と、\ruby{王子}{おうじ}\ruby{様}{さま}は、\ruby{顔}{かお}を\ruby{真}{ま}っ\ruby{赤}{か}にした。

「これはこれは…では、こう\ruby{命令}{めいれい}する。あるときはあくびをし、あるときは…」

\ruby{王様}{おうさま}は\ruby{何}{なに}か\ruby{口}{ぐち}の\ruby{中}{なか}でもぐもぐ\ruby{言}{い}って、\ruby{気}{き}を\ruby{揉}{も}んでいる\ruby{様子}{ようす}でした。

なぜなら、\ruby{王様}{おうさま}はなんでも\ruby{自分}{じぶん}の\ruby{思}{おも}い\ruby{通}{どお}りにしたくて、そこから\ruby{外}{はず}れるものは\ruby{許}{ゆる}せなかった。いわゆる、\ruby{絶対}{ぜったい}の\ruby{王様}{おうさま}ってやつ。でも、\ruby{根}{ね}は\ruby{優}{やさ}しかったので、\ruby{物分}{ものわか}りのいいことしか\ruby{言}{い}いつけなかった。

\ruby{王様}{おうさま}にはこんな\ruby{口}{くち}\ruby{癖}{ぐせ}がある。

「わしが\ruby{大将}{たいしょう}に\ruby{海}{うみ}の\ruby{鳥}{とり}になれと\ruby{命令}{めいれい}したとする。その\ruby{大将}{たいしょう}がわしの\ruby{命令}{めいれい}に\ruby{従}{したが}わないとしても、\ruby{大将}{たいしょう}がいけないわけではないだろう。わしがいけないのだろう。」

「\ruby{座}{すわ}っていい?」と、\ruby{王子}{おうじ}\ruby{様}{さま}は\ruby{気}{き}まずそうに\ruby{言}{い}った。

「うん、\ruby{座}{すわ}んなさい、\ruby{命令}{めいれい}する。」\ruby{王様}{おうさま}は\ruby{毛皮}{けがわ}の\ruby{裾}{すそ}を\ruby{厳}{おごそ}かに\ruby{引}{ひ}いて、\ruby{言}{い}いつけた。

でも、\ruby{王子}{おうじ}\ruby{様}{さま}にはよくわからないことがあった。この\ruby{星}{ほし}はすごくちいちゃい、\ruby{王様}{おうさま}は\ruby{一体}{いったい}、\ruby{何}{なに}を\ruby{治}{おさ}めてるんだろうか。

「\ruby{陛下}{へいか}、すいませんが、\ruby{質問}{しつもん}が…」

「\ruby{訪}{たず}ねなさい、\ruby{命令}{めいれい}する!」と、\ruby{王様}{おうさま}は\ruby{慌}{あわ}てて\ruby{言}{い}った。

「\ruby{陛下}{へいか}は\ruby{何}{なに}を\ruby{治}{おさ}めてるんですか。」

「すべてである。」と、\ruby{王様}{おうさま}は\ruby{当}{あ}たり\ruby{前}{まえ}のように\ruby{答}{こた}えた。

「すべて?」

\ruby{王様}{おうさま}はそっと\ruby{指}{ゆび}を\ruby{出}{だ}して、\ruby{自分}{じぶん}の\ruby{星}{ほし}と、ほかの\ruby{惑星}{わくせい}とか\ruby{星}{ほし}とか、みんなを\ruby{指}{さ}した。

「あれをみんな?」と、\ruby{王子}{おうじ}\ruby{様}{さま}は\ruby{言}{い}った。

「うん、あれをみんな。」と、\ruby{王様}{おうさま}は\ruby{答}{こた}えた。なぜなら、\ruby{絶対}{ぜったい}の\ruby{王様}{おうさま}であるだけでなく、\ruby{宇宙}{うちゅう}の\ruby{王様}{おうさま}でもあったからだ。

「じゃあ、\ruby{星}{ほし}はみんな、\ruby{陛下}{へいか}にしたがっているわけですね?」

「そうだとも。すぐにも\ruby{従}{したが}う。わしは\ruby{不}{ふ}\ruby{規律}{きりつ}を\ruby{許}{ゆる}さんのじゃ。」

あまりにすごい\ruby{力}{ちから}なので、\ruby{王子}{おうじ}\ruby{様}{さま}はびっくりした。\ruby{自分}{じぶん}にもしそれだけの\ruby{力}{ちから}があれば、40\ruby{回}{かい}と\ruby{言}{い}わず、72\ruby{回}{かい}、いや、100\ruby{回}{かい}でも、いやいや、200\ruby{回}{かい}でも、\ruby{夕暮}{ゆうぐ}れがたった一\ruby{日}{にち}の\ruby{間}{あいだ}に\ruby{見}{み}られるんじゃないか。しかも、\ruby{椅子}{いす}も\ruby{動}{うご}かずに。

そう\ruby{考}{かんが}えたとき、ちょっと\ruby{切}{せつ}なくなった。そういえば、\ruby{自分}{じぶん}の\ruby{小}{ちい}さな\ruby{星}{ほし}を\ruby{捨}{す}ててきたんだって。だから、\ruby{思}{おも}い\ruby{切}{き}って、\ruby{王様}{おうさま}にお\ruby{願}{ねが}いをしてみた。

「\ruby{夕暮}{ゆうぐ}れが\ruby{見}{み}たいんです。どうかお\ruby{願}{ねが}いします。\ruby{夕暮}{ゆうぐ}れろって\ruby{言}{い}ってください。」

「わしが\ruby{大将}{たいしょう}に\ruby{向}{む}かって、\ruby{蝶々}{ちょうちょ}みたいに\ruby{花}{はな}から\ruby{花}{はな}へ\ruby{飛}{と}べとか、\ruby{悲劇}{ひげき}を\ruby{書}{か}けとか、\ruby{海}{うみ}の\ruby{鳥}{とり}になれとか、\ruby{命令}{めいれい}するとする。そして、その\ruby{大将}{たいしょう}が\ruby{命令}{めいれい}を\ruby{実行}{じっこう}しないとしたら、\ruby{大将}{たいしょう}とわしと、どっちが\ruby{間違}{まちが}ってるだろうかね。」

「\ruby{王様}{おうさま}の\ruby{方}{ほう}です。」と、\ruby{王子}{おうじ}\ruby{様}{さま}はきっぱり\ruby{言}{い}った。

「その\ruby{通}{とお}り。\ruby{人}{ひと}には\ruby{銘々}{めいめい}その\ruby{人}{ひと}のできることをしてもらわなきゃならん。\ruby{道理}{どうり}の\ruby{土台}{どだい}あっての\ruby{権力}{けんりょく}じゃ。もし、お\ruby{前}{まえ}が\ruby{人民}{じんみん}たちに、\ruby{海}{うみ}に\ruby{行}{い}って\ruby{飛}{と}び\ruby{込}{こ}めと\ruby{命令}{めいれい}したら、\ruby{人民}{じんみん}たちは\ruby{革命}{かくめい}を\ruby{起}{お}こすだろう。わしは\ruby{無理}{むり}の\ruby{命令}{めいれい}をしないのだから、みんなをわしに\ruby{服従}{ふくじゅう}させる\ruby{権力}{けんりょく}があるのじゃ。」

「じゃあ、\ruby{僕}{ぼく}の\ruby{夕暮}{ゆうぐ}れは?」と、\ruby{王子}{おうじ}\ruby{様}{さま}は\ruby{迫}{せま}った。なぜなら、\ruby{王子}{おうじ}\ruby{様}{さま}は\ruby{一度}{いちど}\ruby{聞}{き}いたことは\ruby{絶対}{ぜったい}\ruby{忘}{わす}れない。

「うーん、\ruby{夕日}{ゆうひ}は\ruby{見}{み}せてあげる。わしが\ruby{命令}{めいれい}してやる。だが、\ruby{都合}{つごう}がよくなるまで、\ruby{待}{ま}つとしよう。それがわしの\ruby{政治}{せいじ}のことじゃ。」

「それはいつ?」と、\ruby{王子}{おうじ}\ruby{様}{さま}は\ruby{尋}{たず}ねる。

「うーん…」と、\ruby{王様}{おうさま}は\ruby{言}{い}って、\ruby{分厚}{ぶあつ}い\ruby{暦}{こよみ}を\ruby{調}{しら}べた。「うーん、そうだなあ。\ruby{大体}{だいたい}、\ruby{午後}{ごご}7\ruby{時}{じ}40\ruby{分}{ふん}ぐらいである。まあ、\ruby{見}{み}ていなさい、\ruby{万事}{ばんじ}わしの\ruby{命令}{めいれい}\ruby{通}{どお}りになるから。」

\ruby{王子}{おうじ}\ruby{様}{さま}はあくびをした。\ruby{夕暮}{ゆうぐ}れに\ruby{会}{あ}えなくて、\ruby{残念}{ざんねん}だった。それに、ちょっともううんざりだった。

「ここですることはもうないから。」と、\ruby{王子}{おうじ}\ruby{様}{さま}は\ruby{王様}{おうさま}に\ruby{言}{い}った。「そろそろ\ruby{行}{い}くよ。」

「\ruby{行}{い}くな、\ruby{行}{い}くな!」と、\ruby{王様}{おうさま}は\ruby{言}{い}った。\ruby{家来}{けらい}ができて、それだけ\ruby{嬉}{うれ}しかったんだ。

「\ruby{行}{い}ってはならん!そちを\ruby{大臣}{だいじん}にしてやるぞ!」

「それで\ruby{何}{なに}をするの?」

「うーん、\ruby{人}{ひと}を\ruby{裁}{さば}くであるぞ!」

「でも、\ruby{裁}{さば}くにしても、\ruby{人}{ひと}がいないよ。」

「そりゃ\ruby{分}{わ}からん。わしはまだ、わしの\ruby{国}{くに}を\ruby{回}{まわ}ってみたことがないんでね。\ruby{年}{とし}を\ruby{取}{と}ったし、\ruby{馬車}{ばしゃ}を\ruby{置}{お}く\ruby{場所}{ばしょ}がないんで、\ruby{歩}{ある}くのが\ruby{疲}{つか}れるよ。」

「うーん~でも\ruby{僕}{ぼく}はもう\ruby{見}{み}たよ。」と、\ruby{王子}{おうじ}\ruby{様}{さま}は\ruby{屈}{かが}んで、もう\ruby{一度}{いちど}チラリっと\ruby{星}{ほし}の\ruby{向}{む}こう\ruby{側}{がわ}を\ruby{見}{み}た。「あっちには\ruby{人}{ひと}っ\ruby{子一人}{こひとり}いない。」

「なら、\ruby{自分}{じぶん}を\ruby{裁}{さば}くである。」と、\ruby{王様}{おうさま}は\ruby{答}{こた}えた。「もっと\ruby{難}{むずか}しいぞ、\ruby{自分}{じぶん}を\ruby{裁}{さば}くほうが、\ruby{人}{ひと}を\ruby{裁}{さば}くよりも、はるかに\ruby{難}{むずか}しい。うまく\ruby{自分}{じぶん}を\ruby{裁}{さば}くことができたなら、それは、\ruby{正真}{しょうしん}\ruby{正銘}{しょうめい}\ruby{賢者}{けんじゃ}の\ruby{証}{あかし}だ。」

すると、\ruby{王子}{おうじ}\ruby{様}{さま}は\ruby{言}{い}った。

「\ruby{僕}{ぼく}、どこにいたって、\ruby{自分}{じぶん}を\ruby{裁}{さば}けます。ここに\ruby{住}{す}む\ruby{必要}{ひつよう}はありません。」

「ええとね、わしの\ruby{星}{ほし}には、\ruby{年}{とし}とったねずみがどこかにいるようじゃ。\ruby{夜}{よる}、\ruby{物}{もの}\ruby{音}{おと}がするからな。そのヨボヨボのねずみを\ruby{裁}{さば}けばよい。ときとき、\ruby{死刑}{しけい}にするんである。そうすれば、その\ruby{命}{いのち}はそちの\ruby{裁}{さば}き\ruby{次第}{しだい}である。だが、いつも\ruby{許}{ゆる}してやることだ。\ruby{一匹}{いっぴき}しかいないねずみなんだからね。」

また、\ruby{王子}{おうじ}\ruby{様}{さま}は\ruby{返事}{へんじ}をする。

「\ruby{僕}{ぼく}、\ruby{死刑}{しけい}にするの\ruby{嫌}{きら}いだし、もう、さっさと\ruby{行}{い}きたいんです。」

「ならん!」と、\ruby{王様}{おうさま}は\ruby{言}{い}う。

もう、\ruby{王子}{おうじ}\ruby{様}{さま}はいつでも\ruby{行}{い}けたんだけど、\ruby{年寄}{としよ}りの\ruby{王様}{おうさま}をしょんぼりさせたくなかった。

「もし\ruby{陛下}{へいか}が、\ruby{言}{い}う\ruby{通}{とお}りになるのをお\ruby{望}{のぞ}みなら、\ruby{物分}{ものわか}りのいいことを\ruby{言}{い}いつけられるはずです。ほら、\ruby{一分}{いっぷん}\ruby{以内}{いない}に\ruby{出発}{しゅっぱつ}せよ、とか。\ruby{僕}{ぼく}には、\ruby{都合良}{つごうよ}くなっているように\ruby{思}{おも}うんですけど。」

\ruby{王様}{おうさま}は\ruby{何}{なに}も\ruby{言}{い}わかなった。

\ruby{王子}{おうじ}\ruby{様}{さま}はどうしようかと\ruby{思}{おも}ったけど、ため\ruby{息}{いき}をついて、ついに\ruby{星}{ほし}を\ruby{後}{あと}にした。

「そちをほかの\ruby{星}{ほし}へ\ruby{使}{つか}わせるぞ!」そのとき、\ruby{王様}{おうさま}は\ruby{慌}{あわ}ててこう\ruby{言}{い}った。まったくもって、\ruby{偉}{えら}そうな\ruby{言}{い}い\ruby{方}{かた}だった。

\ruby{大人}{おとな}の\ruby{人}{ひと}って、\ruby{相当}{そうとう}\ruby{変}{か}わってるなあ。と、\ruby{王子}{おうじ}\ruby{様}{さま}は\ruby{旅}{たび}を\ruby{続}{つづ}けながら、そう\ruby{思}{おも}った。




\chapter{第十一章}
\ruby{二番目}{にばんめ}の\ruby{星}{ほし}には\ruby{自惚}{うぬぼ}れ\ruby{男}{おとこ}が\ruby{住}{す}んでいた。

「やあやあ、\ruby{俺}{おれ}に\ruby{感心}{かんしん}している\ruby{人間}{にんげん}がやってきたなあ!」と、\ruby{自惚}{うぬぼ}れ\ruby{男}{おとこ}は\ruby{王子}{おうじ}\ruby{様}{さま}を\ruby{見}{み}かけたなり、\ruby{遠}{とお}くから\ruby{叫}{さけ}んだ。

\ruby{自惚}{うぬぼ}れ\ruby{男}{おとこ}の\ruby{目}{め}から\ruby{見}{み}ると、ほかの\ruby{人}{ひと}はみんな、\ruby{自分}{じぶん}に\ruby{感心}{かんしん}しているのだ。

「こんにちは。\ruby{変}{へん}な\ruby{帽子}{ぼうし}\ruby{被}{かぶ}ってるね。」

「こりゃ\ruby{挨拶}{あいさつ}するための\ruby{帽子}{ぼうし}だ。\ruby{俺}{おれ}をやんやとはやしてくれる\ruby{人}{ひと}がいるときに、\ruby{挨拶}{あいさつ}するための\ruby{帽子}{ぼうし}なんだ。でも、あいにく、\ruby{誰}{だれ}もこっちのほうへやってこないんでね。」

「あっ、そう?」と、\ruby{王子}{おうじ}\ruby{様}{さま}は\ruby{言}{い}ったが、\ruby{相手}{あいて}が\ruby{何}{なに}を\ruby{言}{い}っているのか、わからなかったのだ。

「\ruby{手}{て}を\ruby{叩}{たた}きなさい、パチパチと!」\ruby{自惚}{うぬぼ}れ\ruby{男}{おとこ}は\ruby{言}{い}った。

\ruby{王子}{おうじ}\ruby{様}{さま}は、\ruby{手}{て}をパチパチと\ruby{叩}{たた}いた。すると、\ruby{自惚}{うぬぼ}れ\ruby{男}{おとこ}は\ruby{帽子}{ぼうし}を\ruby{持}{も}ち\ruby{上}{あ}げながら、\ruby{丁寧}{ていねい}にお\ruby{辞儀}{じぎ}をした。

「こりゃ、\ruby{王様}{おうさま}を\ruby{訪}{たず}ねるより\ruby{面白}{おもしろ}いな。」と、\ruby{王子}{おうじ}\ruby{様}{さま}は\ruby{思}{おも}って、また\ruby{手}{て}をパチパチと\ruby{叩}{たた}いた。\ruby{自惚}{うぬぼ}れ\ruby{男}{おとこ}は、また\ruby{帽子}{ぼうし}を\ruby{持}{も}ち\ruby{上}{あ}げながら、お\ruby{辞儀}{じぎ}をした。

五\ruby{分間}{ふんかん}も\ruby{手}{て}を\ruby{叩}{たた}く\ruby{稽古}{けいこ}をしているうちに、\ruby{王子}{おうじ}\ruby{様}{さま}は、することがいつまでも\ruby{同}{おな}じことなので、くたびれた。

「その\ruby{帽子}{ぼうし}を\ruby{落}{お}とすには、どうすればいいの?」\ruby{王子}{おうじ}\ruby{様}{さま}は\ruby{聞}{き}いてみた。

しかし、\ruby{褒}{ほ}め\ruby{言葉}{ことば}しか\ruby{聞}{き}こえない\ruby{自惚}{うぬぼ}れ\ruby{男}{おとこ}には、\ruby{質問}{しつもん}も\ruby{全}{まった}く\ruby{聞}{き}こえない。

「お\ruby{前}{まえ}さんは、\ruby{本当}{ほんとう}に\ruby{俺}{おれ}に\ruby{感心}{かんしん}しているのかね。」と、\ruby{自惚}{うぬぼ}れ\ruby{男}{おとこ}が\ruby{王子}{おうじ}\ruby{様}{さま}に\ruby{訪}{たず}ねました。

「\ruby{感心}{かんしん}するって、それ、\ruby{一体}{いったい}どういうこと?」

「\ruby{感心}{かんしん}するっていうのはね、\ruby{俺}{おれ}がこの\ruby{星}{ほし}のうちで、\ruby{一番}{いちばん}\ruby{美}{うつく}しくて、\ruby{一番}{いちばん}\ruby{立派}{りっぱ}な\ruby{服}{ふく}を\ruby{着}{き}ていて、\ruby{一番}{いちばん}お\ruby{金持}{かねも}ちで、それに、\ruby{一番}{いちばん}\ruby{賢}{かしこ}い\ruby{人}{ひと}だと\ruby{思}{おも}うことだよ。」

「でも、この\ruby{星}{ほし}の\ruby{上}{うえ}にいる\ruby{人}{ひと}ったら、あんたひとりっきりじゃないの?」

「\ruby{頼}{たの}むからね、まあ、とにかく、\ruby{俺}{おれ}に\ruby{感心}{かんしん}してくれ!」

「\ruby{僕}{ぼく}、\ruby{感心}{かんしん}するよ。」と、\ruby{王子}{おうじ}\ruby{様}{さま}はちょっと\ruby{肩}{かた}をすくめながらこう\ruby{言}{い}った。「でも、なぜそんなことに\ruby{拘}{こだわ}るの?」

\ruby{王子}{おうじ}\ruby{様}{さま}はその\ruby{星}{ほし}から\ruby{立}{た}ち\ruby{去}{さ}った。\ruby{大人}{おとな}って、やっぱり\ruby{本当}{ほんとう}に\ruby{奇妙}{きみょう}だな。\ruby{王子}{おうじ}\ruby{様}{さま}は\ruby{旅}{たび}を\ruby{続}{つづ}けながら、そう\ruby{思}{おも}った。




\chapter{第十二章}
お\ruby{弁当}{べんとう}の\ruby{後}{あと}、みんなと\ruby{校庭}{こうてい}で\ruby{走}{はし}り\ruby{回}{まわ}ったトットちゃんが、\ruby{電車}{でんしゃ}の\ruby{教室}{きょうしつ}に\ruby{戻}{もど}ると、女の先生が、

「\ruby{皆}{みな}さん、\ruby{今日}{きょう}は、とてもよく\ruby{勉強}{べんきょう}したから、\ruby{午後}{ごご}は、\ruby{何}{なに}をしたい?」

と\ruby{聞}{き}いた。トットちゃんが、(えーと、\ruby{私}{わたし}のしたいこと、って\ruby{言}{い}えば……)なんて\ruby{考}{かんが}えるより\ruby{前}{まえ}に、みんなが口々に

「\ruby{散歩}{さんぽ}!」

といった。すると先生は、

「じゃ、\ruby{行}{い}きましょう」

といって立ち上がり、みんなも、\ruby{電車}{でんしゃ}のドアを\ruby{開}{あ}けて、\ruby{靴}{くつ}を\ruby{履}{は}いて、\ruby{飛}{と}び\ruby{出}{だ}した。トットちゃんは、パパと犬のロッキーと、\ruby{散歩}{さんぽ}に\ruby{行}{い}ったことはあるけど、学校で、\ruby{散歩}{さんぽ}に\ruby{行}{い}く、って\ruby{知}{し}らなかったから、ビックリした。でも、\ruby{散歩}{さんぽ}は\ruby{大好}{だいす}きだから、トットちゃんも、\ruby{急}{いそ}いで\ruby{靴}{くつ}を\ruby{履}{は}いた。

あとで\ruby{分}{わ}かったことだけど、先生が\ruby{朝}{あさ}の一\ruby{時間}{じかん}目に、その日、一日やる\ruby{時間}{じかん}\ruby{割}{わり}の\ruby{問題}{もんだい}を\ruby{黒板}{こくばん}に\ruby{書}{か}いて、みんなが、\ruby{頑張}{がんば}って、\ruby{午前中}{ごぜんちゅう}に、\ruby{全部}{ぜんぶ}やっちゃうと、\ruby{午後}{ごご}は、たいがい\ruby{散歩}{さんぽ}になるのだった。これは一年生でも、六年生でも\ruby{同}{おな}じだった。

学校の\ruby{門}{もん}を出ると、女の先生を、\ruby{真}{ま}ん\ruby{中}{なか}にして、九人の一年生は、小さい川に\ruby{沿}{そ}って\ruby{歩}{ある}き\ruby{出}{だ}した。川の\ruby{両側}{りょうがわ}には、ついこの\ruby{間}{あいだ}まで\ruby{満開}{まんかい}だった、\ruby{桜}{さくら}の大きい木が、ずーっと\ruby{並}{なら}んでいた。そして、\ruby{見渡}{みわた}す\ruby{限}{かぎ}り、\ruby{菜}{な}の\ruby{花畑}{はなばたけ}だった。\ruby{今}{いま}では、川も\ruby{埋}{う}め\ruby{立}{た}てられ、\ruby{団地}{だんち}やお\ruby{店}{みせ}でギュウヅメの\ruby{自由}{じゆう}が\ruby{丘}{おか}も、この\ruby{頃}{ころ}は、ほとんどが\ruby{畑}{はたけ}だった。

「お\ruby{散歩}{さんぽ}は、\ruby{九品仏}{くほんぶつ}よ」

と、\ruby{兎}{うさぎ}の\ruby{絵}{え}のジャンパースカートの、女の子がいった。この子は、“サッコちゃん”という\ruby{名前}{なまえ}だった。それからサッコちゃんは、

「\ruby{九品仏}{くほんぶつ}の\ruby{池}{いけ}のそばで、この\ruby{前}{まえ}、\ruby{蛇}{へび}を見たわよ」とか、「\ruby{九品仏}{くほんぶつ}のお\ruby{寺}{てら}の\ruby{古}{ふる}い\ruby{井戸}{いど}の中に、\ruby{流}{なが}れ\ruby{星}{ぼし}が\ruby{落}{お}ちてるんだって」

とか\ruby{教}{おし}えてくれた。みんなは、\ruby{勝手}{かって}に、おしゃべりしながら\ruby{歩}{ある}いていく。空は青く、\ruby{蝶々}{ちょうちょ}が、いっぱい、あっちにも、こっちにも、ヒラヒラしていた。\ruby{十分}{じゅっぷん}くらい\ruby{歩}{ある}いたところで、女の先生は、足を\ruby{止}{と}めた。そして、\ruby{黄色}{きいろ}い\ruby{菜}{な}の\ruby{花}{はな}を\ruby{指}{さ}して、

「これは、\ruby{菜}{な}の\ruby{花}{はな}ね。どうして、お花が\ruby{咲}{さ}くか、\ruby{分}{わ}かる?」

といった。そして、それから、メシベとオシベの\ruby{話}{はな}しをした。\ruby{生徒}{せいと}は、みんな\ruby{道}{みち}にしゃがんで、\ruby{菜}{な}の\ruby{花}{はな}を\ruby{観察}{かんさつ}した。先生は、\ruby{蝶々}{ちょうちょ}も、花を\ruby{咲}{さ}かせるお\ruby{手伝}{てつだ}いをしている、といった。\ruby{本当}{ほんとう}に、\ruby{蝶々}{ちょうちょ}は、お\ruby{手伝}{てつだ}いをしているらしく、\ruby{忙}{いそが}しそうだった。それから、また先生は\ruby{歩}{ある}き\ruby{出}{だ}したから、みんなも、\ruby{観察}{かんさつ}はおしまいにして、立ち上がった。\ruby{誰}{だれ}かが、

「オシベと、アカンベは\ruby{違}{ちが}うよね」

とか、いった。トットちゃんは、(\ruby{違}{ちが}うんじゃないかなあー!)と\ruby{思}{おも}ったけど、よく、わかんなかった。でも、オシベとメシベが\ruby{大切}{たいせつ}、ってことは、みんなと\ruby{同}{おな}じように、よく\ruby{分}{わ}かった。

そして、また\ruby{十分}{じゅっぷん}くらい\ruby{歩}{ある}くと、こんもりした小さな森が見えてきて、それが九品仏のお寺だった。

境内に入ると、みんな、見たいもののほうに、キャアキャアいって\ruby{走}{はし}っていった。サッコちゃんが、

「\ruby{流}{なが}れ\ruby{星}{ぼし}の\ruby{井戸}{いど}を見に\ruby{行}{い}かない?」

といったので、もちろん、トットちゃんは、

「うん」

といって、サッコちゃんの\ruby{後}{あと}について\ruby{走}{はし}った。\ruby{井戸}{いど}っていっても、石みたいので\ruby{出来}{でき}ていて、\ruby{二人}{ふたり}の\ruby{胸}{むね}のところくらいまであり、木のふたがしてあった。\ruby{二人}{ふたり}でふたを\ruby{取}{と}って、下をのぞくと中は\ruby{真}{ま}っ\ruby{暗}{くら}で、よく見ると、コンクリートの\ruby{固}{かた}まりか、石の\ruby{固}{かた}まりみたいのが入っているだけで、トットちゃんが\ruby{想像}{そうぞう}してたみたいな、キラキラ\ruby{光}{ひか}る\ruby{星}{ほし}は、どこにも見えなかった。\ruby{長}{なが}いこと、\ruby{頭}{あたま}を\ruby{井戸}{いど}の中に\ruby{突}{つ}っ\ruby{込}{こ}んでいたトットちゃんは、\ruby{頭}{あたま}を上げると、サッコちゃんに\ruby{聞}{き}いた。

「お\ruby{星}{ほし}さま、見た?」

サッコちゃんは、\ruby{頭}{あたま}を\ruby{振}{ふ}ると

「\ruby{一度}{いちど}も、ないの」

といった。トットちゃんは、どうして\ruby{光}{ひか}らないか、お\ruby{考}{かんが}えた。そして、いった。

「お\ruby{星}{ほし}さま、\ruby{今}{いま}、\ruby{寝}{ね}てるんじゃないの?」

サッコちゃんは、大きい目を、もっと大きくしていった。

「お\ruby{星}{ほし}さまって、\ruby{寝}{ね}るの?」

トットちゃんは、あまり\ruby{確信}{かくしん}が\ruby{無}{な}かったから、早口でいった。

「お\ruby{星}{ほし}さまは、\ruby{昼間}{ひるま}、\ruby{寝}{ね}てて、\ruby{夜}{よる}、\ruby{起}{お}きて、\ruby{光}{ひか}るんじゃないか、って\ruby{思}{おも}うんだ」

それから、みんなで、\ruby{仁王}{におう}さまのお\ruby{腹}{なか}を見て\ruby{笑}{わら}ったり、\ruby{薄暗}{うすぐら}いお\ruby{堂}{どう}の中の\ruby{仏}{ほとけ}さまを、(\ruby{少}{すこ}し、こわい)と\ruby{思}{おも}いながらも、のぞいたり、\ruby{天狗}{てんぐ}さまの大きな\ruby{足跡}{あしあと}の\ruby{残}{のこ}ってる石に、\ruby{自分}{じぶん}の足を\ruby{乗}{の}せて\ruby{比}{くら}べてみたり、\ruby{池}{いけ}の\ruby{周}{まわ}りを\ruby{回}{まわ}って、ボートに\ruby{乗}{の}っている人に、「こんちは」といったり、お\ruby{墓}{はか}の\ruby{周}{まわ}りの、\ruby{黒}{くろ}いツルツルの、あぶら石を\ruby{借}{か}りて、\ruby{石蹴}{いしけ}りをしたり、もう\ruby{満足}{まんぞく}するぐらい、\ruby{遊}{あそ}んだ。\ruby{特}{とく}に、\ruby{初}{はじ}めてのトットちゃんは、もう\ruby{興奮}{こうふん}して、\ruby{次}{つぎ}から\ruby{次}{つぎ}と、\ruby{何}{なに}かを\ruby{発見}{はっけん}しては、\ruby{叫}{さけ}び\ruby{声}{ごえ}を上げた。

\ruby{春}{はる}の\ruby{日差}{ひざ}しが、\ruby{少}{すこ}し\ruby{傾}{かたむ}いた。先生は、

「\ruby{帰}{かえ}りましょう」

といって、また、みんな、\ruby{菜}{な}の\ruby{花}{はな}と\ruby{桜}{さくら}の\ruby{木}{き}の\ruby{間}{あいだ}も\ruby{道}{みち}を、\ruby{並}{なら}んで、学校に\ruby{向}{む}かった。\ruby{子供}{こども}たちにとって、\ruby{自由}{じゆう}で、お\ruby{遊}{あそ}びの\ruby{時間}{じかん}と見える、この『\ruby{散歩}{さんぽ}』が、\ruby{実}{じつ}は、\ruby{貴重}{きちょう}な、\ruby{理科}{りか}や、\ruby{歴史}{れきし}や、\ruby{生物}{せいぶつ}の\ruby{勉強}{べんきょう}になっているのだ、ということを、\ruby{子供}{こども}たちは気がついていなかった。

トットちゃんは、もう、すっかり、みんなと\ruby{友達}{ともだち}になっていて、\ruby{前}{まえ}から、ずーっと\ruby{一緒}{いっしょ}にいるような気になっていた。だから、\ruby{帰}{かえ}り\ruby{道}{みち}に

「\ruby{明日}{あした}も、\ruby{散歩}{さんぽ}にしよう!」

と、みんなに大きい\ruby{声}{こえ}で\ruby{言}{い}った。みんなは、とびはねながら、いった。

「そうしよう」

\ruby{蝶々}{ちょうちょ}は、まだまだ\ruby{忙}{いそが}しそうで、\ruby{鳥}{とり}の\ruby{声}{こえ}が、\ruby{近}{ちか}くや\ruby{遠}{とお}くに\ruby{聞}{き}こえていた。

トットちゃんの\ruby{胸}{むね}は、なんか、うれしいもので、いっぱいだった。




\chapter{第十三章}
トットちゃんには、\ruby{本当}{ほんとう}に、\ruby{新}{あたら}しい\ruby{驚}{おどろ}きで、いっぱいの、トモエ\ruby{学園}{がくえん}での\ruby{毎日}{まいにち}が\ruby{過}{す}ぎていった。\ruby{相変}{あいか}わらず、\ruby{学校}{がっこう}に\ruby{早}{はや}く\ruby{行}{い}きたくて、\ruby{朝}{あさ}が\ruby{待}{ま}ちきれなかった。そして、\ruby{帰}{かえ}ってくると、\ruby{犬}{いぬ}のロッキーと、ママとパパに、「\ruby{今日}{きょう}、\ruby{学校}{がっこう}で、どんなことをして、どのくらい\ruby{面白}{おもしろ}かった」とか、「もう、びっくりしちゃった」とか、しまいには、ママが、「\ruby{話}{はなし}は、ちょっとお\ruby{休}{やす}みして、おやつにしたら?」というまで、\ruby{話}{はなし}をやめなかった。そして、これは、どんなにトットちゃんが、\ruby{学校}{がっこう}に\ruby{馴}{な}れてもやっぱり、\ruby{毎日}{まいにち}ように、\ruby{話}{はな}すことは、\ruby{山}{やま}のように、あったのだった。(でも、こんなに\ruby{話}{はな}すことがたくさんあるってことは、\ruby{有難}{ありがた}いこと)と、ママは、\ruby{心}{こころ}から、\ruby{嬉}{うれ}しく\ruby{思}{おも}っていた。ある\ruby{日}{ひ}、トットちゃんは、\ruby{学校}{がっこう}に\ruby{行}{い}く\ruby{電車}{でんしゃ}の\ruby{中}{なか}で、\ruby{突然}{とつぜん}、「あれ?オモエに\ruby{校歌}{こうか}って、あったかな?」と\ruby{考}{かんが}えた。そう\ruby{思}{おも}ったら、もう、\ruby{早}{はや}く\ruby{学校}{がっこう}に\ruby{着}{つ}きたくなって、まだ、あと\ruby{二}{ふた}つも\ruby{駅}{えき}があるのに、ドアのところに\ruby{立}{た}って、\ruby{自由}{じゆう}が\ruby{丘}{おか}に\ruby{電車}{でんしゃ}が\ruby{着}{つ}いたら、すぐ\ruby{出}{で}られるように、ヨーイ・ドンの\ruby{格好}{かっこう}で\ruby{待}{ま}った。ひとつ\ruby{前}{まえ}の\ruby{駅}{えき}で、ドアが\ruby{開}{ひら}いたとき、\ruby{乗}{の}り\ruby{込}{こ}もうとした、おばさんは、\ruby{女}{おんな}の\ruby{子}{こ}が、ドアのところで、ヨーイ・ドンの\ruby{形}{かたち}になってるので、\ruby{降}{お}りるのか、と\ruby{思}{おも}ったら、そのままの\ruby{形}{かたち}で\ruby{動}{うご}かないので、「どうなっちゃってるのかね」といいながら、\ruby{乗}{の}り\ruby{込}{こ}んできた。こんな\ruby{具合}{ぐあい}だったから、\ruby{駅}{えき}に\ruby{着}{つ}いたときの、トットちゃんの\ruby{早}{はや}く\ruby{降}{お}りたことといったら、なかった。\ruby{若}{わか}い\ruby{男}{おとこ}の\ruby{車掌}{しゃしょう}さんが、しゃれたポーズで、まだ、\ruby{完全}{かんぜん}に\ruby{止}{と}まっていない\ruby{電車}{でんしゃ}から、プラットホームに\ruby{片足}{かたあし}をつけて、おりながら、「\ruby{自由}{じゆう}が\ruby{丘}{おか}!お\ruby{降}{ふ}りの\ruby{方}{ほう}は……」といったとき、もう、トットちゃんの\ruby{姿}{すがた}は、\ruby{改札口}{かいさつぐち}から、\ruby{見}{み}えなくなっていた。\ruby{学校}{がっこう}に\ruby{着}{つ}いて、\ruby{電車}{でんしゃ}の\ruby{教室}{きょうしつ}に\ruby{入}{はい}ると、トットちゃんは、\ruby{先}{さき}に\ruby{来}{き}ていた、\ruby{山内}{やまうち}\ruby{泰二}{たいじ}\ruby{君}{くん}に、すぐ\ruby{聞}{き}いた。「ねえ、タイちゃん。この\ruby{学校}{がっこう}って、\ruby{校歌}{こうか}ある?」\ruby{物理}{ぶつり}の\ruby{好}{す}きなタイちゃんは、とても、\ruby{考}{かんが}えそうな\ruby{声}{こえ}で\ruby{答}{こた}えた。「ないんじゃないかな?」「ふーん」と、トットちゃんは、\ruby{少}{すこ}し、もったいをつけて、それから、「あったほうが、いいと\ruby{思}{おも}うんだ。\ruby{前}{まえ}の\ruby{学校}{がっこう}なんて、すごいのが、あったんだから!」といって、\ruby{大}{おお}きな\ruby{声}{こえ}で\ruby{歌}{うた}い\ruby{始}{はじ}めた。「せんぞくいけはあさけれどいじんのむねをふかくくみ(\ruby{洗足池}{せんぞくいけ}は\ruby{浅}{あさ}けれど、\ruby{偉人}{いじん}の\ruby{胸}{むね}を\ruby{深}{ふか}く\ruby{汲}{く}み)」これが、まえの\ruby{学校}{がっこう}の\ruby{校歌}{こうか}だった。ほんの\ruby{少}{すこ}ししか\ruby{通}{かよ}わなかったし、\ruby{一年生}{いちねんせい}には、\ruby{難}{むずか}しい\ruby{言葉}{ことば}だったけど、トットちゃんは、ちゃんと、\ruby{覚}{おぼ}えていた。(ただし、この\ruby{部分}{ぶぶん}だけだったけど)\ruby{聞}{き}き\ruby{終}{お}わると、\ruby{泰}{たい}ちゃんは、\ruby{少}{すこ}し\ruby{感心}{かんしん}したように、\ruby{頭}{あたま}を二\ruby{回}{かい}くらい、\ruby{軽}{かる}く\ruby{振}{ふ}ると、「ふーん」といった。その\ruby{頃}{ころ}には、\ruby{他}{ほか}の\ruby{生徒}{せいと}も\ruby{着}{き}ていて、みんなも、トットちゃんの、\ruby{難}{むずか}しい\ruby{言葉}{ことば}に\ruby{尊敬}{そんけい}と、\ruby{憧}{あこが}れを\ruby{持}{も}ったらしく、「ふーん」といった。トットちゃんは、いった。「ねえ、\ruby{校長}{こうちょう}\ruby{先生}{せんせい}に、\ruby{校歌}{こうか}、\ruby{作}{つく}ってもらおうよ」みんなも、そう\ruby{思}{おも}ったところだったから、「そうしよう、そうしよう」といって、みんなで、ゾロゾロ\ruby{校長}{こうちょう}\ruby{室}{しつ}に\ruby{行}{い}った。\ruby{校長}{こうちょう}\ruby{先生}{せんせい}は、トットちゃんの\ruby{歌}{うた}を\ruby{聞}{き}き、みんなの\ruby{希望}{きぼう}を\ruby{聞}{き}くと、「よし、じゃ、\ruby{明日}{あした}の\ruby{朝}{あさ}までに\ruby{作}{つく}っておくよ」といった。みんなは、「\ruby{約束}{やくそく}だよ」といって、また、ゾロゾロ\ruby{教室}{きょうしつ}に\ruby{戻}{もど}った。さて、\ruby{次}{つぎ}の\ruby{日}{ひ}の\ruby{朝}{あさ}だった。\ruby{各}{かく}\ruby{教室}{きょうしつ}に、\ruby{校長}{こうちょう}\ruby{先生}{せんせい}から、“みんな、\ruby{校庭}{こうてい}に\ruby{集}{あつ}まるように”という、ことづけがあった。トットちゃん\ruby{達}{たち}は、\ruby{期待}{きたい}でむねを、ワクワクさせながら\ruby{校庭}{こうてい}に\ruby{集}{あつ}まった。\ruby{校長}{こうちょう}\ruby{先生}{せんせい}は、\ruby{校庭}{こうてい}の\ruby{真}{ま}ん\ruby{中}{なか}に、\ruby{黒板}{こくばん}を\ruby{運}{はこ}び\ruby{出}{だ}すと、いった。「いいかい、\ruby{君}{きみ}\ruby{達}{たち}の\ruby{学校}{がっこう}、トモエの\ruby{校歌}{こうか}だよ」そして\ruby{黒板}{こくばん}に、五\ruby{線}{せん}を\ruby{書}{か}くと、\ruby{次}{つぎ}のように、オタマジャクシを\ruby{並}{なら}べた。それから、\ruby{校長}{こうちょう}\ruby{先生}{せんせい}は、\ruby{手}{て}を\ruby{指揮者}{しきしゃ}のように、\ruby{大}{おお}きく\ruby{上}{あ}げると、「さあ、\ruby{一緒}{いっしょ}に\ruby{歌}{うた}おう!」といって、\ruby{手}{て}を\ruby{振}{ふ}り\ruby{下}{お}ろした。\ruby{全校}{ぜんこう}\ruby{生徒}{せいと}、五十\ruby{人}{にん}は、みんな、\ruby{先生}{せんせい}の\ruby{声}{こえ}に\ruby{合}{あ}わせて、\ruby{歌}{うた}った。「トモエ、トモエ、トーモエ!」「……これだけ?」ちょっとした\ruby{間}{ま}があって、トットちゃんが\ruby{聞}{き}いた。\ruby{校長}{こうちょう}\ruby{先生}{せんせい}は、\ruby{得意}{とくい}そうに\ruby{答}{こた}えた。「そうだよ」トットちゃんは、ひどく、がっかりした\ruby{声}{こえ}で、\ruby{先生}{せんせい}に\ruby{言}{い}った。「もっと、むずかしいのが、よかったんだ。センゾクイケハアサケレドーみたいなの」\ruby{先生}{せんせい}は、\ruby{顔}{かお}を\ruby{真}{ま}っ\ruby{赤}{か}にして、\ruby{笑}{わら}いながらいった。「いいかい?これ、いいと\ruby{思}{おも}うけどな」\ruby{結局}{けっきょく}、\ruby{他}{ほか}の\ruby{子供}{こども}\ruby{達}{たち}も、「こんなカンタンすぎるのなら、いらない」といって、\ruby{断}{ことわ}った。\ruby{先生}{せんせい}は、ちょっと\ruby{残念}{ざんねん}そうだったけど、\ruby{別}{べつ}に\ruby{怒}{いか}りもしないで、\ruby{黒板}{こくばん}けしで、\ruby{消}{け}してしまった。トットちゃんは、すこし(\ruby{先生}{せんせい}に\ruby{悪}{わる}かったかな)と\ruby{思}{おも}ったけど(ほしかったのは、もっと\ruby{偉}{えら}そうなヤツだったんだもの、\ruby{仕方}{しかた}がないや)と\ruby{考}{かんが}えた。

\ruby{本当}{ほんとう}は、こんなに\ruby{簡単}{かんたん}で『\ruby{学校}{がっこう}を、そして\ruby{子供}{こども}たち』を\ruby{愛}{あい}する\ruby{校長}{こうちょう}\ruby{先生}{せんせい}の\ruby{気持}{きも}ちがこもった\ruby{校歌}{こうか}はなかったのに、\ruby{子供}{こども}\ruby{達}{たち}には、まだ、それが\ruby{分}{わ}からなかった。そして、その\ruby{後}{あと}、\ruby{子供}{こども}たちも\ruby{校歌}{こうか}のことは\ruby{忘}{わす}れ、\ruby{先生}{せんせい}も\ruby{要}{い}らないと\ruby{思}{おも}ったのか、\ruby{黒板}{こくばん}けしで\ruby{消}{け}したまま、\ruby{最後}{さいご}まで、トモエには、\ruby{校歌}{こうか}って、なかった。




\chapter{第十四章}
五\ruby{番目}{ばんめ}の\ruby{星}{ほし}はとても\ruby{珍}{めずら}しい\ruby{星}{ほし}だった。\ruby{星}{ほし}のうちで、\ruby{一番}{いちばん}\ruby{小}{ちい}さな\ruby{星}{ほし}だった。そこには、ちょうど、\ruby{街灯}{がいとう}と\ruby{点灯}{てんとう}\ruby{人}{ひと}がいられるぐらいの\ruby{場所}{ばしょ}しかなかった。\ruby{王子}{おうじ}\ruby{様}{さま}はどうやってもわからなかった。\ruby{空}{そら}のこんな\ruby{場所}{ばしょ}で、\ruby{星}{ほし}に\ruby{家}{いえ}もないし、\ruby{人}{ひと}もいないのに、\ruby{街灯}{がいとう}と\ruby{点灯}{てんとう}\ruby{人}{ひと}がいて、\ruby{何}{なに}のためになるんだろうか。それでも、\ruby{王子}{おうじ}\ruby{様}{さま}は、\ruby{心}{こころ}の\ruby{中}{なか}でこう\ruby{思}{おも}った。

この\ruby{人}{ひと}はバカバカしいかもしれない。でも、\ruby{王様}{おうさま}、\ruby{自惚}{うぬぼ}れ\ruby{男}{おとこ}、\ruby{実業家}{じつぎょうか}や、\ruby{大酒飲}{おおざけの}みよりは、バカバカしくない。そうだとしても、この\ruby{人}{ひと}のやってることには、\ruby{意味}{いみ}がある。\ruby{明}{あ}かりをつけるってことは、\ruby{例}{たと}えるなら、\ruby{星}{ほし}とか、\ruby{花}{はな}とかが、ひとつ\ruby{新}{あたら}しく\ruby{生}{う}まれるってこと。だから、\ruby{灯}{とも}りを\ruby{消}{け}すのは、\ruby{星}{ほし}とか\ruby{花}{はな}をお\ruby{休}{やす}みさせるってこと。とっても\ruby{素敵}{すてき}なお\ruby{勤}{つと}め。\ruby{素敵}{すてき}だから、\ruby{本当}{ほんとう}に、\ruby{誰}{だれ}かのためになる。

\ruby{王子}{おうじ}\ruby{様}{さま}は、\ruby{星}{ほし}に\ruby{足}{あし}を\ruby{踏}{ふ}み\ruby{入}{い}れたとき、\ruby{丁寧}{ていねい}に\ruby{点灯}{てんとう}\ruby{人}{じん}にお\ruby{辞儀}{じぎ}をした。

「こんにちは。なぜ、いま、\ruby{街頭}{がいとう}の\ruby{火}{ひ}を\ruby{消}{け}したの?」

「\ruby{命令}{めいれい}だよ。やあ、おはよう。」と、\ruby{点灯}{てんとう}\ruby{人}{じん}が\ruby{答}{こた}えた。

「どうな\ruby{命令}{めいれい}?」

「\ruby{街頭}{がいとう}の\ruby{火}{ひ}を\ruby{消}{け}すことだよ。やあ、こんばんは。」と\ruby{言}{い}って、\ruby{点灯}{てんとう}\ruby{人}{じん}はまた\ruby{火}{ひ}をつけた。

「だけど、なぜ、また\ruby{火}{ひ}をつけたの?」

「\ruby{命令}{めいれい}だよ。」と、\ruby{点灯}{てんとう}\ruby{人}{じん}が\ruby{答}{こた}えた。

「わからないなあ。」と、\ruby{王子}{おうじ}\ruby{様}{さま}が\ruby{言}{い}った。

「わからなくていいよ、\ruby{命令}{めいれい}は\ruby{命令}{めいれい}だよ。やあ、おはよう。」と\ruby{言}{い}って、\ruby{点灯}{てんとう}\ruby{人}{じん}は\ruby{街頭}{がいとう}の\ruby{火}{ひ}を\ruby{消}{け}した。

それから、おでこを\ruby{赤}{あか}いチェックのハンカチで\ruby{拭}{ふ}いた。

「なにしろ、とんでもない\ruby{仕事}{しごと}だよ。\ruby{昔}{むかし}は\ruby{理屈}{りくつ}にあってたんだがね。\ruby{朝}{あさ}になると\ruby{火}{ひ}を\ruby{消}{け}す、\ruby{夕方}{ゆうがた}になると、\ruby{火}{ひ}をつける。\ruby{昼間}{ひるま}は\ruby{休}{やす}めたし、\ruby{夜}{よる}は\ruby{眠}{ねむ}ったもんだ。」

「で、そのあと、\ruby{命令}{めいれい}が\ruby{変}{か}わったってこと?」

「\ruby{命令}{めいれい}は\ruby{変}{か}わりゃしないよ。それが\ruby{本当}{ほんとう}、ひどい\ruby{話}{はなし}なんだ。この\ruby{星}{ほし}は\ruby{年々}{としどし}、\ruby{回}{まわ}るのがどんどん\ruby{早}{はや}くなるのに、\ruby{命令}{めいれい}は\ruby{変}{か}わらないときてるんだからなあ。」

「つまり?」

「つまり、\ruby{今}{いま}じゃ、この\ruby{星}{ほし}のやつが、一\ruby{分間}{ふんかん}に\ruby{一周}{ひとまわ}りすることになってるんで、\ruby{俺}{おれ}ときたら、一\ruby{秒}{びょう}も\ruby{休}{やす}めなくなったんだよ。一\ruby{分間}{ふんかん}に\ruby{一度}{いちど}、\ruby{火}{ひ}をつけたり、\ruby{消}{け}したりするんだからな。」

「\ruby{変}{へん}だなあ。一\ruby{分間}{ふんかん}が一\ruby{日}{にち}だなんて。」

「ちっとも\ruby{変}{へん}なことなんかないよ。\ruby{俺}{おれ}たちは、もう一\ruby{ヶ月}{かげつ}も\ruby{話}{はな}してるんだぜ。」と、\ruby{点灯}{てんとう}\ruby{人}{じん}が\ruby{言}{い}った。

「一\ruby{ヶ月}{かげつ}?」

「そうだよ。30\ruby{分}{ふん}、だから、30\ruby{日}{にち}さ。やあ、こんばんは。」\ruby{点灯}{てんとう}\ruby{人}{じん}は、また\ruby{街灯}{がいとう}に\ruby{火}{ひ}をつけた。

\ruby{王子}{おうじ}\ruby{様}{さま}は、\ruby{相手}{あいて}の\ruby{顔}{かお}をじっと\ruby{見}{み}た。そして、こんなにも\ruby{命令}{めいれい}をよく\ruby{守}{まも}る\ruby{点灯}{てんとう}\ruby{人}{じん}が\ruby{好}{す}きになった。\ruby{王子}{おうじ}\ruby{様}{さま}は、\ruby{夕暮}{ゆうぐ}れを\ruby{見}{み}たいとき、\ruby{自分}{じぶん}から\ruby{椅子}{いす}を\ruby{動}{うご}かしていたことを\ruby{思}{おも}い\ruby{出}{だ}した。

\ruby{王子}{おうじ}\ruby{様}{さま}は、この\ruby{友達}{ともだち}を\ruby{助}{たす}けたかった。

「ねえ、\ruby{休}{やすみ}みたい\ruby{時}{とき}に\ruby{休}{やす}めるこつ、\ruby{知}{し}ってるよ。」

「いつだって\ruby{休}{やす}みたいよ!」と、\ruby{点灯}{てんとう}\ruby{人}{じん}が\ruby{言}{い}った。

\ruby{人}{ひと}っていうのは、\ruby{真面目}{まじめ}にやってても、\ruby{怠}{なま}けたいものなんだ。

\ruby{王子}{おうじ}\ruby{様}{さま}は、\ruby{言葉}{ことば}を\ruby{続}{つづ}けた。

「\ruby{君}{きみ}の\ruby{星}{ほし}は、\ruby{本当}{ほんとう}に\ruby{小}{ちい}さいんだから、3\ruby{歩}{ほ}\ruby{歩}{ある}けば、ぐるりっと\ruby{回}{まわ}ってしまえるよ。\ruby{相当}{そうとう}ゆっくり\ruby{歩}{ある}いてさえいたら、しょっちゅうお\ruby{日様}{ひさま}を\ruby{眺}{なが}めていられるわけだよ。\ruby{休}{やす}みたくなったら、\ruby{歩}{ある}くんだな。そしたら、\ruby{君}{きみ}がほしいが\ruby{思}{おも}うだけ、\ruby{昼間}{ひるま}が\ruby{続}{つづ}くよ。」

「そうしたかたって、\ruby{俺}{おれ}は\ruby{大}{たい}して\ruby{助}{たす}からないなあ。\ruby{俺}{おれ}がこの\ruby{世}{よ}で\ruby{好}{す}きなのは、\ruby{眠}{ねむ}ることだよ。」

「そりゃ\ruby{困}{こま}ったね。」と、\ruby{王子}{おうじ}\ruby{様}{さま}は\ruby{言}{い}った。

「うん、\ruby{困}{こま}ったよ。やあ、おはよう。」そして、\ruby{点灯}{てんとう}\ruby{人}{じん}は、\ruby{街頭}{がいとう}の\ruby{火}{ひ}を\ruby{消}{け}した。

\ruby{王子}{おうじ}\ruby{様}{さま}は、ずっと\ruby{遠}{とお}くへ\ruby{旅}{たび}を\ruby{続}{つづ}けながら、こんなふうに\ruby{思}{おも}った。

あの\ruby{人}{ひと}、ほかのみんなから、\ruby{馬鹿}{ばか}にされるだろうな。\ruby{王様}{おうさま}、\ruby{自惚}{うぬぼ}れ\ruby{男}{おとこ}、\ruby{大酒飲}{おおざけの}み、\ruby{実業家}{じつぎょうか}から。でも、\ruby{僕}{ぼく}からしてみれば、たった\ruby{一人}{ひとり}、あの\ruby{人}{ひと}だけは、\ruby{変}{へん}だと\ruby{思}{おも}わなかった。それっていうのも、もしかすると、あの\ruby{人}{ひと}が\ruby{自分}{じぶん}じゃないことのために、あくせくしていたからかも。

\ruby{王子}{おうじ}\ruby{様}{さま}は\ruby{残念}{ざんねん}そうにため\ruby{息}{いき}をついた。さらに\ruby{考}{かんが}える。

たった\ruby{一人}{ひとり}、あの\ruby{人}{ひと}だけ、\ruby{僕}{ぼく}は\ruby{友達}{ともだち}になれると\ruby{思}{おも}った。でも、あの\ruby{人}{ひと}の\ruby{星}{ほし}は、\ruby{本当}{ほんとう}に\ruby{小}{ちい}さすぎて、\ruby{二人}{ふたり}も\ruby{入}{はい}らない…ただ、\ruby{王子}{おうじ}\ruby{様}{さま}としては、そうとは\ruby{思}{おも}いたくなかったんだけど、\ruby{実}{じつ}は、この\ruby{星}{ほし}のことも、\ruby{残念}{ざんねん}に\ruby{思}{おも}っていたんだ。だって、なんといっても、24\ruby{時間}{じかん}に1440\ruby{回}{かい}も\ruby{夕暮}{ゆうぐ}れが\ruby{見}{み}られるっていう、\ruby{恵}{めぐ}まれた\ruby{星}{ほし}なんだから。




\chapter{第十五章}
六\ruby{番目}{ばんめ}の\ruby{星}{ほし}は、\ruby{前}{まえ}の\ruby{星}{ほし}より 10 \ruby{倍}{ばい}\ruby{大}{おお}きかった。そこには\ruby{分厚}{ぶあつ}くて\ruby{大}{おお}きな\ruby{本}{ほん}を\ruby{書}{か}く\ruby{老紳士}{ろうしんし}が\ruby{住}{す}んでいた。\ruby{王子}{おうじ}さまを\ruby{見}{み}かけると、「おや、\ruby{探検家}{たんけんか}がやってきた。」と\ruby{大声}{おおごえ}で\ruby{言}{い}った。

\ruby{王子}{おうじ}さまは\ruby{机}{つくえ}に\ruby{腰掛}{こしか}け、\ruby{息}{いき}をついた。\ruby{随分}{ずいぶん}\ruby{旅}{たび}をしてきたものだ。

「あんた、どこから\ruby{来}{き}たのかい?」と、\ruby{老紳士}{ろうしんし}は\ruby{王子}{おうじ}\ruby{様}{さま}は\ruby{言}{い}った。

「その\ruby{大}{おお}きな\ruby{本}{ほん}は\ruby{何}{なに}?ここで\ruby{何}{なに}をしているの?」と、\ruby{王子}{おうじ}\ruby{様}{さま}が\ruby{言}{い}った。

「わしは\ruby{地理}{ちり}\ruby{学者}{がくしゃ}だ。」と、\ruby{老紳士}{ろうしんし}が\ruby{言}{い}った。

「\ruby{地理}{ちり}\ruby{学者}{がくしゃ}って?」

「\ruby{海}{うみ}や\ruby{川}{かわ}や、\ruby{砂漠}{さばく}がどこにあるのか、そんなことを\ruby{知}{し}ってる\ruby{学者}{がくしゃ}のことだよ。」

「そりゃ\ruby{面白}{おもしろ}いなあ、\ruby{本当}{ほんとう}に、そんなのが、\ruby{本当}{ほんとう}の\ruby{仕事}{しごと}ですよ!」

そういって、\ruby{王子}{おうじ}\ruby{様}{さま}は\ruby{自分}{じぶん}の\ruby{周}{まわ}りの\ruby{星}{ほし}の\ruby{上}{うえ}に、チラと\ruby{目}{め}をやった。けど、まだ\ruby{一度}{いちど}も、こんなにも\ruby{堂々}{どうどう}とした\ruby{星}{ほし}を、\ruby{見}{み}たことがなかった。

「あなたの\ruby{星}{ほし}、とても\ruby{綺麗}{きれい}ですね。\ruby{海}{うみ}がありますか、ここには。」

「\ruby{知}{し}らんよ、そんなこと。」と、\ruby{地理}{ちり}\ruby{学者}{がくしゃ}が\ruby{言}{い}った。

「へえ…」\ruby{王子}{おうじ}\ruby{様}{さま}はがっかりした。「じゃあ、\ruby{山}{やま}は?」

「\ruby{知}{し}らんよ、そいつも。」と、\ruby{地理}{ちり}\ruby{学者}{がくしゃ}が\ruby{言}{い}った。

「じゃあ、\ruby{町}{まち}だの、\ruby{川}{かわ}だよ、\ruby{砂漠}{さばく}だのってものは?」

「それも\ruby{知}{し}らんよ。」

「だって、\ruby{地理}{ちり}\ruby{学者}{がくしゃ}でしょう?」

「そりゃそうだ。だが、わしは\ruby{探検家}{たんけんか}じゃない。\ruby{探検家}{たんけんか}なんか、わしには\ruby{全}{まった}くご\ruby{縁}{ふち}がないよ。\ruby{地理}{ちり}\ruby{学者}{がくしゃ}は、\ruby{町}{まち}や\ruby{川}{かわ}や、\ruby{山}{やま}や\ruby{海}{うみ}や、\ruby{大}{おお}きな\ruby{海}{うみ}や、\ruby{大海原}{おおうなばら}や\ruby{砂漠}{さばく}や\ruby{数}{かぞ}えに\ruby{行}{い}くことはない。とても\ruby{大切}{たいせつ}な\ruby{仕事}{しごと}をしてるんだから、そこらをぶらついてなんかおられんのだ。\ruby{自分}{じぶん}の\ruby{机}{つくえ}を\ruby{離}{はな}れることはない。そのかわり、\ruby{探検家}{たんけんか}を\ruby{迎}{むか}えるんだ。\ruby{探検家}{たんけんか}が\ruby{来}{き}たら、いろいろな\ruby{報告}{ほうこく}を\ruby{受}{う}けて、\ruby{相手}{あいて}の\ruby{話}{はなし}をノートに\ruby{取}{と}る。そして、\ruby{相手}{あいて}の\ruby{話}{はなし}を\ruby{面白}{おもしろ}いと\ruby{思}{おも}ったら、\ruby{地理}{ちり}\ruby{学者}{がくしゃ}というものは、その\ruby{探検家}{たんけんか}が\ruby{正直者}{しょうじきもの}かどうかを\ruby{調}{しら}べるんだ。」

「どうして?」

「もし、\ruby{探検家}{たんけんか}が\ruby{嘘}{うそ}をついたら、\ruby{地理}{ちり}の\ruby{本}{ほん}がトンチンカンにならんとも\ruby{限}{かぎ}らんからね。\ruby{探検家}{たんけんか}がやたら\ruby{酒}{さけ}を\ruby{飲}{の}んでも、やっぱり\ruby{同}{おな}じことだよ。」

「どうして?」と、\ruby{王子}{おうじ}\ruby{様}{さま}が\ruby{言}{い}った。

「どうしてって、\ruby{大酒飲}{おおざけの}みのやつには、\ruby{物}{もの}が\ruby{二}{ふた}つに\ruby{見}{み}えるからさ。すると、\ruby{地理}{ちり}\ruby{学者}{がくしゃ}は、\ruby{山}{やま}がひとつしかないところに、\ruby{二}{ふた}つあると\ruby{書}{か}くだろうじゃないか。」

「\ruby{僕}{ぼく}、\ruby{悪}{わる}い\ruby{探検家}{たんけんか}になりそうな\ruby{人}{ひと}、\ruby{知}{し}ってますよ。」

「うん、そんなこともあるものだ。だから、\ruby{地理}{ちり}\ruby{学者}{がくしゃ}というものは、この\ruby{探検家}{たんけんか}は、\ruby{素性}{すじょう}が\ruby{良}{よ}さそうだと\ruby{思}{おも}うと、その\ruby{人}{ひと}の\ruby{発見}{はっけん}したことの\ruby{調査}{ちょうさ}をやるのだ。」

「\ruby{見}{み}に\ruby{行}{い}くの?」

「いや、\ruby{見}{み}には\ruby{行}{い}かんよ。そんなこと、\ruby{面倒}{めんどう}くさいさ。しかし、\ruby{探検家}{たんけんか}から、いろんな\ruby{証拠}{しょうこ}を\ruby{持}{も}ち\ruby{出}{だ}してもらうんだよ。\ruby{例}{たと}えば、\ruby{大}{おお}きな\ruby{山}{やま}を\ruby{発見}{はっけん}したというんだったら、いくつも、\ruby{大}{おお}きな\ruby{石}{いし}を\ruby{持}{も}ってきてもらうわけだ。」

\ruby{地理}{ちり}\ruby{学者}{がくしゃ}は、\ruby{不意}{ふい}にワクワクしだした。

「そうだ、\ruby{君}{きみ}は\ruby{遠}{とお}くから\ruby{来}{き}たんだな。\ruby{探検家}{たんけんか}だ。さあ、わしに、\ruby{君}{きみ}の\ruby{星}{ほし}のことを\ruby{喋}{しゃべ}ってくれないか。」

そうやって、\ruby{地理}{ちり}\ruby{学者}{がくしゃ}はノートを\ruby{開}{ひら}いて、\ruby{鉛筆}{えんぴつ}を\ruby{削}{けず}った。\ruby{地理}{ちり}\ruby{学者}{がくしゃ}というものは、\ruby{探検家}{たんけんか}の\ruby{話}{はなし}をまず、\ruby{鉛筆}{えんぴつ}で\ruby{書}{か}き\ruby{留}{と}める。それから、\ruby{探検家}{たんけんか}が\ruby{信}{しん}じられるだけのものを\ruby{出}{だ}してきたら、やっとインクで\ruby{書}{か}き\ruby{留}{と}めるんだ。

「それで?」と、\ruby{地理}{ちり}\ruby{学者}{がくしゃ}は\ruby{待}{ま}ち\ruby{遠}{どお}しそうに\ruby{言}{い}った。

「ええと、\ruby{僕}{ぼく}のうちですか。\ruby{大}{たい}して\ruby{面白}{おもしろ}いところじゃありません。ちいちゃい\ruby{星}{ほし}なんです。

\ruby{火山}{かざん}が\ruby{三}{みっ}つあります、\ruby{活火山}{かっかざん}が\ruby{二}{ふた}つと、\ruby{死火山}{しかざん}がひとつ。でも、いつ\ruby{爆発}{ばくはつ}するかわかりませんよ。」

「うん、そりゃわからん。」と、\ruby{地理}{ちり}\ruby{学者}{がくしゃ}が\ruby{言}{い}った。

「\ruby{花}{はな}もひとつあるんです。」

「わしたちは、\ruby{花}{はな}のことなんか、\ruby{書}{か}かないよ。」

「どうしてなの?\ruby{一番}{いちばん}\ruby{綺麗}{きれい}だよ。」

「\ruby{花}{はな}っていうものは、\ruby{儚}{はかな}いものなんだからね。」

「\ruby{儚}{はかな}いって?」

「\ruby{地理}{ちり}の\ruby{本}{ほん}はなあ、」と、\ruby{地理}{ちり}\ruby{学者}{がくしゃ}は\ruby{言}{い}う。「すべての\ruby{中}{なか}で\ruby{一番}{いちばん}ちゃんとしておる。\ruby{絶対}{ぜったい}\ruby{古}{ふる}くなったりしないからのう。\ruby{山}{やま}が\ruby{動}{うご}いたりするなんか\ruby{滅多}{めった}にない、\ruby{大海原}{おおうなばら}が\ruby{干上}{ひあ}がるなんか\ruby{滅多}{めった}にない。わしたちは、\ruby{変}{か}わらないものを\ruby{書}{か}くのだ。」

\ruby{王子}{おうじ}\ruby{様}{さま}は、\ruby{横}{よこ}から\ruby{口}{くち}を\ruby{出}{だ}した。

「でも、\ruby{死火山}{しかざん}だった、\ruby{目}{め}を\ruby{覚}{さ}めすことがありますよ。\ruby{儚}{はかな}いってなんのこと?」

「\ruby{火山}{かざん}が\ruby{眠}{ねむ}っていようと、\ruby{目}{め}を\ruby{覚}{さ}ましているようと、わしたちにとっちゃ、\ruby{同}{おな}じことだよ。

わしたちが\ruby{問題}{もんだい}にするのは、\ruby{山}{やま}だ。\ruby{山}{やま}は\ruby{変}{か}わることがないからね。」

「だけど、\ruby{儚}{はかな}いってなんのこと?」\ruby{一度}{いちど}\ruby{何}{なに}か\ruby{聞}{き}き\ruby{出}{だ}すと、しまいまで\ruby{聞}{き}かずにいられない\ruby{王子}{おうじ}\ruby{様}{さま}が、\ruby{繰}{く}り\ruby{返}{かえ}した。

「そりゃ、そのうち\ruby{消}{き}えて、なくなるって\ruby{意味}{いみ}だよ。」

「\ruby{僕}{ぼく}の\ruby{花}{はな}、そのうち\ruby{消}{き}えて、なくなるの?」

「うん、そうだとも。」

「\ruby{僕}{ぼく}の\ruby{花}{はな}は\ruby{儚}{はかな}い\ruby{花}{はな}なのか。\ruby{身}{み}の\ruby{守}{まも}りと\ruby{言}{い}ったら、\ruby{四}{よっ}つの\ruby{棘}{とげ}しか\ruby{持}{も}っていない。それなのに、あの\ruby{花}{はな}を\ruby{僕}{ぼく}の\ruby{星}{ほし}に、\ruby{一人}{ひとり}ぼっちにしてきたんだ。

と、\ruby{王子}{おうじ}\ruby{様}{さま}は\ruby{考}{かんが}えた。\ruby{王子}{おうじ}\ruby{様}{さま}は\ruby{初}{はじ}めて、あの\ruby{花}{はな}が\ruby{懐}{なつ}かしくなった。それでも、\ruby{元気}{げんき}を\ruby{取}{と}り\ruby{戻}{もど}して\ruby{聞}{き}いた。

「\ruby{僕}{ぼく}、\ruby{今度}{こんど}は、どこの\ruby{星}{ほし}を\ruby{見物}{けんぶつ}したらいいでしょうかね?」

「\ruby{地球}{ちきゅう}の\ruby{見物}{けんぶつ}しなさい。なかなか\ruby{評判}{ひょうばん}のいい\ruby{星}{ほし}だ。」と、\ruby{地理}{ちり}\ruby{学者}{がくしゃ}が\ruby{答}{こた}えた。

\ruby{王子}{おうじ}\ruby{様}{さま}は\ruby{遠}{とお}くに\ruby{残}{のこ}してきた\ruby{花}{はな}のことを\ruby{思}{おも}いながら、そこを\ruby{後}{あと}にした。




\chapter{第十六章}
そんなわけで、七\ruby{番目}{ばんめ}の\ruby{星}{ほし}が\ruby{地球}{ちきゅう}だった。この\ruby{地球}{ちきゅう}というのは、どこにでもある\ruby{星}{ほし}なんかじゃない。\ruby{数}{かぞ}えてみると、\ruby{王様}{おうさま}が(もちろん、\ruby{黒}{くろ}い\ruby{顔}{かお}の\ruby{王様}{おうさま}も\ruby{入}{い}れて)110\ruby{人}{にん}、\ruby{地理}{ちり}\ruby{学者}{がくしゃ}が7000\ruby{人}{にん}、\ruby{実業家}{じつぎょうか}が90\ruby{万}{まん}\ruby{人}{にん}、\ruby{大酒飲}{おおざけの}みが750\ruby{万}{まん}\ruby{人}{にん}、\ruby{自惚}{うぬぼ}れが3\ruby{億}{おく}1千100\ruby{万}{まん}\ruby{人}{にん}で、\ruby{合}{あ}わせて\ruby{大体}{だいたい}20\ruby{億}{おく}の\ruby{大人}{おとな}の\ruby{人}{ひと}がいる。

\ruby{地球}{ちきゅう}の\ruby{大}{おお}きさをわかりやすくする、こんな\ruby{話}{はなし}がある。\ruby{電気}{でんき}が\ruby{使}{つか}われるまでは、\ruby{六}{むっ}つの\ruby{大陸}{たいりく}ひっくるめて、なんと、46\ruby{万}{まん}2511\ruby{人}{にん}もの、\ruby{大勢}{おおぜい}の\ruby{点灯}{てんとう}\ruby{人}{ひと}がいなきゃならなかった。

\ruby{遠}{とお}くから\ruby{眺}{なが}めると、\ruby{大変}{たいへん}\ruby{見}{み}ものだ。この\ruby{大勢}{おおぜい}の\ruby{動}{うご}きは、バレーのダンサーみたいに、きちっきちっとしていた。

まずはニュージーランドとオーストラリアの\ruby{点灯}{てんとう}\ruby{人}{じん}の\ruby{出番}{でばん}が\ruby{来}{く}る。そこで、\ruby{自分}{じぶん}のランプをつけると、この\ruby{人}{ひと}たちは\ruby{眠}{ねむ}りにつく。すると、\ruby{次}{つぎ}は\ruby{中国}{ちゅうごく}とシベリアの\ruby{番}{ばん}が\ruby{来}{き}て、この\ruby{動}{うご}きに\ruby{加}{くわ}わって、\ruby{終}{お}わると、\ruby{裏}{うら}に\ruby{引}{ひ}っ\ruby{込}{こ}む。それから、ロシアとインドの\ruby{点灯}{てんとう}\ruby{人}{じん}の\ruby{番}{ばん}になる。

\ruby{次}{つぎ}はアフリカとヨーロッパ。それから\ruby{南}{みなみ}アメリカ、それから\ruby{北}{きた}\ruby{アメリカ}{あめりか}。しかも、この\ruby{人}{ひと}たちは、\ruby{自分}{じぶん}の\ruby{出}{で}る\ruby{順}{じゅん}を、\ruby{絶対}{ぜったい}に\ruby{間違}{まちが}えない。

でも、\ruby{北極}{ほっきょく}にひとつだけ、\ruby{南極}{なんきょく}にもひとつだけ、\ruby{街灯}{がいとう}があるんだけど、そこの\ruby{二人}{ふたり}の\ruby{点灯}{てんとう}\ruby{人}{じん}は、のんべんだらりとした\ruby{毎}{まい}\ruby{日}{ひ}を\ruby{送}{おく}っていた。だって、一\ruby{年}{ねん}に二\ruby{回}{かい}、\ruby{働}{はたら}くだけでいいんだから。




\chapter{第十七章}
うまく\ruby{言}{い}おうとして、ちょっと\ruby{嘘}{うそ}をついてしまうってことがある。\ruby{点灯}{てんとう}\ruby{人}{じん}のことも、\ruby{全部}{ぜんぶ}ありのままってわけきゃないんだ。そのせいで、\ruby{何}{なに}も\ruby{知}{し}らない\ruby{人}{ひと}に、\ruby{僕}{ぼく}らの\ruby{星}{ほし}のことを\ruby{変}{へん}に\ruby{教}{おし}えてしまったかもしれない。\ruby{地球}{ちきゅう}のほんのちょっとしか、\ruby{人間}{にんげん}ものじゃない。\ruby{地球}{ちきゅう}に\ruby{住}{す}んでる20\ruby{億}{おく}のひとに、まっすぐ\ruby{立}{た}ってもらって、\ruby{集会}{しゅうかい}みたいに\ruby{寄}{よ}り\ruby{集}{あつ}まってもらったら、わけもなく、\ruby{縦}{たて}30キロ、\ruby{横}{よこ}30キロの\ruby{広場}{ひろば}に\ruby{収}{おさ}まってしまう。\ruby{太平洋}{たいへいよう}で\ruby{一番}{いちばん}ちっちゃい\ruby{島}{しま}にだって、\ruby{入}{はい}ってしまうはずだ。

でも、\ruby{大人}{おとな}の\ruby{人}{ひと}に、こんな\ruby{事}{こと}を\ruby{言}{い}っても、やっぱり\ruby{信}{しん}じない。いろんなところが、\ruby{自分}{じぶん}たちのものだって\ruby{思}{おも}いたいんだ。\ruby{自分}{じぶん}たちはバオバブくらいでっかいものなんだって、\ruby{考}{かんが}えてる。だから、その\ruby{人}{ひと}たちに、「\ruby{数}{かぞ}えてみてよ」って、いってごらん。\ruby{数字}{すうじ}が\ruby{大好}{だいす}きだから、きっと\ruby{嬉}{うれ}しがる。でも、みんなはそんなつまらない\ruby{事}{こと}で、\ruby{時間}{じかん}をつぶさないように。

くだらない。みんな、\ruby{僕}{ぼく}を\ruby{信}{しん}じて。

\ruby{地球}{ちきゅう}についた\ruby{王子}{おうじ}\ruby{様}{さま}は、\ruby{人}{ひと}っ\ruby{子一人}{こひとり}いないことに、\ruby{驚}{おどろ}いた。もしかして、\ruby{星}{ほし}を\ruby{間違}{まちが}えたかなって、\ruby{不安}{ふあん}になってきた。その\ruby{時}{とき}、\ruby{月色}{げっしょく}の\ruby{輪}{わ}が、\ruby{砂}{すな}の\ruby{中}{なか}でほどけた。\ruby{王子}{おうじ}\ruby{様}{さま}は\ruby{一応}{いちおう}、\ruby{声}{こえ}をかけてみた。

「こんばんは。」

「こんばんは。」

「この\ruby{星}{ほし}は\ruby{何}{なん}という\ruby{星}{ほし}?」

「\ruby{地球}{ちきゅう}だよ、アフリカさ。」

「そうか、それじゃ、\ruby{地球}{ちきゅう}には\ruby{誰}{だれ}もいないの?」

「ここは\ruby{砂漠}{さばく}だからね。\ruby{砂漠}{さばく}には\ruby{誰}{だれ}もいない。\ruby{地球}{ちきゅう}は\ruby{大}{おお}きいんだよ。」

\ruby{王子}{おうじ}さまは\ruby{岩}{いわ}に\ruby{座}{すわ}って、\ruby{空}{そら}を\ruby{見上}{みあ}げた。

「\ruby{星}{ほし}がきらきら\ruby{光}{ひか}っているのは、\ruby{旅}{たび}をしている\ruby{僕}{ぼく}たち\ruby{皆}{みな}が、いつか\ruby{自分}{じぶん}の\ruby{星}{ほし}に\ruby{帰}{かえ}る\ruby{時}{とき}、すぐに\ruby{見}{み}つかるようにかな。\ruby{見}{み}て、あれが\ruby{僕}{ぼく}の\ruby{星}{ほし}、ちょうど\ruby{真上}{まうえ}にある。でも、なんて\ruby{遠}{とお}いんだ。」

「\ruby{綺麗}{きれい}な\ruby{星}{ほし}だね。なぜ\ruby{地球}{ちきゅう}に\ruby{来}{き}たんだい?」

「\ruby{僕}{ぼく}、\ruby{花}{はな}とうまくいっていないんだ。」

「そうか。」

「\ruby{人間}{にんげん}はどこ?\ruby{砂漠}{さばく}ってちょっと\ruby{寂}{さび}しいよね。」

「\ruby{人間}{にんげん}が\ruby{居}{い}ても\ruby{寂}{さび}しいさ。」

「 \ruby{君}{きみ}って\ruby{変}{か}わった\ruby{生}{い}き\ruby{物}{もの}だね。\ruby{指}{ゆび}みたいに\ruby{細}{ほそ}くて。」

「でも、\ruby{王様}{おうさま}の\ruby{指}{ゆび}よりずっと\ruby{強}{つよ}いんだよ。」

「そんなに\ruby{強}{つよ}いはずはないよ。\ruby{足}{あし}もないし、\ruby{旅}{たび}もできないじゃない。」

「\ruby{私}{わたし}は\ruby{船}{ふね}より\ruby{遠}{とお}くにお\ruby{前}{まえ}を\ruby{連}{つ}れて\ruby{行}{おこな}ける。」

\ruby{蛇}{へび}は、\ruby{金}{きん}のブレスレットのように\ruby{王子}{おうじ}さまの\ruby{足首}{あしくび}に\ruby{巻}{ま}き\ruby{付}{つ}いた。

「\ruby{私}{わたし}は\ruby{触}{ふ}れたものを\ruby{皆}{みな}\ruby{土}{つち}へと\ruby{返}{かえ}してやる。しかしお\ruby{前}{まえ}は\ruby{純粋}{じゅんすい}\ruby{無垢}{むく}で、\ruby{星}{ほし}からやってきたという。」

\ruby{王子}{おうじ}さまは\ruby{何}{なに}も\ruby{答}{こた}えなかった。

「\ruby{可愛}{かわい}そうに。この\ruby{岩}{いわ}だらけの\ruby{星}{ほし}で、お\ruby{前}{まえ}はかくも\ruby{弱}{よわ}い。いつか、\ruby{自分}{じぶん}の\ruby{星}{ほし}が\ruby{恋}{こい}しくてたまらなくなったら、\ruby{私}{わたし}が\ruby{力}{ちから}を\ruby{貸}{か}してやろう。」

「わかったよ。でも、どうして\ruby{君}{きみ}はいつも\ruby{謎}{なぞ}めいた\ruby{話}{はな}し\ruby{方}{かた}をするの?」

「\ruby{私}{わたし}には\ruby{全}{すべ}ての\ruby{謎}{なぞ}が\ruby{解}{と}けるからさ。」

そして、どちらも\ruby{黙}{だま}り\ruby{込}{こ}んだ。



\end{document}