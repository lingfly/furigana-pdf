% !TeX encoding = UTF-8
% !TeX program = LuaLaTeX

\documentclass[12pt,a4paper,oneside,openany]{book}

\usepackage{luatexja-ruby}       %假名标注
\usepackage{xcolor}			 %引入颜色
\usepackage[hidelinks]{hyperref} %给目录添加超链接
\usepackage{titlesec}
\usepackage{indentfirst}         %章节首页首行缩进

% 自定义章节标题样式,调整默认垂直间距
\titleformat{\chapter}       	 % 要格式化的章节命令,如 \chapter、\section、\subsection 等。
[hang]                       	 % 标题的形状,可以是 hang(悬挂格式,默认)、block(块格式)
{\normalfont\Huge\bfseries}   	 % format,标题的整体格式,可以包括字体、大小、粗细等
{Chapter\thechapter}          	 % label,标题编号的格式,如 \thechapter、\thesection 等。
{1em}                         	 % Spacing between label and title
{}  

\headheight = 13pt           	 % 页眉高度
\headsep = 30pt		        	 % 页眉和正文的间距
\topmargin = -30pt               % 页眉和页面顶端的间距
\titlespacing{\chapter}{0pt}{-60pt}{10pt} % 调整标题间距
\textheight = 700pt              % 正文高度
\textwidth = 450pt 			 % 正文宽度
\setlength{\hoffset}{-20pt}      % 正文向左偏移20pt

\ltjsetruby{size=0.6}            %设置振假名字号
%\ltjsetruby{fontcmd=\gtfamily}  %设置振假名字体
\ltjsetruby{mode=00}             %设置振假名的「進入」和「突出」模式

\setlength{\parindent}{2em}	 %首行缩进
\marginparwidth = 72pt           %边栏宽度

\linespread{1.8}			 %行距
\selectfont

% 边注
\newcounter{num}[chapter]
\newcommand{\translate}[2]{\addtocounter{num}{1} {\color{orange} #1\textsuperscript{\scriptsize \thenum}}\marginpar{\scriptsize \textsuperscript{\scriptsize \thenum} #2}}

\begin{document}
\tableofcontents

\chapter{序章}
% !TeX encoding = UTF-8
% !TeX program = LuaLaTeX

\documentclass[18pt,a4paper,oneside,openany]{book}
% \raggedright
%\pagestyle{headings}
\setlength{\parindent}{2em}


\usepackage{luatexja-ruby}
\usepackage{indentfirst}

\ltjsetruby{size=0.6}            %设置振假名字号
%\ltjsetruby{fontcmd=\gtfamily}  %设置振假名字体
\ltjsetruby{mode=00}             %设置振假名的「進入」和「突出」模式

\begin{document}


\tableofcontents


\chapter{六歳の時}
\markright{六歳の時}


%\gtfamily                       %设置主体文字字体
%\normalsize
\linespread{1.8}
\selectfont

六\ruby{歳}{さい}の\ruby{時}{とき}\ruby{僕}{ぼく}は、「\ruby{体験談}{たいけんだん}」という\ruby{原生林}{げんせいりん}について\ruby{書}{か}かれた\ruby{本}{ほん}で、\ruby{素晴}{すば}らしい\ruby{挿絵}{さしえ}を\ruby{見}{み}たことがある。それは\ruby{大蛇}{だいじゃ}の\ruby{ボア}{ぼあ}が\ruby{猛獣}{もうじゅう}を\ruby{飲}{の}み\ruby{込}{こ}もうとしている\ruby{絵}{え}だった。\ruby{本}{ほん}にはこんな\ruby{説明}{せつめい}があった。

\ruby{ボア}{ぼあ}は\ruby{獲物}{えもの}を\ruby{噛}{か}まずに\ruby{丸}{まる}ごと\ruby{飲}{の}み\ruby{込}{こ}みます。すると\ruby{動}{うご}けなくなるので、\ruby{獲物}{えもの}を\ruby{消化}{しょうか}する\ruby{半年}{はんとし}もの\ruby{間}{かん}、ずっと\ruby{眠}{ねむ}って\ruby{過}{す}ごします。

\ruby{僕}{ぼく}は\ruby{ジャングル}{じゃんぐる}での\ruby{冒険}{ぼうけん}についていろいろと\ruby{考}{かんが}え、\ruby{自分}{じぶん}でも\ruby{色鉛筆}{いろえんぴつ}を\ruby{使}{つか}って、\ruby{生}{う}まれて\ruby{初}{はじ}めての\ruby{絵}{え}を\ruby{描}{か}き\ruby{上}{あ}げた。その\ruby{傑作}{けっさく}を\ruby{大人}{おとな}たちに\ruby{見}{み}せ、\ruby{怖}{こわ}いかどうか\ruby{聞}{き}いてみた。すると、こんな\ruby{答}{こた}えが\ruby{返}{かえ}ってきた。

どうして\ruby{帽子}{ぼうし}が\ruby{怖}{こわ}いんだい?

\ruby{帽子}{ぼうし}の\ruby{絵}{え}なんかじゃなかった。\ruby{ゾウ}{ぞう}を\ruby{消化}{しょうか}している\ruby{ボア}{ぼあ}を\ruby{描}{えが}いたのだ。でも、\ruby{大人}{おとな}にはわからないらしいので、\ruby{今度}{こんど}は\ruby{ボア}{ぼあ}の\ruby{内側}{うちがわ}の\ruby{絵}{え}を\ruby{描}{か}いてみた。\ruby{大人}{おとな}には\ruby{何時}{なんじ}だって\ruby{説明}{せつめい}が\ruby{必要}{ひつよう}なのだ。\ruby{僕}{ぼく}の\ruby{二番目}{にばんめ}の\ruby{絵}{え}では、ちゃんと\ruby{ボア}{ぼあ}の\ruby{中}{なか}にいる\ruby{ゾウ}{ぞう}が\ruby{見}{み}えていた。しかし\ruby{大人}{おとな}たちは\ruby{中}{なか}が\ruby{見}{み}えようが\ruby{見}{み}えまいが、\ruby{ボア}{ぼあ}の\ruby{絵}{え}は\ruby{片付}{かたづ}けて、\ruby{地理}{ちり}や\ruby{歴史}{れきし}、\ruby{算数}{さんすう}や\ruby{文法}{ぶんぽう}の\ruby{勉強}{べんきょう}をしなさいと、\ruby{僕}{ぼく}を\ruby{嗜}{たしな}めた。

こうして、6\ruby{歳}{さい}にして\ruby{僕}{ぼく}は\ruby{偉大}{いだい}な\ruby{画家}{がか}になるという\ruby{夢}{ゆめ}を\ruby{諦}{あきら}めた。\ruby{作品}{さくひん}\ruby{第}{だい}一\ruby{号}{ごう}と\ruby{第}{だい}二\ruby{号}{ごう}が\ruby{共}{とも}に\ruby{不評}{ふひょう}で、\ruby{気持}{きも}ちが\ruby{挫}{くじ}けてしまったのだ。

\ruby{大人}{おとな}というのは、\ruby{自分}{じぶん}たちとは\ruby{全}{まった}く\ruby{何}{なに}もわかっていないから、いつも\ruby{子供}{こども}の\ruby{方}{ほう}から\ruby{説明}{せつめい}してあげなきゃいけなくて、うんざりする。\ruby{僕}{ぼく}は\ruby{別}{べつ}の\ruby{仕事}{しごと}を\ruby{選}{えら}ぶ\ruby{必要}{ひつよう}に\ruby{迫}{せま}られて、\ruby{飛行機}{ひこうき}の\ruby{操縦士}{そうじゅうし}になった。そして、\ruby{世界}{せかい}\ruby{中}{ちゅう}をあちこち\ruby{飛}{と}び\ruby{回}{まわ}った。\ruby{地理}{ちり}は\ruby{確}{たし}かに\ruby{役}{やく}に\ruby{立}{た}った。\ruby{僕}{ぼく}は\ruby{一目}{ひとめ}で\ruby{中国}{ちゅうごく}と\ruby{アリゾナ}{ありぞな}を\ruby{見分}{みわ}ける\ruby{事}{こと}ができる。\ruby{夜間飛行}{やかんひこう}で\ruby{迷}{まよ}った\ruby{時}{とき}など、そういう\ruby{知識}{ちしき}があると\ruby{本当}{ほんとう}に\ruby{助}{たす}かる。

これまでの\ruby{人生}{じんせい}で、\ruby{僕}{ぼく}はたくさんの\ruby{重要}{じゅうよう}\ruby{人物}{じんぶつ}と\ruby{知}{し}り\ruby{合}{あ}った。\ruby{随分}{ずいぶん}\ruby{多}{おお}くの\ruby{大人}{おとな}たちと\ruby{一緒}{いっしょ}に\ruby{暮}{く}らしたし、\ruby{マジカ}{まじか}にも\ruby{見}{み}てきた。それでも\ruby{僕}{ぼく}の\ruby{考}{かんが}えはあまり\ruby{変}{か}わらなかった。\ruby{僕}{ぼく}は\ruby{物分}{ものわか}りのよさそうな\ruby{人}{ひと}に\ruby{出会}{であ}った\ruby{時}{とき}には\ruby{必}{かなら}ず、\ruby{肌}{はだ}に\ruby{離}{はな}さず\ruby{持}{も}ち\ruby{歩}{ある}いていた\ruby{作品}{さくひん}\ruby{第}{だい}一\ruby{号}{ごう}を\ruby{見}{み}せ、\ruby{実験}{じっけん}していた。その\ruby{人}{ひと}が\ruby{本当}{ほんとう}に\ruby{物事}{ものごと}の\ruby{分}{わ}かる\ruby{人}{ひと}かどうか、\ruby{知}{し}りたかったから。でも、\ruby{答}{こた}えはいつも\ruby{同}{おな}じだった。

\ruby{帽子}{ぼうし}だね。

その\ruby{後}{あと}\ruby{僕}{ぼく}は\ruby{ボア}{ぼあ}の\ruby{話}{はなし}も、\ruby{原生林}{げんせいりん}の\ruby{話}{はなし}も、\ruby{星}{ほし}の\ruby{話}{はなし}もしなかった。\ruby{話}{はなし}を\ruby{合}{あ}わせて、\ruby{ブリッジ}{ぶりっじ}や\ruby{ゴルフ}{ごるふ}や、\ruby{政治}{せいじ}や\ruby{ネクタイ}{ねくたい}の\ruby{話}{はなし}をした。するとその\ruby{大人}{おとな}は\ruby{話}{はなし}が\ruby{分}{わ}かる\ruby{相手}{あいて}と\ruby{知}{し}り\ruby{合}{あ}えたと\ruby{言}{い}って\ruby{喜}{よろこ}ぶのだ。

\chapter{Chapter2}

萨芬奋斗奋斗

\end{document}



\chapter{綿羊}
\ruby{新}{あたら}しい\ruby{学校}{がっこう}の\ruby{門}{もん}をくぐる\ruby{前}{まえ}に、トットちゃんのママが、なぜ\ruby{不安}{ふあん}なのかを\ruby{説明}{せつめい}すると、それはトットちゃんが、\ruby{小学校}{しょうがっこう}一\ruby{年}{ねん}なのにかかわらず、すでに\ruby{学校}{がっこう}を\ruby{退学}{たいがく}になったからだった。\ruby{一年生}{いちねんせい}で!!

つい\ruby{先週}{せんしゅう}のことだった。ママはトットちゃんの\ruby{担任}{たんにん}の\ruby{先生}{せんせい}に\ruby{呼}{よ}ばれて、はっきり、こういわれた。

「お\ruby{宅}{たく}のお\ruby{嬢}{じょう}さんがいると、クラス\ruby{中}{じゅう}の\ruby{迷惑}{めいわく}になります。よその\ruby{学校}{がっこう}にお\ruby{連}{つ}れください!」 \ruby{若}{わか}くて\ruby{美}{うつく}しい\ruby{女}{おんな}の\ruby{先生}{せんせい}は、ため\ruby{息}{いき}をつきながら、\ruby{繰}{く}り\ruby{返}{かえ}した。 「\ruby{本当}{ほんとう}に\ruby{困}{こま}ってるんです!」 ママはびっくりした。(\ruby{一体}{いったい}、どんなことを……。クラス\ruby{中}{じゅう}の\ruby{迷惑}{めいわく}になる、どんなことを、あの\ruby{子}{こ}がするんだろうか……)

\ruby{先生}{せんせい}は、カールしたまつ\ruby{毛}{げ}をパチパチさせ、パーマのかかった\ruby{短}{みじか}い\ruby{内巻}{うちまき}の\ruby{毛}{け}を\ruby{手}{て}でなでながら\ruby{説明}{せつめい}に\ruby{取}{と}り\ruby{掛}{か}かった。

「まず、\ruby{授業}{じゅぎょう}\ruby{中}{ちゅう}に、\ruby{机}{つくえ}のフタを、百ぺんくらい、あけたり\ruby{閉}{し}めたりするんです。そこで\ruby{私}{わたし}が、\ruby{用事}{ようじ}がないのに、\ruby{開}{あ}けたり\ruby{閉}{し}めたりしてはいけませんと\ruby{申}{もう}しますと、お\ruby{宅}{たく}のお\ruby{嬢}{じょう}さんは、ノートから、\ruby{筆箱}{ふでばこ}、\ruby{教科書}{きょうかしょ}、\ruby{全部}{ぜんぶ}を\ruby{机}{つくえ}の\ruby{中}{なか}にしまってしまって、\ruby{一}{ひと}つ\ruby{一}{ひと}つ\ruby{取}{と}り\ruby{出}{だ}すんです。たとえば、\ruby{書}{か}き\ruby{取}{と}りをするとしますね。するとお\ruby{嬢}{じょう}さんは、まずフタを\ruby{開}{あ}けて、ノートを\ruby{取}{と}り\ruby{出}{だ}した、と\ruby{思}{おも}うが\ruby{早}{はや}いか、パタン!とフタを\ruby{閉}{し}めてしまいます。そして、すぐにまた\ruby{開}{あ}けて\ruby{頭}{あたま}を\ruby{中}{なか}につっこんで\ruby{筆箱}{ふでばこ}から“ア”を\ruby{書}{か}くための\ruby{鉛筆}{えんぴつ}を\ruby{出}{だ}すと、\ruby{急}{いそ}いで\ruby{閉}{し}めて、“ア”を\ruby{書}{か}きます。ところが、うまく\ruby{書}{か}けなかったり\ruby{間違}{まちが}えたりしますね。そうすると、フタを\ruby{開}{あ}けて、また\ruby{頭}{あたま}を\ruby{突}{つ}っ\ruby{込}{こ}んで、\ruby{消}{け}し\ruby{ゴム}{ごむ}をだし、\ruby{閉}{し}めると、\ruby{急}{いそ}いで\ruby{消}{け}し\ruby{ゴム}{ごむ}を\ruby{使}{つか}い、\ruby{次}{つぎ}に、すごい\ruby{早}{はや}さで\ruby{開}{あ}けて、\ruby{消}{け}し\ruby{ゴム}{ごむ}をしまって、フタを\ruby{閉}{し}めてしまいます。で、すぐ、また\ruby{開}{あ}けるので\ruby{見}{み}てますと、“ア”ひとつだけ\ruby{書}{か}いて、\ruby{道具}{どうぐ}をひとつひとつ、\ruby{全部}{ぜんぶ}しまうんです。\ruby{鉛筆}{えんぴつ}をしまい、\ruby{閉}{し}めて、また\ruby{開}{あ}けてノートをしまい……というふうに。そして、\ruby{次}{つぎ}の“イ”のときに、また、ノートから\ruby{始}{はじ}まって、\ruby{鉛筆}{えんぴつ}、\ruby{消}{け}し\ruby{ゴム}{ごむ}……その\ruby{度}{たび}に,\ruby{私}{わたし}の\ruby{目}{め}の\ruby{前}{まえ}で、\ruby{目}{め}まぐるしく、\ruby{机}{つくえ}のフタが\ruby{開}{ひら}いたり\ruby{閉}{し}まったり。\ruby{私}{わたし}、\ruby{目}{め}が\ruby{回}{まわ}るんです。でも、\ruby{一応}{いちおう}、\ruby{用事}{ようじ}があるんですから、いけないとは\ruby{申}{もう}せませんけど……」 \ruby{先生}{せんせい}のまつ\ruby{毛}{げ}が、その\ruby{時}{とき}を\ruby{思}{おも}い\ruby{出}{だ}したように、パチパチと\ruby{早}{はや}くなった。

そこで\ruby{聞}{き}いて、ママには、トットちゃんが、なんで、\ruby{学校}{がっこう}の\ruby{机}{つくえ}を、そんなに\ruby{開}{あ}けたり\ruby{閉}{し}めたりするのか、ちょっとわかった。というのは、\ruby{初}{はじ}めて\ruby{学校}{がっこう}に\ruby{行}{い}って\ruby{帰}{かえ}ってきた\ruby{日}{ひ}に、トットちゃんが、ひどく\ruby{興奮}{こうふん}して、こうママに\ruby{報告}{ほうこく}したことを\ruby{思}{おも}い\ruby{出}{だ}したからだった。「ねえ、\ruby{学校}{がっこう}って、すごいの。\ruby{家}{いえ}の\ruby{机}{つくえ}の\ruby{引}{ひ}き\ruby{出}{だ}しは、こんな\ruby{風}{ふう}に、\ruby{引}{ひ}っ\ruby{張}{ぱ}るのだけど、\ruby{学校}{がっこう}のはフタが\ruby{上}{うえ}にあがるの。\ruby{ゴミ}{ごみ}\ruby{箱}{ばこ}のフタと\ruby{同}{おな}じなんだけど、もっとツルツルで、いろんなものが、しまえて、とってもいいんだ!」ママには、\ruby{今}{いま}まで\ruby{見}{み}たことのない\ruby{机}{つくえ}の\ruby{前}{まえ}で、トットちゃんが\ruby{面白}{おもしろ}がって、\ruby{開}{あ}けたり\ruby{閉}{し}めたりしてる\ruby{様子}{ようす}が\ruby{目}{め}に\ruby{見}{み}えるようだった。そして、それは、(そんなに\ruby{悪}{わる}いことではないし、\ruby{第}{だい}一、だんだん\ruby{馴}{な}れてくれば、そんなに\ruby{開}{あ}けたり\ruby{閉}{し}めたりしなくなるだろう)と\ruby{考}{かんが}えたけど、\ruby{先生}{せんせい}には、「よく\ruby{注意}{ちゅうい}しますから」といった。ところが、\ruby{先生}{せんせい}には、それまでの\ruby{調子}{ちょうし}より\ruby{声}{こえ}をもうすこし\ruby{高}{たか}くして、こういった。「それだけなら、よろしいんですけど!」ママは、すこし\ruby{身}{み}がちぢむような\ruby{気}{き}がした。\ruby{先生}{せんせい}は、\ruby{体}{からだ}を\ruby{少}{すこ}し\ruby{前}{まえ}にのり\ruby{出}{だ}すといった。「\ruby{机}{つくえ}で\ruby{音}{おと}を\ruby{立}{た}ててないな、と\ruby{思}{おも}うと、\ruby{今度}{こんど}は、\ruby{授業}{じゅぎょう}\ruby{中}{ちゅう}、\ruby{立}{た}ってるんです。ずーっと!」ママは、またびっくりしたので\ruby{聞}{き}いた。「\ruby{立}{た}ってるって、どこにでございましょうか?」\ruby{先生}{せんせい}はすこし\ruby{怒}{おこ}った\ruby{風}{ふう}にいった。「\ruby{教室}{きょうしつ}の\ruby{窓}{まど}のところです!」ママは、わけが\ruby{分}{わ}からないので、\ruby{続}{つづ}けて\ruby{質問}{しつもん}した。「\ruby{窓}{まど}のところで、\ruby{何}{なに}をしてるんでしょうか?」\ruby{先生}{せんせい}は、\ruby{半分}{はんぶん}、\ruby{叫}{さけ}ぶような\ruby{声}{こえ}で\ruby{言}{い}った。「チンドン\ruby{屋}{や}を\ruby{呼}{よ}び\ruby{込}{こ}むためです。」

\ruby{先生}{せんせい}の\ruby{話}{はなし}を、まとめて\ruby{見}{み}ると、こういうことになるらしかった。一\ruby{時間}{じかん}\ruby{目}{め}に、\ruby{机}{つくえ}をパタパタを、かなりやると、それ\ruby{以後}{いご}は、\ruby{机}{つくえ}を\ruby{離}{はな}れて、\ruby{窓}{まど}のところに\ruby{立}{た}って\ruby{外}{そと}を\ruby{見}{み}ている。そこで、\ruby{静}{しず}かにしていてくれるのなら、\ruby{立}{た}っててもいい、と\ruby{先生}{せんせい}が\ruby{思}{おも}った\ruby{矢先}{やさき}に、\ruby{突然}{とつぜん}、トットちゃんは、\ruby{大}{おお}きい\ruby{声}{こえ}で「チンドン\ruby{屋}{や}さーん!」と\ruby{外}{そと}に\ruby{向}{む}かって\ruby{叫}{さけ}んだ。だいたい、この\ruby{教室}{きょうしつ}の\ruby{窓}{まど}というのが、トットちゃんにっとては\ruby{幸福}{こうふく}なことに、\ruby{先生}{せんせい}にとっては\ruby{不幸}{ふこう}なことに、1\ruby{階}{かい}にあり、しかも\ruby{通}{とお}りは\ruby{目}{め}の\ruby{前}{まえ}だった。そして\ruby{境}{さかい}といえば、\ruby{低}{ひく}い、\ruby{生垣}{いけがき}があるだけだったから、トットちゃんは、\ruby{簡単}{かんたん}に、\ruby{通}{とお}りを\ruby{歩}{ある}いてる\ruby{人}{ひと}と、\ruby{話}{はなし}ができるわけだったのだ。さて、\ruby{通}{とお}りかかったチンドン\ruby{屋}{や}さんは、\ruby{呼}{よ}ばれたから\ruby{教室}{きょうしつ}の\ruby{下}{した}まで\ruby{来}{く}る。するとトットちゃんは、うれしそうに、クラス\ruby{中}{じゅう}の\ruby{皆}{みな}に\ruby{呼}{よ}びかけた。「\ruby{来}{き}たわよー」。\ruby{勉強}{べんきょう}してたクラス\ruby{中}{じゅう}の\ruby{子供}{こども}は、\ruby{全員}{ぜんいん}、その\ruby{声}{こえ}で\ruby{窓}{まど}のところに、\ruby{詰}{つ}め\ruby{掛}{か}けて、\ruby{口々}{くちぐち}に\ruby{叫}{さけ}ぶ。「チンドン\ruby{屋}{や}さーん」。すると、トットちゃんは、チンドン\ruby{屋}{や}さんに\ruby{頼}{たの}む。「ねえ、ちょっとだけで、やってみて?」\ruby{学校}{がっこう}のそばを\ruby{通}{とお}る\ruby{時}{とき}は、\ruby{音}{おと}をおさえめにしているチンドン\ruby{屋}{や}さんも、せっかくの\ruby{頼}{たの}みだからというので\ruby{盛大}{せいだい}に\ruby{始}{はじ}める。クラスネットや\ruby{鉦}{かね}や\ruby{太鼓}{たいこ}や、\ruby{三味線}{さみせん}で。その\ruby{間}{あいだ}、\ruby{先生}{せんせい}がどうしてるか、といえば、\ruby{一段落}{いちだんらく}つくまで、ひとり\ruby{教壇}{きょうだん}で、ジーっと\ruby{待}{ま}ってるしかない。(この一\ruby{曲}{きょく}が\ruby{終}{お}わるまでの\ruby{辛抱}{しんぼう}なんだから)と\ruby{自分}{じぶん}に\ruby{言}{い}い\ruby{聞}{き}かせながら。

さて、一\ruby{曲}{きょく}\ruby{終}{お}わると、チンドン\ruby{屋}{や}さんは\ruby{去}{さ}って\ruby{行}{い}き、\ruby{生徒}{せいと}たちは、それぞれの\ruby{席}{せき}に\ruby{戻}{もど}る。ところが、\ruby{驚}{おどろ}いたことに、トットちゃんは、\ruby{窓}{まど}のところから\ruby{動}{うご}かない。「どうして、まだ、そこにいるのですか?」という\ruby{先生}{せんせい}の\ruby{問}{と}いに、トットちゃんは、\ruby{大真面目}{おおまじめ}に\ruby{答}{こた}えた。「だって、また\ruby{違}{ちが}うチンドン\ruby{屋}{や}さんが\ruby{来}{き}たら、お\ruby{話}{はなし}しなきゃならないし。それから、さっきのチンドン\ruby{屋}{や}さんが、また、\ruby{戻}{もど}ってきたら、\ruby{大変}{たいへん}だからです。」

「これじゃ、\ruby{授業}{じゅぎょう}にならない、ということが、おわかりでしょう?」\ruby{話}{はな}してるうちに、\ruby{先生}{せんせい}は、かなり\ruby{感情的}{かんじょうてき}なってきて、ママに\ruby{言}{い}った。ママは、(なるほど、これでは\ruby{先生}{せんせい}も、お\ruby{困}{こま}りだわ)と\ruby{思}{おも}いかけた。とたん、\ruby{先生}{せんせい}は、また\ruby{一段}{いちだん}と\ruby{大}{おお}きな\ruby{声}{こえ}で、こういった。「それに……」ママはびっくりしながらも、\ruby{情}{なさ}けない\ruby{思}{おも}い\ruby{出}{で}\ruby{先生}{せんせい}に\ruby{聞}{き}いた。「まだ、あるんでございましょうか……」\ruby{先生}{せんせい}は、すぐいった。「“まだ”というように、\ruby{数}{かぞ}えられるくらいなら、こうやって、やめていただきたい、とお\ruby{願}{ねが}いはしません!!」それから\ruby{先生}{せんせい}は、\ruby{少}{すこ}し\ruby{息}{いき}を\ruby{静}{しず}めて、ママの\ruby{顔}{かお}を\ruby{見}{み}て\ruby{言}{い}った。「\ruby{昨日}{きのう}のことですが、\ruby{例}{れい}によって、\ruby{窓}{まど}のところに\ruby{立}{た}っているので、またチンドン\ruby{屋}{や}だと\ruby{思}{おも}って\ruby{授業}{じゅぎょう}をしておりましたら、これが、また\ruby{大}{おお}きな\ruby{声}{こえ}で、いきなり、『\ruby{何}{なに}してるの?』と、\ruby{誰}{だれ}かに、\ruby{何}{なに}かを\ruby{聞}{き}いているんですね。\ruby{相手}{あいて}は、\ruby{私}{わたし}のほうから\ruby{見}{み}えませんので、\ruby{誰}{だれ}だろう、と\ruby{思}{おも}っておりますと、また\ruby{大}{おお}きな\ruby{声}{こえ}で、『ねえ、\ruby{何}{なに}をしてるの?』って。それも、\ruby{今度}{こんど}は、\ruby{通}{とお}りにでなく、\ruby{上}{うえ}のほうに\ruby{向}{む}かって\ruby{聞}{き}いてるんです。\ruby{私}{わたし}も\ruby{気}{き}になりまして、\ruby{相手}{あいて}の\ruby{返事}{へんじ}が\ruby{聞}{き}こえるかした、と\ruby{耳}{みみ}を\ruby{澄}{す}ましてみましたが、\ruby{返事}{へんじ}がないんです。お\ruby{嬢}{じょう}さんは、それでも、さかんに、『ねえ、\ruby{何}{なに}してるの?』を\ruby{続}{つづ}けるので、\ruby{授業}{じゅぎょう}にもさしさわりがあるので、\ruby{窓}{まど}のところに\ruby{行}{い}って、お\ruby{嬢}{じょう}さんの\ruby{話}{はな}しかけてる\ruby{相手}{あいて}が\ruby{誰}{だれ}なのか、\ruby{見}{み}てみようと\ruby{思}{おも}いました。\ruby{窓}{まど}から\ruby{顔}{かお}を\ruby{出}{だ}して\ruby{上}{うえ}を\ruby{見}{み}ましたら、なんと、つばめが、\ruby{教室}{きょうしつ}の\ruby{屋根}{やね}の\ruby{下}{した}に、\ruby{巣}{す}を\ruby{作}{つく}っているんです。その、つばめに\ruby{聞}{き}いてるんですね。そりゃ\ruby{私}{わたし}も、\ruby{子供}{こども}の\ruby{気持}{きも}ちが、\ruby{分}{わ}からないわけじゃありませんから、つばめに\ruby{聞}{き}いてることを、\ruby{馬鹿}{ばか}げている、とは\ruby{申}{もう}しません。\ruby{授業}{じゅぎょう}\ruby{中}{ちゅう}に、あんな\ruby{声}{こえ}で、つばめに、『\ruby{何}{なに}をしてるのか?』と\ruby{聞}{き}かなくてもいいと、\ruby{私}{わたし}は\ruby{思}{おも}うんです」そして\ruby{先生}{せんせい}は、ママが、\ruby{一体}{いったい}なんとお\ruby{詫}{わ}びをしよう、と\ruby{口}{くち}を\ruby{開}{あ}きかけたのより、\ruby{早}{はや}く\ruby{言}{い}った。「それから、こういうことも、ございました。\ruby{初}{はじ}めての\ruby{図画}{ずが}の\ruby{時間}{じかん}のことですが、\ruby{国旗}{こっき}を\ruby{描}{えが}いて\ruby{御覧}{ごらん}なさい、と\ruby{私}{わたし}が\ruby{申}{もう}しましたら、\ruby{他}{ほか}の\ruby{子}{こ}は、\ruby{画用紙}{がようし}に、ちゃんと\ruby{日}{ひ}の\ruby{丸}{まる}を\ruby{描}{えが}いたんですが、お\ruby{宅}{たく}のお\ruby{嬢}{じょう}さんは、\ruby{朝日}{あさひ}\ruby{新聞}{しんぶん}の\ruby{模様}{もよう}のような、\ruby{軍艦旗}{ぐんかんき}を\ruby{描}{えが}き\ruby{始}{はじ}めました。それなら、それでいい、と\ruby{思}{おも}ってましたら、\ruby{突然}{とつぜん}、\ruby{旗}{はた}の\ruby{周}{まわ}りに、ふさを、つけ\ruby{始}{はじ}めたんです。ふさ。よく\ruby{青年}{せいねん}\ruby{団}{だん}とか、そういった\ruby{旗}{はた}についてます。あの、ふさです。で、それも、まあ、どこかで\ruby{見}{み}たのだろうから、と\ruby{思}{おも}っておりました。ところが、ちょっと\ruby{目}{め}を\ruby{離}{はな}したキスに、まあ、\ruby{黄色}{きいろ}のふさを、\ruby{机}{つくえ}にまで、どんどん\ruby{描}{えが}いちゃってるんです。だいたい\ruby{画用紙}{がようし}に、ほぼいっぱいに\ruby{旗}{はた}を\ruby{描}{えが}いたんですから、ふさの\ruby{余裕}{よゆう}は、もともと、あまりなかったんですが、それに、\ruby{黄色}{きいろ}のクレヨンで、ゴシゴシふさを\ruby{描}{えが}いたんですね。それが、はみ\ruby{出}{だ}しちゃって、\ruby{画用紙}{がようし}をどかしたら、\ruby{机}{つくえ}に、ひどい\ruby{黄色}{きいろ}のギザギザが\ruby{残}{のこ}ってしまって、ふいても、こすっても、とれません。まあ、\ruby{幸}{さいわ}いなことは、ギザギザが三\ruby{方向}{ほうこう}だけだった、ってことでしょうか?」ママは、ちぢこまりながらも、\ruby{急}{いそ}いで\ruby{質問}{しつもん}した。「\ruby{三方}{さんぼう}\ruby{向}{むか}っていうのは……」\ruby{先生}{せんせい}は、そろそろ\ruby{疲}{つか}れてきた、という\ruby{様子}{ようす}だったが、それでも\ruby{親切}{しんせつ}にいった。「\ruby{旗竿}{はたざお}を\ruby{左}{ひだり}はじに\ruby{描}{えが}きましたから、\ruby{旗}{はた}のギザギザは、\ruby{三方}{さんぼう}だけだったんでございます」ママは、\ruby{少}{すこ}し\ruby{助}{たす}かった、と\ruby{思}{おも}って、「はあ、それで\ruby{三方}{さんぼう}だけ……」といった。すると、\ruby{先生}{せんせい}は、\ruby{次}{つぎ}に、とっても、ゆっくりの\ruby{口調}{くちょう}で、\ruby{一言}{ひとこと}ずつ\ruby{区切}{くぎ}って「ただし、その\ruby{代}{か}わり、\ruby{旗竿}{はたざお}のはじが、やはり、\ruby{机}{つくえ}に、はみ\ruby{出}{だ}して、\ruby{残}{のこ}っております!!」それから\ruby{先生}{せんせい}は\ruby{立}{た}ち\ruby{上}{あ}がると、かなり\ruby{冷}{つめ}たい\ruby{感}{かん}じで、とどめをさすように\ruby{言}{い}った。「それと、\ruby{迷惑}{めいわく}しているのは、\ruby{私}{わたし}だけではございません。\ruby{隣}{となり}の\ruby{一年生}{いちねんせい}の\ruby{受}{う}け\ruby{持}{も}ちの\ruby{先生}{せんせい}もお\ruby{困}{こま}りのことが、あるそうですから……」ママは、\ruby{決心}{けっしん}しないわけには、いかなかった。(\ruby{確}{たし}かに、これじゃ、\ruby{他}{ほか}の\ruby{生徒}{せいと}さんに、ご\ruby{迷惑}{めいわく}すぎる。どこか、\ruby{他}{ほか}の\ruby{学校}{がっこう}を\ruby{探}{さが}して、\ruby{移}{うつ}したほうが、よさそうだ。\ruby{何}{なん}とか、あの\ruby{子}{こ}の\ruby{性格}{せいかく}がわかっていただけて、\ruby{皆}{みな}と\ruby{一緒}{いっしょ}にやっていくことを\ruby{教}{おし}えてくださるような\ruby{学校}{がっこう}に……)そうして、ママが、あっちこっち、かけずりまわって\ruby{見}{み}つけたのが、これから\ruby{行}{い}こうとしている\ruby{学校}{がっこう}、というわけだったのだ。ママは、この\ruby{退学}{たいがく}のことを、トットちゃんに\ruby{話}{はな}していなかった。\ruby{話}{はな}しても、\ruby{何}{なに}がいけなかったのか、わからないだろうし、また、そんなにことで、トットちゃんが、コンプレックスを\ruby{持}{も}つのも、よくないと\ruby{思}{おも}ったから、(いつか、\ruby{大}{おお}きくなったら、\ruby{話}{はな}しましょう)と、きめていた。ただ、トットちゃんには、こういった。「\ruby{新}{あたら}しい\ruby{学校}{がっこう}に\ruby{行}{い}ってみない?いい\ruby{学校}{がっこう}だって\ruby{話}{はなし}よ」トットちゃんは、\ruby{少}{すこ}し\ruby{考}{かんが}えてから、\ruby{言}{い}った。「\ruby{行}{い}くけど……」ママは、(この\ruby{子}{こ}は、\ruby{今}{いま}\ruby{何}{なに}を\ruby{考}{かんが}えてるのだろうか)と\ruby{思}{おも}った。(うすうす、\ruby{退学}{たいがく}のこと、\ruby{気}{き}がついていたんだろうか……)\ruby{次}{つぎ}の\ruby{瞬間}{しゅんかん}、トットちゃんは、ママの\ruby{腕}{うで}の\ruby{中}{なか}に、\ruby{飛}{と}び\ruby{込}{こ}んで\ruby{来}{き}て、いった。「ねえ、\ruby{今度}{こんど}の\ruby{学校}{がっこう}に、いいチンドン\ruby{屋}{や}さん、\ruby{来}{く}るかな?」とにかく、そんなわけで、トットちゃんとママは、\ruby{新}{あたら}しい\ruby{学校}{がっこう}に\ruby{向}{む}かって、\ruby{歩}{ある}いているのだった。




\chapter{花儿}
\ruby{学校}{がっこう}の\ruby{門}{もん}が、はっきり\ruby{見}{み}えるところまで\ruby{来}{き}て、トットちゃんは、\ruby{立}{た}ち\ruby{止}{どま}った。なぜなら、この\ruby{間}{あいだ}まで\ruby{行}{い}っていた\ruby{学校}{がっこう}の\ruby{門}{もん}は、\ruby{立派}{りっぱ}なコンクリートみたいな\ruby{柱}{はしら}で、\ruby{学校}{がっこう}の\ruby{名前}{なまえ}も、\ruby{大}{おお}きく\ruby{書}{か}いてあった。ところが、この\ruby{新}{あたら}しい\ruby{学校}{がっこう}の\ruby{門}{もん}ときたら、\ruby{低}{ひく}い\ruby{木}{き}で、しかも\ruby{葉}{は}っぱが\ruby{生}{は}えていた。「\ruby{地面}{じめん}から\ruby{生}{は}えてる\ruby{門}{もん}ね」と、トットちゃんはママに\ruby{言}{い}った。そうして、こう、\ruby{付}{つ}け\ruby{加}{くわ}えた。「きっと、どんどんはえて、\ruby{今}{いま}に\ruby{電信柱}{でんしんばしら}より\ruby{高}{たか}くなるわ」\ruby{確}{たし}かに、その二\ruby{本}{ほん}の\ruby{門}{もん}は、\ruby{根}{ね}っこのある\ruby{木}{き}だった。トットちゃんは、\ruby{門}{もん}に\ruby{近}{ちか}づくと、いきなり\ruby{顔}{かお}を、\ruby{斜}{なな}めにした。なぜかといえば、\ruby{門}{もん}にぶら\ruby{下}{さ}げてある\ruby{学校}{がっこう}の\ruby{名前}{なまえ}を\ruby{書}{か}いた\ruby{札}{さつ}が、\ruby{風}{かぜ}に\ruby{吹}{ふ}かれたのか、\ruby{斜}{なな}めになっていたからだった。「トモエがくえん」トットちゃんは、\ruby{顔}{かお}を\ruby{斜}{なな}めにしたまま、\ruby{表札}{ひょうさつ}を\ruby{読}{よ}み\ruby{上}{あ}げた。そして、ママに、「トモエって、なあに?」と\ruby{聞}{き}こうとしたときだった。トットちゃんの\ruby{目}{め}の\ruby{端}{はし}に、\ruby{夢}{ゆめ}としか\ruby{思}{おも}えないものが\ruby{見}{み}えたのだった。トットちゃんは、\ruby{身}{み}をかがめると、\ruby{門}{もん}の\ruby{植}{う}え\ruby{込}{こ}みの、\ruby{隙間}{すきま}に\ruby{頭}{あたま}を\ruby{突}{つ}っ\ruby{込}{こ}んで、\ruby{門}{もん}の\ruby{中}{なか}をのぞいてみた。どうしよう、みえたんだけど!「ママ!あれ、\ruby{本当}{ほんとう}の\ruby{電車}{でんしゃ}?\ruby{校庭}{こうてい}に\ruby{並}{なら}んでるの」それは、\ruby{走}{はし}っていない、\ruby{本当}{ほんとう}の\ruby{電車}{でんしゃ}が六\ruby{台}{だい}、\ruby{教室}{きょうしつ}\ruby{用}{よう}に、\ruby{置}{お}かれてあるのだった。トットちゃんは、\ruby{夢}{ゆめ}のように\ruby{思}{おも}った。“\ruby{電車}{でんしゃ}の\ruby{教室}{きょうしつ}……”

\ruby{電車}{でんしゃ}で\ruby{窓}{まど}が、\ruby{朝}{あさ}の\ruby{光}{ひかり}を\ruby{受}{う}けて、キラキラと\ruby{光}{ひか}っていた。\ruby{目}{め}を\ruby{輝}{かがや}かして、のぞいているトットちゃんの、ホッペタも、\ruby{光}{ひか}っていた。   \ruby{気}{き}に\ruby{入}{い}ったわ\ruby{次}{つぎ}の\ruby{瞬間}{しゅんかん}、トットちゃんは、「わーい」と\ruby{歓声}{かんせい}を\ruby{上}{あ}げると、\ruby{電車}{でんしゃ}の\ruby{教室}{きょうしつ}のほうに\ruby{向}{む}かって\ruby{走}{はし}り\ruby{出}{だ}した。そして、\ruby{走}{はし}りながら、ママに\ruby{向}{む}かって\ruby{叫}{さけ}んだ。「ねえ、\ruby{早}{はや}く、\ruby{動}{うご}かない\ruby{電車}{でんしゃ}に\ruby{乗}{の}ってみよう!」ママは、\ruby{驚}{おどろ}いて\ruby{走}{はし}り\ruby{出}{だ}した。もとバスケットバールの\ruby{選手}{せんしゅ}だったままの\ruby{足}{あし}は、トットちゃんより\ruby{速}{はや}かったから、トットちゃんが、\ruby{後}{あと}、ちょっとでドア、というときに、スカートを\ruby{捕}{つか}まえられてしまった。ママは、スカートのはしを、ぎっちり\ruby{握}{にぎ}ったまま、トットちゃんにいった。「ダメよ。この\ruby{電車}{でんしゃ}は、この\ruby{学校}{がっこう}のお\ruby{教室}{きょうしつ}なんだし、あなたは、まだ、この\ruby{学校}{がっこう}に\ruby{入}{はい}れていただいてないんだから。もし、どうしても、この\ruby{電車}{でんしゃ}に\ruby{乗}{の}りたいんだったら、これからお\ruby{目}{め}にかかる\ruby{校長}{こうちょう}\ruby{先生}{せんせい}とちゃんと、お\ruby{話}{はな}してちょうだい。そして、うまくいったら、この\ruby{学校}{がっこう}に\ruby{通}{とお}えるんだから、\ruby{分}{わ}かった?」トットちゃんは、(\ruby{今}{いま}\ruby{乗}{の}れないのは、とても\ruby{残念}{ざんねん}なことだ)と\ruby{思}{おも}ったけど、ママのいう\ruby{通}{とお}りにしようときめたから、\ruby{大}{おお}きな\ruby{声}{こえ}で、「うん」といって、それから、いそいで、つけたした。「\ruby{私}{わたし}、この\ruby{学校}{がっこう}、とっても\ruby{気}{き}に\ruby{入}{い}ったわ」ママは、トットちゃんが\ruby{気}{き}に\ruby{入}{い}ったかどうかより、\ruby{校長}{こうちょう}\ruby{先生}{せんせい}が、トットちゃんを\ruby{気}{き}に\ruby{入}{い}ってくださるかどうか\ruby{問題}{もんだい}なのよ、といいたい\ruby{気}{き}がしたけど、とにかく、トットちゃんのスカートから\ruby{手}{て}を\ruby{離}{はな}し、\ruby{手}{て}をつないで\ruby{校長}{こうちょう}\ruby{室}{しつ}のほうに\ruby{歩}{ある}き\ruby{出}{だ}した。どの\ruby{電車}{でんしゃ}も\ruby{静}{しず}かで、ちょっと\ruby{前}{まえ}に、一\ruby{時間}{じかん}\ruby{目}{め}の\ruby{授業}{じゅぎょう}が\ruby{始}{はじ}まったようだった。あまり\ruby{広}{ひろ}くない\ruby{校庭}{こうてい}の\ruby{周}{まわ}りには、\ruby{塀}{へい}の\ruby{変}{か}わりに、いろんな\ruby{種類}{しゅるい}の\ruby{木}{き}が\ruby{植}{う}わっていて、\ruby{花壇}{かだん}には、\ruby{赤}{あか}や\ruby{黄色}{きいろ}の\ruby{花}{はな}がいっぱい\ruby{咲}{さ}いていた。\ruby{校長}{こうちょう}\ruby{室}{しつ}は、\ruby{電車}{でんしゃ}ではなく、ちょうど、\ruby{門}{もん}から\ruby{正面}{しょうめん}に\ruby{見}{み}える\ruby{扇形}{おうぎがた}に\ruby{広}{ひろ}がった七\ruby{段}{だん}くらいある\ruby{石}{いし}の\ruby{階段}{かいだん}を\ruby{上}{のぼ}った、その\ruby{右手}{みぎて}にあった。トットちゃんは、ママの\ruby{手}{て}を\ruby{振}{ふ}り\ruby{切}{き}ると、\ruby{階段}{かいだん}を\ruby{駆}{か}け\ruby{上}{あ}がって\ruby{行}{い}ったが、\ruby{急}{きゅう}に\ruby{止}{と}まって、\ruby{振}{ふ}り\ruby{向}{む}いた。だから、\ruby{後}{うし}ろから\ruby{行}{い}ったママは、もう\ruby{少}{すこ}しで、トットちゃんと\ruby{正面}{しょうめん}\ruby{衝突}{しょうとつ}するところだった。「どうしたの?」ママは、トットちゃんの\ruby{気}{き}が\ruby{変}{か}わったのかと\ruby{思}{おも}って、\ruby{急}{いそ}いで\ruby{聞}{き}いた。トットちゃんは、ちょうど\ruby{階段}{かいだん}の\ruby{一番}{いちばん}うえに\ruby{立}{た}った\ruby{形}{かたち}だったけど、まじめな\ruby{顔}{かお}をして、\ruby{小声}{こごえ}でママに\ruby{聞}{き}いた。ママは、かなり\ruby{辛抱}{しんぼう}づよい\ruby{人間}{にんげん}だったから……というか,\ruby{面白}{おもしろ}がりやだったから、やはり\ruby{小声}{こごえ}になって、トットちゃんに\ruby{顔}{かお}をつけて、\ruby{聞}{き}いた。「どうして?」トットちゃんは、ますます\ruby{声}{こえ}をひそめて\ruby{言}{い}った。「だってさ、\ruby{校長}{こうちょう}\ruby{先生}{せんせい}って、ママいったけど、こんなに\ruby{電車}{でんしゃ}、いっぱい\ruby{持}{も}ってるんだから、\ruby{本当}{ほんとう}は、\ruby{駅}{えき}の\ruby{人}{ひと}なんじゃないの?」\ruby{確}{たし}かに、\ruby{電車}{でんしゃ}の\ruby{払}{はら}い\ruby{下}{さ}げを\ruby{校舎}{こうしゃ}にしている\ruby{学校}{がっこう}なんてめずらしいから、トットちゃんの\ruby{疑問}{ぎもん}も、もっとものこと、とママも\ruby{思}{おも}ったけど、この\ruby{際}{さい}、\ruby{説明}{せつめい}してるヒマはないので、こういった。「じゃ、あなた、\ruby{校長}{こうちょう}\ruby{先生}{せんせい}に\ruby{伺}{うかが}って\ruby{御覧}{ごらん}なさい、\ruby{自分}{じぶん}で。それと、あなたのパパのことを\ruby{考}{かんが}えてみて?パパはヴァイオリンを\ruby{弾}{ひ}く\ruby{人}{ひと}で、いくつかヴァイオリンを\ruby{持}{も}ってるけど、ヴァイオリン\ruby{屋}{や}さんじゃないでしょう?そういう\ruby{人}{ひと}もいるのよ」トットちゃんは、「そうか」というと、ママと\ruby{手}{て}をつないだ。




\chapter{第四章}
こうして\ruby{僕}{ぼく}は\ruby{二}{ふた}つ\ruby{目}{め}のとても\ruby{大切}{たいせつ}な\ruby{事}{こと}を\ruby{知}{し}った。\ruby{王子}{おうじ}さまのいた\ruby{星}{ほし}は、\ruby{家}{いえ}\ruby{一軒}{いっけん}よりやや\ruby{大}{おお}きいくらいの\ruby{大}{おお}きさなのだ。それほど\ruby{驚}{おどろ}きはしなかった。\ruby{地球}{ちきゅう}や\ruby{木星}{もくせい}、\ruby{火星}{かせい}、\ruby{金星}{きんせい}の\ruby{様}{よう}に、\ruby{名前}{なまえ}のある\ruby{巨大}{きょだい}な\ruby{星}{ほし}\ruby{以外}{いがい}にも、\ruby{望遠鏡}{ぼうえんきょう}でも\ruby{見}{み}つからないほど\ruby{小}{ちい}さな\ruby{星}{ほし}が、\ruby{何}{なに}百とあることを\ruby{知}{し}っていたからだ。\ruby{天文}{てんもん}\ruby{学者}{がくしゃ}がそんな\ruby{星}{ほし}を\ruby{発見}{はっけん}すると、\ruby{名前}{なまえ}の\ruby{替}{か}わりに\ruby{番号}{ばんごう}をつける。

\ruby{例}{たと}えば、\ruby{小}{しょう}\ruby{惑星}{わくせい}32 5 といった\ruby{様}{よう}に。\ruby{王子}{おうじ}さまがやってきた\ruby{星}{ほし}は、\ruby{小}{しょう}\ruby{惑星}{わくせい}B612だと\ruby{思}{おも}う。1909\ruby{年}{ねん}に、トルコの\ruby{天文}{てんもん}\ruby{学者}{がくしゃ}が\ruby{一度}{いちど}だけ\ruby{望遠鏡}{ぼうえんきょう}で\ruby{観測}{かんそく}した\ruby{星}{ほし}だ。\ruby{天文}{てんもん}\ruby{学者}{がくしゃ}は\ruby{国際}{こくさい}\ruby{天文学}{てんもんがく}\ruby{会}{かい}で、\ruby{自分}{じぶん}の\ruby{発見}{はっけん}について\ruby{堂々}{どうどう}と\ruby{発表}{はっぴょう}した。しかしその\ruby{時}{とき}は\ruby{服装}{ふくそう}のせいで、\ruby{誰}{だれ}にも\ruby{信}{しん}じてもらえなかった。\ruby{大人}{おとな}なんてそんなもんだ。しかし、\ruby{小}{しょう}\ruby{惑星}{わくせい}B612に\ruby{名誉}{めいよ}\ruby{挽回}{ばんかい}の\ruby{幸運}{こううん}が\ruby{訪}{おとず}れた。トルコの\ruby{独裁者}{どくさいしゃ}が\ruby{国民}{こくみん}にヨーロッパ\ruby{風}{ふう}の\ruby{服}{ふく}を\ruby{着}{き}るように\ruby{命令}{めいれい}し、\ruby{従}{したが}わなければ\ruby{死刑}{しけい}という\ruby{事}{こと}になったのだ。そこで\ruby{天文}{てんもん}\ruby{学者}{がくしゃ}は、1920\ruby{年}{ねん}、\ruby{今度}{こんど}はもっと\ruby{専念}{せんねん}された\ruby{服装}{ふくそう}で\ruby{同}{おな}じ\ruby{発表}{はっぴょう}を\ruby{繰}{く}り\ruby{返}{かえ}した。この\ruby{時}{とき}は\ruby{皆}{みな}が\ruby{彼}{かれ}の\ruby{言}{い}う\ruby{事}{こと}を\ruby{信}{しん}じた。

この\ruby{星}{ほし}の\ruby{事}{こと}をこんなに\ruby{詳}{くわ}しく\ruby{話}{はな}して、\ruby{番}{ばん}\ruby{号}{ごう}まで\ruby{教}{おし}えるのは、\ruby{大人}{おとな}たちのせいだ。\ruby{大人}{おとな}は\ruby{数字}{すうじ}が\ruby{好}{す}きだ。\ruby{数字}{すうじ}\ruby{以外}{いがい}には\ruby{興味}{きょうみ}がない。\ruby{新}{あたら}しい\ruby{友達}{ともだち}の\ruby{事}{こと}を\ruby{話}{はな}しても、どんな\ruby{声}{こえ}か、どんな\ruby{遊}{あそ}びが\ruby{好}{す}きか、ちょうちょう\ruby{集}{あつ}めているか、といった\ruby{大切}{たいせつ}な\ruby{事}{こと}は\ruby{何}{なに}も\ruby{聞}{き}いてこない。\ruby{何歳}{なんさい}か、\ruby{何人}{なんにん}\ruby{兄弟}{きょうだい}か、お\ruby{父}{とう}さんの\ruby{年収}{ねんしゅう}はいくらか、といった\ruby{数字}{すうじ}のことばかり\ruby{聞}{き}いてきて、それですっかり\ruby{知}{し}ったつもりになる。

\ruby{王子}{おうじ}さまは\ruby{本当}{ほんとう}にいたよ。\ruby{可愛}{かわい}かったし、\ruby{笑}{わら}っていたし、\ruby{羊}{ひつじ}を\ruby{欲}{ほ}しがっていた。だって、\ruby{羊}{ひつじ}を\ruby{欲}{ほ}しがるって\ruby{事}{こと}は、\ruby{間違}{まちが}えなくその\ruby{人}{ひと}が\ruby{本当}{ほんとう}にいるって\ruby{事}{こと}の\ruby{証拠}{しょうこ}だからね。

こんなふうに\ruby{話}{はな}しても、\ruby{大人}{おとな}は\ruby{肩}{かた}を\ruby{竦}{すく}め、\ruby{子供扱}{こどもあつか}いするだけだ。しかし、\ruby{王子}{おうじ}さまが\ruby{来}{き}た\ruby{星}{ほし}は\ruby{小}{しょう}\ruby{惑星}{わくせい}B612だよ、と\ruby{言}{い}えば、\ruby{大人}{おとな}は\ruby{納得}{なっとく}して、それ\ruby{以上}{いじょう}\ruby{余計}{よけい}な\ruby{事}{こと}は\ruby{聞}{き}いてこない。

\ruby{大人}{おとな}なんてそんなもんだ。でも、\ruby{悪}{わる}く\ruby{思}{おも}ってはいけないよ。\ruby{子供}{こども}は\ruby{大人}{おとな}に\ruby{対}{たい}して、\ruby{広}{ひろ}い\ruby{心}{こころ}で\ruby{接}{せっ}してあげなきゃね。でも、\ruby{生}{い}きるという\ruby{事}{こと}がどういう\ruby{事}{こと}なのか、よくわかっている\ruby{僕}{ぼく}たちには、\ruby{数字}{すうじ}なんかどうでもいい。

\ruby{本当}{ほんとう}だったら\ruby{僕}{ぼく}は、この\ruby{物語}{ものがたり}をお\ruby{伽話}{とぎばなし}のように\ruby{始}{はじ}めたかった。\ruby{昔々}{むかしむかし}、\ruby{自分}{じぶん}よりほんの\ruby{少}{すこ}し\ruby{大}{おお}きいだけの\ruby{星}{ほし}に\ruby{暮}{く}らしている\ruby{小}{ちい}さな\ruby{王子}{おうじ}さまがいました。\ruby{王子}{おうじ}さまは\ruby{友達}{ともだち}をほしがっていました。\ruby{生}{い}きるという\ruby{事}{こと}がどういう\ruby{事}{こと}なのかわかっている\ruby{人}{ひと}には、こういう\ruby{言}{い}い\ruby{方}{かた}のほうがずっと\ruby{本当}{ほんとう}らしく\ruby{聞}{き}こえるだろう。\ruby{僕}{ぼく}はこの\ruby{本}{ほん}を\ruby{軽々}{かるがる}しく\ruby{読}{よ}まれたくない。こういった\ruby{思}{おも}い\ruby{出話}{でばなし}を\ruby{語}{かた}る\ruby{事}{こと}は、\ruby{僕}{ぼく}にとって\ruby{本当}{ほんとう}に\ruby{辛}{から}い。\ruby{僕}{ぼく}の\ruby{友達}{ともだち}が\ruby{羊}{ひつじ}を\ruby{連}{つ}れていってしまって、もう6\ruby{年}{ねん}になる。こうして\ruby{彼}{かれ}の\ruby{事}{こと}を\ruby{書}{か}くのは、\ruby{彼}{かれ}を\ruby{忘}{わす}れないためだ。\ruby{友達}{ともだち}を\ruby{忘}{わす}れてしまうのは\ruby{悲}{かな}しい、\ruby{誰}{だれ}にでも\ruby{友達}{ともだち}がいるわけではない。それに、\ruby{僕}{ぼく}も\ruby{数字}{すうじ}にしか\ruby{興味}{きょうみ}のない\ruby{大人}{おとな}になってしまうかもしれない。そうならないために\ruby{僕}{ぼく}は、\ruby{絵}{え}の\ruby{具}{ぐ}\ruby{箱}{はこ}と\ruby{鉛筆}{えんぴつ}を\ruby{買}{か}った。6\ruby{歳}{さい}でボアの\ruby{外側}{そとがわ}と\ruby{内側}{うちがわ}を\ruby{描}{えが}いて\ruby{以来}{いらい}、\ruby{何}{なに}も\ruby{描}{えが}いていなかった\ruby{僕}{ぼく}にとって、この\ruby{年}{とし}でもう\ruby{一度}{いちど}\ruby{絵}{え}を\ruby{描}{か}くのは\ruby{大変}{たいへん}な\ruby{事}{こと}だった。できるだけ、\ruby{本物}{ほんもの}そっくりな\ruby{肖像画}{しょうぞうが}を\ruby{描}{えが}いてみるつもりだ。

でも、ちゃんと\ruby{描}{えが}けるかどうかは、\ruby{自信}{じしん}がない。\ruby{一枚}{いちまい}いいものが\ruby{描}{えが}けても、その\ruby{次}{つぎ}はまるで\ruby{似}{に}ていないかもしれない。\ruby{背丈}{せたけ}が\ruby{難}{むずか}しいし、\ruby{服}{ふく}の\ruby{色}{いろ}も\ruby{迷}{まよ}ってしまう。\ruby{手探}{てさぐ}りでやってみるが、もっと\ruby{大事}{だいじ}な\ruby{細}{こま}かい\ruby{部分}{ぶぶん}を\ruby{間違}{まちが}えてしまうかもしれない。でも、そこは\ruby{大目}{おおめ}に\ruby{見}{み}てほしい。\ruby{王子}{おうじ}さまは\ruby{詳}{くわ}しい\ruby{事}{こと}は\ruby{何}{なに}も\ruby{説明}{せつめい}してくれなかったのだ。おそらく\ruby{彼}{かれ}は\ruby{僕}{ぼく}の\ruby{事}{こと}を\ruby{自分}{じぶん}と\ruby{同}{おな}じ\ruby{仲間}{なかま}だと\ruby{思}{おも}ったのだろう。しかし\ruby{残念}{ざんねん}ながら\ruby{僕}{ぼく}は、\ruby{箱}{はこ}の\ruby{中}{なか}の\ruby{羊}{ひつじ}を\ruby{見}{み}る\ruby{事}{こと}ができない。\ruby{少}{すこ}しばかり\ruby{大人}{おとな}になってしまったのかもしれない。\ruby{年}{とし}を\ruby{取}{と}ったのだ。




\chapter{第五章}
\ruby{日}{ひ}を\ruby{追}{お}うごとに\ruby{僕}{ぼく}は\ruby{王子}{おうじ}さまの\ruby{星}{ほし}の\ruby{事}{こと}や、そこからの\ruby{旅立}{たびだ}ち、これまでの\ruby{旅}{たび}について\ruby{知}{し}るようになっていった。\ruby{王子}{おうじ}さまが\ruby{偶々}{たまたま}\ruby{口}{くち}にした\ruby{言葉}{ことば}で、\ruby{少}{すこ}しずつ\ruby{様子}{ようす}がわかってきた。こうして\ruby{三日目}{みっかめ}に、バオバブをめぐる\ruby{大}{だい}\ruby{騒動}{そうどう}を\ruby{知}{し}った。これも\ruby{羊}{ひつじ}のおかげだった。\ruby{王子}{おうじ}さまが\ruby{急}{きゅう}に\ruby{心配}{しんぱい}になったらしくて、こう\ruby{聞}{き}いてきたのだ。

\ruby{羊}{ひつじ}が\ruby{小}{ちい}さな\ruby{木}{き}も\ruby{食}{た}べるって、\ruby{本当}{ほんとう}なんでしょう?

うん、\ruby{本当}{ほんとう}だよ。

ああ、よかった。

\ruby{羊}{ひつじ}が\ruby{小}{ちい}さな\ruby{木}{き}を\ruby{食}{た}べる\ruby{事}{こと}がなぜそんなに\ruby{大事}{だいじ}な\ruby{事}{こと}なのか、\ruby{僕}{ぼく}にはわからなかった。しかし、\ruby{王子}{おうじ}さまは\ruby{更}{さら}にこう\ruby{聞}{き}いてきた。

だったら、バオバブも\ruby{食}{た}べるよね。

\ruby{僕}{ぼく}は\ruby{王子}{おうじ}さまにバオバブは\ruby{小}{ちい}さな\ruby{木}{き}じゃなくて、\ruby{教会}{きょうかい}の\ruby{建物}{たてもの}と\ruby{同}{おな}じくらい\ruby{大}{おお}きな\ruby{木}{き}だから、ゾウの\ruby{群}{む}れを\ruby{丸}{まる}ごと\ruby{連}{つ}れてきても、たった一\ruby{本}{ほん}のバオバブも\ruby{食}{た}べきれないだろうと\ruby{教}{おし}えてあげた。ゾウの\ruby{群}{む}れを\ruby{思}{おも}い\ruby{描}{えが}いて、\ruby{王子}{おうじ}さまは\ruby{笑}{わら}った。

\ruby{上}{うえ}に\ruby{上}{うえ}に\ruby{積}{つ}み\ruby{重}{かさ}ねなきゃいけないね。

しかし、\ruby{続}{つづ}けてなかなか\ruby{鋭}{するど}い\ruby{指摘}{してき}をした。

バオバブだって、\ruby{大}{おお}きくなる\ruby{前}{まえ}は、\ruby{小}{ちい}さいんだよね。

そりゃそうだよ。それにしても、どうして\ruby{羊}{ひつじ}に\ruby{小}{ちい}さなバオバブを\ruby{食}{た}べてもらいたいんだい?

\ruby{何}{なに}を\ruby{言}{い}ってるの?そんなの\ruby{当}{あ}たり\ruby{前}{まえ}でしょう。

\ruby{僕}{ぼく}は\ruby{一人}{ひとり}でこの\ruby{難問}{なんもん}を\ruby{解}{と}き\ruby{明}{あ}かす\ruby{事}{こと}になり、\ruby{散々}{さんざん}\ruby{頭}{あたま}を\ruby{捻}{ひね}った。つまり、こういう\ruby{事}{こと}だ。\ruby{王子}{おうじ}さまの\ruby{星}{ほし}には、\ruby{他}{ほか}の\ruby{星}{ほし}と\ruby{同}{おな}じように、よい\ruby{草}{くさ}と\ruby{悪}{わる}い\ruby{草}{くさ}があった。よい\ruby{草}{くさ}はよい\ruby{種}{たね}から\ruby{育}{そだ}ち、\ruby{悪}{わる}い\ruby{草}{くさ}は\ruby{悪}{わる}い\ruby{種}{たね}から\ruby{育}{そだ}つ。しかし、\ruby{種}{たね}は\ruby{目}{め}に\ruby{見}{み}えない。\ruby{土}{つち}の\ruby{中}{なか}でひっそりと\ruby{眠}{ねむ}っている。その\ruby{一}{ひと}つが\ruby{気}{き}まぐれに\ruby{目}{め}を\ruby{覚}{さ}ますと、\ruby{伸}{の}びをして、おずおずとあどけない\ruby{小}{ちい}さな\ruby{茎}{くき}を\ruby{太陽}{たいよう}に\ruby{向}{む}かって\ruby{伸}{の}ばし\ruby{始}{はじ}める。それが \ruby{赤蕪}{あかかぶ} やバラだったら、そのままにしておいて\ruby{構}{かま}わない。でも、\ruby{悪}{わる}い\ruby{草}{くさ}だと\ruby{分}{わ}かったら、すぐに\ruby{抜}{ぬ}き\ruby{取}{と}らなくてはいけない。\ruby{王子}{おうじ}さまの\ruby{星}{ほし}には、そんな\ruby{恐}{おそ}ろしい\ruby{種}{たね}があった。バオバブの\ruby{種}{たね}だ。\ruby{星}{ほし}の\ruby{土}{つち}はどこもかしこもバオバブの\ruby{種}{たね}だらけだった。\ruby{少}{すこ}しでも\ruby{抜}{ぬ}くのが\ruby{遅}{おく}れると、バオバブはもう\ruby{手}{て}がつけられなくなる。\ruby{星}{ほし}\ruby{全体}{ぜんたい}を\ruby{覆}{おお}いつくし、\ruby{根}{ね}っ\ruby{子}{こ}がつき\ruby{抜}{ぬ}け、\ruby{穴}{あな}を\ruby{開}{あ}けてしまう。\ruby{小}{ちい}さな\ruby{星}{ほし}だと\ruby{殖}{ふえ}\ruby{過}{す}ぎたバオバブで\ruby{破裂}{はれつ}してしまう。

\ruby{決}{き}まりにできるかどうかだね。\ruby{毎朝}{まいあさ}、\ruby{自分}{じぶん}の\ruby{身支度}{みじたく}が\ruby{済}{す}んだら、\ruby{星}{ほし}の\ruby{手入}{てい}れに\ruby{取}{と}り\ruby{掛}{か}かる。

\ruby{芽}{め}を\ruby{出}{だ}したばかりのバラとバオバブはよく\ruby{似}{に}ているんだけど、それを\ruby{見分}{みわ}けて、バオバブだと\ruby{分}{わ}かったら、すぐに\ruby{抜}{ぬ}いてしまう。\ruby{手間}{てま}はかかるけど、とっても\ruby{簡単}{かんたん}な\ruby{事}{こと}だよ。\ruby{偶}{ぐう}には\ruby{仕事}{しごと}を\ruby{後回}{あとまわ}しにしても\ruby{大丈夫}{だいじょうぶ}な\ruby{時}{とき}ってあるけど、バオバブでそんな\ruby{事}{こと}をしたら、\ruby{取}{と}り\ruby{返}{かえ}しがつかなくなるんだ。\ruby{例}{たと}えばね、ある\ruby{星}{ほし}に\ruby{怠}{なま}け\ruby{者}{もの}が\ruby{住}{す}んでいたんだけど、その\ruby{人}{ひと}は三本\ruby{さんぼん}のバオバブをほったらかしにしていたばかりに……\ruby{僕}{ぼく}は\ruby{王子}{おうじ}さまの\ruby{話}{はな}す\ruby{通}{とお}りにその\ruby{星}{ほし}の\ruby{絵}{え}を\ruby{描}{か}いた。\ruby{星}{ほし}より\ruby{巨大}{きょだい}な三\ruby{本}{ほん}のバオバブと\ruby{途方}{とほう}に\ruby{暮}{く}れる\ruby{怠}{なま}け\ruby{者}{もの}、お\ruby{説教}{せっきょう}\ruby{臭}{くさ}い\ruby{事}{こと}を\ruby{言}{い}うのはあまり\ruby{好}{す}きじゃないけれど、バオバブの\ruby{脅威}{きょうい}は\ruby{地球}{ちきゅう}ではほとんど\ruby{知}{し}られていないし、\ruby{小}{しょう}\ruby{惑星}{わくせい}で\ruby{道}{みち}に\ruby{迷}{まよ}った\ruby{人}{ひと}が\ruby{危険}{きけん}な\ruby{目}{め}に\ruby{遭}{あ}う\ruby{可能性}{かのうせい}は、あまりにも\ruby{大}{おお}きい。だから\ruby{僕}{ぼく}は\ruby{一度}{いちど}だけ\ruby{普段}{ふだん}の\ruby{慎}{つつし}みを\ruby{忘}{わす}れて、こう\ruby{言}{い}っておこう。

おーい、\ruby{子供}{こども}たち、バオバブに\ruby{気}{き}をつけろ。

\ruby{僕}{ぼく}は\ruby{友人}{ゆうじん}たちに\ruby{警告}{けいこく}を\ruby{与}{あた}えるために、\ruby{一生懸命}{いっしょうけんめい}この\ruby{絵}{え}を\ruby{仕上}{しあ}げた。\ruby{苦労}{くろう}して\ruby{描}{えが}いた\ruby{価値}{かち}はあった。\ruby{他}{た}はこれほどうまくいかなかった。バオバブを\ruby{描}{えが}いた\ruby{時}{とき}は、\ruby{切羽詰}{せっぱつま}って\ruby{気持}{きも}ちが\ruby{高}{たか}ぶっていたのだ。




\chapter{第六章}
トットちゃんは、\ruby{校長}{こうちょう}先生に\ruby{連}{つ}れられて、みんなが、お\ruby{弁当}{べんとう}を\ruby{食}{た}べるところを、見に\ruby{行}{い}くことになった。お\ruby{昼}{ひる}だけは、\ruby{電車}{でんしゃ}でなく、「みんな、\ruby{講堂}{こうどう}に\ruby{集}{あつ}まることになっている」と\ruby{校長}{こうちょう}先生が\ruby{教}{おし}えてくれた。\ruby{講堂}{こうどう}はさっきトットちゃんが上がってきた石の\ruby{階段}{かいだん}の、\ruby{突}{つ}き\ruby{当}{あ}たりにあった。いってみると、\ruby{生徒}{せいと}たちが、\ruby{大騒}{おおさわ}ぎをしながら、\ruby{机}{つくえ}と\ruby{椅子}{いす}を、\ruby{講堂}{こうどう}に、まーるく\ruby{輪}{わ}になるように、\ruby{並}{なら}べているところだった。\ruby{隅}{すみ}っこで、それを見ていたトットちゃんは、\ruby{校長}{こうちょう}先生の\ruby{上着}{うわぎ}を\ruby{引}{ひ}っ\ruby{張}{ぱ}って\ruby{聞}{き}いた。

「\ruby{他}{ほか}の\ruby{生徒}{せいと}は、どこにいるの?」

\ruby{校長}{こうちょう}先生は\ruby{答}{こた}えた。

「これで\ruby{全部}{ぜんぶ}なんだよ」

「\ruby{全部}{ぜんぶ}!?」

トットちゃんは、\ruby{信}{しん}じられない気がした。だって、\ruby{前}{まえ}の学校の一クラスと\ruby{同}{おな}じくらいしか、いないんだもの。そうすると、

「学校中で、五十人くらいなの?」

\ruby{校長}{こうちょう}先生は、「そうだ」といった。トットちゃんは、なにもかも、\ruby{前}{まえ}の学校と\ruby{違}{ちが}ってると\ruby{思}{おも}った。

みんなが\ruby{着席}{ちゃくせき}すると、\ruby{校長}{こうちょう}先生は、

「みんな、\ruby{海}{うみ}のものと、山のもの、もって\ruby{来}{き}たかい?」

と\ruby{聞}{き}いた。

「はーい」

みんな、それぞれの、お\ruby{弁当}{べんとう}の、ふたを\ruby{取}{と}った。

「どれどれ」

\ruby{校長}{こうちょう}先生は、\ruby{机}{つくえ}で\ruby{出来}{でき}た円の中に入ると、ひとりず、お\ruby{弁当}{べんとう}をのぞきながら、\ruby{歩}{ある}いている。

\ruby{生徒}{せいと}たちは、\ruby{笑}{わら}ったり、キイキイいったり、にぎやかだった。

「\ruby{海}{うみ}のものと、山のもの、って、なんだろう」

トットちゃんは、おかしくなった。でも、とっても、とっても、この学校は\ruby{変}{か}わっていて、\ruby{面白}{おもしろ}そう。お\ruby{弁当}{べんとう}の\ruby{時間}{じかん}が、こんなに、\ruby{愉快}{ゆかい}で、\ruby{楽}{たの}しいなんて、\ruby{知}{し}らなかった。トットちゃんは、\ruby{明日}{あした}からは、\ruby{自分}{じぶん}も、あの\ruby{机}{つくえ}に\ruby{座}{すわ}って、『\ruby{海}{うみ}のものと、山のもの』の\ruby{弁当}{べんとう}を、\ruby{校長}{こうちょう}先生に見てもらうんだ、と\ruby{思}{おも}うと、もう、\ruby{嬉}{うれ}しさと、\ruby{楽}{たの}しさで、\ruby{胸}{むね}がいっぱいになり、\ruby{叫}{さけ}びそうになった。 お\ruby{弁当}{べんとう}を、のぞきこんでる\ruby{校長}{こうちょう}先生の\ruby{肩}{かた}に、お\ruby{昼}{ひる}の\ruby{光}{ひかり}が、やわらかく\ruby{止}{と}まっていた。




\chapter{第七章}
きのう、「\ruby{今日}{きょう}から、\ruby{君}{きみ}は、もう、この学校の\ruby{生徒}{せいと}だよ」、そう\ruby{校長}{こうちょう}先生に\ruby{言}{い}われたトットちゃんにとって、こんなに\ruby{次}{つぎ}の日が\ruby{待}{ま}ち\ruby{遠}{どお}しい、ってことは、\ruby{今}{いま}までになかった。だから、いつもなら\ruby{朝}{あさ}、ママが\ruby{叩}{たた}き\ruby{起}{お}こしても、まだベッドの上でぼんやりしてることの\ruby{多}{おお}いトットちゃんが、この日ばかりは、\ruby{誰}{だれ}からも\ruby{起}{お}こされない\ruby{前}{まえ}に、もうソックスまではいて、ランドセルを\ruby{背負}{しょ}って、みんなの\ruby{起}{お}きるのを\ruby{待}{ま}っていた。

ロッキーは、\ruby{途中}{とちゅう}までは、耳をピンと立てて\ruby{神妙}{しんみょう}に\ruby{聞}{き}いていたけど、\ruby{説明}{せつめい}の\ruby{終}{お}わりのところで、\ruby{定期}{ていき}を、ちょっと、なめてみて、それから、あくびをした。それでも、トットちゃんは、\ruby{一生懸命}{いっしょうけんめい}に\ruby{話}{はな}し\ruby{続}{つづ}けた。

「\ruby{電車}{でんしゃ}の\ruby{教室}{きょうしつ}は、\ruby{動}{うご}かないから、お\ruby{教室}{きょうしつ}では、\ruby{定期}{ていき}はいらないと\ruby{思}{おも}うんだ。とにかく、\ruby{今日}{きょう}は\ruby{持}{も}ってるのよ」

たしかにロッキーは、\ruby{今}{いま}まで、\ruby{歩}{ある}いて\ruby{通}{かよ}う学校の\ruby{門}{もん}まで、\ruby{毎日}{まいにち}、トットちゃんと\ruby{一緒}{いっしょ}に\ruby{行}{い}って、\ruby{後}{あと}は、\ruby{一人}{ひとり}で\ruby{家}{いえ}に\ruby{帰}{かえ}ってきていたから、\ruby{今日}{きょう}も、そのつもりでいた。

トットちゃんは、\ruby{定期}{ていき}をロッキーの\ruby{首}{くび}からはずすと、\ruby{大切}{たいせつ}そうに\ruby{自分}{じぶん}の\ruby{首}{くび}にかけると、パパとママに、もう\ruby{一度}{いちど}、 『\ruby{行}{い}ってまいりまーす』というと、\ruby{今度}{こんど}は\ruby{振}{ふ}り\ruby{返}{かえ}らずに、ランドセルをカタカタいわせて\ruby{走}{はし}り\ruby{出}{だ}した。ロッキーも、からだをのびのびさせながら、\ruby{並}{なら}んで\ruby{走}{はし}り\ruby{出}{だ}した。

\ruby{駅}{えき}までの\ruby{道}{みち}は、\ruby{前}{まえ}の学校に\ruby{行}{い}く\ruby{道}{みち}と、ほとんど\ruby{変}{か}わらなかった。だから、\ruby{途中}{とちゅう}でトットちゃんは、\ruby{顔見知}{かおみし}りの犬や\ruby{猫}{ねこ}や、\ruby{前}{まえ}の\ruby{同級}{どうきゅう}生と、すれ\ruby{違}{ちが}った。トットちゃんは、その\ruby{度}{たび}に、「\ruby{定期}{ていき}を見せて、\ruby{驚}{おどろ}かせてやろうかな?」と\ruby{思}{おも}ったけど、(もし\ruby{遅}{おそ}くなったら\ruby{大変}{たいへん}だから、\ruby{今日}{きょう}は、よそう……)と\ruby{決}{き}めて、どんどん\ruby{歩}{ある}いた。

\ruby{駅}{えき}のところに\ruby{来}{き}て、いつもなら左に\ruby{行}{い}くトットちゃんが、右に\ruby{曲}{ま}がったので、\ruby{可哀}{かわい}そうにロッキーは、とても\ruby{心配}{しんぱい}そうに\ruby{立}{た}ち\ruby{止}{どま}って、キョロキョロした。トットちゃんは、\ruby{改札口}{かいさつぐち}のところまで\ruby{行}{い}ったんだけど、\ruby{戻}{もど}ってきて、まだ\ruby{不思議}{ふしぎ}そうな\ruby{顔}{かお}をしてるロッキーにいった。

「もう、\ruby{前}{まえ}の学校には\ruby{行}{い}かないのよ。\ruby{新}{あたら}しい学校に\ruby{行}{い}くんだから」

それからトットちゃんは、ロッキーの\ruby{顔}{かお}に、\ruby{自分}{じぶん}の\ruby{顔}{かお}をくっつけ、ついでにロッキーの耳の中の、においをかいだ。(いつもと\ruby{同}{おな}じくらい、くさいけれど、\ruby{私}{わたし}には、いい、におい!)そう\ruby{思}{おも}うと\ruby{顔}{かお}を\ruby{離}{はな}して、「バイバイ」というと、\ruby{定期}{ていき}を\ruby{駅}{えき}の人に見せて、ちょっと\ruby{高}{たか}い\ruby{駅}{えき}の\ruby{階段}{かいだん}を、\ruby{登}{のぼ}り\ruby{始}{はじ}めた。ロッキーは、小さい\ruby{声}{こえ}で\ruby{鳴}{な}いて、トットちゃんが\ruby{階段}{かいだん}を上がっていくのを、いつまでも\ruby{見送}{みおく}っていた。

この\ruby{家}{いえ}の中で、いちばん、きちんと\ruby{時間}{じかん}を\ruby{守}{まも}るシェパードのロッキーは、トットちゃんの、いつもと\ruby{違}{ちが}う\ruby{行動}{こうどう}に、\ruby{怪訝}{けげん}そうな目を\ruby{向}{む}けながら、それでも、大きく\ruby{伸}{の}びをすると、トットちゃんにぴったりとくっついて、(\ruby{何}{なに}か\ruby{始}{はじ}まるらしい)ことを\ruby{期待}{きたい}した。

ママ\ruby{大変}{たいへん}だった。\ruby{大忙}{おおいそが}しで、『\ruby{海}{うみ}のものと山のもの』のお\ruby{弁当}{べんとう}を\ruby{作}{つく}り、トットちゃんに\ruby{朝}{あさ}ごはんを\ruby{食}{た}べさせ、\ruby{毛糸}{けいと}で\ruby{編}{あ}んだヒモを\ruby{通}{とお}した、セルロイドの\ruby{定期入}{ていきい}れを、トットちゃんの\ruby{首}{くび}にかけた。これは\ruby{定期}{ていき}を、なくさないためだった、パパは「いい子でね」と\ruby{頭}{あたま}をヒシャヒシャにしたまま\ruby{言}{い}った。「もちろん!」と、トットちゃんは\ruby{言}{い}うと、\ruby{玄関}{げんかん}で\ruby{靴}{くつ}を\ruby{履}{は}き、\ruby{戸}{と}を\ruby{開}{あ}けると、クルリと\ruby{家}{いえ}の中を\ruby{向}{む}き、\ruby{丁寧}{ていねい}にお\ruby{辞儀}{じぎ}をして、こういった。

「みなさま、\ruby{行}{い}ってまいります」

\ruby{見送}{みおく}りに立っていたママは、ちょっと\ruby{涙}{なみだ}でそうになった。それは、こんなに生き生きとしてお\ruby{行儀}{ぎょうぎ}よく、\ruby{素直}{すなお}で、\ruby{楽}{たの}しそうにしてるトットちゃんが、つい、このあいだ、「\ruby{退学}{たいがく}になった」、ということを\ruby{思}{おも}い\ruby{出}{だ}したからだった。(\ruby{新}{あたら}しい学校で、うまくいくといい……)ママは\ruby{心}{こころ}からそう\ruby{祈}{いの}った。

ところが、\ruby{次}{つぎ}の\ruby{瞬間}{しゅんかん}、ママは、\ruby{飛}{と}び\ruby{上}{あ}がるほど\ruby{驚}{おどろ}いた。というのは、トットちゃんが、せっかくママが\ruby{首}{くび}からかけた\ruby{定期}{ていき}を、ロッキーの\ruby{首}{くび}にかけているのを見たからだった。ママは、(\ruby{一体}{いったい}どうなるのだろう?)と\ruby{思}{おも}ったけど、だまって、\ruby{成}{な}り\ruby{行}{ゆ}きを見ることにした。トットちゃんは、\ruby{定期}{ていき}をロッキーの\ruby{首}{くび}にかけると、しゃがんで、ロッキーに、こういった。

「いい?この\ruby{定期}{ていき}のヒモは、あんたに、\ruby{合}{あ}わないのよ」

\ruby{確}{たし}かに、ロッキーにはヒモが\ruby{長}{なが}く、\ruby{定期}{ていき}は\ruby{地面}{じめん}を\ruby{引}{ひ}きずっていた。

「わかった?これは\ruby{私}{わたし}の\ruby{定期}{ていき}で、あんたのじゃないから、あんたは\ruby{電車}{でんしゃ}に\ruby{乗}{の}れないの。\ruby{校長}{こうちょう}先生に\ruby{聞}{き}いてみるけど、\ruby{駅}{えき}の人にも。で『いい』っていったら、あんたも学校に\ruby{来}{こ}られるんだけど、どうかなあ」




\chapter{第八章}
トットちゃんが、きのう、\ruby{校長}{こうちょう}先生から\ruby{教}{おし}えていただいた、\ruby{自分}{じぶん}の\ruby{教室}{きょうしつ}である、\ruby{電車}{でんしゃ}のドアに手をかけたとき、まだ\ruby{校庭}{こうてい}には、\ruby{誰}{だれ}の\ruby{姿}{すがた}も見えなかった。\ruby{今}{いま}と\ruby{違}{ちが}って、\ruby{昔}{むかし}の\ruby{電車}{でんしゃ}は、\ruby{外}{そと}から\ruby{開}{あ}くように、ドアに\ruby{取手}{とって}がついていた。\ruby{両手}{りょうて}で、その\ruby{取手}{とって}を\ruby{持}{も}って、右に\ruby{引}{ひ}くと、ドアは、すぐ\ruby{開}{あ}いた。トットちゃんは、ドキドキしながら、そーっと、\ruby{首}{くび}を\ruby{突}{つ}っ\ruby{込}{こ}んで、中を見てみた。

「わあーい」

これなら、\ruby{勉強}{べんきょう}しながら、いつも\ruby{旅行}{りょこう}をしてるみたいじゃない。\ruby{網棚}{あみだな}もあるし、\ruby{窓}{まど}も\ruby{全部}{ぜんぶ}、そのままだし。\ruby{違}{ちが}うところは、\ruby{運転手}{うんてんしゅ}さんの\ruby{席}{せき}のところに\ruby{黒板}{こくばん}があるのと、\ruby{電車}{でんしゃ}の\ruby{長}{なが}い\ruby{腰掛}{こしかけ}を、はずして、\ruby{生徒}{せいと}\ruby{用}{よう}の\ruby{机}{つくえ}と\ruby{腰掛}{こしかけ}が\ruby{進行}{しんこう}\ruby{方向}{ほうこう}に\ruby{向}{む}いて\ruby{並}{なら}んでいるのと、つり\ruby{革}{かわ}が\ruby{無}{な}いところだけ。\ruby{後}{あと}は、\ruby{天井}{てんじょう}も\ruby{床}{ゆか}も、\ruby{全部}{ぜんぶ}、\ruby{電車}{でんしゃ}のままになっていた。トットちゃんは\ruby{靴}{くつ}を\ruby{脱}{ぬ}いで中に入り、\ruby{誰}{だれ}でも\ruby{腰掛}{こしか}けていたいくらい、\ruby{気持}{きも}ちのいい\ruby{椅子}{いす}だった。トットちゃんは、うれしくて、(こんな気に入った学校は、\ruby{絶対}{ぜったい}に、お休みなんかしないで、ずーっとくる)と,\ruby{強}{つよ}く\ruby{心}{こころ}に\ruby{思}{おも}った。

それからトットちゃんは、\ruby{窓}{まど}から\ruby{外}{そと}を見ていた。すると、\ruby{動}{うご}いていないはずの\ruby{電車}{でんしゃ}なのに、\ruby{校庭}{こうてい}の花や木が、\ruby{少}{すこ}し\ruby{風}{かぜ}に\ruby{揺}{ゆ}れているせいか、\ruby{電車}{でんしゃ}が\ruby{走}{はし}っているような\ruby{気持}{きも}ちになった。

「ああ、\ruby{嬉}{うれ}しいなあー」

トットちゃんは、とうとう\ruby{声}{こえ}に出して、そういった。それから、\ruby{顔}{かお}をぺったりガラス\ruby{窓}{まど}にくっつけると、いつも、\ruby{嬉}{うれ}しいとき、そうするように、デタラメ\ruby{歌}{うた}を、うたいはじめた。

 とても うれし

 うれし とても

 どうしてかっていえば……

 そこまで\ruby{歌}{うた}ったとき、\ruby{誰}{だれ}かが\ruby{乗}{の}り\ruby{込}{こ}んできた。女の子だった。その子は、ノートと\ruby{筆箱}{ふでばこ}をランドセルから出して\ruby{机}{つくえ}の上に\ruby{置}{お}くと、\ruby{背伸}{せの}びをして、\ruby{網棚}{あみだな}にランドセルをのせた。それから\ruby{草履}{ぞうり}\ruby{袋}{ぶくろ}も、のせた。トットちゃんは\ruby{歌}{うた}をやめて、\ruby{急}{いそ}いで、まねをした。\ruby{次}{つぎ}に、男の子が\ruby{乗}{の}ってきた。その子は、ドアのところから、バスケットボールのように、ランドセルを、\ruby{網棚}{あみだな}に\ruby{投}{な}げ\ruby{込}{こ}んだ。\ruby{網棚}{あみだな}の、\ruby{網}{あみ}は、大きく\ruby{波}{なみ}うつと、ランドセルを、\ruby{投}{な}げ\ruby{出}{だ}した。ランドセルは、\ruby{床}{ゆか}に\ruby{落}{お}ちた。その男の子は、「\ruby{失敗}{しっぱい}!」というと、またもや、\ruby{同}{おな}じところから、\ruby{網棚}{あみだな}めがけて、\ruby{投}{な}げ\ruby{込}{こ}んだ。\ruby{今度}{こんど}は、うまく、おさまった。『\ruby{成功}{せいこう}!』と、その子は\ruby{叫}{さけ}ぶと、すぐ、「\ruby{失敗}{しっぱい}!」といって、\ruby{机}{つくえ}によじ\ruby{登}{のぼ}ると、\ruby{網棚}{あみだな}のランドセルを\ruby{開}{あ}けて、\ruby{筆箱}{ふでばこ}やノートを出した。そういうのを出すのを\ruby{忘}{わす}れたから、\ruby{失敗}{しっぱい}だったに\ruby{違}{ちが}いなかった。

こうして、九人の\ruby{生徒}{せいと}が、トットちゃんの\ruby{電車}{でんしゃ}に\ruby{乗}{の}り\ruby{込}{こ}んできて、それが、トモエ\ruby{学園}{がくえん}の、一年生の\ruby{全員}{ぜんいん}だった。

そしてそれは、\ruby{同}{おな}じ\ruby{電車}{でんしゃ}で\ruby{旅}{たび}をする、\ruby{仲間}{なかま}だった。




\chapter{第九章}
お\ruby{教室}{きょうしつ}が\ruby{本当}{ほんとう}の\ruby{電車}{でんしゃ}で、“かわってる”と\ruby{思}{おも}ったトットちゃんが、\ruby{次}{つぎ}に“かわってる”と\ruby{思}{おも}ったのは、\ruby{教室}{きょうしつ}で\ruby{座}{すわ}る\ruby{場所}{ばしょ}だった。\ruby{前}{まえ}の学校は、\ruby{誰}{だれ}かさんは、どの\ruby{机}{つくえ}、\ruby{隣}{となり}は\ruby{誰}{だれ}、\ruby{前}{まえ}は\ruby{誰}{だれ}、と\ruby{決}{き}まっていた。ところが、この学校は、どこでも、\ruby{次}{つぎ}の日の\ruby{気分}{きぶん}や\ruby{都合}{つごう}で、\ruby{毎日}{まいにち}、\ruby{好}{す}きなところに\ruby{座}{すわ}っていいのだった。

そこでトットちゃんは、さんざん\ruby{考}{かんが}え、そして\ruby{見回}{みまわ}したあげく、\ruby{朝}{あさ}、トットちゃんの\ruby{次}{つぎ}に\ruby{教室}{きょうしつ}に入ってきた女の子の\ruby{隣}{となり}に\ruby{座}{すわ}ることに\ruby{決}{き}めた。なぜなら、この子が、\ruby{長}{なが}い耳をした\ruby{兎}{うさぎ}の\ruby{絵}{え}のついた、ジャンパースカートをはいていたからだった。

でも、なによりも“かわっていた”のは、この学校の、\ruby{授業}{じゅぎょう}のやりかただった。

\ruby{普通}{ふつう}の学校は、一\ruby{時間}{じかん}目が\ruby{国語}{こくご}なら、\ruby{国語}{こくご}をやって、二\ruby{時間}{じかん}目が\ruby{算数}{さんすう}なら、\ruby{算数}{さんすう}、という\ruby{風}{かぜ}に、\ruby{時間}{じかん}\ruby{割}{わり}の\ruby{通}{とお}りの\ruby{順番}{じゅんばん}なのだけど、この学校は、まるっきり\ruby{違}{ちが}っていた。

\ruby{何}{なに}しろ、一\ruby{時間}{じかん}目が\ruby{始}{はじ}まるときに、その日、一日やる\ruby{時間}{じかん}\ruby{割}{わり}の、\ruby{全部}{ぜんぶ}の\ruby{科目}{かもく}の\ruby{問題}{もんだい}を、女の先生が、\ruby{黒板}{こくばん}にいっぱいに\ruby{書}{か}いちゃって、

「さあ、どれでも\ruby{好}{す}きなのから、\ruby{始}{はじ}めてください」

といったんだ。だから\ruby{生徒}{せいと}は、\ruby{国語}{こくご}であろうと、\ruby{算数}{さんすう}であろうと、\ruby{自分}{じぶん}の\ruby{好}{す}きなのから\ruby{始}{はじ}めていっこうに、かまわないのだった。だから、\ruby{作文}{さくぶん}の\ruby{好}{す}きな子が、\ruby{作文}{さくぶん}を\ruby{書}{か}いていると、\ruby{後}{うし}ろでは、\ruby{物理}{ぶつり}の\ruby{好}{す}きな子が、アルコールランプに火をつけて、フラスコをブクブクやったり、\ruby{何}{なに}かを\ruby{爆発}{ばくはつ}させてる、なんていう\ruby{光景}{こうけい}は、どの\ruby{教室}{きょうしつ}でもみられることだった。この\ruby{授業}{じゅぎょう}のやり\ruby{方}{かた}は、\ruby{上級}{じょうきゅう}になるにしたがって、その\ruby{子供}{こども}の\ruby{興味}{きょうみ}を\ruby{持}{も}っているもの、\ruby{興味}{きょうみ}の\ruby{持}{も}ち\ruby{方}{かた}、\ruby{物}{もの}の\ruby{考}{かんが}え\ruby{方}{かた}、そして、\ruby{個性}{こせい}、といったものが、先生に、はっきり\ruby{分}{わ}かってくるから、先生にとって、\ruby{生徒}{せいと}を\ruby{知}{し}る上で、\ruby{何}{なに}よりの\ruby{勉強法}{べんきょうほう}だった。

また、\ruby{生徒}{せいと}にとっても、\ruby{好}{す}きな\ruby{学科}{がっか}からやっていい、というのは、\ruby{嬉}{うれ}しいことだったし、\ruby{嫌}{きら}いな\ruby{学科}{がっか}にしても、学校が\ruby{終}{お}わる\ruby{時間}{じかん}までに、やればいいのだから、\ruby{何}{なん}とか、やりくり\ruby{出来}{でき}た。\ruby{従}{したが}って、\ruby{自習}{じしゅう}の\ruby{形式}{けいしき}が\ruby{多}{おお}く、いよいよ、\ruby{分}{わ}からなくなってくると、先生のところに\ruby{聞}{き}きに\ruby{行}{い}くか、\ruby{自分}{じぶん}の\ruby{席}{せき}に先生に\ruby{来}{き}ていただいて、\ruby{納得}{なっとく}の\ruby{行}{い}くまで、\ruby{教}{おし}えてもらう。そして、\ruby{例}{れい}\ruby{題}{だい}をもらって、また\ruby{自習}{じしゅう}に入る。これは\ruby{本当}{ほんとう}の\ruby{勉強}{べんきょう}だった。だから、先生の\ruby{話}{はなし}や\ruby{説明}{せつめい}を、ボンヤリ\ruby{聞}{き}く、といった\ruby{事}{こと}は、\ruby{無}{な}いにひとしかった。トットちゃん\ruby{達}{たち}、一年生は、まだ\ruby{自習}{じしゅう}をするほどの\ruby{勉強}{べんきょう}を\ruby{始}{はじ}めていなかったけど、それでも、\ruby{自分}{じぶん}の\ruby{好}{す}きな\ruby{科目}{かもく}から\ruby{勉強}{べんきょう}する、ということには、かわりなかった。カタカナを\ruby{書}{か}く子、\ruby{絵}{え}を\ruby{描}{か}く子。本を\ruby{読}{よ}んでる子。中には、\ruby{体操}{たいそう}をしている子もいた。トットちゃんの\ruby{隣}{となり}の女の子は、もう、ひらがなが\ruby{書}{か}けるらしく、ノートに\ruby{写}{うつ}していた。トットちゃんは、\ruby{何}{なに}もかもが\ruby{珍}{めずら}しくて、ワクワクしちゃって、みんなみたいに、すぐ\ruby{勉強}{べんきょう}、というわけにはいかなかった。そんな\ruby{時}{とき}、トットちゃんの\ruby{後}{うし}ろの\ruby{机}{つくえ}の男の子が立ち上がって、\ruby{黒板}{こくばん}のほうに\ruby{歩}{ある}き\ruby{出}{だ}した。ノートを\ruby{持}{も}って。\ruby{黒板}{こくばん}の\ruby{横}{よこ}の\ruby{机}{つくえ}で、\ruby{他}{ほか}の子に\ruby{何}{なに}かを\ruby{教}{おし}えている先生のところに\ruby{行}{い}くらしかった。その子の\ruby{歩}{ある}くのを、\ruby{後}{うし}ろから見たトットちゃんは、それまでキョロキョロしてた\ruby{動作}{どうさ}をピタリと\ruby{止}{と}めて、\ruby{頬杖}{ほおづえ}をつき、ジーっと、その子を見つめた。その子は、\ruby{歩}{ある}くとき、足を\ruby{引}{ひ}きずっていた。とっても、\ruby{歩}{ある}くとき、\ruby{体}{からだ}が\ruby{揺}{ゆ}れた。\ruby{始}{はじ}めは、わざとしているのか、と\ruby{思}{おも}ったくらいだった。でも、やっぱり、わざとじゃなくて、そういう\ruby{風}{かぜ}になっちゃうんだ、と、しばらく見ていたトットちゃんに\ruby{分}{わ}かった。その子が、\ruby{自分}{じぶん}の\ruby{机}{つくえ}に\ruby{戻}{もど}ってくるのを、トットちゃんは、さっきの、\ruby{頬杖}{ほおづえ}のまま、見た。目と目が\ruby{合}{あ}った。その男の子は、トットちゃんを見ると、ニコリと\ruby{笑}{わら}った。トットちゃんも、あわてて、ニコリとした。その子が、\ruby{後}{うし}ろの\ruby{席}{せき}に\ruby{座}{すわ}ると、――\ruby{座}{すわ}るのも、\ruby{他}{ほか}の子より、\ruby{時間}{じかん}がかかったんだけど――トットちゃんは、クルリと\ruby{振}{ふ}り\ruby{向}{む}いて、その子に\ruby{聞}{き}いた。「どうして、そんな\ruby{風}{ふう}に\ruby{歩}{ある}くの?」その子は、\ruby{優}{やさ}しい\ruby{声}{こえ}で\ruby{静}{しず}かに\ruby{答}{こた}えた。とても\ruby{利口}{りこう}そうな\ruby{声}{こえ}だった。「\ruby{僕}{ぼく}、\ruby{小児}{しょうに}\ruby{麻痺}{まひ}なんだ」「しょうにまひ?」トットちゃんは、それまで、そういう\ruby{言葉}{ことば}を\ruby{聴}{き}いたことが\ruby{無}{な}かったから、\ruby{聞}{き}き\ruby{返}{かえ}した。その子は、\ruby{少}{すこ}し小さい\ruby{声}{こえ}でいった。「そう、\ruby{小児}{しょうに}\ruby{麻痺}{まひ}。足だけじゃないよ。手だって……」そういうと、その子は、\ruby{長}{なが}い\ruby{指}{ゆび}と\ruby{指}{ゆび}が、くっついて、\ruby{曲}{ま}がったみたいになった手を出した。トットちゃんは、その左手を見ながら、「\ruby{直}{なお}らないの?」と\ruby{心配}{しんぱい}になって\ruby{聞}{き}いた。その子は、\ruby{黙}{だま}っていた。トットちゃんは、\ruby{悪}{わる}いことを\ruby{聞}{き}いたのかと\ruby{悲}{かな}しくなった。すると、その子は、\ruby{明}{あか}るい\ruby{声}{こえ}で\ruby{言}{い}った。「\ruby{僕}{ぼく}の\ruby{名前}{なまえ}は、やまもとやすあき。\ruby{君}{きみ}は?」トットちゃんは、その子が\ruby{元気}{げんき}な\ruby{声}{こえ}を出したので、\ruby{嬉}{うれ}しくなって、大きな\ruby{声}{こえ}で\ruby{言}{い}った。「トットちゃんよ」こうして、山本\ruby{泰明}{やすあき}ちゃんと、トットちゃんのお\ruby{友達}{ともだち}づきあいが\ruby{始}{はじ}まった。\ruby{電車}{でんしゃ}の中は、\ruby{暖}{あたた}かい\ruby{日差}{ひざ}しで、\ruby{暑}{あつ}いくらいだった。\ruby{誰}{だれ}かが、\ruby{窓}{まど}を\ruby{開}{ひら}けた。\ruby{新}{あたら}しい\ruby{春}{はる}の\ruby{風}{かぜ}が、\ruby{電車}{でんしゃ}の中を\ruby{通}{とお}り\ruby{抜}{ぬ}け、\ruby{子供}{こども}たちの\ruby{髪}{かみ}の\ruby{毛}{け}が\ruby{歌}{うた}っているように、とびはねた。トットちゃんの、トモエでの\ruby{第}{だい}一目は、こんな\ruby{風}{ふう}に\ruby{始}{はじ}まったのだった。



\end{document}