\ruby{砂}{すな}と\ruby{岩}{いわ}と\ruby{雪}{ゆき}の\ruby{中}{なか}を\ruby{長}{なが}い\ruby{間}{あいだ}\ruby{歩}{ある}いてきた\ruby{王子}{おうじ}さまは、ようやく\ruby一{本}{いっぽん}の\ruby{道}{みち}を\ruby{見}{み}つけた。そして、\ruby{道}{みち}は\ruby{必}{かなら}ず\ruby{人間}{にんげん}が\ruby{居}{い}る\ruby{場所}{ばしょ}へと\ruby{通}{つう}じている。\ruby{王子}{おうじ}さまの\ruby{行}{い}き\ruby{着}{つ}いた\ruby{先}{さき}はバラの\ruby{花}{はな}が\ruby{咲}{さ}き\ruby{揃}{そろ}った\ruby{庭園}{ていえん}だった。

「こんにちは。」

「こんにちは。」

\ruby{王子}{おうじ}さまはバラたちを\ruby{凝視}{ぎょうし}した。どれも\ruby{王子}{おうじ}さまの\ruby{花}{はな}にそっくりだった。

「キミたちは\ruby{誰}{だれ}なの?」

「\ruby{私}{わたし}たちはバラよ。」

「そんな……」

\ruby{王子}{おうじ}さまはとても\ruby{悲}{かな}しい\ruby{気持}{きも}ちになった。\ruby{王子}{おうじ}さまの\ruby{花}{はな}は\ruby{自分}{じぶん}は\ruby{宇宙}{うちゅう}でたった\ruby{一}{ひと}つだけの\ruby{存在}{そんざい}と\ruby{語}{かた}っていた。それなのに、この\ruby{庭園}{ていえん}だけで\ruby{同}{おな}じ\ruby{花}{はな}が五千\ruby{本}{ほん}もあるなんて。

あの\ruby{花}{はな}がこれを\ruby{見}{み}たら、ひどく\ruby{傷}{きず}つくだろうな。\ruby{笑}{わら}いものにならないように、\ruby{激}{はげ}しく\ruby{咳}{せき}をして\ruby{死}{し}んだふりをするかも。そして\ruby{僕}{ぼく}は\ruby{花}{はな}を\ruby{介抱}{かいほう}するふりをしなきゃいけなくなるんだ。

そうしないと\ruby{僕}{ぼく}に\ruby{恥}{は}じ\ruby{入}{はい}らすようとして\ruby{本当}{ほんとう}に\ruby{死}{し}んでしまう。

そして\ruby{王子}{おうじ}さまはこう\ruby{思}{おも}った。

この\ruby{世}{よ}に\ruby{一}{ひと}つしかない\ruby{花}{はな}を\ruby{持}{も}っていて、\ruby{豊}{ゆた}かだと\ruby{思}{おも}っていたけど、\ruby{僕}{ぼく}が\ruby{持}{も}っていたのはただの\ruby{有}{あ}り\ruby{触}{ふ}れたバラの\ruby{花}{はな}だったんだ。\ruby{後}{あと}は\ruby{膝}{ひざ}までの\ruby{高}{たか}さしかない\ruby{三}{みっ}つの\ruby{火山}{かざん}、そのうちの\ruby{一}{ひと}つは\ruby{永久}{えいきゅう}に\ruby{火}{ひ}が\ruby{消}{き}えたままかもしれない。これじゃ\ruby{僕}{ぼく}は\ruby{立派}{りっぱ}な\ruby{王子}{おうじ}にはなれないよ。

そして\ruby{王子}{おうじ}さまは\ruby{草}{くさ}の\ruby{上}{うえ}に\ruby{突}{つ}っ\ruby{伏}{ぷ}して\ruby{泣}{な}いた。


