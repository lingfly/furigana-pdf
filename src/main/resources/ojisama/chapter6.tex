ああ、\ruby{小}{ちい}さな\ruby{王子}{おうじ}さま。こうして\ruby{僕}{ぼく}は\ruby{少}{すこ}しずつ、ささやかで\ruby{憂鬱}{ゆううつ}な\ruby{君}{きみ}の\ruby{人生}{じんせい}を\ruby{理解}{りかい}していった。\ruby{長}{なが}い\ruby{間}{あいだ}、\ruby{君}{きみ}には\ruby{美}{うつく}しい\ruby{夕日}{ゆうひ}しか\ruby{心}{こころ}を\ruby{慰}{なぐさ}める\ruby{物}{もの}がなかった\ruby{事}{こと}も。\ruby{僕}{ぼく}がこの\ruby{秘密}{ひみつ}を\ruby{知}{し}ったのは、\ruby{四日}{よっか}\ruby{目}{め}の\ruby{朝}{あさ}。\ruby{君}{きみ}がこう\ruby{言}{い}った\ruby{時}{とき}だ。

\ruby{僕}{ぼく}、\ruby{夕日}{ゆうひ}が\ruby{大好}{だいす}きなんだ。\ruby{夕日}{ゆうひ}を\ruby{見}{み}に\ruby{行}{い}こうよ。

でも、\ruby{待}{ま}たなきゃね。

\ruby{待}{ま}つって、\ruby{何}{なに}を?

\ruby{日}{ひ}が\ruby{沈}{しず}むのをさ。

\ruby{君}{きみ}はとてもびっくりしたようだった。そして、すぐに\ruby{笑}{わら}い\ruby{出}{だ}した。

\ruby{僕}{ぼく}、まだ\ruby{自分}{じぶん}の\ruby{星}{ほし}にいるつもりになっていたよ。

そうだね。

\ruby{誰}{だれ}もが\ruby{知}{し}っているように、アメリカが\ruby{正午}{しょうご}の\ruby{時}{とき}にはフランスは\ruby{夕暮}{ゆうぐ}れだ。だから、\ruby{一分}{いっぶん}でフランスに\ruby{飛}{と}んで\ruby{行}{おこな}けたら、\ruby{夕日}{ゆうひ}を\ruby{見}{み}る\ruby{事}{こと}ができるけど、\ruby{残念}{ざんねん}ながら、フランスは\ruby{遠}{とお}すぎる。だけど\ruby{君}{きみ}の\ruby{小}{ちい}さな\ruby{星}{ほし}では、ほんの\ruby{少}{すこ}し\ruby{椅子}{いす}を\ruby{動}{うご}かすだけでいい、そうすれば\ruby{見}{み}たい\ruby{時}{とき}に\ruby{何時}{なんじ}でも、\ruby{黄昏}{たそがれ}を\ruby{眺}{なが}めていられる。

\ruby{僕}{ぼく}ね、一\ruby{日}{にち}に44\ruby{回}{かい}も\ruby{夕日}{ゆうひ}を\ruby{見}{み}た\ruby{事}{こと}があるよ。

そう\ruby{言}{い}って、\ruby{暫}{しばら}くしてからこう\ruby{付加}{つけくわ}えた。

ね、\ruby{悲}{かな}しくてたまらない\ruby{時}{とき}って、\ruby{夕日}{ゆうひ}が\ruby{恋}{こい}しくなるよね。

44\ruby{回}{かい}も\ruby{夕日}{ゆうひ}を\ruby{見}{み}た\ruby{日}{ひ}は、\ruby{悲}{かな}しくてたまらなかったのかい?

しかし、\ruby{王子}{おうじ}さまは\ruby{答}{こた}えなかった。


