キツネが\ruby{現}{あらわ}れたのはその\ruby{時}{とき}だった。

「こんにちは。」

「こんにちは。」

\ruby{王子}{おうじ}さまは\ruby{丁寧}{ていねい}に\ruby{答}{こた}えたが、\ruby{振}{ふ}り\ruby{返}{かえ}っても\ruby{誰}{だれ}もいなかった。

「ここだよ。リンゴの\ruby{木}{き}の\ruby{下}{した}さ。」

「\ruby{君}{きみ}は\ruby{誰}{だれ}?とっても\ruby{可愛}{かわい}いね。」

「\ruby{僕}{ぼく}、キツネだよ。」

「\ruby{一緒}{いっしょ}に\ruby{遊}{あそ}ぼう。\ruby{僕}{ぼく}、\ruby{今}{いま}とっても\ruby{悲}{かな}しいんだ。」

「\ruby{君}{きみ}とは\ruby{遊}{あそ}べない。\ruby{飼}{か}い\ruby{馴}{な}らされていないから。」

「ああ、ごめんね。でも、\ruby{飼}{か}い\ruby{馴}{な}らすってどういう\ruby{意味}{いみ}?」

「\ruby{君}{きみ}はこの\ruby{辺}{へん}の\ruby{人}{ひと}じゃないね。\ruby{何}{なに}を\ruby{探}{さが}しているんだい? 」

「\ruby{人間}{にんげん}だよ。ね、\ruby{飼}{か}い\ruby{馴}{な}らすってどういう\ruby{意味}{いみ}?」

「\ruby{人間}{にんげん}は\ruby{銃}{じゅう}を\ruby{持}{も}っていて\ruby{狩}{か}りをする、\ruby{全}{まった}く\ruby{困}{こま}ったものだ。でも、\ruby{鶏}{にわとり}を\ruby{飼}{か}っている。いいところはそこだけかな。\ruby{君}{きみ}、\ruby{鶏}{にわとり}を\ruby{探}{さが}しているの?」

「\ruby{違}{ちが}うよ。\ruby{探}{さが}しているのは\ruby{友達}{ともだち}だ。\ruby{飼}{か}い\ruby{馴}{な}らすって、どういう\ruby{意味}{いみ}?」

「\ruby{皆}{みな}がすっかり\ruby{忘}{わす}れていることだよ。\ruby{絆}{きずな}を\ruby{作}{つく}るって\ruby{意味}{いみ}だ。」

「\ruby{絆}{きずな}を\ruby{作}{つく}る?」

「そうさ、\ruby{僕}{ぼく}にとって\ruby{君}{きみ}はまだ\ruby{他}{ほか}の十万\ruby{人}{にん}の\ruby{男}{おとこ}の\ruby{子}{こ}と\ruby{同}{おな}じ、ただの\ruby{男}{おとこ}の\ruby{子}{こ}だ。\ruby{僕}{ぼく}には\ruby{君}{きみ}は\ruby{必要}{ひつよう}ないし、\ruby{君}{きみ}にも\ruby{僕}{ぼく}は\ruby{必要}{ひつよう}ない。\ruby{君}{きみ}にとって\ruby{僕}{ぼく}はまだ\ruby{他}{ほか}の十万\ruby{匹}{びき}のキツネと\ruby{同}{おな}じ、ただのキツネだからね。だけど、\ruby{君}{きみ}が\ruby{僕}{ぼく}を\ruby{飼}{か}い\ruby{馴}{な}らしたら、\ruby{僕}{ぼく}たちは\ruby{互}{たが}いに\ruby{必要}{ひつよう}\ruby{不可欠}{ふかけつ}な\ruby{存在}{そんざい}になる。\ruby{僕}{ぼく}にとって\ruby{君}{きみ}は\ruby{世界}{せかい}でたった\ruby{一人}{ひとり}だけの\ruby{男}{おとこ}の\ruby{子}{こ}、\ruby{君}{きみ}にとって\ruby{僕}{ぼく}は\ruby{世界}{せかい}でたった一\ruby{匹}{ひき}だけのキツネ。 」

「だんだんわかってきたよ。ある\ruby{花}{はな}のことだけど、その\ruby{花}{はな}は\ruby{僕}{ぼく}を\ruby{飼}{か}い\ruby{馴}{な}らしていたんだと\ruby{思}{おも}うな。」

「そういうこともあるかもね。\ruby{地球}{ちきゅう}では\ruby{何}{なん}でもあるからね。」

「ああ、\ruby{地球}{ちきゅう}の\ruby{話}{はなし}じゃないんだよ。」

「えっ?\ruby{他}{ほか}の\ruby{星}{ほし}?」

「そう。」

「その\ruby{星}{ほし}には、\ruby{猟師}{りょうし}はいる?」

「いないよ。」

「そいつはいいや。\ruby{鶏}{にわとり}はいる? 」

「いないね。」

「\ruby{思}{おも}い\ruby{通}{どお}りにいかないもんだな。まあいいや、\ruby{話}{はなし}を\ruby{続}{つづ}けよう。\ruby{僕}{ぼく}の\ruby{暮}{く}らしは\ruby{単調}{たんちょう}だよ。\ruby{僕}{ぼく}が\ruby{鶏}{にわとり}を\ruby{追}{お}う、\ruby{人間}{にんげん}が\ruby{僕}{ぼく}を\ruby{追}{お}う。\ruby{鶏}{にわとり}は\ruby{皆}{みな}\ruby{同}{おな}じ、\ruby{人間}{にんげん}も\ruby{皆}{みな}\ruby{同}{おな}じ、おかげで、\ruby{些}{いささ}か\ruby{退屈}{たいくつ}しているんだ。でも、もし\ruby{君}{きみ}が\ruby{僕}{ぼく}を\ruby{飼}{か}い\ruby{馴}{な}らしてくれたら、\ruby{僕}{ぼく}の\ruby{暮}{く}らしはお\ruby{日}{ひ}さまが\ruby{当}{あ}たったみたいになるよ。\ruby{僕}{ぼく}は\ruby{足音}{あしおと}が\ruby{聞}{き}き\ruby{分}{わ}けられる。\ruby{誰}{だれ}かの\ruby{足}{あし}\ruby{音}{おと}が\ruby{聞}{き}こえたら、\ruby{僕}{ぼく}は\ruby{慌}{あわ}てて\ruby{地面}{じめん}に\ruby{潜}{もぐ}る。でも\ruby{君}{きみ}の\ruby{足音}{あしおと}は\ruby{音楽}{おんがく}みたいに\ruby{僕}{ぼく}を\ruby{穴}{あな}から\ruby{誘}{さそ}い\ruby{出}{だ}す。それに、ほら、あそこに\ruby{小麦畑}{こむぎばたけ}が\ruby{見}{み}えるでしょう。\ruby{僕}{ぼく}はパンを\ruby{食}{た}べないから、\ruby{小麦}{こむぎ}には\ruby{全}{まった}く\ruby{用}{よう}がないんだ。だから\ruby{小麦畑}{こむぎばたけ}を\ruby{見}{み}ても\ruby{何}{なに}も\ruby{感}{かん}じない。\ruby{悲}{かな}しい\ruby{話}{はなし}だけどね。でも、\ruby{君}{きみ}は\ruby{金色}{きんいろ}の\ruby{髪}{かみ}をしているよね。だから、\ruby{君}{きみ}が\ruby{僕}{ぼく}を\ruby{飼}{か}い\ruby{馴}{な}らしてくれたら、\ruby{素晴}{すば}らしい\ruby{事}{こと}になる。\ruby{金色}{きんいろ}の\ruby{小麦}{こむぎ}を\ruby{見}{み}るたびに、\ruby{僕}{ぼく}は\ruby{君}{きみ}の\ruby{事}{こと}を\ruby{思}{おも}い\ruby{出}{だ}すようになるよ。\ruby{小麦畑}{こむぎばたけ}を\ruby{渡}{わた}っていく\ruby{風}{かぜ}の\ruby{音}{おと}さえ\ruby{好}{す}きになるよ。」

キツネはふと\ruby{黙}{だま}って、\ruby{長}{なが}い\ruby{間}{あいだ}\ruby{王子}{おうじ}さまを\ruby{見}{み}つめていた。

「お\ruby{願}{ねが}い、\ruby{僕}{ぼく}を\ruby{飼}{か}い\ruby{馴}{な}らして。」

「そうしたいんだけど、あんまり\ruby{時間}{じかん}がないんだ。\ruby{友達}{ともだち}を\ruby{見付}{みつ}けて、いろいろたくさん\ruby{学}{まな}ばなきゃいけないし。」

「\ruby{飼}{か}い\ruby{馴}{な}らさなきゃ\ruby{学}{まな}べないよ。\ruby{人間}{にんげん}には\ruby{学}{まな}ぶ\ruby{時間}{じかん}なんかない。お\ruby{店}{みせ}で\ruby{溺愛}{できあい}のものを\ruby{買}{か}ってくるだけさ。でも、\ruby{友達}{ともだち}を\ruby{買}{か}えるお\ruby{店}{みせ}はないから、\ruby{人間}{にんげん}にはもう\ruby{友達}{ともだち}がいないんだ。\ruby{友達}{ともだち}が\ruby{欲}{ほ}しかったら、\ruby{僕}{ぼく}を\ruby{飼}{か}い\ruby{馴}{な}らして。」

「\ruby{僕}{ぼく}はどうすればいいの。」

「とっても\ruby{辛抱強}{しんぼうづよ}くならなきゃね。まず、\ruby{僕}{ぼく}からちょっと\ruby{離}{はな}れて、\ruby{草}{くさ}の\ruby{中}{なか}に\ruby{座}{すわ}るんだ。\ruby{僕}{ぼく}は\ruby{横目}{よこめ}で\ruby{君}{きみ}を\ruby{見}{み}て、\ruby{君}{きみ}は\ruby{何}{なに}も\ruby{言}{い}わない、\ruby{言葉}{ことば}は\ruby{誤解}{ごかい}のもとだから。でも、\ruby{毎日}{まいにち}\ruby{少}{すこ}しずつ、だんだん\ruby{近}{ちか}くに\ruby{座}{すわ}れるようになるんだ。」

\ruby{次}{つぎ}の\ruby{日}{ひ}、\ruby{王子}{おうじ}さまは\ruby{戻}{もど}ってきた。

「できたら、\ruby{同}{おな}じ\ruby{時間}{じかん}に\ruby{戻}{もど}ってきたほうがいいよ。\ruby{例}{たと}えば、4\ruby{時}{じ}に\ruby{君}{きみ}が\ruby{来}{く}るとすると、\ruby{僕}{ぼく}は3\ruby{時}{じ}から\ruby{嬉}{うれ}しくなってくる。\ruby{時間}{じかん}が\ruby{経}{た}つにつれて、ますます\ruby{嬉}{うれ}しくなってくる。4\ruby{時}{じ}になると、そわそわして\ruby{気}{き}も\ruby{漫}{そぞ}ろさ。\ruby{幸福}{こうふく}ってどんなものかを\ruby{知}{し}るんだ。でも、\ruby{君}{きみ}が\ruby{何時}{なんじ}と\ruby{決}{き}めず、\ruby{適当}{てきとう}に\ruby{来}{く}ると、\ruby{何時}{なんじ}に\ruby{心}{こころ}の\ruby{準備}{じゅんび}を\ruby{始}{はじ}めればいいのか\ruby{分}{わ}からなくなる。\ruby{習慣}{しゅうかん}にする\ruby{事}{こと}が\ruby{大事}{だいじ}なんだよ。 」

「\ruby{習慣}{しゅうかん}って、\ruby{何}{なに}なの?」

「\ruby{随分}{ずいぶん}と\ruby{忘}{わす}れがちなもののことさ。ある一\ruby{日}{にち}を\ruby{他}{ほか}の\ruby{日}{ひ}と\ruby{区別}{くべつ}し、ある\ruby{時間}{じかん}を\ruby{他}{ほか}の\ruby{時間}{じかん}と\ruby{区別}{くべつ}するんだ。\ruby{例}{たと}えば、\ruby{僕}{ぼく}を\ruby{追}{お}い\ruby{回}{まわ}す\ruby{猟師}{りょうし}たちにも\ruby{習慣}{しゅうかん}がある。\ruby{毎週}{まいしゅう}\ruby{木曜日}{もくようび}は\ruby{狩}{か}りをせず、\ruby{村}{むら}の\ruby{娘}{むすめ}たちと\ruby{踊}{おど}るのさ。だから\ruby{木曜日}{もくようび}は\ruby{素晴}{すば}らしい\ruby{日}{ひ}だ。\ruby{僕}{ぼく}は\ruby{葡萄}{ぶどう}\ruby{畑}{はたけ}の\ruby{辺}{あた}りまで\ruby{散歩}{さんぽ}に\ruby{行}{おこな}ける。でも、もし\ruby{猟師}{りょうし}たちが\ruby{何時}{いつ}でも\ruby{好}{す}きな\ruby{日}{ひ}に\ruby{踊}{おど}ったら、\ruby{毎日}{まいにち}が\ruby{皆}{みな}\ruby{同}{おな}じになって、\ruby{僕}{ぼく}は\ruby{全}{まった}く\ruby{休暇}{きゅうか}がとれなくなる。」

こうして\ruby{王子}{おうじ}さまはキツネを\ruby{飼}{か}い\ruby{馴}{な}らした。\ruby{出発}{しゅっぱつ}の\ruby{時}{とき}が\ruby{近付}{ちかづ}くと、キツネは\ruby{言}{い}った。

「ああ、\ruby{泣}{な}けてきちゃうよ。

「\ruby{君}{きみ}のせいだよ。\ruby{僕}{ぼく}は\ruby{君}{きみ}を\ruby{困}{こま}らせたくなかったのに、\ruby{君}{きみ}が\ruby{飼}{か}い\ruby{馴}{な}らしてなんて\ruby{言}{い}ったから。」

「そうだよ。その\ruby{通}{とお}りだよ。」

「でも、\ruby{君}{きみ}は\ruby{泣}{な}くんだ。」

「そうだよ。その\ruby{通}{とお}りだよ。」

\ruby{王子}{おうじ}:だったら、\ruby{君}{きみ}は\ruby{損}{そん}しちゃったんじゃないか。

「\ruby{僕}{ぼく}は\ruby{得}{とく}したんだよ。\ruby{小麦色}{こむぎいろ}の\ruby{分}{ぶん}だけ。さあ、もう\ruby{一度}{いちど}\ruby{庭園}{ていえん}に\ruby{足}{あし}を\ruby{運}{はこ}んで、バラたちを\ruby{見}{み}てきてごらん?\ruby{君}{きみ}のバラは\ruby{世界}{せかい}にたった\ruby{一}{ひと}つしかないバラの\ruby{花}{はな}だって、わかるから。そうしたら、\ruby{戻}{もど}ってきて\ruby{僕}{ぼく}にさよならを\ruby{言}{い}って、お\ruby{別}{わか}れに\ruby{秘密}{ひみつ}を\ruby{一}{ひと}つあげるから。」

\ruby{王子}{おうじ}さまはもう\ruby{一度}{いちど}バラたちを\ruby{見}{み}に\ruby{行}{い}った。そして、\ruby{言}{い}った。

「キミたちはどれも\ruby{僕}{ぼく}のバラとは\ruby{全然}{ぜんぜん}\ruby{似}{に}ていないよ。キミたちはまだ\ruby{僕}{ぼく}にとっては\ruby{取}{と}るに\ruby{足}{た}りない\ruby{存在}{そんざい}だ。\ruby{飼}{か}い\ruby{馴}{な}らされていないし、\ruby{飼}{か}い\ruby{馴}{な}らしてもいないもの。\ruby{会}{あ}ったばかりの\ruby{頃}{ころ}の\ruby{僕}{ぼく}のキツネみたいだ。あのキツネは\ruby{他}{ほか}の十万\ruby{匹}{びき}のキツネと\ruby{同}{おな}じ、ただのキツネだった。

でも\ruby{僕}{ぼく}はキツネと\ruby{友達}{ともだち}になった。\ruby{今}{いま}では、\ruby{世界}{せかい}に一\ruby{匹}{ひき}だけのキツネだよ。キミたちは\ruby{綺麗}{きれい}さ。でも、\ruby{空}{から}っぽなんだ。\ruby{誰}{だれ}もキミたちのためには\ruby{死}{し}んでない。\ruby{勿論}{もちろん}、\ruby{普通}{ふつう}の\ruby{通}{とお}りすがりの\ruby{人}{ひと}は\ruby{僕}{ぼく}のバラをキミたちと\ruby{同}{おな}じだと\ruby{思}{おも}うだろう。でも、\ruby{僕}{ぼく}の\ruby{花}{はな}はたった\ruby{一}{ひと}つで、キミたち\ruby{全部}{ぜんぶ}を\ruby{合}{あ}わせたよりも\ruby{大切}{たいせつ}なんだ。だって、\ruby{僕}{ぼく}が\ruby{水}{みず}をかけてあげたのはあの\ruby{花}{はな}だから。

ガラスの\ruby{覆}{おお}い をかぶせてあげたのも、\ruby{衝立}{ついたて}で\ruby{守}{まも}ってあげたのも、ちょうちょになる二三\ruby{匹}{ひき}を\ruby{残}{のこ}して\ruby{毛虫}{けむし}を\ruby{退治}{たいじ}してあげたのも、\ruby{文句}{もんく}を\ruby{言}{い}ったり、\ruby{自慢}{じまん}したり、\ruby{時々}{ときどき}\ruby{黙}{だま}り\ruby{込}{こ}んだりするのにさえ、\ruby{耳}{みみ}を\ruby{傾}{かたむ}けてあげたのも、あの\ruby{花}{はな}だけだから。なぜってあの\ruby{花}{はな}は\ruby{僕}{ぼく}のバラの\ruby{花}{はな}だから。」

そして\ruby{王子}{おうじ}さまはキツネのところに\ruby{戻}{もど}った。

「さよならだね。」

「ああ、さよならだ。じゃ、\ruby{秘密}{ひみつ}を\ruby{教}{おし}えるよ。\ruby{簡単}{かんたん}な\ruby{事}{こと}だ。\ruby{心}{こころ}で\ruby{見}{み}なければ、\ruby{物事}{ものごと}はよく\ruby{見}{み}えない。\ruby{一番}{いちばん}\ruby{大切}{たいせつ}な\ruby{事}{こと}は、\ruby{目}{め}に\ruby{見}{み}えない。」

「\ruby{一番}{いちばん}\ruby{大切}{たいせつ}な\ruby{事}{こと}は、\ruby{目}{め}に\ruby{見}{み}えない。」

「\ruby{君}{きみ}のバラを\ruby{何}{なに}よりも\ruby{大切}{たいせつ}な\ruby{物}{もの}にしたのは、\ruby{君}{きみ}がバラのために\ruby{費}{つい}やした\ruby{時間}{じかん}なんだ。」

「\ruby{僕}{ぼく}がバラのために\ruby{費}{つい}やした\ruby{時間}{じかん}。」

「\ruby{人間}{にんげん}はこの\ruby{真義}{しんぎ}を\ruby{忘}{わす}れてしまった。でも、\ruby{君}{きみ}は\ruby{忘}{わす}れてはいけないよ。\ruby{君}{きみ}は\ruby{飼}{か}い\ruby{馴}{な}らしたものに\ruby{永遠}{えいえん}に\ruby{責任}{せきにん}があるんだ。だから\ruby{君}{きみ}は\ruby{君}{きみ}のバラに\ruby{責任}{せきにん}がある。」

「\ruby{僕}{ぼく}は\ruby{僕}{ぼく}のバラに\ruby{責任}{せきにん}がある。」


